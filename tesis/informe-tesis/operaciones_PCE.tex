\chapter{OPERACIONES PCE}

\section{Introducción}
\cpnote{Porfa tambien esqueleto de la into}
\esqueleto{ 
\begin{itemize}
\item Hablar que es importante estudiar problemas de muchas 
partículas.
\item Con la teoría de los canales cuánticos se pueden estudiar una 
gran variedad de sistemas, siendo uno de ellos los de sistemas de qubits.
\item Hablar qué son los qubits: sistemas cuánticos de dos niveles. 
\item Escribir sobre la estructura del capítulo.
\end{itemize}
}

\section{Operaciones PCE}
%\esqueleto{
%Dos o tres párrafos máximo hablando sobre las operaciones PCE 
%como operaciones proyectivas a la base de productos tensoriales de 
%las matrices de Pauli. Me gustaría poner un problema de motivación
%sólo para hacer más interesante la lectura a partir de acá y dejarle al
%lector algo con lo que pueda entender qué onda con las operaciones 
%PCE. Propongo lo siguiente: cadena de espines. Supongamos una 
%cadena de espines de $N$ sitios y decimos que nos interesa saber 
%si la matriz de densidad del $i$-ésimo espín puede proyectarse 
%al subespacio cuya base son $\sigma_x$ y $\sigma_y$ (lo que 
%quiero decir es que $(r_x,r_y,r_z)\to(r_x,r_y,0)$). Entonces hablo 
%de que hay que considerar que el $i$-ésimo espín puede estar 
%entrelazado con el resto de la cadena, sin embargo, basta con considerar
%que se encuentra entrelazado con cualquiera de sus vecinos y revisar 
%en qué se transforma la matriz de densidad de esos dos espines para 
%averiguar si es posible tal evolución física. Me gustaría poner unas 
%figuritas para hacer interesante esto. 
%}

\cpnote{Siento qe aun no es un esqueleto sino ideas. De la motivación
no me gusta como la abordas. Yo pensaría mejor en hablar un poco de defasing
y de bitflip para uno y muchos qubits. Siento qeu tu propuesta está
aun un poco por fuera de tu alcance y te propondría mantenerla más simple.
Si quieres iteremos la propuesta, y cuando nos pongamos de acuerdo ya haces un esqueleto 
mas a nivel de parrafos.}

\esqueleto{
\begin{itemize}
\item Para introducir el contexto de las operaciones PCE voy mencionar sobre
el defasing y el bitflip channel cuando $p=0.5$.
\item Luego hablar sobre
la extensión de este tipo de canales cuánticos para sistemas de 2 
qubits. Mencionar que es intuitivo entender que localmente, sobre cada qubit,  
las únicas evoluciones físicas son las que ya se conocen para 1 qubit 
(bitflip, defasing y bit-phase flip).
\item Sin embargo, el problema cobra interés
al considerar que el estado de 2 qubits es más que 
sólo la suma del estado del qubit 1 y del qubit 2, ya que aparecen 
las correlaciones cuánticas.
\item Mencionar que 
vamos a entender a este tipo de canales cuánticos como operaciones 
que `borran' las componentes del vector de Bloch generalizado de un 
sistema de qubits.
\item Por último, mencionar que 
el entrelazamiento en sistemas de muchas partículas nos conduce
a preguntarnos de qué forma puede un canal cuántico las correlaciones
cuánticas de un sistema de qubits.
\end{itemize}
}

\section{1 qubit}
%\esqueleto{
%Resumen de los resultados de 1 qubit. Para seguir un orden lógico
%en este capítulo voy a hablar de los resultados de 1 qubit con el 
%problema resuelto analíticamente (está medio desordenado 
%si hablo por acá del método numérico y luego lo vuelvo a hacer 
%en la sección 2.5), a partir de las desigualdades
%de los eigenvalores (el polihedro?).
%}

\cpnote{Está bien. Acá antes de escribir quiero un esqueleto más detallado. Lo mismo para las siguientes secciones, 2.3, 2.4 y 2.5}

\esqueleto{ 
\begin{itemize}
\item Escribir a $\rho$ de 1 qubit en la base de las matrices de Pauli.
\item Enunciar las 8 operaciones PCE de 1 qubit.
\item Desarrollar cómo verificar analíticamente si las operaciones PCE 
de 1 qubit son CP. Es decir, partir de 
\begin{align*}
\left(
\begin{array}{cccc}
 1 & 0 & 0 & 0 \\
 0 & \tau _1 & 0 & 0 \\
 0 & 0 & \tau _2 & 0 \\
 0 & 0 & 0 & \tau _3 \\
\end{array}
\right)
\end{align*}
calcular el estado de Jamiolkowsky y verificar si es positivo. Con esto llegaré 
a las desigualdades.
\item Con las desigualdades revisar los casos para sacar los 5 canales 
cuánticos PCE.
\item Discutir que las operaciones PCE que borran 1 y 2 componentes 
de Bloch (3 operaciones cada una) son equivalentes bajo permutación 
de los elementos de la base de las matrices de Pauli. Esto me servirá 
para retomarlo después con el argumento de que los canales cuánticos
son equivalentes con permutaciones locales de los elementos de la base 
y de los swaps de partículas. 
\item Final: resumencillo de que hay 8 operaciones PCE, pero sólo 5 son 
CP (y por tanto evoluciones físicas).
\end{itemize}
}

\section{El problema de $\mathbf{n}$ qubits}
\esqueleto{ 
\begin{itemize}
\item Escribir a la matriz de densidad de $n$ qubits en la base 
de productos de las matrices de Pauli. Me gustó mucho la notación 
que usé en el último documento para escribir la expresión de los eigenvalores
de $n$ qubits, así que qué te parece escribirla de la siguiente manera?
\begin{align}
\rho = \frac{1}{2^n}\qty(
1
+
\sum_{\underset{(j_1,\ldots,j_n\neq0)}{j_1,\ldots,j_n=0}}^3
r_{j_1,\ldots,j_n}
\sigma_{j_1} \ot \ldots \ot \sigma_{j_n}
).
\end{align}
Yo la siento más sencilla de entender que como está en el Nielsen y Chuang.
\item Las operaciones PCE de $n$ qubits se definen entonces como 
las operaciones lineales que transforman a las componentes 
de la matriz de densidad como $r_{j_1,\ldots,j_n}\to
\tau_{j_1,\ldots,j_n}r_{j_1,\ldots,j_n}$ donde $\tau_{j_1,\ldots,j_n}=0,1$
son los elementos de la diagonal del superoperador $\E$.
\item Hacer énfasis que la única condición que debe satisfacer una operación
PCE para ser un canal cuántico es ser CP. Por lo cual, nuestro estudie 
se reduce a eso: estudiar la CP de las operaciones PCE.
\item Enunciar los pasos para verificar la CP para una operación de $n$ 
qubits verificando la positividad de la matriz de Choi 
a partir del superoperador $\E$ en la base que es diagonal
\item Concluir remarcando la idea clave de nuestro estudio que es investigar 
la CP de las operaciones PCE.
\end{itemize}
}

\section{Solución numérica}
\esqueleto{ 
\begin{itemize}
\item Resolvimos numéricamente la verificación de la CP para 
2 qubits completo y para 3 parcialmente implementando la verificación
de la positividad de la matriz de Choi. Decir que fue imposible verificar 
todas las operaciones de 3 qubits porque son un montón.
\item Hablar de las herramientas que se desarrollaron en Mathematica.
\item Colocar link al repositorio donde habrá un .nb para probar las 
funciones y la verificación de la que estoy hablando en esta sección.
\item Colocar una gráfica de tiempo de ejecución para justificar que 
no teníamos tiempo suficiente para dejar tostando la computadora 
como 1 año para esperar los resultados completos de 3 qubits.
\item Concluir que si bien el método numérico nos dio `poquitos 
resultados', esos resultados sirven muchísimo para extraer mucha 
información y ganar intuición acerca de los canales cuánticos PCE.
\end{itemize}
}
