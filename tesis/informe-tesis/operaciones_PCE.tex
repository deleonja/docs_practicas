\chapter{OPERACIONES PCE}

\section{Introducción} % {{{
\cpnote{Porfa tambien esqueleto de la into}
\esqueleto{ 
\begin{itemize}
\item Hablar que es importante estudiar problemas de muchas 
partículas.
\item Con la teoría de los canales cuánticos se pueden estudiar una 
gran variedad de sistemas, siendo uno de ellos los de sistemas de qubits.
\item Hablar qué son los qubits: sistemas cuánticos de dos niveles. 
\cpnote{Creo que es un poco tarde para hablar de qubits. Ya mencionaste las matrices
de Pauli, y sería buen momento antes, o de plano no decir nada.}
\item Escribir sobre la estructura del capítulo.
\end{itemize}
}

\esqueleto{
\hrule\vspace{10pt}
\h{Párrafos}
\begin{itemize}
\item Los sistemas cuánticos de dos niveles han sido y siguen 
siendo de gran importancia teórica y práctica para la mecánica 
cuántica y sus aplicaciones.
\item La teoría de los canales cuánticos se puede utilizar para estudiar 
la dinámica de un gran variedad de sistemas cuánticos abiertos, pero el objetivo
de este proyecto está centrado en el estudio de los canales cuánticos de qubits.
\item La estructura del capítulo.
\end{itemize}
}

El tipo de operaciones que estudiamos en este trabajo son operaciones 
que actúan sobre sistemas de qubits. Un qubit es un sistema cuántico 
de dos niveles \janote{cita}, como una partícula de espín 1/2. Los 
canales cuánticos de 1 qubit son especialmente útiles para ganar intuición
sobre los canales cuánticos ya que los estados de 1 qubit se 
pueden representar geométricamente en la esfera de Bloch \janote{cita}.
Los estados de 1 qubit en la esfera de Bloch se pueden relacionar con 
su matriz de densidad al escribir esta en la base de las matrices 
de Pauli como 
\begin{align}
\rho=\frac{1}{2}\sum_{i=0}^3r_i\sigma_i
\end{align}

% }}}
\section{Operaciones PCE} % {{{
%\esqueleto{
%Dos o tres párrafos máximo hablando sobre las operaciones PCE 
%como operaciones proyectivas a la base de productos tensoriales de 
%las matrices de Pauli. Me gustaría poner un problema de motivación
%sólo para hacer más interesante la lectura a partir de acá y dejarle al
%lector algo con lo que pueda entender qué onda con las operaciones 
%PCE. Propongo lo siguiente: cadena de espines. Supongamos una 
%cadena de espines de $N$ sitios y decimos que nos interesa saber 
%si la matriz de densidad del $i$-ésimo espín puede proyectarse 
%al subespacio cuya base son $\sigma_x$ y $\sigma_y$ (lo que 
%quiero decir es que $(r_x,r_y,r_z)\to(r_x,r_y,0)$). Entonces hablo 
%de que hay que considerar que el $i$-ésimo espín puede estar 
%entrelazado con el resto de la cadena, sin embargo, basta con considerar
%que se encuentra entrelazado con cualquiera de sus vecinos y revisar 
%en qué se transforma la matriz de densidad de esos dos espines para 
%averiguar si es posible tal evolución física. Me gustaría poner unas 
%figuritas para hacer interesante esto. 
%}

\cpnote{Siento qe aun no es un esqueleto sino ideas. De la motivación
no me gusta como la abordas. Yo pensaría mejor en hablar un poco de defasing
y de bitflip para uno y muchos qubits. Siento qeu tu propuesta está
aun un poco por fuera de tu alcance y te propondría mantenerla más simple.
Si quieres iteremos la propuesta, y cuando nos pongamos de acuerdo ya haces un esqueleto 
mas a nivel de parrafos.}

\esqueleto{
\begin{itemize}
\item Para introducir el contexto de las operaciones PCE voy mencionar sobre
el defasing y el bitflip channel cuando $p=0.5$.
\item Luego hablar sobre
la extensión de este tipo de canales cuánticos para sistemas de 2 
qubits. Mencionar que es intuitivo entender que localmente, sobre cada qubit,  
las únicas evoluciones físicas son las que ya se conocen para 1 qubit 
(bitflip, defasing y bit-phase flip).
\item Sin embargo, el problema cobra interés
al considerar que el estado de 2 qubits es más que 
sólo la suma del estado del qubit 1 y del qubit 2, ya que aparecen 
las correlaciones cuánticas.
\item Mencionar que 
vamos a entender a este tipo de canales cuánticos como operaciones 
que `borran' las componentes del vector de Bloch generalizado de un 
sistema de qubits.
\item Por último, mencionar que 
el entrelazamiento en sistemas de muchas partículas nos conduce
a preguntarnos de qué forma puede un canal cuántico las correlaciones
cuánticas de un sistema de qubits.
\end{itemize}
}


\esqueleto{
\hrule \vspace{10pt}
\h{Párrafos}
\begin{itemize}
\item El defasing y bitflip channel cuando $p=0.5$ se pueden entender 
como canales cuánticos de 1 qubit que borran 2 componentes del 
vector de Bloch y a la componente restante la dejan invariante.
¿Cómo son los canales cuánticos de este tipo, 
que borran las componentes del vector de Bloch, para 2 qubits?
\item Por la matriz de densidad reducida las evoluciones permitidas 
localmente sobre cada qubit son defasing, bitflip y bit-phase fip.
\item Al examinar la matriz de densidad para 2 qubits
se puede distinguir que está compuesta por componentes locales 
de cada qubit + correlaciones entre ambas partículas. 
Esto conduce principalmente a la pregunta 
¿de qué forma se pueden borrar las correlaciones cuánticas tal que 
esa evolución sea física?
\item Recursivamente, al aumentar el número de partículas en 
el sistema, aparecen nuevas correlaciones que no sabemos de 
qué manera se pueden borrar para que ese proceso cumpla con
las condiciones para ser una evolución que físicamente pueda ocurrir.
El `borrado' está condicionado por la completa positividad que 
debe de satisfacer la operación.
\end{itemize}
}

Hablar de qubits y de esfera de Bloch

Una motivación que conduce a definir y estudiar las operaciones 
de Pauli que borran componentes (PCE) es a partir 
de canales cuánticos de 1 qubit como el canal de inversión
de bit o el canal de inversión de fase. Estos dos canales deforman
a la esfera de Bloch a un elipsoide con alguno de los ejes $x$ o $y$ como
eje mayor. En particular nos vamos a enfocar en el caso que ambos canales
mapean la esfera de Bloch a una línea los ejes $x$ o $y$. De esta manera, 
vamos a conducirnos a preguntarnos por la extensión de este tipo 
de operaciones sobre sistemas de muchos qubits.

Vamos a partir analizando el caso del canal de inversión de bit. 
En términos del vector de Bloch, este canal cuántico atenúa las 
componentes $r_y$ y $r_z$ del vector de Bloch en un factor $1-2p$, 
es decir, $(r_x,r_y,r_z)\to\qty(r_x,(1-2p)r_y,(1-2p)r_z)$.
En el caso particular cuando $r_y=r_z=0$ ($p=0.5$) 
este canal mapea la esfera de Bloch a una línea sobre el eje $x$.
Decimos que el canal 'borró' las componentes $r_y$ y $r_z$ de
la matriz de densidad del sistema. De manera similar, el canal 
de inversión de fase transforma al vector de Bloch como 
$(r_x,r_y,r_z)\to\qty((1-2p)r_x,(1-2p)r_y,r_z)$. Cuando tomamos 
el caso $p=0.5$ el canal mapea la esfera de Bloch a una 
línea sobre el eje $z$, el canal 'borró' las componentes $r_x$ y $r_y$
de la matriz de densidad.

\begin{figure}
\centering
\begin{minipage}{.4\textwidth}
    \centering
    \includegraphics[width=4.5cm]{bloch-ball}
\end{minipage}
$\longmapsto$
\begin{minipage}{0.4\textwidth}
    \centering
    \includegraphics[width=4.5cm]{bit-flip}
\end{minipage}
\caption{
Efecto del canal de inversión de bit sobre la esfera de Bloch, para $p=0.3$.}
\label{fig:bit-flip}
\end{figure}

\begin{figure}
\centering
\begin{minipage}{.4\textwidth}
    \centering
    \includegraphics[width=4.5cm]{bloch-ball}
\end{minipage}
$\longmapsto$
\begin{minipage}{0.4\textwidth}
    \centering
    \includegraphics[width=4.5cm]{phase-flip}
\end{minipage}
\caption{
Efecto del canal de inversión de bit sobre la esfera de Bloch, para $p=0.3$.}
\label{fig:phase-flip}
\end{figure}

Recordemos que en el capítulo anterior analizamos la operación de 
1 qubit que mapea la esfera de Bloch a un disco sobre el plano 
$x$$-$$y$ y encontramos que esa operación no es completamente 
positiva, lo que implica que existe al menos un estado entrelazado en 
el espacio extendido de 2 qubits que es mapeado a algo que no es
un estado físico de 2 qubits. Esto implica que no todas 
las operaciones que borran componentes de la matriz de 
densidad escrita en la base de las matrices de Pauli son canales cuánticos.

En general, la matriz de densidad de un sistema de $n$ qubits 
se puede escribir en la base de productos tensoriales de las matrices 
de Pauli. Llamaremos de aquí en adelante operaciones PCE a los 
mapeos que borran las componentes de Pauli de la matriz de densidad
de un sistema de $n$ qubits. En las próximas secciones vamos a
establecer de manera precisa el problema de las operaciones PCE 
para el caso de 1 qubit y luego el de $n$ qubits.

%Un caso particular e interesante de los canales cuánticos de inversión de bit 
%y de inversión de fase es cuando $p=0.5$. En ese caso los canales 
%cuánticos `borran'  dos componentes del vector de Bloch, las 
%componentes en $y$ y $z$, y componentes en $x$ y $y$, respectivamente.
%Geométricamente, ambos canales cuánticos transforman la bola 
%de Bloch a una línea sobre los ejes $x$ y $z$, respectivamente.
%En la representación de Kraus estos canales cuánticos se escriben como 
%\begin{align}
%\E(\rho)=P_0\rho P_0+P_1\rho P_1,
%\end{align}
%donde, para el inversor de bit, 
%$P_0=\dyad{+_x}{+_x}$ y $P_1=\dyad{-_x}{-_x}$ con
%$\ket{\pm_x}=\frac{1}{\sqrt{2}}\qty(\ket{0}\pm\ket{1})$
%los eigenestados de $\sigma_x$; y para el inversor de fase
%$P_0=\dyad{0}{0}$ y $P_1=\dyad{1}{1}$.


% }}}
\section{1 qubit} % {{{
%\esqueleto{
%Resumen de los resultados de 1 qubit. Para seguir un orden lógico
%en este capítulo voy a hablar de los resultados de 1 qubit con el 
%problema resuelto analíticamente (está medio desordenado 
%si hablo por acá del método numérico y luego lo vuelvo a hacer 
%en la sección 2.5), a partir de las desigualdades
%de los eigenvalores (el polihedro?).
%}

\cpnote{Está bien. Acá antes de escribir quiero un esqueleto más detallado. Lo mismo para las siguientes secciones, 2.3, 2.4 y 2.5}

\esqueleto{ 
\begin{itemize}
\item Escribir a $\rho$ de 1 qubit en la base de las matrices de Pauli.
\item Enunciar las 8 operaciones PCE de 1 qubit.
\item Desarrollar cómo verificar analíticamente si las operaciones PCE 
de 1 qubit son CP. Es decir, partir de 
\begin{align*}
\left(
\begin{array}{cccc}
 1 & 0 & 0 & 0 \\
 0 & \tau _1 & 0 & 0 \\
 0 & 0 & \tau _2 & 0 \\
 0 & 0 & 0 & \tau _3 \\
\end{array}
\right)
\end{align*}
calcular el estado de Jamiolkowsky y verificar si es positivo. Con esto llegaré 
a las desigualdades.
\item Con las desigualdades revisar los casos para sacar los 5 canales 
cuánticos PCE.
\item Discutir que las operaciones PCE que borran 1 y 2 componentes 
de Bloch (3 operaciones cada una) son equivalentes bajo permutación 
de los elementos de la base de las matrices de Pauli. Esto me servirá 
para retomarlo después con el argumento de que los canales cuánticos
son equivalentes con permutaciones locales de los elementos de la base 
y de los swaps de partículas. 
\item Final: resumencillo de que hay 8 operaciones PCE, pero sólo 5 son 
CP (y por tanto evoluciones físicas).
\end{itemize}
}

\esqueleto{
\hrule \vspace{10pt}
\h{Párrafos}
\begin{itemize}
\item Las operaciones PCE de 1 qubit transforman a las componentes
de Bloch $r_i$ como $r_i\to \tau_ir_i$, donde $\tau_i=0,1$.
\item Hay 8 operaciones PCE posibles para 1 qubit.
\item La completa positividad está determinada por un set de 4 desigualdades.
\item Las tres operaciones PCE que borran 1 y 2 componentes del vector 
de Bloch, respectivamente, son equivalente bajo permutación de las 
matrices de Pauli $\sigma_i$ en la base 
$\{\sigma_0,\sigma_1,\sigma_2,\sigma_3\}$.
\item Cinco de las ocho operaciones PCE son completamente positivas y,
por consiguiente, canales cuánticos.
\end{itemize}
}
% }}}
\section{El problema de $\mathbf{n}$ qubits} % {{{
\esqueleto{ 
\begin{itemize}
\item Escribir a la matriz de densidad de $n$ qubits en la base 
de productos de las matrices de Pauli. Me gustó mucho la notación 
que usé en el último documento para escribir la expresión de los eigenvalores
de $n$ qubits, así que qué te parece escribirla de la siguiente manera?
\begin{align}
\rho = \frac{1}{2^n}\qty(
1
+
\sum_{\underset{(j_1,\ldots,j_n\neq0)}{j_1,\ldots,j_n=0}}^3
r_{j_1,\ldots,j_n}
\sigma_{j_1} \ot \ldots \ot \sigma_{j_n}
).
\end{align}
Yo la siento más sencilla de entender que como está en el Nielsen y Chuang.
\item Las operaciones PCE de $n$ qubits se definen entonces como 
las operaciones lineales que transforman a las componentes 
de la matriz de densidad como $r_{j_1,\ldots,j_n}\to
\tau_{j_1,\ldots,j_n}r_{j_1,\ldots,j_n}$ donde $\tau_{j_1,\ldots,j_n}=0,1$
son los elementos de la diagonal del superoperador $\E$.
\item Hacer énfasis que la única condición que debe satisfacer una operación
PCE para ser un canal cuántico es ser CP. Por lo cual, nuestro estudie 
se reduce a eso: estudiar la CP de las operaciones PCE.
\item Enunciar los pasos para verificar la CP para una operación de $n$ 
qubits verificando la positividad de la matriz de Choi 
a partir del superoperador $\E$ en la base que es diagonal
\item Concluir remarcando la idea clave de nuestro estudio que es investigar 
la CP de las operaciones PCE.
\end{itemize}
}

\esqueleto{
\hrule \vspace{10pt}
\h{Párrafos}
\begin{itemize}
\item Una operación PCE de $n$ qubits es una operación lineal que 
preserva la traza de la matriz de densidad y que transforma a las
componentes del vector de Bloch generalizado como 
$r_{j_1,\ldots,j_n}\to \tau_{j_1,\ldots,j_n}r_{j_1,\ldots,j_n}$,
donde $\tau_{j_1,\ldots,j_n}=0,1$.
\item La condición para que una operación PCE sea un canal cuántico
es que sea completamente positiva.
\item Una manera de verificar la completa positividad de una operación PCE
es verificar la positividad de su matriz de Choi.
\item El problema de las operaciones PCE consiste en investigar 
cuáles son las condiciones que se deben de satisfacer para que 
la operación sea completamente positiva.
\end{itemize}
}
% }}}
\section{Solución numérica} % {{{
\esqueleto{ 
\begin{itemize}
\item Resolvimos numéricamente la verificación de la CP para 
2 qubits completo y para 3 parcialmente implementando la verificación
de la positividad de la matriz de Choi. Decir que fue imposible verificar 
todas las operaciones de 3 qubits porque son un montón.
\item Hablar de las herramientas que se desarrollaron en Mathematica.
\item Colocar link al repositorio donde habrá un .nb para probar las 
funciones y la verificación de la que estoy hablando en esta sección.
\item Colocar una gráfica de tiempo de ejecución para justificar que 
no teníamos tiempo suficiente para dejar tostando la computadora 
como 1 año para esperar los resultados completos de 3 qubits.
\item Concluir que si bien el método numérico nos dio `poquitos 
resultados', esos resultados sirven muchísimo para extraer mucha 
información y ganar intuición acerca de los canales cuánticos PCE.
\end{itemize}
}

\esqueleto{
\hrule \vspace{10pt}
\h{Párrafos}
\begin{itemize}
\item En una primera aproximación al problema de las operaciones PCE
verificamos numéricamente la positividad de la matriz de Choi de todas las
operaciones PCE de 2 qubits y parcialmente de 3 qubits.
\item Se implementaron rutinas en Mathematica para verificar la
completa positividad de las operaciones PCE.
\item El tiempo de cómputo para verificar la CP de todas las operaciones 
PCE de 3 qubits hizo imposible analizar todas las operaciones posibles.
\item Los resultados de 2 y 3 qubits exhiben características a partir de 
las cuales se puede comenzar a determinar algunas de las condiciones 
que deben de cumplir las operaciones PCE de $n$ qubits para ser 
canales cuánticos.
\end{itemize}
}
% }}}
