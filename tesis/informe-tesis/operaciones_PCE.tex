\chapter{OPERACIONES PCE}

\section{Introducción}
\cpnote{Porfa tambien esqueleto de la into}

\section{Operaciones PCE}
\esqueleto{
Dos o tres párrafos máximo hablando sobre las operaciones PCE 
como operaciones proyectivas a la base de productos tensoriales de 
las matrices de Pauli. Me gustaría poner un problema de motivación
sólo para hacer más interesante la lectura a partir de acá y dejarle al
lector algo con lo que pueda entender qué onda con las operaciones 
PCE. Propongo lo siguiente: cadena de espines. Supongamos una 
cadena de espines de $N$ sitios y decimos que nos interesa saber 
si la matriz de densidad del $i$-ésimo espín puede proyectarse 
al subespacio cuya base son $\sigma_x$ y $\sigma_y$ (lo que 
quiero decir es que $(r_x,r_y,r_z)\to(r_x,r_y,0)$). Entonces hablo 
de que hay que considerar que el $i$-ésimo espín puede estar 
entrelazado con el resto de la cadena, sin embargo, basta con considerar
que se encuentra entrelazado con cualquiera de sus vecinos y revisar 
en qué se transforma la matriz de densidad de esos dos espines para 
averiguar si es posible tal evolución física. Me gustaría poner unas 
figuritas para hacer interesante esto. 
}

\cpnote{Siento qe aun no es un esqueleto sino ideas. De la motivación
no me gusta como la abordas. Yo pensaría mejor en hablar un poco de defasing
y de bitflip para uno y muchos qubits. Siento qeu tu propuesta está
aun un poco por fuera de tu alcance y te propondría mantenerla más simple.
Si quieres iteremos la propuesta, y cuando nos pongamos de acuerdo ya haces un esqueleto 
mas a nivel de parrafos.}

\section{1 qubit}
\esqueleto{
Resumen de los resultados de 1 qubit. Para seguir un orden lógico
en este capítulo voy a hablar de los resultados de 1 qubit con el 
problema resuelto analíticamente (está medio desordenado 
si hablo por acá del método numérico y luego lo vuelvo a hacer 
en la sección 2.5), a partir de las desigualdades
de los eigenvalores (el polihedro?).
}

\section{El problema de $\mathbf{n}$ qubits}
\esqueleto{
Enunciado del problema para $n$ qubits. Creo que esto no da
para más de 1 o 2 párrafos. 
}

\section{Solución numérica}
\esqueleto{ 
Hablar de la solución numérica, similar al informe de prácticas, pero 
enfocado al problema de $n$ qubits. 
Aquí también agregar el link al repositorio y ahí voy a agregar unas cosillas
para mostrar cálculos de 2 y 3 qubits. 
}
