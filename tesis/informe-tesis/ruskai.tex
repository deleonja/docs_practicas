\chapter{CANALES DIAGONALES DE PAULI CONSTANTES SOBRE LOS EJES}

\section{Introducción} % {{{
\esqueleto{Para de último. Cuando estén escritas las secciones.}

% }}}
\section{Canales diagonales de Pauli constantes sobre los ejes} % {{{
%\esqueleto{Introducir la definición de una MUB (es necesario para la definición de 
%mapa de Ruskai) y dar un ejemplos de dos bases que sean MUB, sólo 
%para aterrizar la idea.}

Dos bases ortonormales $\{\ket{\psi_m^J}\}_{m=0}^{d-1}$
y $\{\ket{\psi_n^K}\}_{m=0}^{d-1}$ de un espacio de Hilbert
de dimensión $d$ se dice que son \textit{mutuamente 
imparciales} si se satisface la 
condición~\cite{bengtsson_zyczkowski_2017,nathanson2007pauli}
\begin{align}\label{eq:mub_definition}
	\abs{\braket{\psi_m^J}{\psi_n^K}}^2=
	\left\{ \begin{array}{lcc}
             \frac{1}{d} & si & J\ne K \\
             \delta_{mn} & si & J=K,
             \end{array}
   \right.
\end{align}
para todo $m$ y $n$. Cuando $d$  es una potencia de un 
número primo, el espacio de Hilbert cuenta hasta con $d+1$ bases mutuamente
imparciales \cite{durt2010mutually}. 
Dado que los sistemas cuánticos de nuestro interés en este trabajo son qubits 
(sistemas de $2^n$ niveles) no vamos a considerar los casos de sistemas que 
no posean $d+1$ bases mutuamente imparciales. 

La matriz de densidad de un sistema de qubits puede escribirse en términos 
de operadores que generan a un conjunto de bases mutuamente imparciales\cpnote{No 
entiendo que tiene que ver las MUBS aca. Con que se tenga una base ya se 
puede escribir una matriz de densidad. No entiendo esta anotación}.
Es posible definir, a partir de un conjunto de bases mutuamente imparciales,
$d+1$ operadores unitarios $W_J$ como~\cite{nathanson2007pauli}
\begin{align}\label{eq:W_J}
	W_J=\sum_{k=1}^d e^{2\pi k i/d} \dyad{\psi_k^J}{\psi_k^J}, 
\end{align}
para cada $J$-ésima base $\ket{\psi_k^J}$. Se dice que los operadores 
$W^J$ son generadores del conjunto de bases $\ket{\psi_k^J}$ mutuamente 
imparciales. El conjunto de las $d^2-1$ potencias $\{W_J^m\}_{m=1,\ldots,d-1,
J=1,\ldots,d+1}$ más la identidad forman una base ortogonal
de unitarias del espacio $\mathcal{M}_d$ de matrices 
de dimensión $d\times d$ con norma $\sqrt{d}$
(en el sentido de Hilbert-Schmidt). Por lo tanto, la matriz de densidad
de un sistema de $d$ niveles puede escribirse como
\begin{align}\label{eq:rho_mub}
\rho=\frac{1}{d}\qty(\1+\sum_{J=1}^{d+1}\sum_{j=1}^{d-1} v_{Jj}W_J^j),
\end{align}
con $v_{Jj}$ las proyecciones de $\rho$ sobre cada uno de los 
operadores $W_J^j$.

Dado un conjunto de bases ortonormales mutuamente imparciales del 
espacio de estados de un sistema de $d$ niveles se
puede definir a un canal diagonal de Pauli constante sobre los ejes $\E$ 
según la acción sobre una matriz de densidad como 
en~\eqref{eq:rho_mub}~\cite{nathanson2007pauli},
\begin{align}\label{eq:ruskai_definition}
	\E :  \frac{1}{d}\qty(\1+\sum_{J=1}^{d+1}\sum_{j=1}^{d-1} v_{Jj}W_J^j)
	\longmapsto 
	\frac{1}{d}\qty(\1+\sum_{J=1}^{d+1}\lambda_J\sum_{j=1}^{d-1} v_{Jj}W_J^j).
\end{align}
Es decir que las componentes $v_{Jj}$ de $\rho$ se transforman como 
$v_{Jj}\mapsto\lambda_Jv_{Jj}$, donde $\lambda_0=1$ más $\lambda_J:=s+t_J$ son
los eigenvalores de $\E$. Las 
condiciones para que un \ruskai{}{}{}Map{} sea una operación 
completamente positiva que preserva la traza (CPTP), \textit{i.e.} un canal cuántico,
son
\begin{align}\label{eq:cptp_conditions_ruskai}
	s+\sum_{J}t_J=1, && t_J\geq0 && \text{y} && s\geq\frac{-1}{d-1}.
\end{align}
La primera condición asegura que la operación preserva la traza de la matriz 
de densidad y las últimas dos que la operación sea completamente positiva
\cite{nathanson2007pauli}.
\cpnote{Croe qe es necesario contextualizar un poco mejor estas operaciones. Aca
aparecen como de manera magica. Quizá decir que se han estudiaro y el porque. 
Te toca hacer una buena introduccion, porque por ahora parece raro}

Nótese la similitud entre la definición en \eqref{eq:ruskai_definition} 
de un canal diagonal de Pauli constante
sobre los ejes y la definición de una operación PCE, \eqref{eq:PCE_definition}.
Los dos tipos de operaciones transforman de 
un modo similar a las componentes de la matriz de densidad en una 
base dada del espacio $\mathcal{M}_d$.
No obstante, los \ruskai{} son más generales, en el sentido que son
operaciones que actúan sobre sistemas de $d$ niveles (las 
operaciones PCE actúan sobre sistemas de $2^n$ niveles) y porque 
las $\lambda_J$, para los \ruskai{}, puede tomar cualquier valor real que 
satisfaga las condiciones en \eqref{eq:cptp_conditions_ruskai}, a diferencia
de las operaciones PCE en las que los $\taus$ restringido a los valores 0 o 1.
De hecho, es esta definición más general de los \ruskai{} la que motiva a
investigar si los canales cuánticos PCE están contenidos dentro de ellos.

%Los \ruskai{} pueden escribirse en la forma
%\begin{align}
%	\E=s\1 + \sum_{L}t_L\sum 
%\end{align}
%
%Para $d=2^k$ se ha mostrado que existen $d+1$ bases mutuamente imparciales 
%del espacio de Hilbert. 
%
%Para cualquier base 
%\begin{align}
%	W_J=\sum_{k=1}^d e^{2\pi k i/d} \dyad{\psi_k^K}{\psi_k^K}, 
%\end{align}
%con $i$ la unidad imaginaria. Se sigue que
%\begin{align}
%	\dyad{\psi_k^K}{\psi_k^K} = \frac{1}{d}\sum_{j=0}^{d-1}e^{-2\pi n j i/d}W_J^j 
%	= \frac{1}{d}\qty[\1 + \sum_{j=1}^{d-1}e^{-2\pi n j i/d}W_J^j]
%\end{align}
%
%Bla bla bla... toda la casaca para llegar a que la matriz de densidad se puede 
%escribir como 
%\begin{align}
%\rho=\frac{1}{d}\qty[\1+\sum_{J=1}^{d+1}\sum_{j=1}^{d-1} v_{Jj}W_J^j],
%\end{align}


%donde $W_J$ son operadores unitarios que generan a $d+1$ bases  
%$\{\ket{\psi_m^J}\}_{m=0}^{d-1}$
%mutuamente imparciales y que se definen como
%\begin{align}
%	W_J=\sum_{k=1}^d e^{2\pi k i/d} \dyad{\psi_k^K}{\psi_k^K}, 
%\end{align}
%La identidad $\1$ junto con los operadores $\{ W_J^j\}$ forman 
%una base ortogonal de unitarias
%del espacio $\mathcal{M}_d$ de las matrices de $d\times d$ \janote{revisar 
%notación con el cap 1} que satisfacen la relación de ortogonalidad, 
%en el sentido de Hilbert-Schmidt, $\Tr(W_J^{d-m}W_K^n)=d\delta_{JK}\delta_{mn}$.
%
%%\esqueleto{Introducir qué es un mapa de Ruskai (definición que pusimos 
%%en el documento para Sergei y para Francois).}
%%
%%La acción de un canal de Pauli constante sobre los ejes $\Phi$ se define 
%%como~\cite{nathanson2007pauli}
%%\begin{align}
%%	\Phi :  \frac{1}{d}\qty[\1+\sum_{J=1}^{d+1}\sum_{j=1}^{d-1} v_{Jj}W_J^j
%%	\longmapsto 
%%	\1+\sum_{J=1}^{d+1}\lambda_J\sum_{j=1}^{d-1} v_{Jj}W_J^j],
%%\end{align}
%%de tal forma que $v_{Jj}\mapsto \lambda_Jv_{Jj}$ y $\lambda_J$ los eigenvalores 
%%de la operación $\Phi$.
%
%\esqueleto{Dar algunos ejemplos de mapas de Ruskai}
%
%Los canales PCE de 1 qubit son Ruskai. Las matrices de Pauli son generadores 
%de 3 bases ortonormales mutuamente imparciales.
%
%También hay algunos PCE de 2 qubits que son claramente Ruskai \janote{será??}

% }}}
\section{Relación con canales cuánticos PCE} % {{{
%\esqueleto{Queremos estudiar si los canales PCE son un subconjunto 
%de los mapas de Ruskai.}

En el caso de 1 qubit, los canales PCE sí son un subconjunto de los \ruskai{}. Para $d=2$,
los eigenvectores de las matrices de Pauli $\sigma_1$, $\sigma_2$ y $\sigma_3$ 
satisfacen \eqref{eq:mub_definition} y, por ende, son un conjunto 
de bases mutuamente imparciales. Por lo tanto, siguiendo \eqref{eq:rho_mub}
e identificando $W_J=\sigma_J$, la matriz de densidad de 1 qubit se escribe como
\begin{align} \label{eq:rho_1_qubit_ruskai}
	\rho=\frac{1}{2}\qty(\1+\sum_{J=1}^3v_J\sigma_J),
\end{align}
donde $v_J$ son lo que hemos llamado a lo largo de este trabajo 
las componentes de Pauli.
De acuerdo con \eqref{eq:ruskai_definition}, 
un canal diagonal de Pauli constante sobre los ejes de 1 qubit transforma 
a la matriz de densidad $\rho$ en \eqref{eq:rho_1_qubit_ruskai} como
\begin{align}
	\E :  \frac{1}{2}\qty(\1+\sum_{J=1}^3v_J\sigma_J)
	\longmapsto 
	\frac{1}{2}\qty(\1+\sum_{J=1}^3\lambda_J v_J\sigma_J).
\end{align}
Si $\lambda_J$ se restringe a los valores de 0 o 1, entonces se recupera 
la definición de un operación PCE de 1 qubit. Por lo tanto, los canales PCE 
de 1 qubit son un caso particular de los \ruskai{} cuando $d=2$.

Sin embargo, en general, no existe relación de contención entre los canales cuánticos PCE 
y los \ruskai{}. Para demostrar esta proposición vamos a utilizar como 
argumento la incompatibilidad entre los rangos de los canales PCE
y de los \ruskai{}. Recordemos que el rango de una matriz es igual 
al número de eigenvalores distintos de cero 
\cite{axler1997linear,lang2012introduction}.
Además, de los teoremas elementales de álgebra lineal se puede mostrar que 
rango de una matriz es invariante ante cambios de base \cite{axler1997linear}.
\cpnote{Es mucho mas simple. Los eigenvalores no cambian con 
un cambio de base. Simplifica el argumento.}

Por un lado, de la definición en \eqref{eq:ruskai_definition} es claro que un \ruskaiMap{}
tiene hasta $d+1$ eigenvalores $\lambda_J$ distintos de cero 
con degeneración $d-1$. Por lo tanto,
el rango de un \ruskaiMap{} $1+l(d-1)$. Por otro lado, 
de la regla $2^k$ discutida en la sección \ref{sec:ch3_discussion},
los canales cuánticos PCE son matrices de rango $2^k$. Para que los canales PCE
y los \ruskai{} tengan el mismo rango se debe cumplir 
\begin{align}\label{eq:ruskai_fucked_up}
2^k=1+l(2^n-1),\quad \forall \ k=0,1,\ldots,2n; l=0,1,2,\ldots,2^n+1.
\end{align}
Sin embargo, es fácil encontrar un contraejemplo para mostrar que esta
ecuación no siempre se satisface. Veamos por ejemplo, para $n=2$ y $k=3$
(2 qubits y 8 componentes de Pauli invariantes),
\begin{align}
l=\frac{2^3-1}{2^2-1}=\frac{7}{3},
\end{align}
lo que contradice los posibles valores que puede tomar $l$ 
en \eqref{eq:ruskai_fucked_up}. De hecho, la ecuación sólo se cumple 
para $k=0,n,2n$. En conclusión, la intersección de los \ruskai{}
y los canales PCE contiene al canal depolatizante, la identidad y canales 
que dejan invariantes $2^n$ componentes de Pauli. En particular, para 2 qubits 
los canales que son intersección entre los \ruskai{} y los canales PCE son los 
elementos de las clases de equivalencia C${}_4^2$ y C${}_4^4$
(ver \Fref{fig:2qubits_PCEChannels_figs}). 

Ya que probamos que los 
rangos de los \ruskai{} y de los canales cuánticos PCE son incompatibles, 
se sigue que los canales PCE no son un subconjunto dentro de los 
\ruskai{}, ni viceversa.


% }}}


