\chapter{CANALES DIAGONALES DE PAULI CONSTANTES 
SOBRE LOS EJES}

\section{Introducción}
\esqueleto{Para de último. Cuando estén escritas las secciones.}

\section{Canales diagonales de Pauli constantes sobre los ejes}
%\esqueleto{Introducir la definición de una MUB (es necesario para la definición de 
%mapa de Ruskai) y dar un ejemplos de dos bases que sean MUB, sólo 
%para aterrizar la idea.}

Dos bases ortonormales $\{\ket{\psi_m^J}\}_{m=0}^{d-1}$
y $\{\ket{\psi_n^K}\}_{m=0}^{d-1}$ de un espacio de Hilbert
de dimensión $d$ se dice que son \textit{mutuamente 
imparciales} si se satisface la 
condición~\cite{bengtsson_zyczkowski_2017,nathanson2007pauli}
\begin{align}\label{eq:mub_definition}
	\abs{\braket{\psi_m^J}{\psi_n^K}}^2=
	\left\{ \begin{array}{lcc}
             \frac{1}{d} & si & J\ne K \\
             \delta_{mn} & si & J=K,
             \end{array}
   \right.
\end{align}
para todo $m$ y $n$. Cuando la dimensión $d$  es una potencia de un 
número primo el espacio de Hilbert cuenta hasta con $d+1$ bases mutuamente
imparciales. Ya que los sistemas cuánticos de nuestro interés son qubits, 
sistemas de $2^n$ niveles, vamos a suponer de aquí en adelante que 
siempre siempre existen $d+1$ bases mutuamente imparciales. 
Existen operadores unitarios $W_J$, definidos como~\cite{nathanson2007pauli}
\begin{align}\label{eq:W_J}
	W_J=\sum_{k=1}^d e^{2\pi k i/d} \dyad{\psi_k^K}{\psi_k^K}, 
\end{align}
que son generadores de las $d+1$
bases mutuamente imparciales. El conjunto de las potencias de los 
generadores de las bases mutuamete imparciales $W_J^m$,
desde $m=0$ hasta $d-1$, más la identidad forman una base ortogonal
de unitarias del espacio $\mathcal{M}_d$ de matrices de 
de dimensión $d\times d$ con norma $\sqrt{d}$
en el sentido de Hilbert-Schmidt. Por lo tanto, la matriz de densidad
de un sistema de $d$ niveles puede escribirse como
\begin{align}\label{eq:rho_mub}
\rho=\frac{1}{d}\qty(\1+\sum_{J=1}^{d+1}\sum_{j=1}^{d-1} v_{Jj}W_J^j),
\end{align}
con $v_{Jj}$ las proyecciones de $\rho$ sobre cada uno de los 
operadores $W_J^j$.

Dado un conjunto de bases ortonormales mutuamente imparciales se
puede definir un canal diagonal de Pauli constante sobre los ejes $\E$ 
según el efecto sobre una matriz de densidad como en \eqref{eq:rho_mub}
\cite{nathanson2007pauli},
\begin{align}\label{eq:ruskai_definition}
	\E :  \frac{1}{d}\qty(\1+\sum_{J=1}^{d+1}\sum_{j=1}^{d-1} v_{Jj}W_J^j)
	\longmapsto 
	\frac{1}{d}\qty(\1+\sum_{J=1}^{d+1}\lambda_J\sum_{j=1}^{d-1} v_{Jj}W_J^j),
\end{align}
tal que las componentes $v_{Jj}$ de $\rho$ se transforman como 
$v_{Jj}\mapsto\lambda_Jv_{Jj}$, donde $\lambda_J:=s+t_J$
\begin{align}\label{eq:cptp_conditions_ruskai}
	s+\sum_{J}t_J=1, && t_J\geq0 && \text{y} && s\geq\frac{-1}{d-1}.
\end{align}
Se sigue que $\lambda_J$ son los 
eigenvalores de un canal diagonal de Pauli constante sobre los ejes $\E$.
Nótese la similitud entre la definición de un canal diagonal de Pauli constante
sobre los ejes, según cómo transforma a $v_{Jj}$, y 
la definición de una operación PCE, de acuerdo a cómo transforma a las 
componentes de Pauli de la matriz de densidad de un sistema de qubits.
Claro que la definición de los \ruskai es más general porque es para
operaciones que actúan sobre sistemas de $d$ niveles en general y porque 
$\lambda_J$ no está restringido a 0 o 1, como en el caso de los 
elementos $\taus$ las operaciones PCE. De hecho, es esta definición 
más general de lso \ruskai la que nos hace preguntarnos si los canales cuánticos 
PCE son un subconjunto de los \ruskai.

En el caso de 1 qubit, los canales PCE sí son \ruskai{}. Para $d=2$,
los eigenvectores de las matrices de Pauli $\sigma_1$, $\sigma_2$ y $\sigma_3$ 
satisfacen \eqref{eq:mub_definition} y, por ende, son un conjunto 
de bases mutuamente imparciales. Por lo tanto, se dice que las 
matrices de Pauli $\sigma_J$ son las generadoras de un conjunto de
bases mutuamente imparciales para el espacio de Hilbert de 1 qubit.
Es decir, $W_J=\sigma_J$ y la matriz de densidad se escribe como
\begin{align} \label{eq:rho_1_qubit_ruskai}
	\rho=\frac{1}{2}\qty(\1+\sum_{J=1}^3v_J\sigma_J),
\end{align}
donde $v_J$ son las componentes de Pauli.
De acuerdo con \eqref{eq:ruskai_definition}, 
un canal diagonal de Pauli constante sobre los ejes de 1 qubit transforma 
a $\rho$ en \eqref{eq:rho_1_qubit_ruskai} como
\begin{align}
	\E :  \frac{1}{2}\qty(\1+\sum_{J=1}^3v_J\sigma_J)
	\longmapsto 
	\frac{1}{2}\qty(\1+\sum_{J=1}^3\lambda_J v_J\sigma_J).
\end{align}
Si $\lambda_J$ se restringe a los valores de 0 o 1, entonces se recupera 
la definición de un operación PCE de 1 qubit. De esta cuenta, 
en la siguiente sección discutiremos sin los canales cuáncitos
PCE de $n$ qubits son un subconjunto contenido dentro de los \ruskai{}.

%Los \ruskai pueden escribirse en la forma
%\begin{align}
%	\E=s\1 + \sum_{L}t_L\sum 
%\end{align}
%
%Para $d=2^k$ se ha mostrado que existen $d+1$ bases mutuamente imparciales 
%del espacio de Hilbert. 
%
%Para cualquier base 
%\begin{align}
%	W_J=\sum_{k=1}^d e^{2\pi k i/d} \dyad{\psi_k^K}{\psi_k^K}, 
%\end{align}
%con $i$ la unidad imaginaria. Se sigue que
%\begin{align}
%	\dyad{\psi_k^K}{\psi_k^K} = \frac{1}{d}\sum_{j=0}^{d-1}e^{-2\pi n j i/d}W_J^j 
%	= \frac{1}{d}\qty[\1 + \sum_{j=1}^{d-1}e^{-2\pi n j i/d}W_J^j]
%\end{align}
%
%Bla bla bla... toda la casaca para llegar a que la matriz de densidad se puede 
%escribir como 
%\begin{align}
%\rho=\frac{1}{d}\qty[\1+\sum_{J=1}^{d+1}\sum_{j=1}^{d-1} v_{Jj}W_J^j],
%\end{align}


%donde $W_J$ son operadores unitarios que generan a $d+1$ bases  
%$\{\ket{\psi_m^J}\}_{m=0}^{d-1}$
%mutuamente imparciales y que se definen como
%\begin{align}
%	W_J=\sum_{k=1}^d e^{2\pi k i/d} \dyad{\psi_k^K}{\psi_k^K}, 
%\end{align}
%La identidad $\1$ junto con los operadores $\{ W_J^j\}$ forman 
%una base ortogonal de unitarias
%del espacio $\mathcal{M}_d$ de las matrices de $d\times d$ \janote{revisar 
%notación con el cap 1} que satisfacen la relación de ortogonalidad, 
%en el sentido de Hilbert-Schmidt, $\Tr(W_J^{d-m}W_K^n)=d\delta_{JK}\delta_{mn}$.
%
%%\esqueleto{Introducir qué es un mapa de Ruskai (definición que pusimos 
%%en el documento para Sergei y para Francois).}
%%
%%La acción de un canal de Pauli constante sobre los ejes $\Phi$ se define 
%%como~\cite{nathanson2007pauli}
%%\begin{align}
%%	\Phi :  \frac{1}{d}\qty[\1+\sum_{J=1}^{d+1}\sum_{j=1}^{d-1} v_{Jj}W_J^j
%%	\longmapsto 
%%	\1+\sum_{J=1}^{d+1}\lambda_J\sum_{j=1}^{d-1} v_{Jj}W_J^j],
%%\end{align}
%%de tal forma que $v_{Jj}\mapsto \lambda_Jv_{Jj}$ y $\lambda_J$ los eigenvalores 
%%de la operación $\Phi$.
%
%\esqueleto{Dar algunos ejemplos de mapas de Ruskai}
%
%Los canales PCE de 1 qubit son Ruskai. Las matrices de Pauli son generadores 
%de 3 bases ortonormales mutuamente imparciales.
%
%También hay algunos PCE de 2 qubits que son claramente Ruskai \janote{será??}

\section{Relación con canales cuánticos PCE}
\esqueleto{Queremos estudiar si los canales PCE son un subconjunto 
de los mapas de Ruskai.}

Nos interesa comprobar o refutar la hipótesis de que los canales PCE sean 
un subconjunto de los \ruskai. Sin embargo, no lo son. Nuestra prueba 
viene después. 

De los eigenvalores de los mapas de Ruksai, se deduce que el rango de un 
mapa de Ruskai es $1+k(2^n-1)$. Los PCE tienen rango $2^n$.
\begin{align}
2^k=1+l(2^n-1)\\
l=\frac{2^k-1}{2^n-1},\\
l=\frac{2^k-1}{2^n-1}
\end{align}
En general, este número es un número racional y, por lo tanto, no es 
posible que los canales PCE sean un subconjunto de los Ruskai o viceversa.

Hay 15 canales cuánticos que son intersección entre los PCE y los de Ruskai.

\esqueleto{Pues nel. Escribir la demostración que pusimos en el 
documento para Francois pero un poco más digerida. En nuestra prueba
mostramos que el rango de los mapas de Ruskai no es $2^k$, como en los PCE, 
razón por la cual queda descartado que los PCE estén completamente 
contenidos dentro de los mapas de Ruskai.}
