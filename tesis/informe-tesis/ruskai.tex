\chapter{CANALES DIAGONALES DE PAULI CONSTANTES 
SOBRE LOS EJES}

\section{Introducción}
\esqueleto{Para de último. Cuando estén escritas las secciones.}

\section{Canales diagonales de Pauli constantes sobre los ejes}
\esqueleto{Introducir la definición de una MUB (es necesario para la definición de 
mapa de Ruskai) y dar un ejemplos de dos bases que sean MUB, sólo 
para aterrizar la idea.}

\esqueleto{Introducir qué es un mapa de Ruskai (definición que pusimos 
en el documento para Sergei y para Francois).}

\esqueleto{Dar algunos ejemplos de mapas de Ruskai}

\section{Relación con canales cuánticos PCE}
\esqueleto{Queremos estudiar si los canales PCE son un subconjunto 
de los mapas de Ruskai.}

\esqueleto{Pues nel. Escribir la demostración que pusimos en el 
documento para Francois pero un poco más digerida. En nuestra prueba
mostramos que el rango de los mapas de Ruskai no es $2^k$, como en los PCE, 
razón por la cual queda descartado que los PCE estén completamente 
contenidos dentro de los mapas de Ruskai.}