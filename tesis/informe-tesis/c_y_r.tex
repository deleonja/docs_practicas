%%% INCLUYA SUS CONCLUSIONES Y RECOMENDACIONES


\chapter{CONCLUSIONES}
En este trabajo de tesis propusimos estudiar un nuevo tipo de operaciones 
que generalizan el proceso de decoherencia cuántica para sistemas de $n$ qubits 
(sistemas de dos niveles). Para 1 qubit, el proceso de decoherencia
puede describirse por medio de canales cuánticos bien conocidos como 
el \textit{bit-flip} (inversor de bit), operación que proyecta el estado del 
sistema a alguno de los eigenestados de $\sigma_z$
\cite{bengtsson_zyczkowski_2017,nathanson2007pauli,nielsen_chuang_2011}. 
Para $n$ qubits, 
introdujimos la definición de una operación que borra las componentes de Pauli
(PCE por sus siglas en ingles, \textit{Pauli component erasing}) como una 
operación diagonal de Pauli que preserva o borra por completo 
las proyecciones de la matriz de densidad sobre los elementos de la
base de productos tensoriales de las matrices de Pauli (componentes de Pauli). 
Diseñamos un método 
númerico para implementar la búsqueda de los \textit{canales cuánticos PCE},
\textit{i.e.} operaciones PCE que son completamente positivas, y nuestros 
resultados muestran evidencia que estos canales cuánticos podrían tener
una estructura matemática propia. 
Por un lado, los canales cuánticos PCE pueden clasificarse en clases de equivalencia, 
es decir, subconjuntos dentro de los cuales todos los elementos están 
conectados vía operaciones unitarias. 
Además, se pueden identificar que las operaciones PCE 
que satisfacen la completa positividad obedecen dos reglas, (1) \textit{regla
$\mathit{2^k}$}: preservan una cantidad de componentes de Pauli
que es una potencia de dos, y (2) \textit{regla espejo}:
existe la misma cantidad de canales cuánticos PCE que preservan $2^k$ 
y $2^{2n-k}$ componentes de Pauli.
Además, probamos que los canales cuánticos PCE no son un subconjunto
de otro tipo de canales cuánticos que se han estudiado antes 
\cite{nathanson2007pauli}. 
En resumen, este trabajo de tesis aporta pruebas numéricas
de que los canales cuánticos PCE
poseen una caracterización 
propia, y abre la posibilidad a preguntas más fundamentales acerca de este 
tipo de canales cuánticos.
\cpnote{Esto lo pondria mas arriba. Siento que acá estorba: (una generalización de los canales cuánticos
que describen la decoherencia
para sistemas de $n$ qubits}

% \cpnote{Primero, la introduccion de un nuevo tipo de canales que generalizan 
% las decoherencias basicas de un qubit. A esto tienen que ir dos frases, y también 
% recoerdando al lector que son los PCEs}
% \cpnote{Plantea las cosas un poco mas generales. Es decir, puedes hacer un planteamiento
% general que contextualice la importancia de tus resultados en general, y que luego 
% se aplican a los PCEs. Itntenta plantear un poco las cosas como en la ultima frase de 
% este parrafo y luego dices en particular lo qeu hacemos. Como por cada frase de aca
% pon una frase que la anteceda tipo la ultima. Quizá vale la pena platicar de esto, 
% pareciera confuso}
% 
% \noindent
% \esqueleto{Ideas:
% \begin{itemize}
% 	\item Buscamos generalizar las decoherencias básicas de un qubit [?]
% 	\item Con esto introducimos la definición de una operación PCE
% 	\item Implementamos computacionalmente un método numérico para 
% 	encontrar canales cuánticos PCE
% 	\item Encontramos tales propiedades 
% 	\item Encontramos que se pueden clasificar
% \end{itemize}
% }

\chapter{RECOMENDACIONES}
\begin{enumerate}
	\item Recomendación 1.
	\item Recomendación 2.
	\item Recomendación 3.
\end{enumerate}
