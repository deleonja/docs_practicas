%%% INCLUYA SUS CONCLUSIONES Y RECOMENDACIONES


\chapter{CONCLUSIONES}
\cpnote{Primero, la introduccion de un nuevo tipo de canales que generalizan 
las decoherencias basicas de un qubit. A esto tienen que ir dos frases, y también 
recoerdando al lector que son los PCEs}
La contribución de este trabajo de tesis fue aportar pruebas numéricas
de que los canales cuánticos PCE de $n$ qubits tienen una caracterización propia 
y abrió la posibilidad a preguntas más fundamentales acerca de este 
tipo de canales cuánticos.
Nuestros resultados numéricos de la búsqueda de canales PCE de 2 y 3 qubits 
muestran principalmente dos cosas: (1) que los canales cuánticos PCE pueden 
ordenarse en clases de equivalencia, lo que reduce el número de canales cuánticos
que no están conectados vía operaciones unitarias, y (2) que las operaciones
PCE que satisfacen la condición de completa positividad obedecen reglas muy
específicas. 
\cpnote{Plantea las cosas un poco mas generales. Es decir, puedes hacer un planteamiento
general que contextualice la importancia de tus resultados en general, y que luego 
se aplican a los PCEs. Itntenta plantear un poco las cosas como en la ultima frase de 
este parrafo y luego dices en particular lo qeu hacemos. Como por cada frase de aca
pon una frase que la anteceda tipo la ultima. Quizá vale la pena platicar de esto, 
pareciera confuso}
Por un lado, 
la \textit{regla $2^k$} establece que el número de 
componentes de Pauli invariantes por un canal PCE debe ser una potencia de 2. 
Por otro lado, la \textit{regla espejo} establece que debe existir el mismo número de 
canales cuánticos PCE que dejan invariantes $2^k$ y $2^{2n-k}$ componentes 
de Pauli. Además, probamos que los canales cuánticos PCE no son un subconjunto
de otro tipo de canales cuánticos que se han estudiado antes. Todo esto, la clasificación,
propiedades y la no contención dentro de otro conjunto de canales cuánticos, 
abre la posibilidad y justifica preguntarse si existe una estructura matemática bien 
definida para los canales cuánticos PCE.

\chapter{RECOMENDACIONES}
\begin{enumerate}
	\item Recomendación 1.
	\item Recomendación 2.
	\item Recomendación 3.
\end{enumerate}
