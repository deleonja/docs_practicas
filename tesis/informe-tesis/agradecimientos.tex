%%% Haga el diseño que más le guste
\chapter*{AGRADECIMIENTOS}
\thispagestyle{empty}
\janote{Iterar:}

A mi mamá, la Ednita Moda, por ser mi mayor soporte durante toda mi vida,
por todos los sacrificios realizados para darme las mejores condiciones para 
vivir y estudiar, por animarme a nunca dejar de aspirar alto, por todas las 
lecciones de vida que me han formado como persona y por el amor incondicional
de madre que jamás me ha faltado.

A mi papá, Luis Alfredo de León Q.E.P.D, por las lecciones de vida, por el 
carácter y cualidades heredadas de él, por velar por mi educación incluso 
en los momentos de su enfermedad, y por ser un modelo de vida para mí.

A Carlos Pineda por dirigir este trabajo de graduación, por su paciencia,
por sus consejos más allá plano académico y por su confianza en mi capacidad 
durante estos años.

A Alejandro Fonseca y David Dávalos por sus ideas y aportes para el progreso
de este trabajo de investigación.

A Juan Diego Chang por ser un fabuloso catedrático que no subestima la capacidad
de sus estudiantes; por sus excelentes clases de Mecánica Cuántica, en las cuáles 
me transmitió su gusto por la materia y me plantó la semilla de la curiosidad;
por todo su apoyo durante mi proceso de graduación;  
y por las incontables conversaciones que tuvimos en el café afuera del T-1
para arreglar la ciencia de Guatemala.

Al Ing. Rodolfo Samayoa, por su excelente labor como docente y administrativo, 
que durante toda su carrera ha dirigido sus esfuerzos al progreso de la carrera 
de Física y Matemática dentro de la Universidad de San Carlos de Guatemala. 
Siempre tendrá mi mayor respeto y admiración.

A los profesores de la ECFM Rodrigo, Giovanni, Damián, Mapache, Ronald, Héctor, Maynor, 
por su dedicación para enseñar, especialmente por el esfuerzo que supone 
no tener formación docente. De alguna u otra manera, marcaron mi vida 
académica durante la licenciatura. 

A Claudia y Norma por siempre tener la mejor disposición para ayudarme 
con todos los trámites administrativos que alguna vez requerí. 

A mi novia, Cindy, por haber sido la motivación que me hizo retomar mis 
objetivos académicos cuando empezaba a perder el rumbo, por todas las porras, 
por todas las palabras de ánimo, por escucharme en los momentos de frustración, 
por vivir la frustración conmigo, por ponerme atención siempre que hablo de 
\textit{los qubits}, por 

A mis mejores amigos los H's, Benja, Gómez y Papaya, por los cuchubales de
los martes en Applebee's (\textit{Vamos a tu casa pues, cerote}), por siempre 
traerme a mi casa cuando me da sueño y me quedo dormido en las fiestas, pero
lo más importante, por su amistad incondicional y desinteresada.

A mis amigos de la U, Sub, Luz, Félix, Elser, Damián, Wendel, Miguel, Ester, Mack, Alex, 
Sarceño, por todos los momentos graciosos, roces, chismes, y validas de verga que vivimos,
siempre los tendré en mi memoria como los recuerdos memorables y 
anécdotas de la licenciatura. A Félix: las risas no faltaron. A Sub: échele ganas mija.
% 