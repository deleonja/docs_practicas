%%% Haga el diseño que más le guste
\chapter*{AGRADECIMIENTOS}
\thispagestyle{empty}
A mi mamá, la Ednita Moda, por ser mi mayor soporte durante toda mi vida,
por su amor incondicional de madre que jamás me ha faltado,
por todos los sacrificios que hizo para siempre proveerme de 
las mejores condiciones para vivir y estudiar, por enseñarme y motivarme
a soñar en grande y por todas sus lecciones de vida que me han hecho
crecer personalmente.

A mi papá, Luis Alfredo de León (Q.E.P.D), por haber sido un papá extraordinario que 
estuvo siempre al pendiente de mi bienestar, aún en los momentos difíciles 
de enfermedad, y por haber sido y seguir siendo un ejemplo de vida para mí. 

A Carlos Pineda le agradezco por las oportunidades que me ha abierto para 
involucrarme en investigación, por su paciencia, y por sus valiosos consejos 
en el plano académico y profesional, y por su confianza en mi capacidad 
durante estos años.

A Alejandro Fonseca y David Dávalos por sus contribuciones y roles fundamentales
en el progreso de este trabajo de investigación, así como también por
los consejos, desde su experiencia, para contribuir a mi formación académica.

A Juan Diego Chang por ser un fabuloso catedrático que se empeña por transmitir
el gusto por sus cursos en cada clase, por no subestimar la capacidad de sus 
estudiantes y ofrecer un contenido más avanzado al estándar. 
Le doy gracias por su apoyo durante mi proceso de graduación
y por las incontables conversaciones que tuvimos en el café afuera del T-1
para arreglar la física en Guatemala.

Al Ing. Rodolfo Samayoa, por su excelente labor docente y administrativa, 
por siempre anteponer el progreso académico de los estudiantes antes que 
la burocracia universitaria, y por su papel clave en la creación de la ECFM
que, sin duda alguna, catapultó el desarrollo de la física y matemática en la USAC.
Siempre tendrá mi mayor respeto y admiración.

A los profesores de la ECFM ,con quienes recibí clases,
Rodrigo, Giovanni, Damián, Mapache, Ronald, Héctor y Maynor.
Por su dedicación y paciencia para enseñar, especialmente por 
el esfuerzo extra que supone no tener formación docente. 
De alguna u otra manera, marcaron positivamente mi formación 
académica de licenciatura. 

A Claudia y Norma por siempre tener la mejor disposición para ayudarme 
con todos los trámites administrativos que alguna vez requerí. 

A mi novia, Cindy, por haber sido mi motivación para retomar mis 
objetivos académicos en un momento turbulento de mi vida, 
por estar en los momentos de alegría y de frustración aún en la distancia, 
por sus palabras de aliento, por escucharme siempre que lo necesito y 
por su sincero amor.

A mis mejores amigos los H's: Benja, Gómez y Papaya. Por los cuchubales de
los martes en Applebee's, por traerme a mi casa todas las veces que 
me quedé dormido en las reuniones sociales con bebidas que marean, 
pero lo más importante, por su amistad 
incondicional y desinteresada.

A mis compañeros y amigos de la U: Sub, Melissa, Pablo Escobar, 
Luz, Félix, Elser, Wendel, Luciano, Miguel, Ester,  Alex y Sarceño.
Por todos los momentos graciosos, roces, chismes, y sufridas académicas 
que vivimos, siempre los tendré en mi memoria como los recuerdos memorables y 
anécdotas de la licenciatura. A Félix, Elser, Wendel, Luciano, Miguel y Ester: 
por lo menos, las risas no faltaron. A Sub: échele ganas mija, ¡tú podés! 
A Pablo y Melissa: ¿tienen tramadol? A Luz: siempre chairos y cero miedo 
a los comunicados. A Alex y Sarceño: ¿quién vos? 
% 