%%% INCLUYA LA SIMBOLOGÍA NECESARIA EN ESTE APARTADO
%%% NO CAMBIAR LA DEFINICIÓN DE LA TABLA LARGA


\chapter{LISTA DE SÍMBOLOS}

\begin{longtable}{@{}l@{\extracolsep{\fill}} p{4.75in} @{}}  %%%	NO CAMBIAR ESTA LÍNEA
  \textsf{Símbolo} & \textsf{Significado}\\[12pt]
  \endhead
  $\ket{\psi}$ &  \textit{ket}, vector de estado en la notación de Dirac \\
  $\bra{\psi}$ & \textit{bra}, funcional en la notación de Dirac\\
  $p_i$ & probabilidad $i$-ésima\\
  $\qty{p_i,\ket{¸\psi_i}}$ & ensamble de estados \\
  $\Lambda$ &  operador que actúa sobre el espacio de Hilber\\
  $\expval{\Lambda}$ & valor esperado del operador $\Lambda$ en la notación de Dirac\\
  $\matrixel{\psi_i}{\Lambda}{\psi_j}$ & elemento de matriz $\Lambda_{ij}$\\
  $\braket{\psi}{\phi}$ & \textit{braket}, producto interno en la notación de Dirac\\
  $\dyad{\psi}{\phi}$ & producto externo entre $\ket{\psi}$ y $\ket{\phi}$ en la notación de Dirac \\
  $\rho$ & matriz de densidad\\
  $\Tr$ & traza \\
	$\E$ & canal cuántico \\
	$U$ & operador unitario \\
	$\otimes$ & producto tensorial\\
	$\ket{\psi}\ket{\phi}$ & producto tensorial $\ket{\psi}\otimes\ket{\phi}$\\
	$\1$ & operador identidad \\
	$\sigma_i$ & matrices de Pauli\\
	$\mapsto$ & ``se mapea a''\\
	$\vec\rho$ & matriz de densidad vectorizada\\
	$\mathcal{M}_d$ & espacio de las matrices de $d\times d$\\
	$\mathcal{HS}$ & espacio de Hilbert-Schmidt\\
	$\Lambda^{\dagger}$ & operador adjunto de $\Lambda$\\
	$\delta_{ij}$ & delta de Kronecker\\
	$\taus$ & elementos de la diagonal de un superoperdador PCE en la base de Pauli\\
	$\lambda_i$ & eigenvalores
\end{longtable}
\janote{aquí hay que completar a mano}
