%%
%%	AVISO IMPORTANTE
%%	Formato optimizado para el sistema operativo GNU/Linux 64 bits
%%	usar TexLive 2016 (o superior), http://www.ctan.org/tex-archive/systems/texlive/Images/
%%	usar TeXstudio 2.11, http://texstudio.sourceforge.net/

\documentclass[letterpaper,12pt]{thesisECFM}
\usepackage{macros}

%%	NO OLVIDE INCLUIR FUENTE DE LAS TABLAS Y FIGURAS

% Decomentar para anular recuadros en los hiperenlaces dentro del pdf
% \hypersetup{pdfborder={0 0 0}}

% Teoremas ---------------------------------------------------------
% estos ambientes son para teoremas, lemas, corolarios, otros
% si no los utiliza los puede obviar en su trabajo de graduación
\theoremstyle{plain}
\newtheorem{thm}{Teorema}[section]
\newtheorem{cor}{Corolario}[chapter]
\newtheorem{lem}{Lema}[chapter]
\newtheorem{prp}{Proposición}[chapter]

\theoremstyle{definition}
\newtheorem{exa}{Ejemplo}[chapter]
\newtheorem{defn}{Definición}[chapter]
\newtheorem{axm}{Axioma}[chapter]

\theoremstyle{remark}
\newtheorem{rem}{Nota}[chapter]


% --------------------------------------------------------------------------------------------------
%                    Mi preámbulo usual (José Alfredo)
% --------------------------------------------------------------------------------------------------
\usepackage{natbib}
\usepackage{ctable} % for \specialrule command
\usepackage{bbold}

\usepackage{lscape}
\newcolumntype{P}[1]{>{\centering\arraybackslash}m{#1}}
\usepackage{amssymb}

\usepackage{physics}
\usepackage{fancybox}
\usepackage{colortbl}
\usepackage{amsbsy}
\usepackage{float}
\usepackage[draft,inline,nomargin]{fixme} \fxsetup{theme=color}
\FXRegisterAuthor{cp}{acp}{\color{blue}CP}
\definecolor{mycolor}{RGB}{200,40,0} \FXRegisterAuthor{ja}{aja}{\color{mycolor}JA}

\usepackage{graphicx}
\graphicspath{ {./img/} }


\usepackage[]{lineno}  \linenumbers
\setlength\linenumbersep{3pt}

\newcommand{\fref}[1]{fig.~\ref{#1}}   \newcommand{\tref}[1]{tabla~\ref{#1}}
\newcommand{\Fref}[1]{Fig.~\ref{#1}}  \newcommand{\Tref}[1]{Tabla~\ref{#1}}
\newcommand{\Cref}[1]{Cuadro~\ref{#1}}

\newcommand{\psii}{\psi_i}
\newcommand{\Pk}[1]{\ket{\psi_{#1} }}
\newcommand{\Pb}[1]{\bra{\psi_{#1} }}
\newcommand{\pk}{\ket{\psi}}
\newcommand{\M}{\mathcal{M}^{(N)}}
\newcommand{\E}{\mathcal{E}}
\newcommand{\Erho}{\mathcal{E}(\rho)}
\newcommand{\1}{\mathbb{1}}
\newcommand{\ten}{\otimes}
\newcommand{\h}[1]{\colorbox{yellow}{#1}}
\newcommand{\hi}{\mathcal{H}}
\newcommand{\txt}[1]{\text{#1}}
\newcommand{\here}{\h{\hspace{15cm}} }
\newcommand{\rhoi}{\dyad{\psii}{\psii}}
\newcommand{\ind}[2]{{{}^{#1}_{#2}}}
\newcommand{\rc}[1]{r_{#1}}
\newcommand{\pauli}[2]{\sigma_{#1}\otimes\sigma_{#2}}
\newcommand{\esqueleto}[1]{\textcolor{mycolor}
{\noindent\textbf{Esqueleto:} #1}}
\newcommand{\ot}{\otimes}
\newcommand{\m}{\textcolor{mycolor}{|}}
\newcommand{\taus}{\tau_{j_1,\ldots ,j_n}}
\newcommand{\tauij}[2]{\boldmath{$\tau_{#1#2}$}}

% Para que funcione mejor la numeración {{{
% https://tex.stackexchange.com/questions/43648/why-doesnt-lineno-number-a-paragraph-when-it-is-followed-by-an-align-equation
\newcommand*\patchAmsMathEnvironmentForLineno[1]{%
  \expandafter\let\csname old#1\expandafter\endcsname\csname #1\endcsname
  \expandafter\let\csname oldend#1\expandafter\endcsname\csname end#1\endcsname
  \renewenvironment{#1}%
     {\linenomath\csname old#1\endcsname}%
     {\csname oldend#1\endcsname\endlinenomath}}% 
\newcommand*\patchBothAmsMathEnvironmentsForLineno[1]{%
  \patchAmsMathEnvironmentForLineno{#1}%
  \patchAmsMathEnvironmentForLineno{#1*}}%
\AtBeginDocument{%
\patchBothAmsMathEnvironmentsForLineno{equation}%
\patchBothAmsMathEnvironmentsForLineno{align}%
\patchBothAmsMathEnvironmentsForLineno{flalign}%
\patchBothAmsMathEnvironmentsForLineno{alignat}%
\patchBothAmsMathEnvironmentsForLineno{gather}%
\patchBothAmsMathEnvironmentsForLineno{multline}%
}
% }}}	

% --------------------------------------------------------------------------------------------------
% --------------------------------------------------------------------------------------------------



% Cuerpo de la tesis -----------------------------------------------

\begin{document}

%% Datos generales del trabajo de graduación
\datosThesis%
{1}%						% física 1; matemática 2
{Mapeos proyectivos\\en sistemas de varios qubits}%		% Título del trabajo de graduación
{José Alfredo de León Garrido}%			% autor
{M.Sc. Juan Diego Chang\\y Dr. Carlos Francisco
Pineda Zorrilla}%			% asesor
{abril de 2021}		% mes y año de la orden de impresión
{2}							% femenino 1; masculino 2

%% Datos generales del examen general privado
\examenPrivado%
{M.Sc. Jorge Marcelo Ixquiac Cabrera}%	% director ECFM
{M.Sc. Edgar Anibal Cifuentes Anléu}%		% secretario académico
{Perengano}%		% examinador 1
{Zutano}%		% examinador 2
{Fulano 2}%		% examinador 3

{\onehalfspacing	% interlineado 1 1/2

\OrdenImpresion{ordenImpresion}		% incluye orden de impresión, guardada en pdf

\Agrade{agradecimientos}			% Agradecimientos

\Dedica{dedicatoria}				% Dedicatoria
asdfds
\par}
 
\frontmatter    % --------------------------------------------------  Hojas preliminares

{\onehalfspacing	% interlineado 1 1/2

\tableofcontents    % Índice general vinculado

%%% \figurasYtablas{ lista_figuras }{ lista_tablas }; con valor 1 se incluye la lista,
%%% cualquier otro valor no la genera
\figurasYtablas{1}{1}

%%% INCLUYA LA SIMBOLOGÍA NECESARIA EN ESTE APARTADO
%%% NO CAMBIAR LA DEFINICIÓN DE LA TABLA LARGA


\chapter{LISTA DE SÍMBOLOS}

\begin{longtable}{@{}l@{\extracolsep{\fill}} p{4.75in} @{}}  %%%	NO CAMBIAR ESTA LÍNEA
  \textsf{Símbolo} & \textsf{Significado}\\[12pt]
  \endhead
  $\ket{\psi}$ &  \textit{ket}, vector de estado en la notación de Dirac \\
  $\bra{\psi}$ & \textit{bra}, funcional en la notación de Dirac\\
  $p_i$ & probabilidad $i$-ésima\\
  $\qty{p_i,\ket{¸\psi_i}}$ & ensamble de estados \\
  $\Lambda$ &  operador que actúa sobre el espacio de Hilber\\
  $\expval{\Lambda}$ & valor esperado del operador $\Lambda$ en la notación de Dirac\\
  $\matrixel{\psi_i}{\Lambda}{\psi_j}$ & elemento de matriz $\Lambda_{ij}$\\
  $\braket{\psi}{\phi}$ & \textit{braket}, producto interno en la notación de Dirac\\
  $\dyad{\psi}{\phi}$ & producto externo entre $\ket{\psi}$ y $\ket{\phi}$ en la notación de Dirac \\
  $\rho$ & matriz de densidad\\
  $\Tr$ & traza \\
	$\E$ & canal cuántico \\
	$U$ & operador unitario \\
	$\otimes$ & producto tensorial\\
	$\ket{\psi}\ket{\phi}$ & producto tensorial $\ket{\psi}\otimes\ket{\phi}$\\
	$\1$ & operador identidad \\
	$\sigma_i$ & matrices de Pauli\\
	$\mapsto$ & ``se mapea a''\\
	$\vec\rho$ & matriz de densidad vectorizada\\
	$\mathcal{M}_d$ & espacio de las matrices de $d\times d$\\
	$\mathcal{HS}$ & espacio de Hilbert-Schmidt\\
	$\Lambda^{\dagger}$ & operador adjunto de $\Lambda$\\
	$\delta_{ij}$ & delta de Kronecker\\
	$\taus$ & elementos de la diagonal de un superoperdador PCE en la base de Pauli\\
	$\lambda_i$ & eigenvalores
\end{longtable}
\janote{aquí hay que completar a mano}
  % Lista de símbolos

%%% Haga el diseño que más le guste
\chapter{OBJETIVOS}
\section*{General}
Estudiar los mapeos de borrado de componentes de Pauli (PCE por
sus siglas en inglés) en sistemas de 2 y 3 qubits.


\section*{Específicos}

\begin{enumerate}
\item Estudiar numéricamente la completa positividad de los mapeos 
PCE en sistemas de 2 y 3 qubits.

\item Estudiar las características de los canales PCE.

\item Comparar los canales PCE con otros canales de Pauli que han 
sido previamente estudiados.

\item Desarrollar una herramienta geométrica para entender los 
mapeos PCE.
\end{enumerate}

      % Resumen y objetivos

%%% Haga el diseño que más le guste
\chapter{INTRODUCCIÓN}
\esqueleto{Ningún sistema cuántico en la vida real es un sistema completamente
aislado del resto del universo. En realidad, todos los sistemas cuánticos interactúan,
en menor o mayor grado, con un sistema cuántico externo que se conoce 
como entorno.}


\esqueleto{La decoherencia es un proceso al que irremediablemente están 
sujetos los sistemas cuánticos abiertos. La decoherencia es el colapso de
la superposición de estados de un sistema a sólo uno de los estados de la 
superposición.}


\esqueleto{La teoría de los canales cuánticos en un marco conceptual que puede  
capturar la dinámica de los sistemas abiertos. Un canal cuántico
es una operación que modela un Hamiltoniano o un circuito cuántico. Así mismo, 
el proceso de decoherencia puede modelarse con canales cuánticos.}

\esqueleto{Los sistemas de dos niveles son por excelencia los sistemas cuánticos
más sencillos. Esto los hace los sistemas perfectos para comenzar a estudiar
herramientas para describir la decoherencia. Para 1 qubit, la decoherencia
puede modelarse con canales cuánticos de 1 qubit que han sido ampliamente 
estudiados en el pasado.}

\esqueleto{En este trabajo nuestro objetivo es estudia los canales cuánticos 
que modelan la decoherencia de sistemas de $n$ qubits.}
      % Introducción

\mainmatter     % --------------------------------------------------  Cuerpo del Trabajo de Graduación

\chapter{FUNDAMENTOS TEÓRICOS}
\section{Introducción} % {{{
El estudio de los sistemas cuánticos cerrados, es decir sistemas que no
interactúan con su entorno, nos ha permitido entender bastante bien muchos
fenómenos cuánticos. No obstante, una descripción más precisa 
requiere de considerar que los sistemas cuánticos reales son sistemas abiertos
que interactúan con su entorno. Para estudiar los 
sistemas cuánticos abiertos será útil revisar un formalismo distinto
al del vector de estado para describir a los estados cuánticos. 
Este formalismo es el de la matriz de densidad y tiene la ventaja 
de describir de manera más apropiada a los estados mixtos. 
Para la descripción de la evolución dinámica  
vamos a estudiar la teoría de los canales cuánticos, 
que es un marco teórico en el cual se considera que los estados 
cuánticos (matriz de densidad) evolucionan de forma discreta.

La estructura de este capítulo es la siguiente.
En la sección \ref{sec:ensambles} revisaremos una motivación para introducir 
a la matriz de densidad a partir de un ensamble de estados. 
Seguidamente, en la sección \ref{sec:density-matrices-properties},
estudiaremos las propiedades que debe cumplir una matriz 
de densidad para representar un estado cuántico
y cómo se reformulan los postulados 
de la mecánica cuántica utilizando este nuevo formalismo.
En la sección \ref{sec:qtm-channels} vamos a revisar las
condiciones para que un canal cuántico describa la evolución 
física de la matriz de densida. Por último, en la sección
\ref{sec:qtm-channels-representation}, vamos a estudiar 
la representación de superoperador y de Kraus de un canal cuántico.

% }}}
\section{Ensambles de estados cuánticos} \label{sec:ensambles} % {{{
% \esqueleto{Un copy-paste de la sección 1.1 del informe final de 
% prácticas. Voy a retocar alguna parte si fuera necesario, como ser
% más formal o agregar alguna prueba.}

Para introducir la definición de la matriz de densidad presentamos la 
motivación que exponen Sakurai y Napolitano \cite{sakurai_napolitano_2017}.
Consideremos un sistema cuántico que se encuentra en alguno de los estados 
$\Pk{i}$ con probabilidad $p_i$. Esto induce naturalmente el 
ensamble de estados del sistema $\{p_i, \ket{\psi_i} \}$. 
Supongamos que realizamos 
la medición de algún observable $\Lambda$ sobre el ensamble. El 
valor esperado al medir $\Lambda$ sobre este sistema es
\begin{align}\label{eq:expVal-expanded-Lambda}
	\expval{\Lambda} &= \sum_i p_i \matrixel{\psii}{\Lambda}{\psii}
	= \sum_{i,j,k} p_i 
	\bra{\psii}\dyad{\phi_j}{\phi_j}\Lambda\dyad{\phi_k}{\phi_k}\Pk{i},
\end{align}
con $\ket{\phi_j}$ es una base ortonormal del
espacio de Hilbert del sistema. Si se reordena
\eqref{eq:expVal-expanded-Lambda} de manera apropiada
se obtiene una expresión 
que motiva claramente la definición de la matriz de densidad $\rho$,
\begin{align}\label{eq:expVa-Lambda-wRho}
	\expval{\Lambda}&= \sum _{j,k}\bra{\phi_k}\qty(\sum_ip_i \dyad{\psii}{\psii} 
	)\ket{\phi_j}	\matrixel{\phi_j}{\Lambda}{\phi_k}.
\end{align}
La matriz que se encuentra entre paréntesis se define como 
la matriz de densidad 
$\rho$~\cite{nielsen_chuang_2011, sakurai_napolitano_2017}
\begin{equation}\label{eq:rho_def}
	\rho \equiv \sum _i p_i\dyad{\psi_i}{\psi_i}.
\end{equation}
Notemos que un elemento de matriz de $\rho$, escrita en la base 
$\ket{\phi_j}$, es
\begin{align}
	\matrixel{\phi_k}{\rho}{\phi_j} = 
	\sum_ip_i \braket{\phi_k}{\psii}\braket{\psii}{\phi_j},
\end{align}
por lo tanto, sustituyendo en la ecuación \eqref{eq:expVa-Lambda-wRho}
se tiene que el valor promedio del observable $\Lambda$ es
\begin{align}\label{eq:Tr(rhoLambda)}
	\expval{\Lambda}=\sum _k \matrixel{\phi_k}{\rho \Lambda}{\phi_k} 
	= \Tr \qty(\rho\Lambda).
\end{align} 
%Este resultado muestra cómo a partir de la matriz de densidad $\rho$ 
%se puede calcular toda la información física disponible de un sistema al 
%realizar una medición. 
Este resultado es interesante porque muestra que es posible 
calcular el valor promedio de un observable utilizando la 
matriz de densidad del sistema. En virtud de este resultado
vale la pena investigar a continuación cómo evoluciona la 
matriz de densidad de un sistema.
%Desde luego la matriz de densidad es una herramienta con la que
%se puede formular matemáticamente la mecánica cuántica como 
%con el vector de estado.
% \cpnote{Esta frase la pondría
% al principio del capítulo y lo que sigue del párrafo lo pondría al mismo 
% nivel conceptual que la discusión anterior. }. 
% \janote{De acuerdo. En la última iteración que hice de la introducción
% tome en cuenta este comentario tuyo.}
% 
% \janote{Me habías dejado los siguientes dos comentarios entre el texto
% que tenía antes. Así que opté por reescribir el desarrollo de la evolución
% dinámica de una vez con el operador $U$.}
% \cpnote{En aras de la simplicidad, 
% yo formularía el ejemplo directamente con al $U$. No hay necesidad de traer el 
% Hamiltoniano acá.}
% \cpnote{El título 
% de la siguiente sección contradice la ultima frase. quizá vale la pena que revises
% este caputulo y lo leas todo a ver si hay mas problemas como de estructura. }
% 

Consideremos un sistema que se encuentra en el ensamble 
de estados inicial $\{ p_i, \ket{\psi_i(0)}\}$
y que evoluciona según algún operador unitario $U(t)$. Es decir, 
el ensamble de estados en cualquier tiempo $t>0$ está dado por 
$\{p_i, U(t)\ket{\psi_i(0)}\}$. Entonces, utilizando la definición 
\eqref{eq:rho_def} recién introducida, la matriz de
densidad $\rho(t)$ del sistema será
\begin{align} \label{eq:Rho-evolution-H}
	\rho(t) &= \sum_i p_i\dyad{\psi_i(t)}{\psi_i(t)}\nonumber\\
	&= \sum_i p_i U(t) \dyad{\psi_i(0)}{\psi_i(0)}U^{\dagger}(t)
	\nonumber\\
	&= U(t)\rho(0)U^{\dagger}(t),
\end{align}
donde $\rho(0)$ es la matriz de densidad del ensamble 
de estados inicial $\{p_i, \ket{\psii(0)}\}$. Hemos probado así
que la descripción dinámica de un sistema que evoluciona 
según un operador unitario puede hacerse utilizando 
su matriz de densidad. En la sección que sigue 
estudiaremos las propiedades que deben satisfacer 
las matrices de densidad en general y revisaremos cómo 
formular los postulados de la mecánica cuántica con la matriz de densidad.

%
%Por ejemplo, veamos a continuación cómo 
%describir la evolución de un sistema cuántico cerrado.
%Consideremos un sistema cuyo Hamiltoniano es $H$, y es
%independiente del tiempo. Bajo estas condiciones la evolución
%del sistema está descrita por 
%$\ket{\psi (t)}=U(t)\ket{\psi(0)}$~\cite{sakurai2010modern}. 
%Sin embargo, consideremos que el sistema se encuentra inicialmente
%en un ensamble de estados $\{p_i, \ket{\psii(0)}\}$, por lo cual
%el estado final del sistema será 
%$\{p_i, \ket{\psii(t)}\}$.
%Por consiguiente, el matriz de densidad final $\rho(t)$  es 
%\begin{align} \label{eq:Rho-evolution-H}
%	\rho(t) &= \sum_j p_j\dyad{\psi_j(t)}{\psi_j(t)}\nonumber\\
%	&= \sum_j p_j U(t) \dyad{\psi_j(0)}{\psi_j(0)}U^{\dagger}(t)
%	\nonumber\\
%	&= e^{-iHt}\rho(0)e^{iHt},
%\end{align}
%donde $\rho(0)$ es la matriz de densidad que describe al ensamble 
%de estados inicial $\{p_i, \ket{\psii(0)}\}$.	 Aunque hemos desarrollado 
%un ejemplo para la evolución de un sistema cuyo Hamiltoniano es 
%independiente del tiempo, es sencillo de ver que en general la dinámica  
%de un sistema cerrado se describe como 
%\begin{align}\label{eq:rho-ClosedEvolution}
%\rho(0) \longrightarrow U\rho(0)U^{\dagger}.
%\end{align}
%Con esto, se ha asegurado que la dinámica de un sistema cuántico puede 
%describirse utilizando su matriz de densidad. 


%Las ecuaciones \eqref{eq:Tr(rhoLambda)} y \eqref{eq:rho-ClosedEvolution}
%muestran que la matriz de densidad puede utilizarse para la descripción 
%de la medición y la evolución de los estados cuánticos. 
%En la siguiente sección veremos la formulación de los postulados 
%de la mecánica cuántica utilizando la matriz de densidad. 


% }}}
\section{Propiedades de la matriz de densidad} % {{{
% \cpnote{Aca voy. Lo anterior ya queda listo}
\label{sec:density-matrices-properties}
% \esqueleto{Revisando esta sección en el informe de prácticas 
% veo que me gustaría ir aquí más al grano y mandar al lector 
% a las pruebas en el Chuang (para no copiar otra vez las pruebas
% aquí). Puntualizaré: (1) caracterización de la matriz de densidad, (2)
% postulados de la mecánica cuántica usando la matriz de densidad y
% (3) matriz de densidad reducida.}


A pesar de que contamos con la definición \eqref{eq:rho_def} de 
la matriz de densidad dado un ensamble de estados, será útil 
estudiar cuáles son las condiciones que una matriz debe cumplir 
para ser una matriz de densidad. Luego de esto, estamos listos 
para revisar la formulación de los postulados de la mecánica cuántica
utilizando la matriz de densidad. Por último, vamos a estudiar la
matriz de densidad reducida, la manera para describir a los
subsistemas de sistemas compuestos con el formalismo 
de la matriz de densidad.
% \cpnote{La matriz de densidad es una
% matriz, no una aplicación (en su manera más simple\ldots). Quizá te 
% quieras referir a la traza parcial?}\janote{No. Me refiero a una 
% aplicación del formalismo de la matriz de densidad. Reformulé 
% esa frase.}

Las propiedades que una matriz debe tener para que esta represente al 
estado físico de un sistema se establecen a 
continuación~\cite{nielsen_chuang_2011}:
\begin{thm}\label{teo:density-matrix}
Una matriz $\rho$  que actúa sobre el espacio de Hilbert de un sistema 
es la matriz de densidad asociada a algún ensamble 
$\{p_i, \ket{\psi _i} \}$ si y sólo si satisface las condiciones:
\begin{enumerate}
\item $\Tr \rho = 1$.
\item $\matrixel{\psi}{\rho}{\psi} \geq 0$, $\forall\ket{\psi} \in \hi$.
\end{enumerate}	
\end{thm} 
\begin{proof} Consultar~\cite[p.~101]{nielsen_chuang_2011} \end{proof}

La condición de traza unitaria de la matriz de densidad es análoga
a la condición de normalización de la función de onda, las probabilidades
de medir al sistema en un estado u otro deben sumar uno. 
Por otro lado, los eigenvalores $\lambda_i$ y eigenvectores $\ket{\psi_i}$ de la 
matriz de densidad definen a uno de los posibles ensambles de estados
$\{\lambda_i, \ket{\psi_i} \}$ del sistema. Entonces,
la condición de positividad de la matriz de densidad asegura que las
probabilidades $\lambda_i$ de encontrar al sistema en el estado $\ket{\psi_i}$
son cero o positivas, como esperaríamos que fuese. 
% \cpnote{Siengo que la notación
% $A\ge 0$ no es obvia. Quizá quieras decir en este párrafo que es. Lo dices 
% implicitamente, pero creo qe hay que ser tantito mas explicito.}\janote{Qué
% tal mejor sólo modificar la notación para la positividad en el teorema 1.3.1
% así como lo hice?}

Ahora que que ya contamos con una definición de la matriz de densidad 
que no parte de conocer \textit{apriori} el ensamble de estados del sistema
nos ocupamos de la formulación de los postulados de la mecánica cuántica 
utilizando este nuevo formalismo~\cite[p.~102]{nielsen_chuang_2011}.
\begin{itemize}
	\item[] \textbf{Postulado 1.} \textit{Estado del sistema.} 
	Un sistema físico tiene asociado un espacio vectorial complejo
	con producto interno conocido como el espacio de Hilbert $\hi$ del
	sistema. Los estados del sistema están descritos por el conjunto 
	de matrices de densidad que actúan sobre $\hi$.
	\item[] \textbf{Postulado 2.} \textit{Evolución unitaria.}
	La evolución de un sistema cuántico cerrado, en un intervalo 
	de tiempo $[t_1,t_2]$, está descrita	por una transformación unitaria $U$
	de la siguiente manera 
	\begin{equation} \label{eq:postulate-ClosedEvolution}
	\rho(t_1)\longrightarrow U\rho(t_1) U^{\dagger}=\rho(t_2).
	\end{equation}
	\item[] \textbf{Postulado 3.} \textit{Medición.}
	Las mediciones sobre el estado $\rho$ de un sistema 
	están descritas por un conjunto de operadores $\{M_m\}$. 
	Estos son operadores que actúan sobre $\hi$ y el 
	índice $m$ refiere a los posibles
	resultados de la medición. Si el estado del sistema es $\rho$ 
	inmediatamente antes de la medición, entonces la probabilidad
	$p(m)$ de obtener el resultado $m$ es
	\begin{equation} \label{eq:post_MeasureProb}
	p(m)=\Tr \qty(M_m^{\dagger}M_m\rho),
	\end{equation}						
	y el estado del sistema inmediatamente después de la medición será
	\begin{equation} \label{eq:post_MeasureTrasnfState}
	\rho'=\frac{M_m\rho M_m^{\dagger}}{\tr \qty(M_m^{\dagger}M_m\rho)}.
	\end{equation}	
	Los operadores $M_m$ deben satisfacer la ecuación de completitud
	\begin{equation} \label{eq:post_MeasureMCompleteness}
	\sum _m M_m^{\dagger}M_m=\mathbb{1}.
	\end{equation}
	\item[] \textbf{Postulado 4.} \textit{Sistemas de partículas.}
	El espacio de Hilbert de un sistema 	de varias partículas se compone del
	producto tensorial de los espacios de Hilbert 	individuales.
	Es decir, si el sistema total se compone de $N$ partículas, 
	entonces el espacio de Hilbert total es
	\begin{align}\label{eq:postulado4}
		\hi_{\txt{total}} = \hi_1\ten \hi_2 \ten \ldots \ten \hi_N.
	\end{align}
	Los estados del sistema total están descritos por las matrices de 
	densidad que actúan sobre $\hi _{\text{total}}$.
\end{itemize}

% Respecto al último postulado debemos resaltar que no todos 
% los estados del sistema total son de la forma $\rho=\rho_1
% \ot \ldots \ot \rho_N$. Es decir, no todos los estados del sistema 
% total son factorizables \cpnote{Acá tenemos que hablar. Hay varios tipos de estados, 
% los factorizables, los separables y los enlazados. Siento que no conoces la diferencia 
% entre los primeros dos. Hablamos y corriges este parrafo por favor}
% \janote{Ve directo al siguiente comentario}, pues el
% conjunto de los estados del sistema total también está compuesto por estados 
% no factorizables, también llamados entrelazados.
% Son justo los estados entrelazados los que conducen a hacernos preguntas
% como ¿cuál es el estado en el que se encuentra sólo una parte del sistema total?
% Seremos más claros con un ejemplo. Si un sistema bipartito 
% se encuentra en un estado factorizable $\rho\ot \sigma$ sabemos 
% que una parte del sistema se encuentra en el estado $\rho$ y la otra 
% en el estado $\sigma$. No obstante, ¿cuál es el estado
% de bipartición del sistema cuando el estado total es uno entrelazado
% ($\rho_{\text{total}}\neq\rho\ot \sigma$)? O,
% al menos, es deseable conocer cuál es la información accesible 
% para un observable de sólo una parte del sistema. Para responder a 
% estas preguntas revisaremos en lo que resta de esta sección
% la matriz de densidad reducida. Esta aplicación de la matriz de densidad 
% proporciona una herramienta para la descripción de subsistemas 
% de sistemas cuánticos compuestos~\cite{nielsen_chuang_2011}.
% \cpnote{Creo que igual este párrafo sobra. De nuevo, lo platicamos}
% \janote{Quisiera mantenerlo porque hace una transición de hablar 
% de los postulados a la matriz de densidad reducida. Lo iteré y es 
% el párrafo nuevo que le sigue a este comentario:}

Debemos resaltar que aunque el espacio de Hilbert de muchas partículas 
es de la forma \eqref{eq:postulado4} no todas las matrices de densidad
que actúan sobre son de la forma 
\begin{align}
\rho_{\text{total}}=\rho_1\ot \rho_2\ot \ldots\ot \rho_n,
\end{align}
lo cual nos conduce a preguntarnos que dada una matriz de densidad de
un sistema compuesto $\rho_{\text{total}}$, ¿cuál es el estado 
en el que se encuentra cada uno de los subsistemas que lo componen?
Por esta razón, para averiguar cómo describir a los subsistemas con 
el formalismo de la matriz de densidad es que vamos a revisar a continuación
cómo introducir la matriz de densidad reducida utilizando una herramienta
conocida como la traza parcial \cite{nielsen_chuang_2011}. 
% \janote{Si este párrafo te pareció bien habría que borrar el anterior.}



Supongamos dos subsistemas $A$ y $B$ cuyo estado total es 
$\rho^{AB}$ (en general $\rho^{AB}\neq \rho^A\ot \rho^B$). 
Consideremos una base ortonormal $\ket{\psi_i}$ de $\hi_A$ y una 
base ortonormal $\ket{\phi_j}$ de $\hi_B$. Una base ortonormal 
del sistema total $\hi_{\text{total}}$ está dado 
por $\ket{\psi_i}\ot \ket{\phi_j}$.
En esta base un elemento de matriz de $\rho^{AB}$ está dado por
\begin{align}
	\bra{\psi_i}\bra{\phi_j}\rho^{AB}\ket{\psi_k}\ket{\phi_l},
\end{align}
donde hemos utilizado la notación 
$\ket{\psi}\ot\ket{\phi}=\ket{\psi}\ket{\phi}$.
Supongamos que $\Omega_A\ot \1$ es un observable que actúa sobre $A$
en $\hi_{\text{total}}$ y el valor esperado 
$\expval{\Omega_A\ot \1}$, utilizando~\eqref{eq:Tr(rhoLambda)},~es
\begin{align}
	\Tr \qty(\rho^{AB}\qty(\Omega_A\ot \mathbb{1})) &= \sum _{i,j} 
	\bra{\psi_i}\bra{\phi_j}\rho^{AB}\qty(\Omega_A\ot \mathbb{1})
	\ket{\psi_i}\ket{\phi_j} 
	\nonumber\\
	&= \sum _{i,j} 
	\bra{\psi_i} \bra{\phi_j}\rho^{AB}
	\qty(\sum _{k,l} \ket{\psi_k} \ket{\phi_l} \bra{\psi_k}\bra{\phi_l} )
	\qty(\Omega_A\ot \mathbb{1})\ket{\psi_i} \ket{\phi_j}\nonumber\\
	&= \sum _{i,j,k,l} 
	\bra{\psi_i} \bra{\phi_j}\rho^{AB}\ket{\psi_k} \ket{\phi_l}
	\matrixel{\psi_k}{\Omega_A}{\psi_i}\delta _{lj}\nonumber\\
	&= \sum_{i,k}\qty(\sum _j 
	\bra{\psi_i} \bra{\phi_j}\rho^{AB}\ket{\psi_k} \ket{\phi_j})
	\matrixel{\psi_k}{\Omega_A}{\psi_i}.
	\label{eq:almost-reducedRho}
\end{align}
% donde hemos utilizado $\ket{\psi}\ot\ket{\phi}=\ket{\psi}\ket{\phi}$\cpnote{Esto
% es mas bien una definicipon que de hecho puedes hacer más arriba}
% \janote{Listo. Lo definí después de la ecuación 1.13. Si te parece bien 
% porfa lo que sigue aquí}.
La matriz entre paréntesis define a la matriz de densidad
reducida $\rho^A$ del subsistema $A$ como~\cite{chandra2013quantum}
\begin{align}
	\sum _j \bra{\psi_i}\bra{\phi_j}\rho^{AB}\ket{\psi_k}\ket{\phi_j}
	= \matrixel{\psi_i}{\rho^A}{\psi_k}.
	\label{eq:reducedRho-def1}
\end{align}
y retomando el cálculo de $\expval{\Omega_A\ot \1}$
en \eqref{eq:almost-reducedRho} se obtiene, en función de $\rho^A$,
\begin{align}
	\Tr \qty(\rho^{AB}\qty(\Omega_A\ot \mathbb{1}))&= \sum _{i,k}\nonumber
	\matrixel{\psi_i}{\rho^A}{\psi_k} \matrixel{\psi_k}{\Omega_A}{\psi_i}
	\nonumber\\
	&= \sum_i \matrixel{\psi_i}{\rho^A\Omega_A}{\psi_i}\nonumber\\
	&= \Tr \qty(\rho^A\Omega_A). \label{eq:reduced-works}
\end{align}
Notemos la similitud de este resultado con el que se obtuvo en la ecuación
\eqref{eq:Tr(rhoLambda)}. El valor esperado de un observable que actúa 
sobre sólo uno de los subsistemas puede calcularse con su matriz de 
densidad reducida.

Ya que mostramos la utilidad que puede tener la matriz de densidad
reducida, ahora vamos a discutir con detalle la operación que 
la define en \eqref{eq:reducedRho-def1}. 
Esta operación se conoce como traza parcial. El adjetivo `parcial'
es porque la operación de traza se realiza sólo sobre los grados 
de libertad de alguno de los subsistemas. En la ecuación 
\eqref{eq:reducedRho-def1} denotamos a la matriz de densidad 
reducida $\rho^A$ como la traza parcial sobre $B$ de $\rho^{AB}$,
\begin{align} 
	\sum _j \bra{\psi_i}\bra{\phi_j}\rho^{AB}\ket{\psi_k}\ket{\phi_j}
	&=\matrixel{\psi_i}{\Tr_B\qty(\rho^{AB})}{\psi_k}
	= \matrixel{\psi_i}{\rho^A}{\psi_k}.
	\label{eq:partialTrace-def}
\end{align}
En general, la traza parcial es una operación lineal
que se define mediante la relación~\cite{nielsen_chuang_2011}
\begin{align}
	\Tr_B (\dyad{\alpha_i}{\alpha_j}\otimes \dyad{\beta_k}{\beta_l})
	\equiv
	\dyad{\alpha_i}{\alpha_j}\Tr \qty(\dyad{\beta_k}{\beta_l}),
	\label{eq:part_trace-def}
\end{align}
donde $\ket{\alpha_i}$ son vectores ortonormales del subespacio $\hi_A$
y $\ket{\beta_j}$ del resto del espacio de Hilbert total.

Ahora que contamos con las propiedades de la matriz de densidad, 
los postulados reescritos con este formalismo y la aplicación de la matriz
de densidad reducida para describir subsistemas ya disponemos de
las herramientas adecuadas para describir a los estados cuánticos 
utilizando a la matriz de densidad. En la siguiente sección continuamos 
con la evolución de la matriz de densidad utilizando la teoría de los 
canales cuánticos.

% }}}
\section{Canales cuánticos}\label{sec:qtm-channels} % {{{

La teoría de los canales cuánticos es un marco teórico con el cual 
se puede describir la evolución de los sistemas cuánticos abiertos.
Un canal cuántico es una operación lineal que debe preservar las 
propiedades de la matriz de densidad y que, cuando el sistema forma
parte de un sistema más grande junto con un \textit{ancilla}, la
extensión del canal que actúa sobre el sistema total debe también 
de preservar la traza y positividad de la matriz de densidad total.

Matemáticamente se escribe la acción de un canal cuántico $\E$ 
sobre un estado $\rho$ como
\begin{align} \label{eq:E(rho)}
\rho' = \E (\rho),
\end{align} 
donde $\E$ es una operación lineal y $\rho'$ es una matriz de
traza unitaria y positiva. Para que $\E$ sea un canal cuántico 
se requiere también que sea operación completamente positiva (CP). 
Antes de enunciar una de las definiciones de la completa positividad
vamos a elaborar un ejemplo con el objetivo de exponer mejor la 
implicación física de esta condición.

El ejemplo que vamos a discutir es el de una operación que actúa 
sobre $\rho$ como en \eqref{eq:E(rho)} pero que no es CP. Por lo tanto,
la extensión de la operación que actúa sobre un sistema que interactúa con un 
\textit{ancilla} no transforma al estado máximamente entrelazado en 
un estado físico. La operación que consideraremos 
actúa sobre partículas de espín 1/2, i.e. sistemas cuánticos de
dos niveles. En computación e información cuántica
se conoce a estos sistemas como qubits. Una manera
de escribir a la matriz de densidad de 1 qubit es
\begin{align}
\rho&=\frac{1}{2}\sum_{i=0}^{3} r_i\sigma_i,
\label{eq:rho-1qubit}
\end{align}
donde $\sigma_0=\1$ y el resto de $\sigma_i$ las 3 matrices de Pauli.
Imponemos que $r_0=1$ para que $\Tr(\rho)=1$.
Esta forma de escribir a la matriz de densidad de 1 qubit es útil porque
las componentes $r_1$, $r_2$ y $r_3$ especifican las coordenadas 
cartesianas de un punto en la esfera de Bloch, esfera unitaria que 
se utiliza para representar a los estados de 1 qubit. Similarmente, 
la matriz de densidad de 2 qubits se escribe como~\cite{nielsen_chuang_2011}
\begin{align}\label{eq:rho-2qubits}
\rho=\frac{1}{4}\sum _{i,j=0}^{3}r_{ij}\sigma_i\otimes\sigma_j,
\end{align}
con $r_{00}=1$.
Consideremos entonces la operación lineal $\E_z$ de 1 qubit
que mapea la esfera de Bloch a un disco sobre el plano $x$-$y$ 
como se muestra en la \Fref{fig:qtm-op-motivation}.
En términos de las componentes $r_i$, la operación $\E_z$ 
transforma a una matriz de densidad $\rho$ como en 
\eqref{eq:rho-1qubit} de la siguiente manera
\begin{align}
\qty(1,r_1,r_2,r_3)\mapsto\qty(1,r_1,r_2,0).
\end{align}
\begin{figure}% {{{
\centering
\begin{minipage}{.4\textwidth}
\centering
\includegraphics[width=5cm]{bloch.png}
\end{minipage}
$\stackrel{\E_{z}\otimes\1 \vspace{1cm}}{\longmapsto}$
\begin{minipage}{0.4\textwidth}
\centering
\includegraphics[width=6cm]{DiskXY}
\end{minipage}
\caption{Deformación de la esfera de Bloch a un disco sobre el plano $XY$. \ep}
\label{fig:qtm-op-motivation}
\end{figure} % }}}
Aunque podemos revisar explícitamente que las propiedades 
de traza unitaria y positividad de $\rho$ en \eqref{eq:rho-1qubit}
se preservan, también podemos argumentar geométricamente
que, dado que el disco final se contiene en la esfera de Bloch, 
las propiedades de la matriz de densidad de 1 qubit se preservan.
Sin embargo, veremos que la acción de $\E_z\ot \1$ 
sobre el estado máximamente entrelazado de 2 qubits 
se transforma en una matriz no positiva.

El estado máximamente entrelazado de 2 qubits es
$\ket{\phi}=\qty(\ket{00}+\ket{11})/\sqrt{2}$~\cite{bengtsson_zyczkowski_2017}.
Dado que conocemos cómo transforma $\E_z$ a la matriz de densidad 
escrita en la base de matrices de Pauli será necesario calcular 
la representación de $\dyad{\phi}{\phi}$ escrita como en \eqref{eq:rho-2qubits}
para luego calcular $\E_z\ot \1(\dyad{\phi}{\phi})$.
Las componentes $r_{ij}$ se calculan usando el producto interno
de Hilbert-Schmidt $\Tr\qty(\pauli{i}{j}\dyad{\phi}{\phi})$.
Así encontramos 
\begin{align}
\dyad{\phi}{\phi}=\frac{1}{4}\qty(
\pauli{0}{0}+\pauli{1}{1}-\pauli{2}{2}+\pauli{3}{3}).
\end{align}
Recordemos que $\E_z$ borra la componente $r_3$ de la
matriz de densidad de 1 qubit en \eqref{eq:rho-1qubit}. 
Al calcular el superoperador $\E_z\ot  \1$, la representación matricial de 
un canal cuántico que veremos en la próxima sección, se encuentra
que $\E_z\otimes\1$ actuando sobre una matriz de densidad de 2 qubits
escrita como en \eqref{eq:rho-2qubits} borra
las componentes de la forma $r_{3j}$. Por consiguiente
\begin{align}
\E_z\otimes\1 \qty(\dyad{\phi}{\phi})=\frac{1}{4}\qty(
\pauli{0}{0}+\pauli{1}{1}-\pauli{2}{2}),
\end{align}
que al calcular su representación matricial se tiene
\begin{align}
\E_z\otimes\1 \qty(\dyad{\phi}{\phi})=
\mqty( 
\frac{1}{4} & 0 & 0 & \frac{1}{2} \\
0 & \frac{1}{4} & 0 & 0 \\
0 & 0 & \frac{1}{4} & 0 \\
\frac{1}{2} & 0 & 0 & \frac{1}{4} \\
).
\end{align}

La matriz  $\E_z\otimes\1\qty(\dyad{\phi}{\phi})$ tiene un 
eigenvalor igual a $-1/4$ y, por lo tanto, no satisface la condición 
de positividad para ser una matriz de densidad.  
En otras palabras, $\E_z\otimes\1\qty(\dyad{\phi}{\phi})$ 
no representa a un estado físico de 2 qubits y 
se dice entonces que $\E_z$ no es una operación completamente 
positiva porque existen estados en el espacio extendido de 2 qubits para 
el cual la extensión $\E_z\ot \1$ no preserva la positividad de esos estados. 
De esta manera acabamos de mostrar que la condición de completa
positividad surge de la posibilidad que un sistema tiene de encontrarse 
en un estado entrelazado con un \textit{ancilla}.

Ya que elaboramos la implicación física de la completa positividad 
vamos a establecer una definición precisa de esta condición con la
que debe de cumplir un canal cuántico.
Se dice que una operación $\E$ es CP si 
y sólo si, para cualquier extensión arbitraria de dimensión $K$ 
del espacio de Hilbert $(\hi_N \rightarrow \hi_N \otimes \hi_K)$ 
el operador $\E\otimes\1_K$ es positivo~\cite{bengtsson_zyczkowski_2017}. 
Con esta definición hemos terminado de revisar las dos condiciones que 
debe de satisfacer una operación lineal para ser un 
canal cuántico.

Un canal cuántico es un mapeo lineal  (1) que preserva las características
de la matriz de densidad y (2) que es completamente positivo. 
En la literatura se suele utilizar el término operaciones 
completamente positivas que preservan la traza 
(CPTP) para referirse a los canales 
cuánticos~\cite{bengtsson_zyczkowski_2017}. 
En la siguiente y última sección de este capítulo estudiaremos 
algunas representaciones de canales cuánticos. 
% }}}
\section{Representaciones de los canales cuánticos} % {{{
\label{sec:qtm-channels-representation}
En esta sección estudiaremos dos representaciones distintas de 
los canales cuánticos: los superoperadores y la representación 
de suma de operadores de Kraus. Para cada una de ellas, vamos a revisar
cómo escribir las condiciones que hacen a una operación lineal un 
canal cuántico. La representación como superoperador 
será especialmente útil para investigar el problema
que vamos a plantear en el siguiente capítulo. No obstante, los 
operadores de Kraus siguen siendo una parte importante dentro 
de la teoría de los canales cuánticos y por eso dedicamos brevemente
la última parte de esta sección a estudiarlos.

Los canales cuánticos pueden entenderse como operadores 
que actúan sobre otros operadores. 
La matriz de densidad actúa sobre 
el espacio de Hilbert y un canal cuántico actúa sobre las matrices de 
densidad, que forman parte del espacio de Hilbert-Schmidt. 
La diferencia es entonces que las matrices de densidad 
son operadores que actúan sobre el espacio de 
Hilbert y los canales cuánticos actúan sobre el espacio 
de Hilbert-Schmidt. Para hacer una diferencia en 
la naturaleza de los dos tipos de operadores, los canales cuánticos 
pueden recibir el nombre de superoperadores~\cite{preskill1998lecture}.

Vamos a revisar primero cómo `vectorizar' a la matriz de densidad 
para después exponer cómo es la representación matricial de
un canal cuántico como un superoperador.
Consideremos una matriz de densidad $\rho$ de dimensión $d\times d$.
Para vectorizar a $\rho$, los elementos de matriz $\rho_{ij}$ 
se deben acomodar en un vector columna $\vec{\rho}$, con componentes 
$\rho_k$, siguiendo la ecuación~\citep{bengtsson_zyczkowski_2017}
\begin{align}
\rho_k=\rho_{ij};\hspace{2cm} k=\qty(i-1)d+j,
\label{eq:matrix-to-vector}
\end{align}
donde $i,j=,1,\ldots,d$. Por ejemplo, la matriz de densidad
\begin{align}
\rho=\mqty(\rho_{00}&\rho_{01}\\ \rho_{10}&\rho_{11})
\end{align}
que actúa sobre un espacio de Hilbert de dimensión 2 se vectoriza como 
\begin{align}
\vec{\rho}=\mqty(\rho_{00}\\ \rho_{01}\\ \rho_{10}\\ \rho_{11}).
\end{align}
En general, una matriz de densidad $\rho$ que actúa 
sobre un espacio de Hilbert de dimensión $d$ es 
una matriz de dimensión $d\times d$. Si se vectoriza a $\rho$, la matriz 
de densidad se transforma en un vector $\vec{\rho}$ de $d^2$ 
componentes. Un canal cuántico $\E$ que actúa sobre $\rho$ se puede
representar entonces como una matriz de dimensión $d^2\times d^2$ 
que actúa sobre el vector $\vec{\rho}$. Esta representación matricial de 
$\E$ es la que se conoce como superoperador.

Antes de estudiar las condiciones que un superoperador $\E$
debe satisfacer para ser un canal cuántico es necesario que revisemos
dos herramientas algebraicas: (1) una notación de cuatro índices 
para los elementos de matriz de un operador, y (2) la transformación 
algebraica de \textit{reshuffle}. Primero, para introducir la nueva notación
consideremos a un superoperador $\E$ que actúa sobre un espacio de 
Hilbert-Schmidt de la forma $\hi\mathcal{S}_M\otimes\hi\mathcal{S}_N$. 
Sea $\ket{m}$ una base ortonormal de $\hi\mathcal{S}_M$, $\ket{\mu}$ 
una base ortonormal de $\hi\mathcal{S}_N$ y, por lo tanto, 
$\ket{m}\otimes\ket{\mu}$ una base ortonormal de 
$\hi\mathcal{S}_M\ot\hi\mathcal{S}_N$.
Nótese el uso de letras latinas para los índices del
primer subsistema y letras griegas para los índices del segundo. 
Utilizando cuatro índices, un elemento de matriz de $\E$ se calcula como
\begin{align}
\E_\ind{m\mu}{n\nu}=\matrixel{m\otimes \mu}{\E}{n\otimes \nu}.
\label{eq:4indices}
\end{align}
Por último, vamos a introducir la transformación algebraica de \textit{reshuffle}
de un superoperador $\E$ utilizando la nueva notación de cuatro índices 
en \eqref{eq:4indices}.
El \textit{reshuffle} $\E^R$ de un superoperador $\E$ se define 
como~\cite{bengtsson_zyczkowski_2017}
\begin{align}
\E^R_\ind{m\mu}{n\nu} = \E_\ind{mn}{\mu\nu}.
\label{eq:Reshuffle_4-indices}
\end{align}
El siguiente ejemplo ilustra cómo utilizar la notación de cuatro 
índices en \eqref{eq:4indices} y cómo hacer la transformación 
de \textit{reshuffle} en \eqref{eq:Reshuffle_4-indices} de un 
superoperador que actúa sobre un espacio de Hilbert-Schmidt de 
dimensión 4 de la forma $\mathcal{HS}_2\ot \mathcal{HS}_2$,
%\begin{align}
%\left(
%\begin{array}{cccc}
% \E_{11} & \E_{12} & \E_{13} & \E_{14} \\
% \E_{21} & \E_{22} & \E_{23} & \E_{24} \\
% \E_{31} & \E_{32} & \E_{33} & \E_{34} \\
% \E_{41} & \E_{42} & \E_{43} & \E_{44} \\
%\end{array}
%\right)^R
%=
%\left(
%\begin{array}{cccc}
% \E_{11} & \E_{12} & \E_{21} & \E_{22} \\
% \E_{13} & \E_{14} & \E_{23} & \E_{24} \\
% \E_{31} & \E_{32} & \E_{41} & \E_{42} \\
% \E_{33} & \E_{34} & \E_{43} & \E_{44} \\
%\end{array}
%\right).
%\end{align}
\begin{align}
\left(
\begin{array}{cccc}
 \E_{\ind{00}{00}} & \E_{\ind{00}{01}} & \E_{\ind{00}{10}} & \E_{\ind{00}{11}} \\
 \E_{\ind{01}{00}} & \E_{\ind{01}{01}} & \E_{\ind{01}{10}} & \E_{\ind{01}{11}} \\
 \E_{\ind{10}{00}} & \E_{\ind{10}{01}} & \E_{\ind{10}{10}} & \E_{\ind{10}{11}} \\
 \E_{\ind{11}{00}} & \E_{\ind{11}{01}} & \E_{\ind{11}{10}} & \E_{\ind{11}{11}} \\
\end{array}
\right)^R
=
\left(
\begin{array}{cccc}
 \E_{\ind{00}{00}} & \E_{\ind{00}{01}} & \E_{\ind{01}{00}} & \E_{\ind{01}{01}} \\
 \E_{\ind{00}{10}} & \E_{\ind{00}{11}} & \E_{\ind{01}{10}} & \E_{\ind{01}{11}} \\
 \E_{\ind{10}{00}} & \E_{\ind{10}{01}} & \E_{\ind{11}{00}} & \E_{\ind{11}{01}} \\
 \E_{\ind{10}{10}} & \E_{\ind{10}{11}} & \E_{\ind{11}{10}} & \E_{\ind{11}{11}} \\
\end{array}
\right).
\end{align}
La transformación de \textit{reshuffle} nos será útil específicamente 
para entender la condición de completa positividad que un superoperador $\E$
de cumplir para ser un canal cuántico.

Ahora que ya contamos con las herramientas algebraicas necesarias
estamos listos para enunciar las condiciones que debe cumplir un 
superoperador $\E$ para ser un canal cuántico. 
%Un superoperador 
%$\E$ es una matriz que actúa sobre una matriz de densidad vectorizada
%$\vec{\rho}$\cpnote{Quitaría esta frase. Para ser superoperador, actua sobre cualquier
%operador. Ya lo dijiste arriba y creo que acá no mas estas metiendo ruido}.
Recordemos de la sección anterior que un canal cuántico 
es una operación lineal que actúa sobre una matriz de densidad 
como en \eqref{eq:E(rho)}, donde $\rho'$
es una matriz de densidad y $\E$ es una operación completamente positiva.
La condición de positividad con la que debe cumplir la matriz de densidad 
implica que además es una matriz Hermítica. 
Por lo tanto, que un canal cuántico preserve la Hermiticidad, traza unitaria 
y positividad de la matriz de densidad impone tres condiciones 
sobre el superoperador $\E$~\cite{bengtsson_zyczkowski_2017}:
% \janote{Sobre la hermiticidad de $\rho$: es una propiedad que se puede 
% deducir de la positividad: si $\rho$ es positiva tiene descomposición 
% espectral $\sum_i\lambda_i\dyad{\psi_i}{\psi_i}$,
% entonces se sigue que $\rho=\rho^{\dagger}$. Por eso, arreglé este
% párrafo que acabas de terminar de leer para introducir la Hermiticidad
% como una propiedad que se deduce de las características de la 
% matriz de densidad que establece el teorema \ref{teo:density-matrix}.}
% \cpnote{No has definido rho prima (al menos no cerca), para que tngan sntido
% estas definiciones, tienes que decir de neuvo que rho prima es E sobre rho}
% \janote{Listo. Leete de nuevo este párrafo. Borré el enunciado que 
% comentaste que hacía ruido.}
\begin{align}
\txt{(i)}&& \rho'&=\qty(\rho')^{\dagger}&\Leftrightarrow
    && \E_\ind{m\mu}{n\nu}=\E_\ind{\mu m}{\nu n}^*,&&
    \label{eq:H-condition}\\
\txt{(ii)}&&\Tr(\rho')&=1
    &\Leftrightarrow&&  \sum_{m}\E_\ind{mm}{n\nu}=\delta_{n\nu},&
    \label{eq:Tr-condition}\\     
\txt{(iii)}&&\matrixel{\psi}{\rho'}{\psi}&\geq0,\ \forall \ket{\psi} \in \hi
    &\Leftrightarrow&&  \sum_{n\nu}\E_{\ind{m\mu}{n\nu}}\rho_{n\nu}\geq0,& 
    & \rho>0.
    \label{eq:positivity-condition}
\end{align}
De esta manera, sólo nos hace falta revisar la condición que debe cumplir
un superoperador $\E$ para ser una operación completamente positiva.
% \cpnote{no entiendo esta frase en este contexto. estamos evaluando algo? Ya
% evaluamos algo? Además creo que aca conviene separar parrafo pues pasas
% de algo general a lo particular. Como que siento qeu aca el flujo esta chistoson. 
% Dale una mirada a esto y le sigo luego} \janote{Listo. Reformulé lo que 
% quería decir con `evaluar'}
Para eso, se introduce a la matriz de Choi $D_{\E}$ de $\E$,
que se define como~\cite{bengtsson_zyczkowski_2017}
\begin{align}\label{eq:ChoiMatrix-via-reshuffle}
D_{\E}=\E^{R},
\end{align}
con $\E^R$ la transformación de \textit{reshuffle} definida en 
\eqref{eq:Reshuffle_4-indices}. El siguiente teorema establece 
cuándo $\E$ es una operación completamente positiva:
\begin{thm}{Teorema de Choi.}\label{thm:choi-CP}
Un superoperador lineal $\E$ es completamente positivo si y sólo si 
su matriz de Choi asociada $D_{\E}$ es positiva semidefinida.
\end{thm}
\begin{proof}
Vamos a presentar la demostración expuesta por Bengtsson
en~\cite[p. 281]{bengtsson_zyczkowski_2017}.
La matriz de Choi $D_{\E}$ de un 
canal cuántico $\E$ es una matriz Hermítica que 
actúa sobre el espacio de Hilbert-Schmidt $\mathcal{H}_{N^2}$
(espacio de las matrices de dimensión $N^2\times N^2$). 
Por el teorema de descomposición espectral, 
la matriz $D_{\E}$ se puede escribir~\cite{nielsen_chuang_2011}
\begin{align}\label{eq:choiThmProof_braketChoi}
D_{\E}&=\sum_{i}\lambda_i\dyad{\chi_i}{\chi_i},
\end{align}
con $\lambda_i$ sus eigenvalores. Dado que la matriz de Choi es una 
matriz Hermítica, $\lambda_i\in\mathbb{R}$. En la notación 
de cuatro índices introducida en \eqref{eq:4indices},
\eqref{eq:choiThmProof_braketChoi} se escribe
\begin{align}
D_{\ind{mn}{\mu\nu}}&=\sum_i\lambda_i
\chi^i_{mn}\qty(\chi_{\mu\nu}^i)^*.
\end{align}
Consideremos la acción de $\E\ot \1_N$ sobre un 
estado puro $z_{nn'}z^*_{\nu\nu'}$ de $\mathcal{H}_{N^2}$,
\begin{align}
\rho'_{mm'\mu\mu'}&=
\sum_{n,n',\nu,\nu'}
\E_{\ind{m\mu}{n\nu}}\delta_{\ind{m'\mu'}{n'	\nu'}}z_{nn'}z^*_{\nu\nu'}
=
\sum_{n,\nu}
\E_{\ind{m\mu}{n\nu}}z_{nm'}z^*_{\nu\mu'},
\end{align}
donde $\delta_{\ind{m'\mu'}{n'	\nu'}}=\delta_{m'n'}\delta_{\mu'\nu'}$,
\begin{align}
\sum_{n,\nu}\E_{\ind{m\mu}{n\nu}}z_{nm'}z^*_{\nu\mu'}&=
\sum_{n,\nu}D_{\ind{mn}{\mu\nu}}z_{nm'}z^*_{\nu\mu'}\nonumber\\
\rho'_{mm'\mu\mu'}&=
\sum_{n,\nu,i}\lambda_{i}
\chi^i_{mn}z_{nm'}\qty(\chi_{\mu\nu}^iz_{\nu\mu'})^*.
\label{eq:rho'_ChoiThmProof}
\end{align}
%%%%%%%%%%
% \here
% Nótese que utilizamos la definición de la matriz de Choi via 
% el \textit{reshuffle} del superoperador $\E$, 
% $\E_{\ind{m\mu}{n\nu}}=\E^R_{\ind{mn}{\mu\nu}}
% =D_{\ind{mn}{\mu\nu}}$.
% Finalmente, para determinar \cpnote{me parece que acá va ``determinar''
% (moidificando un poco la frase que sigue) o ``si requerimos\ldots'' algo asi.
% No lo cambio porque quiero resaltar que creo que abusas de la palabra
% ``evaluar''}\janote{Por favor revisa de nuevo esta demostración completa.
% Con tus comentarios me di cuenta que tenía errores, 
% hice cambios para que todo el flujo de ideas se entendiera mejor.
% Si está bien porga quita lo que se encuentra entre los \texttt{$\setminus$here}} si $\rho'$ es una matriz positiva 
% revisamos que el valor esperado de ella con un vector arbitrario~$y_{mm'}$
% \cpnote{Uno calcula los elementos de matriz con respecto a una base. Quizá
% lo que quieres hacer es calcular un valor esperado.}
% es no negativo,
% \here
%%%%%%%%%%
Por último, utilizamos el hecho de que $\rho'$ debe ser una matriz 
de densidad y, por ende, una matriz positiva semidefinida,
\begin{align}
\sum_{m,m',\mu,\mu'}
y_{mm'}\rho'_{mm'\mu\mu'}y^*_{\mu\mu'}
\geq0,
\end{align}
con $y_{mm'}$ un vector arbitrario de $\mathcal{H}_{N^2}$.
Utilizamos \eqref{eq:rho'_ChoiThmProof} para sustituir $\rho'$,
\begin{align}
\sum_{m,m',n,n'}\lambda_i\abs{y_{mn'}\chi^i_{mn}z_{nm'}}^2\geq0.
\end{align}
Para que esto sea cierto para cualesquiera vectores $z_{mm'}$ y  
´$y_{mm'}$, se debe de cumplir que $\lambda_i\geq0$. Recordemos
que $\lambda_i$ son los eigenvalores de $D_{\E}$.
En conclusión, para que $\rho'$ sea una matriz positiva, la matriz de Choi 
$D_{\E}$ también debe ser una matriz positiva. 
%%%%%%%%%%
% \here
% Por consiguiente, la matriz 
% de Choi $D_{\E}$ debe ser positiva semidefinida para que $\rho'$ 
% también lo sea. \cpnote{No lo veo. Explicamelo, o si puedes 
% poner una ecuación matricial, estaría padre.}\newline \noindent
% \here
%%%%%%%%%%
\end{proof}
En resumen, un superoperador $\E$ es un canal cuántico si 
cumple con las tres condiciones establecidas en las ecuaciones 
\eqref{eq:H-condition}, \eqref{eq:Tr-condition}
y \eqref{eq:positivity-condition}, además de lo que establece 
el teorema \ref{thm:choi-CP}, que su matriz de Choi sea 
positiva semidefinida. 

Como prometimos al inicio de esta sección, vamos a terminar 
este primer capítulo con la representación en operadores de Kraus 
de un canal cuántico. Los operadores de Kraus fueron 
introducidos por Karl Kraus en 1971 como una consecuencia
deducida por él del teorema de Stinespring para las operaciones 
completamente positivas~\cite{bengtsson_zyczkowski_2017}. 
Aunque la representación en operadores de Kraus y los
superoperadores puedan parecer fundamentalmente distintos, 
ambas representaciones guardan conexión mediante 
la matriz de Choi de un canal cuántico.

A diferencia de un superoperador, los operadores de Kraus representan
a un canal cuántico con operadores que actúan sobre el espacio 
de Hilbert, igual que las matrices de densidad. En la representación 
de Kraus, un canal cuántico $\E$ actúa sobre la matriz 
de densidad como
\begin{align}\label{eq:KrausRepresentation}
\E\qty(\rho)=\sum_k E_k\rho E_k^{\dagger},
\end{align}
para algún conjunto de operadores $\{E_k\}$ que satisfacen la 
condición~\cite{nielsen_chuang_2011}
\begin{align}
\sum _kE_k^{\dagger}E_k\leq\1,
\end{align} 
donde los operadores $E_k$ reciben el nombre de operadores de Kraus.
La libertad de la representación de Kraus permite que 
varios conjuntos distintos de operadores $\{E_k\}$ sean la representación
de un mismo canal cuántico. Por ello, es de interés
determinar un conjunto canónico de operadores de Kraus.

El conjunto canónico se puede determinar para todas las operaciones 
completamente positivas a partir de su matriz de Choi. Un mapeo 
completamente positivo $\E:\mathcal{M}_d\footnote{Espacio de 
las matrices de dimensión $d\times d$.}\to\mathcal{M}_d$ 
se puede escribir como~\citep{bengtsson_zyczkowski_2017}
\begin{align}
\E(\rho) = 
\sum_{k=1}^{r\leq d^2}\lambda_k\chi_k\rho\chi_k^{\dagger}
%= \sum_{i=1}^rE_k\rho E_k^{\dagger},
\label{eq:Kraus-canonical}
\end{align}
con $\lambda_k$ y $\chi_k$ los eigenvalores y eigenvectores 
normalizados de la matriz de Choi $D_{\E}$, así como $r$ el rango 
de $D_{\E}$. 
Estrictamente hablando, $\chi_k$ son los eigenvectores reordenados 
como matrices según \eqref{eq:matrix-to-vector}.
Comparando las ecuaciones \eqref{eq:KrausRepresentation} y
\eqref{eq:Kraus-canonical}, los operadores de Kraus canónicos 
se definen como $E_k=\sqrt{\lambda_k}\chi_k$ y deben satisfacer
la condición
\begin{align}
  \Tr\qty(E_i^{\dagger}E_j)=d_i\delta_{ij}.
  \label{eq:Tr-Kraus-canonical}
\end{align}
Por último, ya que $\E$ es una operación que preserva la traza de 
la matriz de densidad, debe imponerse la ecuación de completitud 
sobre los operadores canónicos de Kraus,
\begin{align}
  \sum_kE_k^{\dagger}E_k=\1.
%  \hspace{0.5cm}
%  \Rightarrow\hspace{0.5cm}
%  \sum_{k=1}^r\lambda_k=N .
  \label{eq:completeness-Kraus-canonical}
\end{align}
% \cpnote{no entiendo ni la flecha de la ecuación anterior ni el hecho de uqe sumen 
% a $N$} \janote{Era un ``por consiguiente''. Estuve pensando un rato 
% cómo explicar que era 'obvio' que suman N, pero no encontré cómo 
% así que lo comenté.
% Yo creo que no quita nada si sólo omito esa parte. Qué dices? }
En pocas palabras, un canal cuántico en la representación canónica de 
Kraus se escribe en la forma de \eqref{eq:Kraus-canonical}
siempre que los operadores canónicos de Kraus cumplan con 
la condiciones en \eqref{eq:Tr-Kraus-canonical} y 
\eqref{eq:completeness-Kraus-canonical}. 

En este capítulo introdujimos brevemente 
% \cpnote{Creo que es un poco mucho decir que 
% revisamos esos formalismos. Mas bine se introdujeron brevemente o algo así.}
% \janote{Jajaja sí, estoy de acuerdo. Apenas y escupí la representación de Kraus.} 
el formalismo de la matriz de densidad
y la teoría de los canales cuánticos. Por un lado, la matriz de densidad es una 
matriz positiva de traza unitaria que sirve de herramienta para 
representar a los estados cuánticos. 
Por otro lado, la teoría de los canales cuánticos es un marco teórico 
para describir la evolución discreta de los sistemas cuánticos abiertos. 
Un canal cuántico es una operación completamente positiva que 
preserva la traza de la matriz de densidad y se puede representar 
como superoperador o en la representación de 
suma de operadores de Kraus. La matriz de densidad y los canales 
cuánticos son el marco teórico dentro del cual estudiaremos, en 
el resto de este trabajo, un tipo de operaciones muy particulares 
que actúan sobre sistemas de qubits, operaciones que hemos bautizado 
con el nombre de operaciones que borran las componentes de Pauli.

% \cpnote{Bien, va quedando bonita la tesis.}
% }}}

      % Cap. 1 

\chapter{OPERACIONES PCE}

\section{Introducción}
\cpnote{Porfa tambien esqueleto de la into}

\section{Operaciones PCE}
\esqueleto{
Dos o tres párrafos máximo hablando sobre las operaciones PCE 
como operaciones proyectivas a la base de productos tensoriales de 
las matrices de Pauli. Me gustaría poner un problema de motivación
sólo para hacer más interesante la lectura a partir de acá y dejarle al
lector algo con lo que pueda entender qué onda con las operaciones 
PCE. Propongo lo siguiente: cadena de espines. Supongamos una 
cadena de espines de $N$ sitios y decimos que nos interesa saber 
si la matriz de densidad del $i$-ésimo espín puede proyectarse 
al subespacio cuya base son $\sigma_x$ y $\sigma_y$ (lo que 
quiero decir es que $(r_x,r_y,r_z)\to(r_x,r_y,0)$). Entonces hablo 
de que hay que considerar que el $i$-ésimo espín puede estar 
entrelazado con el resto de la cadena, sin embargo, basta con considerar
que se encuentra entrelazado con cualquiera de sus vecinos y revisar 
en qué se transforma la matriz de densidad de esos dos espines para 
averiguar si es posible tal evolución física. Me gustaría poner unas 
figuritas para hacer interesante esto. 
}

\cpnote{Siento qe aun no es un esqueleto sino ideas. De la motivación
no me gusta como la abordas. Yo pensaría mejor en hablar un poco de defasing
y de bitflip para uno y muchos qubits. Siento qeu tu propuesta está
aun un poco por fuera de tu alcance y te propondría mantenerla más simple.
Si quieres iteremos la propuesta, y cuando nos pongamos de acuerdo ya haces un esqueleto 
mas a nivel de parrafos.}

\section{1 qubit}
\esqueleto{
Resumen de los resultados de 1 qubit. Para seguir un orden lógico
en este capítulo voy a hablar de los resultados de 1 qubit con el 
problema resuelto analíticamente (está medio desordenado 
si hablo por acá del método numérico y luego lo vuelvo a hacer 
en la sección 2.5), a partir de las desigualdades
de los eigenvalores (el polihedro?).
}

\cpnote{Está bien. Acá antes de escribir quiero un esqueleto más detallado. Lo mismo para las siguientes secciones, 2.3, 2.4 y 2.5}

\section{El problema de $\mathbf{n}$ qubits}
\esqueleto{
Enunciado del problema para $n$ qubits. Creo que esto no da
para más de 1 o 2 párrafos. 
}

\section{Solución numérica}
\esqueleto{ 
Hablar de la solución numérica, similar al informe de prácticas, pero 
enfocado al problema de $n$ qubits. 
Aquí también agregar el link al repositorio y ahí voy a agregar unas cosillas
para mostrar cálculos de 2 y 3 qubits. 
}
      % Cap. 2 

\chapter{RESULTADOS DE 2 Y 3 QUBITS}
\janote{\textbf{Idea principal del capítulo:} los resultados gritan que los 
canales PCE sí se pueden caracterizar, pero nos hace falta LA idea (la de 
Francois, jajaja) para poder formalizar la caracterización general. Para mientras,
tenemos un listado de características que tienen sustento en los resultados
numéricos.}
\section{Introducción}

\section{Resultados}
\noindent
\esqueleto{Con el método numérico descrito en la sección 
\ref{sec:ch2_solucionNumerica} es posible analizar el caso de 2 qubits
completo. Por otro lado, el caso de 3 qubits es imposible de resolver 
completo a fuerza bruta.}

Para encontrar los canales cuánticos PCE de 2 y 3 qubits analizamos
numéricamente la completa positividad de todas las operaciones PCE de 2
qubits y parcialmente las de 3 qubits. 
Por un lado, de las $32,768$ operaciones PCE de 2 qubits encontramos 
$67$ canales cuánticos. La proporción de canales cuánticos PCE
a operaciones PCE $67:31,768$ da una idea de lo restrictivo 
que resulta que una operación PCE sea una operación físicamente 
realizable. 
Por otro lado, para el caso de 3 qubits 
estudiamos las operaciones PCE que dejan invariantes 1, 2, 3 y 4 
componentes de Pauli. No estudiamos más alla de 4 componentes de 
Pauli¨ 
invariantes porque el número de operaciones PCE es tan grande que 
el tiempo de cómputo y uso de memoria por parte de las herramientas
computacionales suponen un obstáculo. 

\noindent
\esqueleto{Los resultados de 2 qubits son... (una tabla con 
las listas de 1's y 0's de $\tau_{ij}$ por número $k$ de componentes 
de Pauli invariantes). «\textit{La idea con esta tabla es motivar las 
figuritas de la siguiente sección porque las listas de 1's y 0's no dicen 
ni madres.}»}

En la \Tref{tab:2qubitsPCEChannel1sAnd0s} mostramos los canales cuánticos 
PCE de 2 qubits obtenidos con las herramientas que describimos en las 
últimas dos secciones del capítulo anterior. La información que se muestra
en la tabla es (1) la configuración de 1's y 0's de los elementos
$\tau_{ij}$ que caracterizan cómo actúa el canal PCE sobre las 
componentes de Pauli de la matriz de densidad de un sistema de 
2 qubits, y (2) el número de componentes de Pauli que el canal PCE
deja invariante. 

Muy poco se puede inferir sobre las características de los canales 
PCE de 2 qubits a partir 
de la \Tref{tab:2qubitsPCEChannel1sAnd0s}. Las listas de 1's y 0's, 
como se presentan en la tabla, no permiten identificar casi ninguna 
propiedad de los canales PCE, a excepción de la cantidad de componentes 
de Pauli (cantidad de 1's) que el canal deja invariante. Además, a 
diferencia de las operaciones PCE de 1 qubit, la acción de las operaciones PCE
de 2 qubits no se pueden interpretar con ayuda de alguna herramienta 
geométrica como la esfera de Bloch. Por esa razón, en la sección 
\ref{sec:ch3_geometric_representation} discutiremos una 
herramienta geométrica con la cual estudiar las características de 
las listas de 1's y 0's de los canales PCE de 1, 2 y 3 qubits.
\begin{table}[]
\centering
\resizebox{\textwidth}{!}{%
\begin{tabular}{|P{0.6cm}|P{0.65cm}|P{0.65cm}|P{0.65cm}|P{0.65cm}|P{0.65cm}|P{0.65cm}|P{0.65cm}|P{0.65cm}|P{0.65cm}|P{0.65cm}|P{0.65cm}|P{0.65cm}|P{0.65cm}|P{0.65cm}|P{0.65cm}|P{0.65cm}|P{2.9cm}|}
\hline
\textbf{No.} 										 & \tauij{0}{0}		& \tauij{0}{1}   & \tauij{0}{2} 	 & \tauij{0}{3} 	& \tauij{1}{0}	 & \tauij{1}{1}		 & \tauij{1}{2} 	& \tauij{1}{3}	 & \tauij{2}{0}		& \tauij{2}{1} 		& \tauij{2}{2} 	 & \tauij{2}{3} 	& \tauij{3}{0} 	 & \tauij{3}{1}   & \tauij{3}{2}   & \tauij{3}{3} & \bf{Componentes de Pauli \boldmath{$r_{ij}$} invariantes} \\ \hline
\textbf{1}                         & 1                     & 0                     & 0                     & 0                     & 0                     & 0                     & 0                     & 0                     & 0                     & 0                     & 0                     & 0                     & 0                     & 0                     & 0                     & 0                     & 1                     \\ \hline
\textbf{2}                         & 1                     & 1                     & 0                     & 0                     & 0                     & 0                     & 0                     & 0                     & 0                     & 0                     & 0                     & 0                     & 0                     & 0                     & 0                     & 0                     & 2                     \\ \hline
\textbf{3}                         & 1                     & 0                     & 1                     & 0                     & 0                     & 0                     & 0                     & 0                     & 0                     & 0                     & 0                     & 0                     & 0                     & 0                     & 0                     & 0                     & 2                     \\ \hline
\textbf{4}                         & 1                     & 0                     & 0                     & 1                     & 0                     & 0                     & 0                     & 0                     & 0                     & 0                     & 0                     & 0                     & 0                     & 0                     & 0                     & 0                     & 2                     \\ \hline
\textbf{5}                         & 1                     & 0                     & 0                     & 0                     & 1                     & 0                     & 0                     & 0                     & 0                     & 0                     & 0                     & 0                     & 0                     & 0                     & 0                     & 0                     & 2                     \\ \hline
\textbf{6}                         & 1                     & 0                     & 0                     & 0                     & 0                     & 1                     & 0                     & 0                     & 0                     & 0                     & 0                     & 0                     & 0                     & 0                     & 0                     & 0                     & 2                     \\ \hline
\textbf{7}                         & 1                     & 0                     & 0                     & 0                     & 0                     & 0                     & 1                     & 0                     & 0                     & 0                     & 0                     & 0                     & 0                     & 0                     & 0                     & 0                     & 2                     \\ \hline
\textbf{8}                         & 1                     & 0                     & 0                     & 0                     & 0                     & 0                     & 0                     & 1                     & 0                     & 0                     & 0                     & 0                     & 0                     & 0                     & 0                     & 0                     & 2                     \\ \hline
\textbf{9}                         & 1                     & 0                     & 0                     & 0                     & 0                     & 0                     & 0                     & 0                     & 1                     & 0                     & 0                     & 0                     & 0                     & 0                     & 0                     & 0                     & 2                     \\ \hline
\textbf{10}                        & 1                     & 0                     & 0                     & 0                     & 0                     & 0                     & 0                     & 0                     & 0                     & 1                     & 0                     & 0                     & 0                     & 0                     & 0                     & 0                     & 2                     \\ \hline
\textbf{11}                        & 1                     & 0                     & 0                     & 0                     & 0                     & 0                     & 0                     & 0                     & 0                     & 0                     & 1                     & 0                     & 0                     & 0                     & 0                     & 0                     & 2                     \\ \hline
\textbf{12}                        & 1                     & 0                     & 0                     & 0                     & 0                     & 0                     & 0                     & 0                     & 0                     & 0                     & 0                     & 1                     & 0                     & 0                     & 0                     & 0                     & 2                     \\ \hline
\textbf{13}                        & 1                     & 0                     & 0                     & 0                     & 0                     & 0                     & 0                     & 0                     & 0                     & 0                     & 0                     & 0                     & 1                     & 0                     & 0                     & 0                     & 2                     \\ \hline
\textbf{14}                        & 1                     & 0                     & 0                     & 0                     & 0                     & 0                     & 0                     & 0                     & 0                     & 0                     & 0                     & 0                     & 0                     & 1                     & 0                     & 0                     & 2                     \\ \hline
\textbf{15}                        & 1                     & 0                     & 0                     & 0                     & 0                     & 0                     & 0                     & 0                     & 0                     & 0                     & 0                     & 0                     & 0                     & 0                     & 1                     & 0                     & 2                     \\ \hline
\textbf{16}                        & 1                     & 0                     & 0                     & 0                     & 0                     & 0                     & 0                     & 0                     & 0                     & 0                     & 0                     & 0                     & 0                     & 0                     & 0                     & 1                     & 2                     \\ \hline
\textbf{17}                        & 1                     & 1                     & 1                     & 1                     & 0                     & 0                     & 0                     & 0                     & 0                     & 0                     & 0                     & 0                     & 0                     & 0                     & 0                     & 0                     & 4                     \\ \hline
\textbf{18}                        & 1                     & 1                     & 0                     & 0                     & 1                     & 1                     & 0                     & 0                     & 0                     & 0                     & 0                     & 0                     & 0                     & 0                     & 0                     & 0                     & 4                     \\ \hline
\textbf{19}                        & 1                     & 1                     & 0                     & 0                     & 0                     & 0                     & 1                     & 1                     & 0                     & 0                     & 0                     & 0                     & 0                     & 0                     & 0                     & 0                     & 4                     \\ \hline
\textbf{20}                        & 1                     & 1                     & 0                     & 0                     & 0                     & 0                     & 0                     & 0                     & 1                     & 1                     & 0                     & 0                     & 0                     & 0                     & 0                     & 0                     & 4                     \\ \hline
\textbf{21}                        & 1                     & 1                     & 0                     & 0                     & 0                     & 0                     & 0                     & 0                     & 0                     & 0                     & 1                     & 1                     & 0                     & 0                     & 0                     & 0                     & 4                     \\ \hline
\textbf{22}                        & 1                     & 1                     & 0                     & 0                     & 0                     & 0                     & 0                     & 0                     & 0                     & 0                     & 0                     & 0                     & 1                     & 1                     & 0                     & 0                     & 4                     \\ \hline
\textbf{23}                        & 1                     & 1                     & 0                     & 0                     & 0                     & 0                     & 0                     & 0                     & 0                     & 0                     & 0                     & 0                     & 0                     & 0                     & 1                     & 1                     & 4                     \\ \hline
\textbf{24}                        & 1                     & 0                     & 1                     & 0                     & 1                     & 0                     & 1                     & 0                     & 0                     & 0                     & 0                     & 0                     & 0                     & 0                     & 0                     & 0                     & 4                     \\ \hline
\textbf{25}                        & 1                     & 0                     & 1                     & 0                     & 0                     & 1                     & 0                     & 1                     & 0                     & 0                     & 0                     & 0                     & 0                     & 0                     & 0                     & 0                     & 4                     \\ \hline
\textbf{26}                        & 1                     & 0                     & 1                     & 0                     & 0                     & 0                     & 0                     & 0                     & 1                     & 0                     & 1                     & 0                     & 0                     & 0                     & 0                     & 0                     & 4                     \\ \hline
\textbf{27}                        & 1                     & 0                     & 1                     & 0                     & 0                     & 0                     & 0                     & 0                     & 0                     & 1                     & 0                     & 1                     & 0                     & 0                     & 0                     & 0                     & 4                     \\ \hline
\textbf{28}                        & 1                     & 0                     & 1                     & 0                     & 0                     & 0                     & 0                     & 0                     & 0                     & 0                     & 0                     & 0                     & 1                     & 0                     & 1                     & 0                     & 4                     \\ \hline
\textbf{29}                        & 1                     & 0                     & 1                     & 0                     & 0                     & 0                     & 0                     & 0                     & 0                     & 0                     & 0                     & 0                     & 0                     & 1                     & 0                     & 1                     & 4                     \\ \hline
\textbf{30}                        & 1                     & 0                     & 0                     & 1                     & 1                     & 0                     & 0                     & 1                     & 0                     & 0                     & 0                     & 0                     & 0                     & 0                     & 0                     & 0                     & 4                     \\ \hline
\textbf{31}                        & 1                     & 0                     & 0                     & 1                     & 0                     & 1                     & 1                     & 0                     & 0                     & 0                     & 0                     & 0                     & 0                     & 0                     & 0                     & 0                     & 4                     \\ \hline
\textbf{32}                        & 1                     & 0                     & 0                     & 1                     & 0                     & 0                     & 0                     & 0                     & 1                     & 0                     & 0                     & 1                     & 0                     & 0                     & 0                     & 0                     & 4                     \\ \hline
\textbf{33}                        & 1                     & 0                     & 0                     & 1                     & 0                     & 0                     & 0                     & 0                     & 0                     & 1                     & 1                     & 0                     & 0                     & 0                     & 0                     & 0                     & 4                     \\ \hline
\textbf{34}                        & 1                     & 0                     & 0                     & 1                     & 0                     & 0                     & 0                     & 0                     & 0                     & 0                     & 0                     & 0                     & 1                     & 0                     & 0                     & 1                     & 4                     \\ \hline
\textbf{35}                        & 1                     & 0                     & 0                     & 1                     & 0                     & 0                     & 0                     & 0                     & 0                     & 0                     & 0                     & 0                     & 0                     & 1                     & 1                     & 0                     & 4                     \\ \hline
\textbf{36}                        & 1                     & 0                     & 0                     & 0                     & 1                     & 0                     & 0                     & 0                     & 1                     & 0                     & 0                     & 0                     & 1                     & 0                     & 0                     & 0                     & 4                     \\ \hline
\textbf{37}                        & 1                     & 0                     & 0                     & 0                     & 1                     & 0                     & 0                     & 0                     & 0                     & 1                     & 0                     & 0                     & 0                     & 1                     & 0                     & 0                     & 4                     \\ \hline
\textbf{38}                        & 1                     & 0                     & 0                     & 0                     & 1                     & 0                     & 0                     & 0                     & 0                     & 0                     & 1                     & 0                     & 0                     & 0                     & 1                     & 0                     & 4                     \\ \hline
\textbf{39}                        & 1                     & 0                     & 0                     & 0                     & 1                     & 0                     & 0                     & 0                     & 0                     & 0                     & 0                     & 1                     & 0                     & 0                     & 0                     & 1                     & 4                     \\ \hline
\textbf{40}                        & 1                     & 0                     & 0                     & 0                     & 0                     & 1                     & 0                     & 0                     & 1                     & 0                     & 0                     & 0                     & 0                     & 1                     & 0                     & 0                     & 4                     \\ \hline
\textbf{41}                        & 1                     & 0                     & 0                     & 0                     & 0                     & 1                     & 0                     & 0                     & 0                     & 1                     & 0                     & 0                     & 1                     & 0                     & 0                     & 0                     & 4                     \\ \hline
\textbf{42}                        & 1                     & 0                     & 0                     & 0                     & 0                     & 1                     & 0                     & 0                     & 0                     & 0                     & 1                     & 0                     & 0                     & 0                     & 0                     & 1                     & 4                     \\ \hline
\textbf{43}                        & 1                     & 0                     & 0                     & 0                     & 0                     & 1                     & 0                     & 0                     & 0                     & 0                     & 0                     & 1                     & 0                     & 0                     & 1                     & 0                     & 4                     \\ \hline
\textbf{44}                        & 1                     & 0                     & 0                     & 0                     & 0                     & 0                     & 1                     & 0                     & 1                     & 0                     & 0                     & 0                     & 0                     & 0                     & 1                     & 0                     & 4                     \\ \hline
\textbf{45}                        & 1                     & 0                     & 0                     & 0                     & 0                     & 0                     & 1                     & 0                     & 0                     & 1                     & 0                     & 0                     & 0                     & 0                     & 0                     & 1                     & 4                     \\ \hline
\textbf{46}                        & 1                     & 0                     & 0                     & 0                     & 0                     & 0                     & 1                     & 0                     & 0                     & 0                     & 1                     & 0                     & 1                     & 0                     & 0                     & 0                     & 4                     \\ \hline
\textbf{47}                        & 1                     & 0                     & 0                     & 0                     & 0                     & 0                     & 1                     & 0                     & 0                     & 0                     & 0                     & 1                     & 0                     & 1                     & 0                     & 0                     & 4                     \\ \hline
\textbf{48}                        & 1                     & 0                     & 0                     & 0                     & 0                     & 0                     & 0                     & 1                     & 1                     & 0                     & 0                     & 0                     & 0                     & 0                     & 0                     & 1                     & 4                     \\ \hline
\textbf{49}                        & 1                     & 0                     & 0                     & 0                     & 0                     & 0                     & 0                     & 1                     & 0                     & 1                     & 0                     & 0                     & 0                     & 0                     & 1                     & 0                     & 4                     \\ \hline
\textbf{50}                        & 1                     & 0                     & 0                     & 0                     & 0                     & 0                     & 0                     & 1                     & 0                     & 0                     & 1                     & 0                     & 0                     & 1                     & 0                     & 0                     & 4                     \\ \hline
\end{tabular}
}
\caption{Canales cuánticos PCE de 2 qubits obtenidos de aplicar 
el método númerico descrito en la sección \ref{sec:ch2_solucionNumerica}
para evaluar la completa positividad de cada operación PCE.
\janote{Seguramente habrá que modificar cómo poner esta 
tabla porque no cabe en una página.}}
\label{tab:2qubitsPCEChannel1sAnd0s}
\end{table}
\begin{table}[h!]
\centering
\resizebox{\textwidth}{!}{%
\begin{tabular}{|P{0.6cm}|P{0.65cm}|P{0.65cm}|P{0.65cm}|P{0.65cm}|P{0.65cm}|P{0.65cm}|P{0.65cm}|P{0.65cm}|P{0.65cm}|P{0.65cm}|P{0.65cm}|P{0.65cm}|P{0.65cm}|P{0.65cm}|P{0.65cm}|P{0.65cm}|P{2.9cm}|}
\hline
\textbf{No.} 										 & \tauij{0}{0}		& \tauij{0}{1}   & \tauij{0}{2} 	 & \tauij{0}{3} 	& \tauij{1}{0}	 & \tauij{1}{1}		 & \tauij{1}{2} 	& \tauij{1}{3}	 & \tauij{2}{0}		& \tauij{2}{1} 		& \tauij{2}{2} 	 & \tauij{2}{3} 	& \tauij{3}{0} 	 & \tauij{3}{1}   & \tauij{3}{2}   & \tauij{3}{3} & \bf{Componentes de Pauli \boldmath{$r_{ij}$} invariantes} \\ \hline
\textbf{51}                        & 1                     & 0                     & 0                     & 0                     & 0                     & 0                     & 0                     & 1                     & 0                     & 0                     & 0                     & 1                     & 1                     & 0                     & 0                     & 0                     & 4                     \\ \hline
\textbf{52}                        & 1                     & 1                     & 1                     & 1                     & 1                     & 1                     & 1                     & 1                     & 0                     & 0                     & 0                     & 0                     & 0                     & 0                     & 0                     & 0                     & 8                     \\ \hline
\textbf{53}                        & 1                     & 1                     & 1                     & 1                     & 0                     & 0                     & 0                     & 0                     & 1                     & 1                     & 1                     & 1                     & 0                     & 0                     & 0                     & 0                     & 8                     \\ \hline
\textbf{54}                        & 1                     & 1                     & 1                     & 1                     & 0                     & 0                     & 0                     & 0                     & 0                     & 0                     & 0                     & 0                     & 1                     & 1                     & 1                     & 1                     & 8                     \\ \hline
\textbf{55}                        & 1                     & 1                     & 0                     & 0                     & 1                     & 1                     & 0                     & 0                     & 1                     & 1                     & 0                     & 0                     & 1                     & 1                     & 0                     & 0                     & 8                     \\ \hline
\textbf{56}                        & 1                     & 1                     & 0                     & 0                     & 1                     & 1                     & 0                     & 0                     & 0                     & 0                     & 1                     & 1                     & 0                     & 0                     & 1                     & 1                     & 8                     \\ \hline
\textbf{57}                        & 1                     & 1                     & 0                     & 0                     & 0                     & 0                     & 1                     & 1                     & 1                     & 1                     & 0                     & 0                     & 0                     & 0                     & 1                     & 1                     & 8                     \\ \hline
\textbf{58}                        & 1                     & 1                     & 0                     & 0                     & 0                     & 0                     & 1                     & 1                     & 0                     & 0                     & 1                     & 1                     & 1                     & 1                     & 0                     & 0                     & 8                     \\ \hline
\textbf{59}                        & 1                     & 0                     & 1                     & 0                     & 1                     & 0                     & 1                     & 0                     & 1                     & 0                     & 1                     & 0                     & 1                     & 0                     & 1                     & 0                     & 8                     \\ \hline
\textbf{60}                        & 1                     & 0                     & 1                     & 0                     & 1                     & 0                     & 1                     & 0                     & 0                     & 1                     & 0                     & 1                     & 0                     & 1                     & 0                     & 1                     & 8                     \\ \hline
\textbf{61}                        & 1                     & 0                     & 1                     & 0                     & 0                     & 1                     & 0                     & 1                     & 1                     & 0                     & 1                     & 0                     & 0                     & 1                     & 0                     & 1                     & 8                     \\ \hline
\textbf{62}                        & 1                     & 0                     & 1                     & 0                     & 0                     & 1                     & 0                     & 1                     & 0                     & 1                     & 0                     & 1                     & 1                     & 0                     & 1                     & 0                     & 8                     \\ \hline
\textbf{63}                        & 1                     & 0                     & 0                     & 1                     & 1                     & 0                     & 0                     & 1                     & 1                     & 0                     & 0                     & 1                     & 1                     & 0                     & 0                     & 1                     & 8                     \\ \hline
\textbf{64}                        & 1                     & 0                     & 0                     & 1                     & 1                     & 0                     & 0                     & 1                     & 0                     & 1                     & 1                     & 0                     & 0                     & 1                     & 1                     & 0                     & 8                     \\ \hline
\textbf{65}                        & 1                     & 0                     & 0                     & 1                     & 0                     & 1                     & 1                     & 0                     & 1                     & 0                     & 0                     & 1                     & 0                     & 1                     & 1                     & 0                     & 8                     \\ \hline
\textbf{66}                        & 1                     & 0                     & 0                     & 1                     & 0                     & 1                     & 1                     & 0                     & 0                     & 1                     & 1                     & 0                     & 1                     & 0                     & 0                     & 1                     & 8                     \\ \hline
\textbf{67}                        & 1                     & 1                     & 1                     & 1                     & 1                     & 1                     & 1                     & 1                     & 1                     & 1                     & 1                     & 1                     & 1                     & 1                     & 1                     & 1                     & 16                    \\ \hline
\end{tabular}
}
\end{table}

En las tablas \janote{tal y tal} mostramos los canales cuánticos de 3
qubits que dejan 1, 2 y 4 componentes invariantes. \janote{bla bla bla...}
 
Analizar numéricamente, una por una, todas las operaciones PCE de 3 qubits 
es una tarea imposible. El número total de operaciones PCE para el caso 
de 3 qubits es de alrededor de $9\times10^{18}$. Supongamos por un momento
que contamos con una computadora promedio para un estudiante de física
en Guatemala, pero con memoria ilimitada que puede analizar 
la completa positividad de 100 operaciones PCE de 3 qubits por segundo. A esa 
computadora le tomaría entre $1/4$ y $1/5$ de la edad del universo 
($13.7\times10^9$ años) analizar todas las operaciones PCE de 3 qubits.
Ahora bien, si consideramos una computadora real, con memoria limitada, 
analizar todas las operaciones PCE de 3 qubits que dejan 
5 componentes de Pauli invariantes supone un problema de memoria. En las
tablas \janote{tal y tal} se muestran el uso de memoria y tiempo de cómputo.

\noindent
\esqueleto{Con nuestro método numérico de fuerza bruta es posible 
resolver el caso de 3 qubits hasta 4 componentes invariantes. Los resultados   
son.... «con 3 qubits está todavía más jalado inferir características de 
los canales PCE a partir de los 1's y 0's»}

\noindent
\esqueleto{Para justificar que es computacionalmente imposible 
resolver numéricamente el problema de las operaciones PCE de 3 qubits, 
más allá de 4 componentes de Pauli invariantes, en el tiempo de este 
trabajo de tesis mostramos gráficas del tiempo de 
cómputo (tiempo vs. cantidad de operaciones PCE) y yo esperaría 
mostrar, por lo menos, que esa curva no es lineal y que el ajuste a la curva 
calcula un chingo de tiempo que no tenemos durante la tesis.}

\section{Una representación geométrica}\label{sec:ch3_geometric_representation}
\esqueleto{Motivados en lo intricado de inferir qué características 
comparten los canales PCE a partir de las listas de 1's y 0's, se nos ocurrió 
una forma de representar geométricamente a las operaciones PCE que hace
más sencillo el análisis de resultados.}

\esqueleto{La figura asociada con una operación PCE de 1 qubit 
es una columna de 
cuadritos.. bla bla y con figuritas, haciendo referencia a lo que se resolvió 
en el capítulo anterior, etc.}

\esqueleto{Para una operación PCE de 2 qubits, los dos índices en las $\tau$
sugieren que ahora la figura asociada debería ser de dos dimensiones. Así, 
los tableritos representan a estas operaciones. Figuritas para explicar y demás.}

\esqueleto{En este punto, ya es más o menos obvio cómo es la representación 
geométrica de 3 qubits y que a partir de 4 qubits ya no podremos utilizar 
esta herramienta geométrica. Mostrar algunas figuras de PCEs de 3 qubits
y hablar de cómo hacer diferencia entre correlaciones y componentes 
locales en esas figuras según los colores 
(sólo para 3 qubits, porque las de 1 y 2 qubits 
las voy a poner en negro).}

\noindent
\esqueleto{Armados con esta potente herramienta geométrica, ahora 
es mucho más sencillo ganar intuición de las operaciones PCE e inferir 
características de los canales PCE. Entonces ahora mostraré los resultados 
de la sección anterior, pero usando las figuritas.}

\section{Discusión de resultados}\label{sec:ch3_discussion}
\janote{Con el fin de hacer más eficiente la redacción, voy a partir del 
documento que preparamos para Sergey (justo coincide los resultados
que iban ahí con lo que vamos a poner en la tesis) y voy a iterar.}

\esqueleto{Esta sección es la que posee el contenido más importante 
de este manuscrito, después de la motivación y planteamiento del 
problema. Después de esta sección, lo único que haremos será 
estudiar si las operaciones PCE son un subconjunto de otras operaciones 
que fueron estudiadas por Ruskai.}

\esqueleto{Las figuras de los canales PCE exhiben patrones que todos 
comparten, parecen respetar alguna simetría...}

\esqueleto{Todos los canales PCE obedecen la regla de $2^k$...}

\esqueleto{Existen familias de canales PCE equivalentes. En las figuritas, 
esto se ve como transposiciones y permutaciones de filas y columnas. 
Físicamente, estos son swaps de partículas y cambios de base local.
Aquí yo creería que vale la pena discutir en palabritas, como en el documento
para sergey, pero también echarle algunas expresiones matemáticas como 
la de aplicar un swap, el PCE, y otro swap, por ejemplo...}

\esqueleto{La familia más sencilla de analizar es la de los PCE de 1 qubit
que dejan dos componentes de Pauli invariantes. Todos se pueden entender 
como la misma operación, pero conectados por rotaciones.}

\esqueleto{Hay correspondencia en el número de canales PCE que 
dejan $2^k$ y $2^{2n-k}$ componentes de Pauli invariantes.}

\esqueleto{Amarrado a la correspondencia de ``arcoiris'' van las reglas 
empíricas que formulamos con Alejo. Esta es otra prueba empírica que 
respalda la hipótesis de una conexión/correspondencia entre canales PCE.}

\esqueleto{Listo, hagamos un resumen de las características puntuales 
que inferimos de los canales PCE: ta ta ta.... Ahora sólo nos hace falta 
formalizar todo esto y hacer conexión formal entre todas las características. 
Además, sería deseable buscar alternativas para poder 
explorar numéricamente el caso completo de 3 e incluso de 4 qubits.}      % Cap. 3 

\chapter{CANALES DIAGONALES DE PAULI CONSTANTES SOBRE LOS EJES}

\section{Introducción} % {{{
\esqueleto{Para de último. Cuando estén escritas las secciones.}

% }}}
\section{Canales diagonales de Pauli constantes sobre los ejes} % {{{
%\esqueleto{Introducir la definición de una MUB (es necesario para la definición de 
%mapa de Ruskai) y dar un ejemplos de dos bases que sean MUB, sólo 
%para aterrizar la idea.}

Dos bases ortonormales $\{\ket{\psi_m^J}\}_{m=0}^{d-1}$
y $\{\ket{\psi_n^K}\}_{m=0}^{d-1}$ de un espacio de Hilbert
de dimensión $d$ se dice que son \textit{mutuamente 
imparciales} si se satisface la 
condición~\cite{bengtsson_zyczkowski_2017,nathanson2007pauli}
\begin{align}\label{eq:mub_definition}
	\abs{\braket{\psi_m^J}{\psi_n^K}}^2=
	\left\{ \begin{array}{lcc}
             \frac{1}{d} & si & J\ne K \\
             \delta_{mn} & si & J=K,
             \end{array}
   \right.
\end{align}
para todo $m$ y $n$. Cuando $d$  es una potencia de un 
número primo, el espacio de Hilbert cuenta hasta con $d+1$ bases mutuamente
imparciales \cite{durt2010mutually}. 
Dado que los sistemas cuánticos de nuestro interés en este trabajo son qubits 
(sistemas de $2^n$ niveles) no vamos a considerar los casos de sistemas que 
no posean $d+1$ bases mutuamente imparciales. 

La matriz de densidad de un sistema de qubits puede escribirse en términos 
de operadores que generan a un conjunto de bases mutuamente imparciales\cpnote{No 
entiendo que tiene que ver las MUBS aca. Con que se tenga una base ya se 
puede escribir una matriz de densidad. No entiendo esta anotación}.
Es posible definir, a partir de un conjunto de bases mutuamente imparciales,
$d+1$ operadores unitarios $W_J$ como~\cite{nathanson2007pauli}
\begin{align}\label{eq:W_J}
	W_J=\sum_{k=1}^d e^{2\pi k i/d} \dyad{\psi_k^J}{\psi_k^J}, 
\end{align}
para cada $J$-ésima base $\ket{\psi_k^J}$. Se dice que los operadores 
$W^J$ son generadores del conjunto de bases $\ket{\psi_k^J}$ mutuamente 
imparciales. El conjunto de las $d^2-1$ potencias $\{W_J^m\}_{m=1,\ldots,d-1,
J=1,\ldots,d+1}$ más la identidad forman una base ortogonal
de unitarias del espacio $\mathcal{M}_d$ de matrices 
de dimensión $d\times d$ con norma $\sqrt{d}$
(en el sentido de Hilbert-Schmidt). Por lo tanto, la matriz de densidad
de un sistema de $d$ niveles puede escribirse como
\begin{align}\label{eq:rho_mub}
\rho=\frac{1}{d}\qty(\1+\sum_{J=1}^{d+1}\sum_{j=1}^{d-1} v_{Jj}W_J^j),
\end{align}
con $v_{Jj}$ las proyecciones de $\rho$ sobre cada uno de los 
operadores $W_J^j$.

Dado un conjunto de bases ortonormales mutuamente imparciales del 
espacio de estados de un sistema de $d$ niveles se
puede definir a un canal diagonal de Pauli constante sobre los ejes $\E$ 
según la acción sobre una matriz de densidad como 
en~\eqref{eq:rho_mub}~\cite{nathanson2007pauli},
\begin{align}\label{eq:ruskai_definition}
	\E :  \frac{1}{d}\qty(\1+\sum_{J=1}^{d+1}\sum_{j=1}^{d-1} v_{Jj}W_J^j)
	\longmapsto 
	\frac{1}{d}\qty(\1+\sum_{J=1}^{d+1}\lambda_J\sum_{j=1}^{d-1} v_{Jj}W_J^j).
\end{align}
Es decir que las componentes $v_{Jj}$ de $\rho$ se transforman como 
$v_{Jj}\mapsto\lambda_Jv_{Jj}$, donde $\lambda_0=1$ más $\lambda_J:=s+t_J$ son
los eigenvalores de $\E$. Las 
condiciones para que un \ruskai{}{}{}Map{} sea una operación 
completamente positiva que preserva la traza (CPTP), \textit{i.e.} un canal cuántico,
son
\begin{align}\label{eq:cptp_conditions_ruskai}
	s+\sum_{J}t_J=1, && t_J\geq0 && \text{y} && s\geq\frac{-1}{d-1}.
\end{align}
La primera condición asegura que la operación preserva la traza de la matriz 
de densidad y las últimas dos que la operación sea completamente positiva
\cite{nathanson2007pauli}.
\cpnote{Croe qe es necesario contextualizar un poco mejor estas operaciones. Aca
aparecen como de manera magica. Quizá decir que se han estudiaro y el porque. 
Te toca hacer una buena introduccion, porque por ahora parece raro}

Nótese la similitud entre la definición en \eqref{eq:ruskai_definition} 
de un canal diagonal de Pauli constante
sobre los ejes y la definición de una operación PCE, \eqref{eq:PCE_definition}.
Los dos tipos de operaciones transforman de 
un modo similar a las componentes de la matriz de densidad en una 
base dada del espacio $\mathcal{M}_d$.
No obstante, los \ruskai{} son más generales, en el sentido que son
operaciones que actúan sobre sistemas de $d$ niveles (las 
operaciones PCE actúan sobre sistemas de $2^n$ niveles) y porque 
las $\lambda_J$, para los \ruskai{}, puede tomar cualquier valor real que 
satisfaga las condiciones en \eqref{eq:cptp_conditions_ruskai}, a diferencia
de las operaciones PCE en las que los $\taus$ restringido a los valores 0 o 1.
De hecho, es esta definición más general de los \ruskai{} la que motiva a
investigar si los canales cuánticos PCE están contenidos dentro de ellos.

%Los \ruskai{} pueden escribirse en la forma
%\begin{align}
%	\E=s\1 + \sum_{L}t_L\sum 
%\end{align}
%
%Para $d=2^k$ se ha mostrado que existen $d+1$ bases mutuamente imparciales 
%del espacio de Hilbert. 
%
%Para cualquier base 
%\begin{align}
%	W_J=\sum_{k=1}^d e^{2\pi k i/d} \dyad{\psi_k^K}{\psi_k^K}, 
%\end{align}
%con $i$ la unidad imaginaria. Se sigue que
%\begin{align}
%	\dyad{\psi_k^K}{\psi_k^K} = \frac{1}{d}\sum_{j=0}^{d-1}e^{-2\pi n j i/d}W_J^j 
%	= \frac{1}{d}\qty[\1 + \sum_{j=1}^{d-1}e^{-2\pi n j i/d}W_J^j]
%\end{align}
%
%Bla bla bla... toda la casaca para llegar a que la matriz de densidad se puede 
%escribir como 
%\begin{align}
%\rho=\frac{1}{d}\qty[\1+\sum_{J=1}^{d+1}\sum_{j=1}^{d-1} v_{Jj}W_J^j],
%\end{align}


%donde $W_J$ son operadores unitarios que generan a $d+1$ bases  
%$\{\ket{\psi_m^J}\}_{m=0}^{d-1}$
%mutuamente imparciales y que se definen como
%\begin{align}
%	W_J=\sum_{k=1}^d e^{2\pi k i/d} \dyad{\psi_k^K}{\psi_k^K}, 
%\end{align}
%La identidad $\1$ junto con los operadores $\{ W_J^j\}$ forman 
%una base ortogonal de unitarias
%del espacio $\mathcal{M}_d$ de las matrices de $d\times d$ \janote{revisar 
%notación con el cap 1} que satisfacen la relación de ortogonalidad, 
%en el sentido de Hilbert-Schmidt, $\Tr(W_J^{d-m}W_K^n)=d\delta_{JK}\delta_{mn}$.
%
%%\esqueleto{Introducir qué es un mapa de Ruskai (definición que pusimos 
%%en el documento para Sergei y para Francois).}
%%
%%La acción de un canal de Pauli constante sobre los ejes $\Phi$ se define 
%%como~\cite{nathanson2007pauli}
%%\begin{align}
%%	\Phi :  \frac{1}{d}\qty[\1+\sum_{J=1}^{d+1}\sum_{j=1}^{d-1} v_{Jj}W_J^j
%%	\longmapsto 
%%	\1+\sum_{J=1}^{d+1}\lambda_J\sum_{j=1}^{d-1} v_{Jj}W_J^j],
%%\end{align}
%%de tal forma que $v_{Jj}\mapsto \lambda_Jv_{Jj}$ y $\lambda_J$ los eigenvalores 
%%de la operación $\Phi$.
%
%\esqueleto{Dar algunos ejemplos de mapas de Ruskai}
%
%Los canales PCE de 1 qubit son Ruskai. Las matrices de Pauli son generadores 
%de 3 bases ortonormales mutuamente imparciales.
%
%También hay algunos PCE de 2 qubits que son claramente Ruskai \janote{será??}

% }}}
\section{Relación con canales cuánticos PCE} % {{{
%\esqueleto{Queremos estudiar si los canales PCE son un subconjunto 
%de los mapas de Ruskai.}

En el caso de 1 qubit, los canales PCE sí son un subconjunto de los \ruskai{}. Para $d=2$,
los eigenvectores de las matrices de Pauli $\sigma_1$, $\sigma_2$ y $\sigma_3$ 
satisfacen \eqref{eq:mub_definition} y, por ende, son un conjunto 
de bases mutuamente imparciales. Por lo tanto, siguiendo \eqref{eq:rho_mub}
e identificando $W_J=\sigma_J$, la matriz de densidad de 1 qubit se escribe como
\begin{align} \label{eq:rho_1_qubit_ruskai}
	\rho=\frac{1}{2}\qty(\1+\sum_{J=1}^3v_J\sigma_J),
\end{align}
donde $v_J$ son lo que hemos llamado a lo largo de este trabajo 
las componentes de Pauli.
De acuerdo con \eqref{eq:ruskai_definition}, 
un canal diagonal de Pauli constante sobre los ejes de 1 qubit transforma 
a la matriz de densidad $\rho$ en \eqref{eq:rho_1_qubit_ruskai} como
\begin{align}
	\E :  \frac{1}{2}\qty(\1+\sum_{J=1}^3v_J\sigma_J)
	\longmapsto 
	\frac{1}{2}\qty(\1+\sum_{J=1}^3\lambda_J v_J\sigma_J).
\end{align}
Si $\lambda_J$ se restringe a los valores de 0 o 1, entonces se recupera 
la definición de un operación PCE de 1 qubit. Por lo tanto, los canales PCE 
de 1 qubit son un caso particular de los \ruskai{} cuando $d=2$.

Sin embargo, en general, no existe relación de contención entre los canales cuánticos PCE 
y los \ruskai{}. Para demostrar esta proposición vamos a utilizar como 
argumento la incompatibilidad entre los rangos de los canales PCE
y de los \ruskai{}. Recordemos que el rango de una matriz es igual 
al número de eigenvalores distintos de cero 
\cite{axler1997linear,lang2012introduction}.
Además, de los teoremas elementales de álgebra lineal se puede mostrar que 
rango de una matriz es invariante ante cambios de base \cite{axler1997linear}.
\cpnote{Es mucho mas simple. Los eigenvalores no cambian con 
un cambio de base. Simplifica el argumento.}

Por un lado, de la definición en \eqref{eq:ruskai_definition} es claro que un \ruskaiMap{}
tiene hasta $d+1$ eigenvalores $\lambda_J$ distintos de cero 
con degeneración $d-1$. Por lo tanto,
el rango de un \ruskaiMap{} $1+l(d-1)$. Por otro lado, 
de la regla $2^k$ discutida en la sección \ref{sec:ch3_discussion},
los canales cuánticos PCE son matrices de rango $2^k$. Para que los canales PCE
y los \ruskai{} tengan el mismo rango se debe cumplir 
\begin{align}\label{eq:ruskai_fucked_up}
2^k=1+l(2^n-1),\quad \forall \ k=0,1,\ldots,2n; l=0,1,2,\ldots,2^n+1.
\end{align}
Sin embargo, es fácil encontrar un contraejemplo para mostrar que esta
ecuación no siempre se satisface. Veamos por ejemplo, para $n=2$ y $k=3$
(2 qubits y 8 componentes de Pauli invariantes),
\begin{align}
l=\frac{2^3-1}{2^2-1}=\frac{7}{3},
\end{align}
lo que contradice los posibles valores que puede tomar $l$ 
en \eqref{eq:ruskai_fucked_up}. De hecho, la ecuación sólo se cumple 
para $k=0,n,2n$. En conclusión, la intersección de los \ruskai{}
y los canales PCE contiene al canal depolatizante, la identidad y canales 
que dejan invariantes $2^n$ componentes de Pauli. En particular, para 2 qubits 
los canales que son intersección entre los \ruskai{} y los canales PCE son los 
elementos de las clases de equivalencia C${}_4^2$ y C${}_4^4$
(ver \Fref{fig:2qubits_PCEChannels_figs}). 

Ya que probamos que los 
rangos de los \ruskai{} y de los canales cuánticos PCE son incompatibles, 
se sigue que los canales PCE no son un subconjunto dentro de los 
\ruskai{}, ni viceversa.


% }}}


      % Cap. 4

{\backmatter     %	Capítulos no van numerados --------------------------------------------------  Apartados finales

%%% INCLUYA SUS CONCLUSIONES Y RECOMENDACIONES


\chapter{CONCLUSIONES}
\cpnote{Primero, la introduccion de un nuevo tipo de canales que generalizan 
las decoherencias basicas de un qubit. A esto tienen que ir dos frases, y también 
recoerdando al lector que son los PCEs}
La contribución de este trabajo de tesis fue aportar pruebas numéricas
de que los canales cuánticos PCE de $n$ qubits tienen una caracterización propia 
y abrió la posibilidad a preguntas más fundamentales acerca de este 
tipo de canales cuánticos.
Nuestros resultados numéricos de la búsqueda de canales PCE de 2 y 3 qubits 
muestran principalmente dos cosas: (1) que los canales cuánticos PCE pueden 
ordenarse en clases de equivalencia, lo que reduce el número de canales cuánticos
que no están conectados vía operaciones unitarias, y (2) que las operaciones
PCE que satisfacen la condición de completa positividad obedecen reglas muy
específicas. 
\cpnote{Plantea las cosas un poco mas generales. Es decir, puedes hacer un planteamiento
general que contextualice la importancia de tus resultados en general, y que luego 
se aplican a los PCEs. Itntenta plantear un poco las cosas como en la ultima frase de 
este parrafo y luego dices en particular lo qeu hacemos. Como por cada frase de aca
pon una frase que la anteceda tipo la ultima. Quizá vale la pena platicar de esto, 
pareciera confuso}
Por un lado, 
la \textit{regla $2^k$} establece que el número de 
componentes de Pauli invariantes por un canal PCE debe ser una potencia de 2. 
Por otro lado, la \textit{regla espejo} establece que debe existir el mismo número de 
canales cuánticos PCE que dejan invariantes $2^k$ y $2^{2n-k}$ componentes 
de Pauli. Además, probamos que los canales cuánticos PCE no son un subconjunto
de otro tipo de canales cuánticos que se han estudiado antes. Todo esto, la clasificación,
propiedades y la no contención dentro de otro conjunto de canales cuánticos, 
abre la posibilidad y justifica preguntarse si existe una estructura matemática bien 
definida para los canales cuánticos PCE.

\chapter{RECOMENDACIONES}
\begin{enumerate}
	\item Recomendación 1.
	\item Recomendación 2.
	\item Recomendación 3.
\end{enumerate}
     % Conclusiones y recomendaciones

\bibliographystyle{abbrv}
\bibliography{references}   % Bibliografía

}

% Descomentar en el caso de necesitar incluir apéndices
%\appendix			% Apéndices

%\chapter{METODOLOGÍA}
% \esqueleto{
% \begin{itemize}
% \item Hacer un recordatorio del trabajo de prácticas porque es la base 
% teórica de este trabajo
% \item Método numérico para 2 y 3 qubits
% \item Análisis los resultados del numérico
% \item Comparación con los mapeos de Ruskai
% \item Trabajo futuro
% \end{itemize}
% }
\cpnote{Creo que esta seccion está mal. Como esta es lo mismo que la siguiente. 
Yo creo que acá mas bien se deben discutir los métodos que usaras. 
Porfa aclarame eso. }

El primer capítulo contendrá las bases teóricas necesarias para 
el estudio de las operaciones PCE. Se expondrán de manera puntual el 
formalismo de la matriz de densidad y la teoría de los canales cuánticos.
Se utilizará como referencias bibliografías libros especializados en 
el tema: Sakurai \cite{sakurai_napolitano_2017}, 
el texto introductorio estándar para información y computación cuántica de 
Nielsen y Chuang \cite{nielsen_chuang_2011}, Bengtsson 
\cite{bengtsson_zyczkowski_2017} y Preskill \cite{preskill1998lecture}.

En el segundo capítulo se definirán las operaciones PCE, el caso de 
1 qubit y se establecerá el problema para sistemas de $n$ qubits.
Para este capítulo se utilizarán los resultados y el estudio realizado 
durante el trabajo de práctica final, se hará un resumen con los 
aspectos más relevantes ya que son la base de este trabajo. 
Finalmente, en este capítulo se discutirá el uso del método numérico, 
que fue diseñado en la práctica final, para evaluar los casos de 2 y 3 qubits
que son el objetivo de este trabajo.

En el tercer capítulo se presentarán los resultados de 2 y 3 qubits. Se 
analizarán y discutirán los resultados, al mismo tiempo que se desarrollará 
una herramienta geométrica que permita entender de manera sencilla 
los canales cuánticos PCE de 2 y 3 qubits.

En el cuarto capítulo se discutirá la relación de las operaciones PCE con 
los canales diagonales de Pauli constantes sobre los ejes 
\cite{nathanson2007pauli}. Lo que buscamos es saber si los canales cuánticos
PCE están contenidos dentro del conjunto de los canales cuánticos que
estudian Nathanson y Ruskai. 
\cpnote{Yo escribiría porque nos interesa discutir eso, ponlo igual en la siguiente seccion}




\par}               % termina interlineado 1 1/2

\end{document}
