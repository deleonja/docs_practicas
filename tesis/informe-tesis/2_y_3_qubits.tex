\chapter{RESULTADOS DE 2 Y 3 QUBITS}
\janote{\textbf{Idea principal del capítulo:} los resultados gritan que los 
canales PCE sí se pueden caracterizar, pero nos hace falta LA idea (la de 
Francois, jajaja) para poder formalizar la caracterización general. Para mientras,
tenemos un listado de características que tienen sustento en los resultados
numéricos.}
\section{Introducción}

\section{Resultados}
\esqueleto{Con el método numérico descrito en la sección 
\ref{sec:ch2_solucionNumerica} es posible analizar el caso de 2 qubits
completo. Por otro lado, el caso de 3 qubits es imposible de resolver 
completo a fuerza bruta.}

\esqueleto{Los resultados de 2 qubits son... (una tabla con 
las listas de 1's y 0's de $\tau_{ij}$ por número $k$ de componentes 
de Pauli invariantes). «\textit{La idea con esta tabla es motivar las 
figuritas de la siguiente sección porque las listas de 1's y 0's no dicen 
ni madres.}»}

\esqueleto{Con nuestro método numérico de fuerza bruta es posible 
resolver el caso de 3 qubits hasta 4 componentes invariantes. Los resultados
son.... «con 3 qubits está todavía más jalado inferir características de 
los canales PCE a partir de los 1's y 0's»}

\esqueleto{Para justificar que es computacionalmente imposible 
resolver numéricamente el problema de las operaciones PCE de 3 qubits, 
más allá de 4 componentes de Pauli invariantes, en el tiempo de este 
trabajo de tesis mostramos gráficas del tiempo de 
cómputo (tiempo vs. cantidad de operaciones PCE) y yo esperaría 
mostrar, por lo menos, que esa curva no es lineal y que el ajuste a la curva 
calcula un chingo de tiempo que no tenemos durante la tesis.}

\section{Una representación geométrica}
\esqueleto{Motivados en lo intricado de inferir qué características 
comparten los canales PCE a partir de las listas de 1's y 0's, se nos ocurrió 
una forma de representar geométricamente a las operaciones PCE que hace
más sencillo el análisis de resultados.}

\esqueleto{La figura asociada con una operación PCE de 1 qubit 
es una columna de 
cuadritos.. bla bla y con figuritas, haciendo referencia a lo que se resolvió 
en el capítulo anterior, etc.}

\esqueleto{Para una operación PCE de 2 qubits, los dos índices en las $\tau$
sugieren que ahora la figura asociada debería ser de dos dimensiones. Así, 
los tableritos representan a estas operaciones. Figuritas para explicar y demás.}

\esqueleto{En este punto, ya es más o menos obvio cómo es la representación 
geométrica de 3 qubits y que a partir de 4 qubits ya no podremos utilizar 
esta herramienta geométrica. Mostrar algunas figuras de PCEs de 3 qubits
y hablar de cómo hacer diferencia entre correlaciones y componentes 
locales en esas figuras según los colores 
(sólo para 3 qubits, porque las de 1 y 2 qubits 
las voy a poner en negro).}

\esqueleto{Armados con esta potente herramienta geométrica, ahora 
es mucho más sencillo ganar intuición de las operaciones PCE e inferir 
características de los canales PCE. Entonces ahora mostraré los resultados 
de la sección anterior, pero usando las figuritas.}

\section{Discusión de resultados}
\janote{Con el fin de hacer más eficiente la redacción, voy a partir del 
documento que preparamos para Sergey (justo coincide los resultados
que iban ahí con lo que vamos a poner en la tesis) y voy a iterar.}

\esqueleto{Esta sección es la que posee el contenido más importante 
de este manuscrito, después de la motivación y planteamiento del 
problema. Después de esta sección, lo único que haremos será 
estudiar si las operaciones PCE son un subconjunto de otras operaciones 
que fueron estudiadas por Ruskai.}

\esqueleto{Las figuras de los canales PCE exhiben patrones que todos 
comparten, parecen respetar alguna simetría...}

\esqueleto{Todos los canales PCE obedecen la regla de $2^k$...}

\esqueleto{Existen familias de canales PCE equivalentes. En las figuritas, 
esto se ve como transposiciones y permutaciones de filas y columnas. 
Físicamente, estos son swaps de partículas y cambios de base local.
Aquí yo creería que vale la pena discutir en palabritas, como en el documento
para sergey, pero también echarle algunas expresiones matemáticas como 
la de aplicar un swap, el PCE, y otro swap, por ejemplo...}

\esqueleto{La familia más sencilla de analizar es la de los PCE de 1 qubit
que dejan dos componentes de Pauli invariantes. Todos se pueden entender 
como la misma operación, pero conectados por rotaciones.}

\esqueleto{Hay correspondencia en el número de canales PCE que 
dejan $2^k$ y $2^{2n-k}$ componentes de Pauli invariantes.}

\esqueleto{Amarrado a la correspondencia de ``arcoiris'' van las reglas 
empíricas que formulamos con Alejo. Esta es otra prueba empírica que 
respalda la hipótesis de una conexión/correspondencia entre canales PCE.}

\esqueleto{Listo, hagamos un resumen de las características puntuales 
que inferimos de los canales PCE: ta ta ta.... Ahora sólo nos hace falta 
formalizar todo esto y hacer conexión formal entre todas las características. 
Además, sería deseable buscar alternativas para poder 
explorar numéricamente el caso completo de 3 e incluso de 4 qubits.}