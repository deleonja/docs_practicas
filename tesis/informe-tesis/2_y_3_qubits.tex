\chapter{RESULTADOS DE 2 Y 3 QUBITS}
\janote{\textbf{Idea principal del capítulo:} los resultados gritan que los 
canales PCE sí se pueden caracterizar, pero nos hace falta LA idea (la de 
Francois, jajaja) para poder formalizar la caracterización general. Para mientras,
tenemos un listado de características que tienen sustento en los resultados
numéricos.}
\section{Introducción}

\section{Resultados}
\noindent
\esqueleto{Con el método numérico descrito en la sección 
\ref{sec:ch2_solucionNumerica} es posible analizar el caso de 2 qubits
completo. Por otro lado, el caso de 3 qubits es imposible de resolver 
completo a fuerza bruta.}

Para encontrar los canales cuánticos PCE de 2 y 3 qubits analizamos
numéricamente la completa positividad de todas las operaciones PCE de 2
qubits y parcialmente las de 3 qubits. 
Por un lado, de las $32,768$ operaciones PCE de 2 qubits encontramos 
$67$ canales cuánticos. La proporción de canales cuánticos PCE
a operaciones PCE $67:31,768$ da una idea de lo restrictivo 
que resulta que una operación PCE sea una operación físicamente 
realizable. 
Por otro lado, para el caso de 3 qubits 
estudiamos las operaciones PCE que dejan invariantes 1, 2, 3 y 4 
componentes de Pauli. No estudiamos más alla de 4 componentes de 
Pauli¨ 
invariantes porque el número de operaciones PCE es tan grande que 
el tiempo de cómputo y uso de memoria por parte de las herramientas
computacionales suponen un obstáculo. 

\noindent
\esqueleto{Los resultados de 2 qubits son... (una tabla con 
las listas de 1's y 0's de $\tau_{ij}$ por número $k$ de componentes 
de Pauli invariantes). «\textit{La idea con esta tabla es motivar las 
figuritas de la siguiente sección porque las listas de 1's y 0's no dicen 
ni madres.}»}

En la \Tref{tab:2qubitsPCEChannel1sAnd0s} mostramos los canales cuánticos 
PCE de 2 qubits obtenidos con las herramientas que describimos en las 
últimas dos secciones del capítulo anterior. La información que se muestra
en la tabla es (1) la configuración de 1's y 0's de los elementos
$\tau_{ij}$ que caracterizan cómo actúa el canal PCE sobre las 
componentes de Pauli de la matriz de densidad de un sistema de 
2 qubits, y (2) el número de componentes de Pauli que el canal PCE
deja invariante. 

Muy poco se puede inferir sobre las características de los canales 
PCE de 2 qubits a partir 
de la \Tref{tab:2qubitsPCEChannel1sAnd0s}. Las listas de 1's y 0's, 
como se presentan en la tabla, no permiten identificar casi ninguna 
propiedad de los canales PCE, a excepción de la cantidad de componentes 
de Pauli (cantidad de 1's) que el canal deja invariante. Además, a 
diferencia de las operaciones PCE de 1 qubit, la acción de las operaciones PCE
de 2 qubits no se pueden interpretar con ayuda de alguna herramienta 
geométrica como la esfera de Bloch. Por esa razón, en la sección 
\ref{sec:ch3_geometric_representation} discutiremos una 
herramienta geométrica con la cual estudiar las características de 
las listas de 1's y 0's de los canales PCE de 1, 2 y 3 qubits.
\begin{table}[]
\centering
\resizebox{\textwidth}{!}{%
\begin{tabular}{|P{0.6cm}|P{0.65cm}|P{0.65cm}|P{0.65cm}|P{0.65cm}|P{0.65cm}|P{0.65cm}|P{0.65cm}|P{0.65cm}|P{0.65cm}|P{0.65cm}|P{0.65cm}|P{0.65cm}|P{0.65cm}|P{0.65cm}|P{0.65cm}|P{0.65cm}|P{2.9cm}|}
\hline
\textbf{No.} 										 & \tauij{0}{0}		& \tauij{0}{1}   & \tauij{0}{2} 	 & \tauij{0}{3} 	& \tauij{1}{0}	 & \tauij{1}{1}		 & \tauij{1}{2} 	& \tauij{1}{3}	 & \tauij{2}{0}		& \tauij{2}{1} 		& \tauij{2}{2} 	 & \tauij{2}{3} 	& \tauij{3}{0} 	 & \tauij{3}{1}   & \tauij{3}{2}   & \tauij{3}{3} & \bf{Componentes de Pauli \boldmath{$r_{ij}$} invariantes} \\ \hline
\textbf{1}                         & 1                     & 0                     & 0                     & 0                     & 0                     & 0                     & 0                     & 0                     & 0                     & 0                     & 0                     & 0                     & 0                     & 0                     & 0                     & 0                     & 1                     \\ \hline
\textbf{2}                         & 1                     & 1                     & 0                     & 0                     & 0                     & 0                     & 0                     & 0                     & 0                     & 0                     & 0                     & 0                     & 0                     & 0                     & 0                     & 0                     & 2                     \\ \hline
\textbf{3}                         & 1                     & 0                     & 1                     & 0                     & 0                     & 0                     & 0                     & 0                     & 0                     & 0                     & 0                     & 0                     & 0                     & 0                     & 0                     & 0                     & 2                     \\ \hline
\textbf{4}                         & 1                     & 0                     & 0                     & 1                     & 0                     & 0                     & 0                     & 0                     & 0                     & 0                     & 0                     & 0                     & 0                     & 0                     & 0                     & 0                     & 2                     \\ \hline
\textbf{5}                         & 1                     & 0                     & 0                     & 0                     & 1                     & 0                     & 0                     & 0                     & 0                     & 0                     & 0                     & 0                     & 0                     & 0                     & 0                     & 0                     & 2                     \\ \hline
\textbf{6}                         & 1                     & 0                     & 0                     & 0                     & 0                     & 1                     & 0                     & 0                     & 0                     & 0                     & 0                     & 0                     & 0                     & 0                     & 0                     & 0                     & 2                     \\ \hline
\textbf{7}                         & 1                     & 0                     & 0                     & 0                     & 0                     & 0                     & 1                     & 0                     & 0                     & 0                     & 0                     & 0                     & 0                     & 0                     & 0                     & 0                     & 2                     \\ \hline
\textbf{8}                         & 1                     & 0                     & 0                     & 0                     & 0                     & 0                     & 0                     & 1                     & 0                     & 0                     & 0                     & 0                     & 0                     & 0                     & 0                     & 0                     & 2                     \\ \hline
\textbf{9}                         & 1                     & 0                     & 0                     & 0                     & 0                     & 0                     & 0                     & 0                     & 1                     & 0                     & 0                     & 0                     & 0                     & 0                     & 0                     & 0                     & 2                     \\ \hline
\textbf{10}                        & 1                     & 0                     & 0                     & 0                     & 0                     & 0                     & 0                     & 0                     & 0                     & 1                     & 0                     & 0                     & 0                     & 0                     & 0                     & 0                     & 2                     \\ \hline
\textbf{11}                        & 1                     & 0                     & 0                     & 0                     & 0                     & 0                     & 0                     & 0                     & 0                     & 0                     & 1                     & 0                     & 0                     & 0                     & 0                     & 0                     & 2                     \\ \hline
\textbf{12}                        & 1                     & 0                     & 0                     & 0                     & 0                     & 0                     & 0                     & 0                     & 0                     & 0                     & 0                     & 1                     & 0                     & 0                     & 0                     & 0                     & 2                     \\ \hline
\textbf{13}                        & 1                     & 0                     & 0                     & 0                     & 0                     & 0                     & 0                     & 0                     & 0                     & 0                     & 0                     & 0                     & 1                     & 0                     & 0                     & 0                     & 2                     \\ \hline
\textbf{14}                        & 1                     & 0                     & 0                     & 0                     & 0                     & 0                     & 0                     & 0                     & 0                     & 0                     & 0                     & 0                     & 0                     & 1                     & 0                     & 0                     & 2                     \\ \hline
\textbf{15}                        & 1                     & 0                     & 0                     & 0                     & 0                     & 0                     & 0                     & 0                     & 0                     & 0                     & 0                     & 0                     & 0                     & 0                     & 1                     & 0                     & 2                     \\ \hline
\textbf{16}                        & 1                     & 0                     & 0                     & 0                     & 0                     & 0                     & 0                     & 0                     & 0                     & 0                     & 0                     & 0                     & 0                     & 0                     & 0                     & 1                     & 2                     \\ \hline
\textbf{17}                        & 1                     & 1                     & 1                     & 1                     & 0                     & 0                     & 0                     & 0                     & 0                     & 0                     & 0                     & 0                     & 0                     & 0                     & 0                     & 0                     & 4                     \\ \hline
\textbf{18}                        & 1                     & 1                     & 0                     & 0                     & 1                     & 1                     & 0                     & 0                     & 0                     & 0                     & 0                     & 0                     & 0                     & 0                     & 0                     & 0                     & 4                     \\ \hline
\textbf{19}                        & 1                     & 1                     & 0                     & 0                     & 0                     & 0                     & 1                     & 1                     & 0                     & 0                     & 0                     & 0                     & 0                     & 0                     & 0                     & 0                     & 4                     \\ \hline
\textbf{20}                        & 1                     & 1                     & 0                     & 0                     & 0                     & 0                     & 0                     & 0                     & 1                     & 1                     & 0                     & 0                     & 0                     & 0                     & 0                     & 0                     & 4                     \\ \hline
\textbf{21}                        & 1                     & 1                     & 0                     & 0                     & 0                     & 0                     & 0                     & 0                     & 0                     & 0                     & 1                     & 1                     & 0                     & 0                     & 0                     & 0                     & 4                     \\ \hline
\textbf{22}                        & 1                     & 1                     & 0                     & 0                     & 0                     & 0                     & 0                     & 0                     & 0                     & 0                     & 0                     & 0                     & 1                     & 1                     & 0                     & 0                     & 4                     \\ \hline
\textbf{23}                        & 1                     & 1                     & 0                     & 0                     & 0                     & 0                     & 0                     & 0                     & 0                     & 0                     & 0                     & 0                     & 0                     & 0                     & 1                     & 1                     & 4                     \\ \hline
\textbf{24}                        & 1                     & 0                     & 1                     & 0                     & 1                     & 0                     & 1                     & 0                     & 0                     & 0                     & 0                     & 0                     & 0                     & 0                     & 0                     & 0                     & 4                     \\ \hline
\textbf{25}                        & 1                     & 0                     & 1                     & 0                     & 0                     & 1                     & 0                     & 1                     & 0                     & 0                     & 0                     & 0                     & 0                     & 0                     & 0                     & 0                     & 4                     \\ \hline
\textbf{26}                        & 1                     & 0                     & 1                     & 0                     & 0                     & 0                     & 0                     & 0                     & 1                     & 0                     & 1                     & 0                     & 0                     & 0                     & 0                     & 0                     & 4                     \\ \hline
\textbf{27}                        & 1                     & 0                     & 1                     & 0                     & 0                     & 0                     & 0                     & 0                     & 0                     & 1                     & 0                     & 1                     & 0                     & 0                     & 0                     & 0                     & 4                     \\ \hline
\textbf{28}                        & 1                     & 0                     & 1                     & 0                     & 0                     & 0                     & 0                     & 0                     & 0                     & 0                     & 0                     & 0                     & 1                     & 0                     & 1                     & 0                     & 4                     \\ \hline
\textbf{29}                        & 1                     & 0                     & 1                     & 0                     & 0                     & 0                     & 0                     & 0                     & 0                     & 0                     & 0                     & 0                     & 0                     & 1                     & 0                     & 1                     & 4                     \\ \hline
\textbf{30}                        & 1                     & 0                     & 0                     & 1                     & 1                     & 0                     & 0                     & 1                     & 0                     & 0                     & 0                     & 0                     & 0                     & 0                     & 0                     & 0                     & 4                     \\ \hline
\textbf{31}                        & 1                     & 0                     & 0                     & 1                     & 0                     & 1                     & 1                     & 0                     & 0                     & 0                     & 0                     & 0                     & 0                     & 0                     & 0                     & 0                     & 4                     \\ \hline
\textbf{32}                        & 1                     & 0                     & 0                     & 1                     & 0                     & 0                     & 0                     & 0                     & 1                     & 0                     & 0                     & 1                     & 0                     & 0                     & 0                     & 0                     & 4                     \\ \hline
\textbf{33}                        & 1                     & 0                     & 0                     & 1                     & 0                     & 0                     & 0                     & 0                     & 0                     & 1                     & 1                     & 0                     & 0                     & 0                     & 0                     & 0                     & 4                     \\ \hline
\textbf{34}                        & 1                     & 0                     & 0                     & 1                     & 0                     & 0                     & 0                     & 0                     & 0                     & 0                     & 0                     & 0                     & 1                     & 0                     & 0                     & 1                     & 4                     \\ \hline
\textbf{35}                        & 1                     & 0                     & 0                     & 1                     & 0                     & 0                     & 0                     & 0                     & 0                     & 0                     & 0                     & 0                     & 0                     & 1                     & 1                     & 0                     & 4                     \\ \hline
\textbf{36}                        & 1                     & 0                     & 0                     & 0                     & 1                     & 0                     & 0                     & 0                     & 1                     & 0                     & 0                     & 0                     & 1                     & 0                     & 0                     & 0                     & 4                     \\ \hline
\textbf{37}                        & 1                     & 0                     & 0                     & 0                     & 1                     & 0                     & 0                     & 0                     & 0                     & 1                     & 0                     & 0                     & 0                     & 1                     & 0                     & 0                     & 4                     \\ \hline
\textbf{38}                        & 1                     & 0                     & 0                     & 0                     & 1                     & 0                     & 0                     & 0                     & 0                     & 0                     & 1                     & 0                     & 0                     & 0                     & 1                     & 0                     & 4                     \\ \hline
\textbf{39}                        & 1                     & 0                     & 0                     & 0                     & 1                     & 0                     & 0                     & 0                     & 0                     & 0                     & 0                     & 1                     & 0                     & 0                     & 0                     & 1                     & 4                     \\ \hline
\textbf{40}                        & 1                     & 0                     & 0                     & 0                     & 0                     & 1                     & 0                     & 0                     & 1                     & 0                     & 0                     & 0                     & 0                     & 1                     & 0                     & 0                     & 4                     \\ \hline
\textbf{41}                        & 1                     & 0                     & 0                     & 0                     & 0                     & 1                     & 0                     & 0                     & 0                     & 1                     & 0                     & 0                     & 1                     & 0                     & 0                     & 0                     & 4                     \\ \hline
\textbf{42}                        & 1                     & 0                     & 0                     & 0                     & 0                     & 1                     & 0                     & 0                     & 0                     & 0                     & 1                     & 0                     & 0                     & 0                     & 0                     & 1                     & 4                     \\ \hline
\textbf{43}                        & 1                     & 0                     & 0                     & 0                     & 0                     & 1                     & 0                     & 0                     & 0                     & 0                     & 0                     & 1                     & 0                     & 0                     & 1                     & 0                     & 4                     \\ \hline
\textbf{44}                        & 1                     & 0                     & 0                     & 0                     & 0                     & 0                     & 1                     & 0                     & 1                     & 0                     & 0                     & 0                     & 0                     & 0                     & 1                     & 0                     & 4                     \\ \hline
\textbf{45}                        & 1                     & 0                     & 0                     & 0                     & 0                     & 0                     & 1                     & 0                     & 0                     & 1                     & 0                     & 0                     & 0                     & 0                     & 0                     & 1                     & 4                     \\ \hline
\textbf{46}                        & 1                     & 0                     & 0                     & 0                     & 0                     & 0                     & 1                     & 0                     & 0                     & 0                     & 1                     & 0                     & 1                     & 0                     & 0                     & 0                     & 4                     \\ \hline
\textbf{47}                        & 1                     & 0                     & 0                     & 0                     & 0                     & 0                     & 1                     & 0                     & 0                     & 0                     & 0                     & 1                     & 0                     & 1                     & 0                     & 0                     & 4                     \\ \hline
\textbf{48}                        & 1                     & 0                     & 0                     & 0                     & 0                     & 0                     & 0                     & 1                     & 1                     & 0                     & 0                     & 0                     & 0                     & 0                     & 0                     & 1                     & 4                     \\ \hline
\textbf{49}                        & 1                     & 0                     & 0                     & 0                     & 0                     & 0                     & 0                     & 1                     & 0                     & 1                     & 0                     & 0                     & 0                     & 0                     & 1                     & 0                     & 4                     \\ \hline
\textbf{50}                        & 1                     & 0                     & 0                     & 0                     & 0                     & 0                     & 0                     & 1                     & 0                     & 0                     & 1                     & 0                     & 0                     & 1                     & 0                     & 0                     & 4                     \\ \hline
\end{tabular}
}
\caption{Canales cuánticos PCE de 2 qubits obtenidos de aplicar 
el método númerico descrito en la sección \ref{sec:ch2_solucionNumerica}
para evaluar la completa positividad de cada operación PCE.
\janote{Seguramente habrá que modificar cómo poner esta 
tabla porque no cabe en una página.}}
\label{tab:2qubitsPCEChannel1sAnd0s}
\end{table}
\begin{table}[h!]
\centering
\resizebox{\textwidth}{!}{%
\begin{tabular}{|P{0.6cm}|P{0.65cm}|P{0.65cm}|P{0.65cm}|P{0.65cm}|P{0.65cm}|P{0.65cm}|P{0.65cm}|P{0.65cm}|P{0.65cm}|P{0.65cm}|P{0.65cm}|P{0.65cm}|P{0.65cm}|P{0.65cm}|P{0.65cm}|P{0.65cm}|P{2.9cm}|}
\hline
\textbf{No.} 										 & \tauij{0}{0}		& \tauij{0}{1}   & \tauij{0}{2} 	 & \tauij{0}{3} 	& \tauij{1}{0}	 & \tauij{1}{1}		 & \tauij{1}{2} 	& \tauij{1}{3}	 & \tauij{2}{0}		& \tauij{2}{1} 		& \tauij{2}{2} 	 & \tauij{2}{3} 	& \tauij{3}{0} 	 & \tauij{3}{1}   & \tauij{3}{2}   & \tauij{3}{3} & \bf{Componentes de Pauli \boldmath{$r_{ij}$} invariantes} \\ \hline
\textbf{51}                        & 1                     & 0                     & 0                     & 0                     & 0                     & 0                     & 0                     & 1                     & 0                     & 0                     & 0                     & 1                     & 1                     & 0                     & 0                     & 0                     & 4                     \\ \hline
\textbf{52}                        & 1                     & 1                     & 1                     & 1                     & 1                     & 1                     & 1                     & 1                     & 0                     & 0                     & 0                     & 0                     & 0                     & 0                     & 0                     & 0                     & 8                     \\ \hline
\textbf{53}                        & 1                     & 1                     & 1                     & 1                     & 0                     & 0                     & 0                     & 0                     & 1                     & 1                     & 1                     & 1                     & 0                     & 0                     & 0                     & 0                     & 8                     \\ \hline
\textbf{54}                        & 1                     & 1                     & 1                     & 1                     & 0                     & 0                     & 0                     & 0                     & 0                     & 0                     & 0                     & 0                     & 1                     & 1                     & 1                     & 1                     & 8                     \\ \hline
\textbf{55}                        & 1                     & 1                     & 0                     & 0                     & 1                     & 1                     & 0                     & 0                     & 1                     & 1                     & 0                     & 0                     & 1                     & 1                     & 0                     & 0                     & 8                     \\ \hline
\textbf{56}                        & 1                     & 1                     & 0                     & 0                     & 1                     & 1                     & 0                     & 0                     & 0                     & 0                     & 1                     & 1                     & 0                     & 0                     & 1                     & 1                     & 8                     \\ \hline
\textbf{57}                        & 1                     & 1                     & 0                     & 0                     & 0                     & 0                     & 1                     & 1                     & 1                     & 1                     & 0                     & 0                     & 0                     & 0                     & 1                     & 1                     & 8                     \\ \hline
\textbf{58}                        & 1                     & 1                     & 0                     & 0                     & 0                     & 0                     & 1                     & 1                     & 0                     & 0                     & 1                     & 1                     & 1                     & 1                     & 0                     & 0                     & 8                     \\ \hline
\textbf{59}                        & 1                     & 0                     & 1                     & 0                     & 1                     & 0                     & 1                     & 0                     & 1                     & 0                     & 1                     & 0                     & 1                     & 0                     & 1                     & 0                     & 8                     \\ \hline
\textbf{60}                        & 1                     & 0                     & 1                     & 0                     & 1                     & 0                     & 1                     & 0                     & 0                     & 1                     & 0                     & 1                     & 0                     & 1                     & 0                     & 1                     & 8                     \\ \hline
\textbf{61}                        & 1                     & 0                     & 1                     & 0                     & 0                     & 1                     & 0                     & 1                     & 1                     & 0                     & 1                     & 0                     & 0                     & 1                     & 0                     & 1                     & 8                     \\ \hline
\textbf{62}                        & 1                     & 0                     & 1                     & 0                     & 0                     & 1                     & 0                     & 1                     & 0                     & 1                     & 0                     & 1                     & 1                     & 0                     & 1                     & 0                     & 8                     \\ \hline
\textbf{63}                        & 1                     & 0                     & 0                     & 1                     & 1                     & 0                     & 0                     & 1                     & 1                     & 0                     & 0                     & 1                     & 1                     & 0                     & 0                     & 1                     & 8                     \\ \hline
\textbf{64}                        & 1                     & 0                     & 0                     & 1                     & 1                     & 0                     & 0                     & 1                     & 0                     & 1                     & 1                     & 0                     & 0                     & 1                     & 1                     & 0                     & 8                     \\ \hline
\textbf{65}                        & 1                     & 0                     & 0                     & 1                     & 0                     & 1                     & 1                     & 0                     & 1                     & 0                     & 0                     & 1                     & 0                     & 1                     & 1                     & 0                     & 8                     \\ \hline
\textbf{66}                        & 1                     & 0                     & 0                     & 1                     & 0                     & 1                     & 1                     & 0                     & 0                     & 1                     & 1                     & 0                     & 1                     & 0                     & 0                     & 1                     & 8                     \\ \hline
\textbf{67}                        & 1                     & 1                     & 1                     & 1                     & 1                     & 1                     & 1                     & 1                     & 1                     & 1                     & 1                     & 1                     & 1                     & 1                     & 1                     & 1                     & 16                    \\ \hline
\end{tabular}
}
\end{table}

En las tablas \janote{tal y tal} mostramos los canales cuánticos de 3
qubits que dejan 1, 2 y 4 componentes invariantes. \janote{bla bla bla...}
 
Analizar numéricamente, una por una, todas las operaciones PCE de 3 qubits 
es una tarea imposible. El número total de operaciones PCE para el caso 
de 3 qubits es de alrededor de $9\times10^{18}$. Supongamos por un momento
que contamos con una computadora promedio para un estudiante de física
en Guatemala, pero con memoria ilimitada que puede analizar 
la completa positividad de 100 operaciones PCE de 3 qubits por segundo. A esa 
computadora le tomaría entre $1/4$ y $1/5$ de la edad del universo 
($13.7\times10^9$ años) analizar todas las operaciones PCE de 3 qubits.
Ahora bien, si consideramos una computadora real, con memoria limitada, 
analizar todas las operaciones PCE de 3 qubits que dejan 
5 componentes de Pauli invariantes supone un problema de memoria. En las
tablas \janote{tal y tal} se muestran el uso de memoria y tiempo de cómputo.

\noindent
\esqueleto{Con nuestro método numérico de fuerza bruta es posible 
resolver el caso de 3 qubits hasta 4 componentes invariantes. Los resultados   
son.... «con 3 qubits está todavía más jalado inferir características de 
los canales PCE a partir de los 1's y 0's»}

\noindent
\esqueleto{Para justificar que es computacionalmente imposible 
resolver numéricamente el problema de las operaciones PCE de 3 qubits, 
más allá de 4 componentes de Pauli invariantes, en el tiempo de este 
trabajo de tesis mostramos gráficas del tiempo de 
cómputo (tiempo vs. cantidad de operaciones PCE) y yo esperaría 
mostrar, por lo menos, que esa curva no es lineal y que el ajuste a la curva 
calcula un chingo de tiempo que no tenemos durante la tesis.}

\section{Una representación geométrica}\label{sec:ch3_geometric_representation}
\esqueleto{Motivados en lo intricado de inferir qué características 
comparten los canales PCE a partir de las listas de 1's y 0's, se nos ocurrió 
una forma de representar geométricamente a las operaciones PCE que hace
más sencillo el análisis de resultados.}

\esqueleto{La figura asociada con una operación PCE de 1 qubit 
es una columna de 
cuadritos.. bla bla y con figuritas, haciendo referencia a lo que se resolvió 
en el capítulo anterior, etc.}

\esqueleto{Para una operación PCE de 2 qubits, los dos índices en las $\tau$
sugieren que ahora la figura asociada debería ser de dos dimensiones. Así, 
los tableritos representan a estas operaciones. Figuritas para explicar y demás.}

\esqueleto{En este punto, ya es más o menos obvio cómo es la representación 
geométrica de 3 qubits y que a partir de 4 qubits ya no podremos utilizar 
esta herramienta geométrica. Mostrar algunas figuras de PCEs de 3 qubits
y hablar de cómo hacer diferencia entre correlaciones y componentes 
locales en esas figuras según los colores 
(sólo para 3 qubits, porque las de 1 y 2 qubits 
las voy a poner en negro).}

\noindent
\esqueleto{Armados con esta potente herramienta geométrica, ahora 
es mucho más sencillo ganar intuición de las operaciones PCE e inferir 
características de los canales PCE. Entonces ahora mostraré los resultados 
de la sección anterior, pero usando las figuritas.}

\section{Discusión de resultados}\label{sec:ch3_discussion}
\janote{Con el fin de hacer más eficiente la redacción, voy a partir del 
documento que preparamos para Sergey (justo coincide los resultados
que iban ahí con lo que vamos a poner en la tesis) y voy a iterar.}

\esqueleto{Esta sección es la que posee el contenido más importante 
de este manuscrito, después de la motivación y planteamiento del 
problema. Después de esta sección, lo único que haremos será 
estudiar si las operaciones PCE son un subconjunto de otras operaciones 
que fueron estudiadas por Ruskai.}

\esqueleto{Las figuras de los canales PCE exhiben patrones que todos 
comparten, parecen respetar alguna simetría...}

\esqueleto{Todos los canales PCE obedecen la regla de $2^k$...}

\esqueleto{Existen familias de canales PCE equivalentes. En las figuritas, 
esto se ve como transposiciones y permutaciones de filas y columnas. 
Físicamente, estos son swaps de partículas y cambios de base local.
Aquí yo creería que vale la pena discutir en palabritas, como en el documento
para sergey, pero también echarle algunas expresiones matemáticas como 
la de aplicar un swap, el PCE, y otro swap, por ejemplo...}

\esqueleto{La familia más sencilla de analizar es la de los PCE de 1 qubit
que dejan dos componentes de Pauli invariantes. Todos se pueden entender 
como la misma operación, pero conectados por rotaciones.}

\esqueleto{Hay correspondencia en el número de canales PCE que 
dejan $2^k$ y $2^{2n-k}$ componentes de Pauli invariantes.}

\esqueleto{Amarrado a la correspondencia de ``arcoiris'' van las reglas 
empíricas que formulamos con Alejo. Esta es otra prueba empírica que 
respalda la hipótesis de una conexión/correspondencia entre canales PCE.}

\esqueleto{Listo, hagamos un resumen de las características puntuales 
que inferimos de los canales PCE: ta ta ta.... Ahora sólo nos hace falta 
formalizar todo esto y hacer conexión formal entre todas las características. 
Además, sería deseable buscar alternativas para poder 
explorar numéricamente el caso completo de 3 e incluso de 4 qubits.}