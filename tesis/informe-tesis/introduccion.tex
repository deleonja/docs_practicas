%%% Haga el diseño que más le guste
\chapter{INTRODUCCIÓN}
Para hacer una descripción completa de los sistemas cuánticos 
con frecuencia se debe considerar la interacción del sistema con su entorno. Por un lado, 
en la naturelaza no existen los sistemas cuánticos ideales 
aislados completamente del resto del universo. 
Por el otro, en sistemas de muchos cuerpos es necesario un formalismo
con el que se pueda describir la evolución de un subsistema, considerando
que en general interactúa con la partición restante del sistema.
Por ejemplo, para hacer una descripción más precisa de un átomo en el laboratorio se
necesita considerar su interacción con la radiación de fondo, 
o para describir a un átomo atrapado en una red óptica se debe considerar, 
en una primera aproximación, la interacción con sus próximos vecinos.
Los canales cuánticos y la ecuación maestra en la forma de Lindblad son algunas
de las herramientas disponibles para describir la evolución de la 
matriz densidad de los sistemas abiertos~\cite{nielsen_chuang_2011}. 

La decoherencia es un fenómeno con mucho interés por entender. 
Este fenómeno ha sido discutido desde los inicios de la mecánica cuántica,
en la década de 1930, por físicos como
Von Neumann~\cite{von2018mathematical} y Heisenberg~\cite{bacciagaluppi2003role}.
No obstante, los fundamentos modernos de los modelos
de decoherencia fueron introducidos en las décadas de 1970 y 1980 
por H. Dieter Zeh~\cite{zeh1970interpretation}  y W. Zurek~\cite{zurek1981pointer}. 
De manera muy general, la decoherencia es un proceso mediante el cual 
un sistema pierde su coherencia cuántica. 
Algunos ejemplos son el colapso de 
la función de onda, la disipación de energía 
de un sistema hasta llevarlo a su estado fundamental y la transformación de los 
estados puros a estados mixtos, entre otros.
Desde un punto de vista teórico, ha habido un debate acerca de la afirmación
que la decoherencia resuelve el problema de la medida en 
la mecánica cuántica~\cite{bacciagaluppi2003role}.
Por otro lado, en un sentido práctico y de aplicación, 
el estudio y caracterización de la decoherencia 
ayuda a entender el impacto que tiene en los protocolos de información 
cuántica como teleportación cuántica y codificación 
superdensa~\cite{pepino2011open}.

En este trabajo introducimos a las operaciones que borran
las componentes de Pauli (PCE por sus siglas en inglés,
\textit{Pauli Component Erasing}), que generalizan las decoherencias de una
partícula de dos niveles (qubits) para sistemas de $n$ qubits.  Para evaluar si
una operación PCE es un canal cuántico, i.e. si modela una evolución física de
un sistema abierto, la operación debe ser completamente positiva y preservar
las propiedades de la matriz densidad~\citep{bengtsson_zyczkowski_2017},
como lo discutiremos en el capítulo 1.  El
caso de 1 qubit es sencillo e ilustrativo para entender el punto de partida de
nuestro problema.  Las operaciones PCE de 1 qubit borran o preservan las
componentes del vector de Bloch, y, de todas, son canales cuánticos sólo
aquellas que mapean la esfera de Bloch al origen de coordenadas, a una línea
sobre los ejes cartesianos y a sí misma (la operación identidad). Estos canales
cuánticos modelan la decoherencia de un sistema de dos niveles cuando su estado
se transforma al estado máximamente mixto, cuando colapsa a alguno de los
eigenestados de las matrices de Pauli y cuando el estado no se modifica.
Específicamente, en este trabajo buscamos numéricamente las operaciones PCE de
2 y, parcialmente, de 3 qubits que satisfacen la completa positividad y
analizamos las características en común para buscar pistas de una
caracterización general de este tipo de canales cuánticos. Además, estudiamos
los canales de Pauli constantes sobre los ejes, de Nathanson y
Ruskai~\cite{nathanson2007pauli}, para investigar si los canales cuánticos PCE
pudieran ser un subconjunto que pertenece a otros canales de Pauli que ya han
sido estudiados. 



