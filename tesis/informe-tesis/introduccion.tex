%%% Haga el diseño que más le guste
\chapter{INTRODUCCIÓN}
\esqueleto{Sistemas abiertos}

Una descripción completa de un sistema cuántico requiere incluir la interacción
con su entorno, otro sistema cuántico externo. Se conoce a estos sistemas 
como sistemas cuánticos abiertos. Ningún sistema cuántico en la
naturaleza está completamente aislado del resto del universo.  
Por ejemplo, para describir con mayor generalidad a un átomo en una red óptica, se debe
considerar que el átomo fue inicialmente preparado en un estado \textit{compartido} 
con los demás átomos de la red. En consecuencia,
la evolución de este tipo de sistemas no es unitaria, en general, como 
la evolución de los sistemas ideales que no interactúan con su entorno. En ese sentido, 
los canales cuánticos proporcionan una teoría que captura esta propiedad de 
los sistemas abiertos. 

\esqueleto{Qubits y decoherencia de 1 qubit}

La decoherencia es un proceso al que irremediablemente están sujetos los 
sistemas cuánticos abiertos. 
Este fenómeno es el proceso mediante 
el cual la superposición de estados en el que se encuentra un sistema colapsa a 
sólo uno de los estados de la superposición (pierde su coherencia cuántica)
a causa de la interacción con su entorno. 
Los sistemas de dos niveles son 
los más sencillos y con mucho interés teórico como para estudiar 
la decoherencia de este tipo de sistemas. En información y computación 
cuántica se conoce a estos sistemas como qubits, y son de gran importancia 
porque son el análogo cuántico de los bits clásicos en 
la implementación de la computación cuántica. 
Algunos ejemplos de sistemas físicos que implementan a un qubit son el espín
del electrón o la polarización de un fotón. Para un qubit,
un tipo de decoherencia se puede entender como el colapso de su 
estado cuántico $\ket{\psi}$ 
%que en general es una superposición de dos 
%estados ortonormales 
%($\ket{0}$ y $\ket{1}$, o $\ket{+}$ y $\ket{-}$, etc),
a un estado de alguna base ortonormal de un espacio complejo de dimensión dos. 

\esqueleto{Operaciones PCE}

En este trabajo queremos estudiar una generalización para sistemas de varios 
qubits de las operaciones que modelan el proceso de decoherencia de 1 qubit. 