%%% Haga el diseño que más le guste
\chapter{INTRODUCCIÓN}
\esqueleto{Ningún sistema cuántico en la vida real es un sistema completamente
aislado del resto del universo. En realidad, todos los sistemas cuánticos interactúan,
en menor o mayor grado, con un sistema cuántico externo que se conoce 
como entorno.}


\esqueleto{La decoherencia es un proceso al que irremediablemente están 
sujetos los sistemas cuánticos abiertos. La decoherencia es el colapso de
la superposición de estados de un sistema a sólo uno de los estados de la 
superposición.}


\esqueleto{La teoría de los canales cuánticos en un marco conceptual que puede  
capturar la dinámica de los sistemas abiertos. Un canal cuántico
es una operación que modela un Hamiltoniano o un circuito cuántico. Así mismo, 
el proceso de decoherencia puede modelarse con canales cuánticos.}

\esqueleto{Los sistemas de dos niveles son por excelencia los sistemas cuánticos
más sencillos. Esto los hace los sistemas perfectos para comenzar a estudiar
herramientas para describir la decoherencia. Para 1 qubit, la decoherencia
puede modelarse con canales cuánticos de 1 qubit que han sido ampliamente 
estudiados en el pasado.}

\esqueleto{En este trabajo nuestro objetivo es estudia los canales cuánticos 
que modelan la decoherencia de sistemas de $n$ qubits.}
