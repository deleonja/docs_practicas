%%% Haga el diseño que más le guste
\chapter{INTRODUCCIÓN}
%\esqueleto{Sistemas abiertos}

Una descripción completa de un sistema cuántico requiere incluir la interacción
con su entorno, es decir, considerar a los sistemas como abiertos. Ningún sistema cuántico en la
naturaleza está completamente aislado del resto del universo.  
Por ejemplo, para describir con generalidad a un átomo en una red óptica, se debe
considerar que el átomo se encuentra inicialmente en un estado \textit{compartido} 
con los demás átomos de la red. En consecuencia,
la evolución de este tipo de sistemas no es unitaria, en general, como 
la de los sistemas ideales que no interactúan con su entorno. En ese sentido, 
los canales cuánticos proporcionan una herramienta que captura 
la no unitariedad de la dinámica de los sistemas abiertos.

%\esqueleto{Qubits y decoherencia de 1 qubit}

La decoherencia es un proceso al que irremediablemente están sujetos los 
sistemas cuánticos abiertos. 
Este fenómeno es el proceso mediante 
el cual la superposición de estados en el que se encuentra un sistema colapsa a 
sólo uno de los estados de la superposición (pierde su coherencia cuántica)
a causa de la interacción con su entorno. 
Los sistemas de dos niveles son 
los más sencillos y con mucho interés teórico como para estudiar 
la decoherencia de este tipo de sistemas. En información y computación 
cuántica se conoce a estos sistemas como qubits, y son de gran importancia 
porque son el análogo cuántico de los bits clásicos en 
la implementación de la computación cuántica. 
Algunos ejemplos de sistemas físicos que implementan a un qubit son el espín
del electrón o la polarización de un fotón. 
Existe un tipo de decoherencia de 1 qubit que se puede entender como 
el colapso de su estado cuántico $\ket{\psi}$ a alguno de los dos eigenestados
del operador de espín en la dirección \textit{z}.

%\esqueleto{Operaciones PCE}

Nuestro interés se enfoca en entender la generalización
para sistemas de $n$ qubits de las operaciones que modelan el proceso de 
decoherencia de 1 qubit. Para esto, introduciremos la definición de una 
operación que borra las componentes de Pauli, PCE por sus siglas en inglés
(\textit{Pauli component erasing}). Una operación PCE es una operación lineal 
que preserva o borra por completo las proyecciones de la matriz de densidad de $n$ qubits
sobre la base de productos tensoriales de las matrices de Pauli. Vamos a investigar 
las características en común del subconjunto de las operaciones PCE que son 
completamente positivas y, por consiguiente, canales cuánticos que describen
diferentes tipos de decoherencia de un sistema de $n$ qubits. Vamos a 
discutir cómo nuestros resultados muestran que este tipo particular de 
canales cuánticos, los \textit{canales cuánticos PCE}, podrían poseer 
una estructura matemática. 