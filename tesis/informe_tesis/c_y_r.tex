%%% INCLUYA SUS CONCLUSIONES Y RECOMENDACIONES


\chapter{CONCLUSIONES}
En este trabajo de tesis propusimos estudiar un nuevo tipo de operaciones, dentro
del marco teórico de los canales cuánticos, 
que pueden modelar el proceso de decoherencia cuántica para sistemas de $n$ qubits 
(sistemas de dos niveles). 
En este manuscrito aportamos pruebas que evidencian
una caracterización propia de este tipo de canales cuánticos
y, con esto, se abre la posibilidad a preguntas más fundamentales acerca de estas 
operaciones.
Para 1 qubit, el proceso de decoherencia
puede describirse por medio de canales cuánticos bien conocidos como 
el \textit{bit-flip} (inversor de bit), operación que proyecta el estado del 
sistema a alguno de los eigenestados de $\sigma_z$
\cite{bengtsson_zyczkowski_2017,nathanson2007pauli,nielsen_chuang_2011}. 
Para $n$ qubits, 
introdujimos la definición de una operación que borra las componentes de Pauli
(PCE por sus siglas en ingles, \textit{Pauli Component Erasing}) como una 
operación diagonal de Pauli que preserva o borra por completo 
las proyecciones de la matriz densidad sobre los elementos de la
base de productos tensoriales de las matrices de Pauli (componentes de Pauli). 
Diseñamos un método 
númerico para implementar la búsqueda de los \textit{canales cuánticos PCE},
\textit{i.e.} operaciones PCE que son completamente positivas, y nuestros 
resultados muestran evidencia que estos canales cuánticos podrían tener
una estructura matemática propia. 
Por un lado, los canales cuánticos PCE pueden clasificarse en clases de equivalencia, 
es decir, subconjuntos dentro de los cuales todos los elementos están 
conectados vía operaciones unitarias. 
Además, identificamos que las operaciones PCE 
que satisfacen la completa positividad obedecen dos reglas, (1) \textit{regla
$\mathit{2^k}$}: preservan una cantidad de componentes de Pauli
que es una potencia de dos, y (2) \textit{regla espejo}:
existe la misma cantidad de canales cuánticos PCE que preservan $2^k$ 
y $2^{2n-k}$ componentes de Pauli.
Por úlitmo, probamos que los canales cuánticos PCE no son un subconjunto
de otro tipo de canales cuánticos de Pauli que se han estudiado antes 
\cite{nathanson2007pauli}. 


% \cpnote{Esto lo pondria mas arriba. Siento que acá estorba: (una generalización de los canales cuánticos
% que describen la decoherencia
% para sistemas de $n$ qubits}
% \janote{Listo. Lo moví para el segundo enunciado. Creí que allí queda bien con el flujo.}

% \cpnote{Primero, la introduccion de un nuevo tipo de canales que generalizan 
% las decoherencias basicas de un qubit. A esto tienen que ir dos frases, y también 
% recoerdando al lector que son los PCEs}
% \cpnote{Plantea las cosas un poco mas generales. Es decir, puedes hacer un planteamiento
% general que contextualice la importancia de tus resultados en general, y que luego 
% se aplican a los PCEs. Itntenta plantear un poco las cosas como en la ultima frase de 
% este parrafo y luego dices en particular lo qeu hacemos. Como por cada frase de aca
% pon una frase que la anteceda tipo la ultima. Quizá vale la pena platicar de esto, 
% pareciera confuso}
% 
% \noindent
% \esqueleto{Ideas:
% \begin{itemize}
% 	\item Buscamos generalizar las decoherencias básicas de un qubit [?]
% 	\item Con esto introducimos la definición de una operación PCE
% 	\item Implementamos computacionalmente un método numérico para 
% 	encontrar canales cuánticos PCE
% 	\item Encontramos tales propiedades 
% 	\item Encontramos que se pueden clasificar
% \end{itemize}
% }

\chapter{TRABAJO FUTURO}
La evidencia que muestran nuestros resultados sobre una estructura 
matemática de los canales cuánticos PCE apunta hacia una dirección 
muy clara para el trabajo futuro inmediato: estudiar analíticamente la caracterización 
general de los canales cuánticos PCE de $n$ qubits. Específicamente, hay algunas 
tareas que proponemos realizar:
\begin{itemize}
\item Estudiar la diagonalización exacta de la matriz de Choi de las operaciones 
PCE de $n$ qubits. Los eigenvectores de la matriz de Choi $D_{\E}$
son un conjunto de operadores de Kraus del canal cuántico $\E$ 
\cite{bengtsson_zyczkowski_2017}. Por lo tanto, 
estudiar la representación de Kraus de los canales cuánticos PCE de 1, 2 y 3 qubits,
que encontramos en este trabajo, podría proporcionar intuición 
sobre los operadores de Kraus en el caso general y, por consiguiente, 
sobre los eigenvectores de la matriz de Choi $D_{\E}$. 
\item Estudiar la relación entre los conjuntos de índices $j_1,\ldots,j_n$
de los $\taus=1$ de los canales cuánticos PCE. Esto fue lo que comenzamos 
a hacer al elaborar las \textit{reglas empíricas} para construir canales PCE de 2 qubits
al final de la sección \ref{sec:ch3_discussion}. Dado que un canal cuántico PCE 
está completamente caracterizado por los valores 1's y 0's del conjunto $\qty{\taus}$, 
entonces vale la pena explorar si existe alguna conexión entre los índices $j_1,\ldots,j_n$. de los 
elementos $\taus=1$, y las características de los canales 
cuánticos PCE discutidas en la sección \ref{sec:ch3_discussion}.
\item Investigar la existencia de un conjunto generador de canales cuánticos 
PCE. En la \Fref{fig:2qubits_PCEChannels_figs} es sencillo ver que la superposición
de las dos primeras figuras PCE de la clase C${}_8^1$ dan como resultado 
la primera figura PCE de la clase C${}_4^1$. Esto se traduce a que la 
concatenación de dos canales PCE da como resultado otro canal PCE. Por lo tanto,
sería interesante explorar más a fondo esta idea para investigar la existencia 
de algún conjunto cuyos elementos generen al resto de canales cuánticos PCE 
mediante alguna operación como la concatenación.
\end{itemize}
Por otro lado, a más largo plazo, otra línea futura de investigación puede ser 
la de estudiar una generalización de los canales cuánticos PCE para sistemas 
de $d$ niveles. Existen otras bases del espacio $\mathcal{M}_d$ de las matrices de 
$d\times d$, como las matrices de GellMann o de Weyl, 
que podrían utilizarse para estudiar operaciones proyectivas, es decir
del tipo \textit{component erasing}, que actúan sobre sistemas de $d$ niveles.
