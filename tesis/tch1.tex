\chapter{FUNDAMENTOS TEÓRICOS}

\section{Introducción}
\esqueleto{Escribiré que la matriz de densidad
y la teoría de canales cuánticos es el marco teórico para el problema 
de las operaciones PCE. Luego un enunciado que resuma de manera
introductoria la idea de cada uno de los conceptos. \newline\newline
Por último enunciar la estructura del capítulo.}

\section{Ensambles de estados cuánticos}
\esqueleto{Un copy-paste de la sección 1.1 del informe final de 
prácticas. Voy a retocar alguna parte si fuera necesario, como ser
más formal o agregar alguna prueba.}

Para introducir la definición del operador de densidad presentamos la 
motivación que exponen Sakurai y Napolitano \cite{sakurai_napolitano_2017}.
Consideremos un sistema cuántico que se encuentra en alguno de los estados 
$\Pk{i}$ con probabilidad $p_i$. El conjunto de todos los posibles estados
conforman un ensamble de estados del sistema. Supongamos que realizamos 
la medición de algún observable $\Lambda$ sobre el ensamble. El 
valor esperado al medir $\Lambda$ sobre este sistema se encuentra
\begin{align}\label{eq:expVal-expanded-Lambda}
	\expval{\Lambda} &= \sum_i p_i \matrixel{\psii}{\Lambda}{\psii}\nonumber\\
	&= \sum_{i,j,k} p_i 
	\bra{\psii}\dyad{\phi_j}{\phi_j}\Lambda\dyad{\varphi_k}{\varphi_k}\Pk{i},
\end{align}
con $\ket{\phi_j}$ y $\ket{\varphi_k}$ dos bases ortonormales distintas del
espacio de Hilbert del sistema. Ahora, al reordenar 
\eqref{eq:expVal-expanded-Lambda} se obtiene una expresión 
que motiva claramente la definición de la matriz de densidad $\rho$,
\begin{align}\label{eq:expVa-Lambda-wRho}
	\expval{\Lambda}&= \sum _{j,k}\bra{\varphi_k}\qty(\sum_ip_i \dyad{\psii}{\psii} 
	)\ket{\phi_j}	\matrixel{\phi_j}{\Lambda}{\varphi_k}.
\end{align}
La matriz que se encuentra dentro del paréntesis se define como 
la matriz de densidad 
$\rho$~\cite{nielsen_chuang_2011, sakurai_napolitano_2017}
\begin{equation}
	\rho \equiv \sum _i p_i\dyad{\psi_i}{\psi_i}.
	\label{eq:rho_def}
\end{equation}
Notemos que 
\begin{align}
	\matrixel{\varphi_k}{\rho}{\phi_j} = 
	\sum_ip_i \braket{\varphi_k}{\psii}\braket{\psii}{\phi_j},
\end{align}
por lo tanto, sustituyendo en la ecuación \eqref{eq:expVa-Lambda-wRho},
se tiene
\begin{align}\label{eq:Tr(rhoLambda)}
	\expval{\Lambda}=\sum _k \matrixel{\varphi_k}{\rho \Lambda}{\varphi_k} 
	= \Tr \qty(\rho\Lambda).
\end{align} 
Este resultado muestra que la matriz de densidad $\rho$ 
contiene toda la información física disponible de un sistema al 
realizar una medición.

Desde luego la matriz de densidad es una herramienta con la que
se puede formular matemáticamente la mecánica cuántica como 
con el vector de estado. Por ejemplo, veamos a continuación cómo 
describir la evolución de un sistema cuántico cerrado.
Consideremos un sistema cuyo Hamiltoniano es $H$, y es
independiente del tiempo. Bajo estas condiciones la evolución
del sistema está descrita por 
$\ket{\psi (t)}=e^{-iHt/\hbar}\ket{\psi(0)}$~\cite{sakuarai}. 
Sin embargo, consideremos que el sistema se encuentra inicialmente
en un ensamble de estados $\{p_i, \ket{\psii(0)}\}$, por lo cual
el estado final del sistema será 
$\{p_i, \ket{\psii(t)}\}$.
Por consiguiente, el operador de densidad final $\rho(t)$  es 
\begin{align} \label{eq:Rho-evolution-H}
	\rho(t) &= \sum_j p_j\dyad{\psi_j(t)}{\psi_j(t)}\nonumber\\
	&= \sum_j p_j e^{-iHt} \dyad{\psi_j(0)}{\psi_j(0)}\qty(e^{-iHt})^{\dagger}
	\nonumber\\
	&= e^{-iHt}\rho(0)e^{iHt},
\end{align}
donde $\rho(0)$ es la matriz de densidad que describe al ensamble 
de estados inicial $\{p_i, \ket{\psii(0)}\}$.	 Aunque hemos desarrollado 
un ejemplo para la evolución de un sistema cuyo Hamiltoniano es 
independiente del tiempo, es sencillo de ver que en general la dinámica  
de un sistema cerrado se describe como 
\begin{align}\label{eq:rho-ClosedEvolution}
\rho(0) \longrightarrow U\rho(0)U^{\dagger}.
\end{align}
Con esto, se ha asegurado que la dinámica de un sistema cuántico puede 
describirse utilizando su matriz de densidad.

Las ecuaciones \eqref{eq:Tr(rhoLambda)} y \eqref{eq:rho-ClosedEvolution}
muestran que la matriz de densidad puede utilizarse para la descripción 
de la medición y la evolución de los estados cuánticos. 
En la siguiente sección veremos la formulación de los postulados 
de la mecánica cuántica utilizando la matriz de densidad.

\section{Propiedades de la matriz de densidad}
\esqueleto{Revisando esta sección en el informe de prácticas 
veo que me gustaría ir aquí más al grano y mandar al lector 
a las pruebas en el Chuang (para no copiar otra vez las pruebas
aquí). Puntualizaré: (1) caracterización de la matriz de densidad, (2)
postulados de la mecánica cuántica usando la matriz de densidad y
(3) matriz de densidad reducida. Para la matriz de densiddad reducida
voy a omitir el ejemplo que coloqué en el informe de prácticas.}



Según Nielsen y Chuang \cite{nielsen_chuang_2011} las matrices
de densidad están caracterizadas por el siguiente teorema:
\begin{thm}\label{teo:density-operator}
Un operador $\rho$  que actúa sobre el espacio de Hilbert de un sistema 
es el operador de densidad asociado a algún ensamble 
$\{p_i, \ket{\psi _i} \}$ si y sólo si satisface las condiciones:
\begin{enumerate}
\item $\Tr \rho = 1$.
\item $\rho \geq 0$.
\end{enumerate}	
\end{thm} 
\begin{proof} Consultar~\cite[p.~101]{nielsen_chuang_2011} \end{proof}

La condición de traza unitaria de la matriz de densidad cumple la misma
función que la condición de normalización del vector de estado. La
suma de las probabilidades de todas las posibles mediciones 
de un observable deben sumar uno. La condición de positividad 
implica que los eigenvalores $\lambda_i$ y eigenvectores
$\ket{i}$ dan lugar a $\{\lambda_i, \ket{i}\}$, uno de todos los 
posibles ensambles que están asociados a la matriz de densidad $\rho$.

Ahora que hemos establecido de manera precisa a la matriz de densidad
nos ocupamos de la formulación de los postulados de la mecánica cuántica 
utilizando la matriz de densidad \cite[p.~102]{nielsen_chuang_2011}.
\begin{itemize}
	\item[] \textbf{Postulado 1.} \textit{Estado del sistema.} 
	Un sistema físico tiene asociado un espacio vectorial complejo
	con producto interno que se conoce como el espacio de Hilbert $\hi$ del
	sistema. Los estados del sistema están descritos por el conjunto 
	de matrices de densidad en el espacio de Hilbert-Schmidt $\mathcal{HS}$	
	que actúan sobre el espacio de Hilbert $\hi$ del sistema. 
	\item[] \textbf{Postulado 2.} \textit{Evolución unitaria.}
	La evolución de un sistema cuántico cerrado en un intervalo 
	de tiempo $[t_1,t_2]$ está descrita 	por una transformación unitaria
	de la siguiente manera 
	\begin{equation} \label{eq:postulate-ClosedEvolution}
	\rho(t_1)\longrightarrow U\rho(t_1) U^{\dagger}=\rho(t_2).
	\end{equation}
	\item[] \textbf{Postulado 3.} \textit{Medición.}
	Las mediciones están descritas por un conjunto de 
	operadores $\{M_m\}$. Estos son operadores que actúan sobre el espacio 
	de Hilbert $\hi$ del sistema. El índice $m$ refiere a los posibles
	resultados de la medición. Si el estado del sistema es $\rho$ 
	inmediatamente antes de la medición, entonces la probabilidad
	de medir $m$ es
	\begin{equation} \label{eq:post_MeasureProb}
	p(m)=\Tr \qty(M_m^{\dagger}M_m\rho),
	\end{equation}						
	y el estado del sistema después de la medición será
	\begin{equation} \label{eq:post_MeasureTrasnfState}
	\rho'=\frac{M_m\rho M_m^{\dagger}}{\tr \qty(M_m^{\dagger}M_m\rho)}.
	\end{equation}	
	Los operadores $M_m$ deben satisfacer la ecuación de completitud
	\begin{equation} \label{eq:post_MeasureMCompleteness}
	\sum _m M_m^{\dagger}M_m=\mathbb{1}.
	\end{equation}
%	Este postulado confirma lo que en la sección anterior aseguramos
%	de que la formulación de cualquier medición proyectiva utilizando el 
%	operador de densidad es posible. Debemos notar que las mediciones
%	proyectivas son un caso especial de las mediciones enunciadas en
%	este postulado.
%	Una medición proyectiva está descrita por un observable
%	$\Omega$, que es un operador Hermítico que actúa sobre $\hi$. 
%	$\Omega$ tiene una descomposición espectral \cite{nielsen_chuang_2011}
%	\begin{align}
%		\Omega = \sum _i \lambda_iP_i,
%	\end{align}
%	donde $P_i$ es el proyector al autoespacio de $\Omega$ 
%	con autovalor $\lambda_i$.
%	De acuerdo con este postulado si el sistema se encuentra en el estado
%	$\rho$, la probabilidad de medir $\lambda_i$ es
%	\begin{align}
%		p(i) = \Tr \qty(P_i^{\dagger}P_i\rho) = \Tr \qty(P_i\rho).
%	\end{align}
%	Además, dado que se midió $\lambda_i$ el estado del sistema inmediatamente 
%	luego de realizar la medición es 
%	\begin{align}
%		\rho'&=\frac{P_i\rho P_i}{\Tr \qty(P_i \rho)}.
%	\end{align}
	\item[] \textbf{Postulado 4.} \textit{Sistemas de partículas.}
	El espacio de Hilbert de un sistema 	de varias partículas se compone del
	producto tensorial de los espacios de Hilbert 	individuales.
	Es decir, si el sistema total se compone de $N$ partículas, 
	entonces el sistema total es
	\begin{align}
		\hi_{\txt{total}} = \hi_1\ten \hi_2 \ten \ldots \ten \hi_N.
	\end{align}
	Los estados del sistema total están descritos por las matrices de 
	densidad que actúan sobre $\hi _{\text{total}}$.
\end{itemize}

Por último en esta sección discutiremos sobre la matriz de densidad
reducida, una herramienta especialmente útil para los sistemas 
de varias partículas. La matriz de densidad es una herramienta 
que proporciona toda la información física disponible al hacer 
una medición sobre cualquier parte del sistema total. 

Supongamos dos subsistemas $A$ y $B$ cuyo estado total es 
$\rho^{AB}$ (en general, $\rho^{AB}\neq \rho^A\ot \rho^B$). 
Consideremos una base ortonormal $\ket{\psi_i}$ de $A$ y una 
base ortonormal $\ket{\phi_j}$ de $B$. Una base ortonormal 
del sistema total está dado por el producto tensorial entre los elementos
de la base de A y de B. En esta base, un elemento de matriz
de $\rho^{AB}$ está dado por
\begin{align}
	\bra{\psi_i}\ten\bra{\phi_j}\rho^{AB}\ket{\psi_k}\ten\ket{\phi_l}.
\end{align}
Supongamos que $\Omega_A$ es un observable únicamente de $A$
que actúa sobre el espacio de Hilbert total. El valor esperado de 
$\Omega_A$, utilizando \eqref{eq:Tr(rhoLambda)}, es
\begin{align}
	\Tr \qty(\rho^{AB}\qty(\Omega_A\ot \mathbb{1})) &= \sum _{i,j} 
	\bra{\psi_i}\ten\bra{\phi_j}\rho^{AB}\qty(\Omega_A\ot \mathbb{1})
	\ket{\psi_i}\ten\ket{\phi_j} 
	\nonumber\\
	&= \sum _{i,j} 
	\bra{\psi_i}\ten\bra{\phi_j}\rho^{AB}
	\qty(\sum _{k,l} \qty(\ket{\psi_k}\ten\ket{\phi_l})
	\qty(\bra{\psi_k}\ten\bra{\phi_l}))
	\qty(\Omega_A\ot \mathbb{1})\ket{\psi_i}\ten\ket{\phi_j}\nonumber\\
	&= \sum _{i,j,k,l} 
	\bra{\psi_i}\ten\bra{\phi_j}\rho^{AB}\ket{\psi_k}\ten\ket{\phi_l}
	\matrixel{\psi_k}{\Omega_A}{\psi_i}\delta _{lj}\nonumber\\
	&= \sum_{i,k}\qty(\sum _j 
	\bra{\psi_i}\ten\bra{\phi_j}\rho^{AB}\ket{\psi_k}\ten\ket{\phi_j})
	\matrixel{\psi_k}{\Omega_A}{\psi_i},
	\label{eq:almost-reducedRho}
\end{align}
y de lo que está entre paréntesis definamos a la matriz de densidad
reducida $\rho^A$~\cite{chandra2013quantum}
\begin{align}
	\matrixel{\psi_i}{\rho^A}{\psi_k} = 
	\sum _j \bra{\psi_i}\ten\bra{\phi_j}\rho^{AB}\ket{\psi_k}\ten\ket{\phi_j}.
	\label{eq:reducedRho-def1}
\end{align}
Con $\rho^A$ definido retomemos  \eqref{eq:almost-reducedRho}
\begin{align}
	\Tr \qty(\rho^{AB}\qty(\Omega_A\ot \mathbb{1}))&= \sum _{i,k}
	\matrixel{\psi_i}{\rho^A}{\psi_k} \matrixel{\psi_k}{\Omega_A}{\psi_i}\\
	&= \sum_i \matrixel{\psi_i}{\rho^A\Omega_A}{\psi_i}\\
	&= \Tr \qty(\rho^A\Omega_A). \label{eq:reduced-works}
\end{align}
Este resultado es similar al que se obtuvo en la ecuación
\eqref{eq:Tr(rhoLambda)}, por lo que la conclusión también es similar:
la matriz de densidad reducida $\rho^A$ contiene la información 
que se puede extraer de $A$ al hacer una medición sobre ese subsistema.

Volviendo a la ecuación \eqref{eq:reducedRho-def1} se define a la 
matriz de densidad reducida de manera más precisa utilizando
la traza parcial
\begin{align}
	\matrixel{\psi_i}{\rho^A}{\psi_k} &= 
	\sum _j \bra{\psi_i}\ten\bra{\phi_j}\rho^{AB}\ket{\psi_k}\ten\ket{\phi_j}\\
	&=\matrixel{\psi_i}{\Tr_B\qty(\rho^{AB})}{\psi_k},
	\label{eq:partialTrace-def}
\end{align}
donde $\Tr_B$ es la traza parcial sobre el sistema $B$. 

\janote{aquí voy} \janote{hablar de que la traza parcial es trazar 
sobre los grados de libertad que no intersan}
En general, la traza parcial se define como \cite{nielsen_chuang_2011}
\begin{align}
	\Tr_B (\dyad{\alpha_i}{\alpha_j}\otimes \dyad{\beta_k}{\beta_l})
	\equiv
	\dyad{\alpha_i}{\alpha_j}\braket{\beta_k}{\beta_l},
	\label{eq:part_trace-def}
\end{align}
donde $\ket{\alpha_i}$ y $\ket{\beta_j}$ son cualesquiera dos vectores
ortonormales en
$\mathcal{H}_A$, y $\mathcal{H}_B$, respectivamente. Es sencillo probar
que, ya que la traza es una operación lineal, la traza parcial también 
lo es. 
Además, debe notarse en la deducción que se presentó para motivar 
al operador de densidad recudido, 
que la ecuación \eqref{eq:reducedRho-def1} evidencia 
el uso de la traza parcial, definida en \eqref{eq:part_trace-def}, 
como la única operación que da lugar al operador de densidad reducido. 

Antes de concluir revisaremos un ejemplo del cálculo del 
operador de densidad reducido que evidencia una de las
\textit{extrañas} consecuencias del
entrelazamiento cuántico.
Consideremos un sistema de 2 qubits que se encuentra en el estado de Bell
$\qty(\ket{00}-\ket{11})/\sqrt{2}$, un estado entrelazado. El operador
de densidad para el sistema compuesto es
\begin{align}
	\rho &= \qty(\frac{\ket{00}-\ket{11}}{\sqrt{2}})
	\qty(\frac{\bra{00}-\bra{11}}{\sqrt{2}}) \\
			 &= \frac{\dyad{00}{00}-\dyad{11}{00}-\dyad{00}{11}
                 +\dyad{11}{11}}{2}.
\end{align}
Calculamos ahora el operador de densidad reducido para el qubit 1.  
Haciendo la traza parcial sobre el qubit 2 se tiene
\begin{align}
	\rho^A &= \Tr _B(\rho) \\
			 	 &= \frac{\Tr _B(\dyad{00}{00})-\Tr _B(\dyad{11}{00})
			 	 -\Tr _B(\dyad{00}{11})+\Tr _B(\dyad{11}{11})}{2} \\
			 	 &= \frac{\dyad{0}{0} + \dyad{1}{1}}{2} \\
			 	 &= \frac{\mathbb{1}}{2}.
\end{align}
Notemos que el qubit 1 se encuentra en un estado mixto
$\qty(\Tr \qty(\mathbb{1}/2)^2<1)$. De hecho, si hiciéramos la 
traza parcial sobre el qubit 1 llegaríamos al mismo estado del qubit 2.
Es decir, aunque contamos con la información completa del estado 
de los dos qubits como sistema compuesto, tenemos incompleta la 
información del estado individual de cualquiera de los dos qubits
en el sistema. Esta es una de las consecuencias del entrelazamiento
cuántico.

Con el operador de densidad reducido hemos completado
el marco teórico del lenguaje del operador de densidad necesario
para el objetivo de este proyecto. Comenzamos
motivando el operador de densidad como herramienta 
para describir ensambles de
estados cuánticos, luego presentamos las características
del operador de densidad y la reformulación de los postulados 
de la mecánica cuántica.
Finalmente, concluimos introduciendo el operador de densidad reducido y
motivando el uso de la operación de traza parcial. Ahora, dirigimos
nuestra atención hacia la descripción de la dinámica de los estados
de sistemas cuánticos abiertos, el tipo de sistemas que son de interés
para este proyecto.

Esta formulación de la mecánica cuántica nos permite 
dos cosas que no sería posible utilizando la formulación del 
vector de estado:
(1) manipular mezclas estadísticas de estados de una manera más 
adecuada y
(2) 
describir estados de subsistemas individuales que forman parte de 
un sistema cuántico compuesto, como lo veremos
en la siguiente sección. 

\section{Canales cuánticos}
\esqueleto{Copy-paste del informe final, sección mapeos 
completamente positivos.}

\section{Representaciones de los canales cuánticos}
\esqueleto{Enunciar que existen las representaciones de 
Kraus y de superoperador. Copy-paste de las secciones en las 
que hablamos de las representaciones en el informe final. Planeo
dejar sólo un ejemplo y matar los ejemplos de las dos representaciones
en un tiro.}

%%%%%%%%%%%%%%%%%%%%%%%%%%%%%%%%%%%%%%
%%% Lo que está comentado abajo era parte del modelo, puede ser útil 
%%% después 
%%%%%%%%%%%%%%%%%%%%%%%%%%%%%%%%%%%%%%

%En adición a los conjuntos de puntos que se trabajan con normalidad
%en Matemáticas ~---puras y aplicadas---, se tendrá que hacer uso
%frecuentemente de los conjuntos de conjuntos, si por ejemplo $X$ es
%la recta real, como un intervalo es un conjunto de puntos, es decir
%un subconjuto de $X$, se tendrá que el conjunto de todos los
%intervalos es un conjunto de conjuntos.
%
%En especial, cuando una clase hace referencia a subconjuntos del
%conjunto $X$, la llamaremos \textbf{familia}. En especial el
%\textbf{conjunto potencia} $\mathcal{P} (X)=\{A\mid A\subseteq X\}$
%es una familia de $X$. Asimismo, se definirá el \textbf{complemento}
%de $A$ por \[A^c=\{x\mid x\notin A\}.\]
%
%\begin{defn}\label{dfcp28} Una \textbf{función de selección}
%para un conjunto $X$ es una función $f$ la cual asocia a cada
%subconjunto no vacío $E$ de $X$ un elemento de $E$: $f(E)\in E$.
%\end{defn}
%
%\begin{axm}[de selección]\label{choice} Para cualquier conjunto
%existe una función de selección.
%\end{axm}
%
%\begin{rem} Con frecuencia el axioma~\ref{choice} se presenta en
%la forma: para cada $E\in \mathcal{P}(X)\setminus\{\nada\}$,
%elegimos un elemento $x\in E$. Asimismo, es equivalente al lema de
%Zorn, para más detalles consultar
%~\cite[p. 97]{Halmo},~\cite[p. 338]{Haus} o~\cite[p. 14]{Hewit}.
%\end{rem}
%
%Una \textbf{clase disjunta} es una clase {\boldmath $A$} de
%conjuntos tal que para cualquier par de conjuntos distintos de
%{\boldmath $A$} son disjuntos, en este caso nos referiremos a la
%unión de conjuntos de {\boldmath $A$} como \textbf{unión disjunta}.
%
%Si $E$ es un subconjunto de $X$, la función $\chi _E$ definida para
%$x\in X$ por la relación:
%\begin{equation}\label{eq0}
%    \chi_E(x)= \begin{cases}
%    1, & \mbox{si } x\in E, \\
%    0, & \mbox{si } x\notin E.
%    \end{cases}
%\end{equation}Es llamada \textbf{función característica}
%del conjunto $E$. La correspondencia entre los conjuntos y sus
%funciones características es inyectiva, y todas las propiedades de
%conjuntos y operaciones entre conjuntos pueden ser expresadas por
%medio de funciones características. 
%
%%%%	TABLA CORTA, ÚNICAMENTE LLEVA LÍNEAS HORIZONTALES PARA SEPARAR BLOQUES
%
%\begin{table}[ht]
%\caption[título optativo de la tabla]{Propiedades de los espacios $L^p$. Fuente: tomada de \cite[Cap. 3]{Brez}.}\label{tablaLp}
%\centering
%\begin{tabular}{cccc}
%\hline
%Espacio & Reflexivo\footnote{En el sentido topológico.} & Separable & Dual \\ \hline %\hline
%$L^p$, $1<p<\infty$  & Si & Si & $L^q$, $1/p+1/q=1$ \\
%$L^1$ & No & Si & $L^\infty$ \\
%$L^\infty$ & No & No & $L^1 \varsubsetneq (L^\infty)’$ \\ \hline
%\end{tabular}
%\end{table}\footnotetext{En el sentido topológico.} 
%
%\begin{defn}\label{dfcp1}Si $\su En$ es una sucesión de
%conjuntos, definiremos los conjuntos $\overline{\lim } E_n$ y
%$\underline{\lim } E_n$ de la siguiente forma:
%\[\begin{array}{cc}
%  \overline{\lim} E_n=\limsup \limits_{n\rightarrow \infty}E_n =
%  \bigcap \limits_{n=1}^\infty \Union Ein \infty ,&
%  \underline{\lim} E_n=\liminf \limits_{n\rightarrow \infty}E_n =
%  \bigcup \limits_{n=1}^\infty \Inter Ein \infty \\
%\end{array}\] y los llamaremos \textbf{límite superior} y
%\textbf{límite inferior}, respectivamente, de la sucesión $\su En$.
%Si tenemos $\overline{\lim} E_n = \underline{\lim} E_n$, usaremos la
%notación $\lim_n E_n$ para este conjunto. Si la sucesión es tal que
%$E_n\subset E_{n+1},\ n=1,2,\dots$ le llamaremos \textbf{creciente}
%y se denotará por $E_n\!\!\uparrow$ y su límite será $\lim
%\limits_{n\rightarrow \infty}E_n = \union En1 \infty$; si es tal que
%$E_n\supset E_{n+1},\ n=1,2,\dots$ le llamaremos
%\textbf{decreciente} y se denotará por $E_n\!\!\downarrow$ y su
%límite será $\lim \limits_ {n\rightarrow \infty}E_n = \inter En1
%\infty$. En ambos casos nos referiremos a ella como
%\textbf{monótona}.\end{defn}
%
%
%\begin{defn}\label{dfcp2}Sea $f$ una aplicación definida del conjunto
%$X$ al conjunto $Y$, es decir $f:X\To Y$. Para cualquier subconjunto
%$T\subseteq Y$, definimos la \textbf{imagen inversa} de $T$, bajo
%$f$, denotada por $\ff (T)$, como sigue:
%\[\ff (T)=\{s\in X\mid f(s)\in T\}.\]\end{defn} 
%
%\begin{thm}\label{thcp1}Para la aplicación $\ff :{\cal{P}} (Y)
%\To {\cal{P}} (X)$ se tienen las propiedades siguientes:
%\begin{enumerate}
%    \vspace{0pt} \item $\ff(\bigcup_j T_j) = \bigcup_j
%    \ff(T_j)$.
%    \vspace{0pt} \item $\ff(\bigcap_j T_j) = \bigcap_j
%    \ff(T_j)$.
%    \vspace{-6pt} \item Si $T_1\cap T_2=\nada$, entonces
%    $\ff(T_1)\cap \ff(T_2) = \nada$.
%    \vspace{-6pt} \item $\ff(T^c) = [\ff(T)]^c$.
%    \vspace{-6pt} \item $\ff(\nada) = \nada$.
%    \vspace{-6pt} \item $\ff(Y) = X$.
%\end{enumerate}
%\end{thm}
%
%
%Las propiedades (1) y (3) del teorema~\ref{thcp1} establecen las
%condiciones para la unión disjunta en una familia en $Y$. Sea ahora
%$\Df$ una familia cualquiera de subconjuntos de $Y$, y definamos la
%familia $\ff (\Df)$ de subconjuntos de $X$ como
%sigue:\begin{equation}\label{eq1}
%    \ff(\Df) = \{A\subseteq X \mid
%A = \ff(T) \mbox{ para algún } T\in \Df\} = \{\ff(T) \mid T\in
%\Df\}.
%\end{equation}
%
%El sistema de numeros reales extendido o \textbf{recta real
%extendida} es el conjunto definido por $\RR\df\R \cup \{-\infty,
%+\infty\}$, con la siguiente relación de orden: para $a\in \R$
%tenemos $-\infty <a< +\infty$. La topología para este conjunto se
%define por declarar como abiertos a los siguientes conjuntos:
%$(a,b),\ [-\infty,b),\ (a,+\infty]$ y cualquier unión de conjuntos
%de este tipo. Cuando se haga referencia a los \textbf{numeros reales
%extendidos} o \textbf{valor real extendido}, se estará hablando de
%los numeros reales y de los símbolos $\pm\infty$. Cuando trabajamos
%con teoría de la integración, nos encontraremos con $\infty$, una
%razón es que algunas veces trataremos de integrar sobre conjuntos de
%medida infinita, este es caso de la recta real.
%
%Por tal motivo, se hacen las siguientes definiciones para
%facilitar su manejo: $a+\infty = \infty+a \df \infty$ si $0\leq a
%\leq \infty$,~y
%\begin{equation}\label{eq4}
%    a\cdot \infty = \infty \cdot a
%\df\begin{cases}
%  \infty, & \mbox{si } 0 < a \leq \infty \\
%  0, & \mbox{si } a = 0 \\
%\end{cases}
%\end{equation}las leyes de cancelación se tratan
%así: $a+b = a+c\ \Rightarrow\ b=c\ $ y $a\cdot b =a\cdot c\
%\Rightarrow\ b=c$ sólo cuando $0 < a < \infty$. 
%
%\begin{defn}\label{dfcp3}Sea $\su aj$ una sucesión en la recta
%real extendida, y sean $b_k = \sup\{a_k,a_{k+1},a_{k+2}\dots\},\
%k=1,2,3,\dots$, y $\beta = \inf\{b_1,b_2,b_3,\dots\}$. Entonces
%llamaremos a $\beta$ el \textbf{límite superior} de $\su aj$, y
%escribiremos $\beta = \limsup \limits_{j\rightarrow \infty}(a_j)$.
%El \textbf{límite inferior} se define análogamente, al intercambiar
%$\sup$ e $\inf$ en las anteriores definiciones; notemos que \[\liminf
%\limits_{j\rightarrow \infty}(a_j) = -\limsup \limits_{j\rightarrow
%\infty}(-a_j).\] Si $\su aj$ converge, entonces tenemos $\liminf
%\limits_{j\rightarrow \infty}(a_j) = \limsup \limits_{j\rightarrow
%\infty}(a_j) = \lim \limits_{j\rightarrow
%\infty}(a_j)$.\end{defn}
%
%\begin{prp}\label{prcp1}Si $0\leq a_1\leq a_2\leq \cdots$,
%$0\leq b_1\leq b_2\leq \cdots$, tales que $a_j \To a$ y $b_j \To
%b$. Entonces $a_j b_j \To ab$.
%\end{prp}
%
%\begin{defn}\label{dfcp4}Supongamos que $\su fj$ es una sucesión de
%funciones de valor real extendido en un conjunto $X$. Entonces
%$\sup_j f_j$ y $\limsup \limits_{j\rightarrow \infty} f_j$ son las
%funciones definidas en $X$ por:
%\[\left(\sup_j f_j\right)(x)\df \sup_j(f_j(x)),\quad \left(\limsup
%\limits_{j \rightarrow \infty} f_j\right)(x)\df \limsup \limits_{j
%\rightarrow \infty}(f_j(x)).\] Si $f(x)=\lim \limits_{j \rightarrow
%\infty} f_j(x)$, y asumimos que el límite existe para cualquier
%$x\in X$, entonces llamaremos a $f$ el \textbf{límite puntual} de la
%sucesión $\su fj$ y hablaremos de \textbf{convergencia puntual} en
%este contexto.\end{defn}
%
%\begin{figure}[ht]
%\centering
%\label{fig:analisisGraficoModelo3}
%\includegraphics[width=0.9\linewidth]{analisisGraficoModelo3}\\
%\caption[Titulo en el índice de figuras (opcional)]{Título en el 
%documento. Las imágenes pueden ser raster (de preferencia jpg, png 
%con buena resolución para imprimir) o vectorial (convertir a pdf, en 
%este caso la resolución no afecta) Fuente: imagen tomada de~\cite{liu}.}
%\end{figure}
%
%
%\begin{lem}\label{lmcp11}Sean $z,w \in \C$, $1 < p \leq 2$ y $1/p +
%1/q = 1$. Entonces tenemos \[\abs{z+w}^q + \abs{z-w}^q \leq 2 (|z|^p
%+ |w|^p)^{\frac{1}{p-1}}.\]
%\end{lem}
%
%\begin{proof} Consultar~\cite[p. 227]{Hewit}. \end{proof}
%
%\begin{defn}\label{dfcp5}Sea $f:X\To \RR$ una aplicación. Se definen
%las aplicaciones $f^+\!\df\max\{f,0\}$, $f^-\!\df-\min\{f,0\}$, a
%$f^+$ y $f^-$ se les llama la \textbf{parte positiva y negativa} de
%$f$, respectivamente.
%\end{defn}
%
%
%\begin{prp}\label{prcp2}Para cualquier aplicación $f:X\To \R$
%denotaremos su valor absoluto con $\abs f$, entonces tenemos
%\[\abs f=f^++f^-,\quad f=f^+-f^-.\]
%\end{prp}
%
%% --------------->  
%
%\section{Tablas y Gráficas}
%
%Las tablas y gráficas deben tener un título \verb|\caption{text}| que la identifique, debe especificar la \textbf{fuente}, y una etiqueta \verb|\label{text}| para hacer referencias cruzadas dentro del documento.
%
%%% TABLAS LARGAS LLEVAN TODAS LAS DIVISIONES DE LOS BLOQUES
%\subsection{Tablas}
%
%\begin{longtable}{|l|l|l|l|l|}
%\caption[]{Diccionario de datos, tabla \textit{marn} (continuación)} \\ \hline
%
%\multicolumn{1}{|c|}{\textbf{Name}} & \multicolumn{1}{c|}{\textbf{Data type}} & \multicolumn{1}{c|}{\textbf{Not Null?}} & \multicolumn{1}{c|}{\textbf{Primary key?}} & \multicolumn{1}{c|}{\textbf{Default}} \\ \hline \endhead
%	\caption[Diccionario de datos, tabla \textit{marn}]{Diccionario de datos, tabla \textit{marn}. Fuente: obtenida de pgAdminIII}\label{data:marn} \\ \hline
%
%	\multicolumn{1}{|c|}{\textbf{Name}} & \multicolumn{1}{c|}{\textbf{Data type}} & \multicolumn{1}{c|}{\textbf{Not Null?}} & \multicolumn{1}{c|}{\textbf{Primary key?}} & \multicolumn{1}{c|}{\textbf{Default}} \\ \hline \endfirsthead 
%
%	id & \textit{integer} & \textit{Yes} & \textit{Yes} & \textit{nextval('marn\_id\_seq'} \\ %\hline
%
%	 &  &  &  & \textit{::regclass)}\footnote{Note que la tabla es mas ancha que lo preestablecido. Procure diseñar elementos acordes con el espacio preestablecido.} \\ \hline
%
%	\multicolumn{ 5}{|l|}{Clave primaria que  obtendrá su valor de forma secuencial al ingresar un nuevo registro} \\ \hline
%		lista\_tax & \textit{text} & \textit{No} & \textit{No} & \textit{} \\ \hline
%
%	\multicolumn{ 5}{|l|}{Clasificación del proyecto en base al Listado Taxativo del MARN} \\ \hline
%		no\_marn & \textit{text} & \textit{No} & \textit{No} & \textit{} \\ \hline
%
%	\multicolumn{ 5}{|l|}{Numero de expediente asignado por el MARN} \\ \hline
%		date0 & \textit{date} & \textit{No} & \textit{No} & \textit{} \\ \hline
%
%	\multicolumn{ 5}{|l|}{Día del ingreso del expediente del proyecto (instrumento ambiental) en el MARN} \\ \hline
%		notas & \textit{text} & \textit{No} & \textit{No} & \textit{} \\ \hline
%
%	\multicolumn{ 5}{|l|}{Observaciones} \\ \hline
%		no\_res\_ap & \textit{text} & \textit{No} & \textit{No} & \textit{} \\ \hline
%
%	\multicolumn{ 5}{|l|}{Numero de resolución aprobatoria del proyecto por el MARN%
%	\footnote{Note que en esta línea la tabla se corta y continua en la siguiente página. 
%	Utilizar paquete \textsf{longtable} y ambiente \textit{longtable}.}} \\ \hline
%		date\_res\_ap & \textit{date} & \textit{No} & \textit{No} & \textit{} \\ \hline
%
%	\multicolumn{ 5}{|l|}{Día de emisión de la resolución aprobatoria por el MARN} \\ \hline
%		date0\_fianza & \textit{date} & \textit{No} & \textit{No} & \textit{} \\ \hline
%
%	\multicolumn{ 5}{|l|}{Día de emisión de fianza del proyecto.} \\ \hline
%		no\_res\_fianza & \textit{text} & \textit{No} & \textit{No} & \textit{} \\ \hline
%
%	\multicolumn{ 5}{|l|}{Numero de la resolución de aceptación de fianza por el MARN} \\ \hline
%		date1\_fianza & \textit{date} & \textit{No} & \textit{No} & \textit{} \\ \hline
%
%	\multicolumn{ 5}{|l|}{Fecha de inicio de fianza} \\ \hline
%		date2\_fianza & \textit{date} & \textit{No} & \textit{No} & \textit{} \\ \hline
%
%	\multicolumn{ 5}{|l|}{Fecha de finalización de fianza (renovación)} \\ \hline
%		lic\_ambiental & \textit{text} & \textit{No} & \textit{No} & \textit{} \\ \hline
%
%	\multicolumn{ 5}{|l|}{Numero de licencia ambiental} \\ \hline
%		date\_lic\_ambiental & \textit{date} & \textit{No} & \textit{No} & \textit{} \\ \hline
%
%	\multicolumn{ 5}{|l|}{Fecha de finalización de ultima licencia ambiental} \\ \hline
%		proyecto\_id & \textit{integer} & \textit{Yes} & \textit{No} & \textit{} \\ \hline
%
%	\multicolumn{ 5}{|l|}{Enlace con la tabla proyecto\_id} \\ \hline
%\end{longtable}


