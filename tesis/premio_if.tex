%---------------------------------------------------------------------------%

% Letter class
\documentclass[letterpaper, 11pt]{letter}
\usepackage[left=2cm,right=2cm,top=2cm,bottom=2cm]{geometry}
\usepackage[spanish]{babel}
\usepackage[utf8]{inputenc}
\usepackage{hyperref}
\hypersetup{
    colorlinks=true,
    linkcolor=blue,
    filecolor=magenta,      
    urlcolor=blue,
    pdftitle={Overleaf Example},
    pdfpagemode=FullScreen,
    }

% Name of sender
\name{Joe Fox}

% Signature of sender
%\signature{José Alfredo de León}
\signature{
\begin{tabular}{c p{5.1cm} c}
~ & ~ & ~ \\
~ & ~ & ~ \\
José Alfredo de León Garrido & ~ & Vo.Bo. \\
~ & ~ & Dr. Carlos Francisco Pineda Zorrilla\\
\end{tabular}}

%% Address of sender
%\address
%{
%    Fox Books,\\
%    Times Square, NYC.
%}

%-----------------------------------------------------------------------------%
\date{Ciudad de México, \today}
\begin{document}

% Name and address of receiver
\begin{letter}
{}

% Opening statement
\opening{
Comité Evaluador,\\
Premio 2022 ``Juan Manuel Lozano Mejía",\\
}

% Letter body

Reciban un cordial saludo. Me dirijo a ustedes para postularme como candidato
para medalla o diploma al premio 2022 ``Juan Manuel Lozano Mejía'' en el nivel
de licenciatura. El trabajo con el que deseo concursar es mi tesis de 
licenciatura. El trabajo se titula ``Mapeos proyectos en sistemas de varios qubits'' y
estuvo bajo la dirección del Dr. Carlos Francisco Pineda Zorrilla del Insituto
de Física. 

Permítanme elaborar un breve resumen de la tesis.
La motivación de este trabajo fue estudiar, con la teoría 
de los canales cuánticos, aquellos procesos  
de decoherencia y disipación de sistemas de muchos qubits que se pueden describir 
como proyecciones de la matriz de densidad a las matrices de Pauli. 
En el caso de 1 qubit estas transformaciones, que han sido extensamente estudiadas en la literatura, son las proyecciones de la esfera de Bloch a un plano,
a un eje o al origen de coordenadas. 

Para generalizar las proyecciones de la esfera de Bloch
a muchas partículas introdujimos 
los mapeos \textit{Pauli component erasing} (PCE), mapeos que preservan o borran
por completo las componentes de la matriz de densidad en la base de las 
matrices de Pauli. Introdujimos diagramas para representar a los mapeos 
PCE y estudiamos numéricamente 
el subconjunto de los canales cuánticos de 1, 2 y 3 qubits, es decir, 
el subconjunto de los mapeos PCE que son completamente positivos y, por tanto,
describen evoluciones físicas.

Por otro lado, también estudiamos la propuesta de Nathanson y Ruskai de los 
canales diagonales de Pauli constantes sobre los ejes para investigar la 
intersección entre estos canales cuánticos y los canales cuánticos PCE.
Los canales diagonales de Pauli constantes sobre los
ejes son canales que actúan sobre las componentes de la matriz de densidad
de un sistema de qubits, escrita en una base de operadores unitarios que
se definen a partir de bases \textit{mutuamente imparciales} del espacio de Hilbert del sistema. En el manuscrito
discutimos cuál es el subconjunto que se construye de la intersección entre 
los canales cuánticos PCE y los canales diagonales de Pauli constantes sobre 
los ejes. 

Para este trabajo se diseñaron e implementaron varias herramientas computacionales en Mathamatica, que ahora se encuentran disponibles een línea en \href{https://github.com/deleonja/projective_maps}{https://github.com/deleonja/projective\_maps}.

Además, vale la pena resaltar que los resultados que se consiguieron
en este trabajo fueron la piedra angular para la solución analítica 
de los canales cuánticos PCE, en la que continuamos trabajando después.
Sobre estos resultados escribimos un artículo en el cual, dada mi contribución, 
soy el primer autor. Este manuscrito puede encontrarse en línea como \href{https://arxiv.org/abs/2205.05808}{arXiv:2205.05808} y está \href{https://journals.aps.org/pra/accepted/b5079Nf1F601ce2282552ec77ee4339c9ce815341}{aceptado para publicarse en Physical Review A} (adjunto documento probatorio). 

% Closing statement
\closing{Sin más, me despido.}

\end{letter}
\end{document}