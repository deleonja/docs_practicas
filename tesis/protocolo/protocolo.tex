%%
%%	AVISO IMPORTANTE
%%	Formato optimizado para el sistema operativo GNU/Linux 64 bits
%%	usar TexLive 2016 (o superior), http://www.ctan.org/tex-archive/systems/texlive/Images/
%%	usar TeXstudio 2.11, http://texstudio.sourceforge.net/

\documentclass[letterpaper,12pt]{thesisECFM}
\usepackage{macros}

%%	NO OLVIDE INCLUIR FUENTE DE LAS TABLAS Y FIGURAS

% Decomentar para anular recuadros en los hiperenlaces dentro del pdf
% \hypersetup{pdfborder={0 0 0}}

% Teoremas ---------------------------------------------------------
% estos ambientes son para teoremas, lemas, corolarios, otros
% si no los utiliza los puede obviar en su trabajo de graduación
\theoremstyle{plain}
\newtheorem{thm}{Teorema}[section]
\newtheorem{cor}{Corolario}[chapter]
\newtheorem{lem}{Lema}[chapter]
\newtheorem{prp}{Proposición}[chapter]

\theoremstyle{definition}
\newtheorem{exa}{Ejemplo}[chapter]
\newtheorem{defn}{Definición}[chapter]
\newtheorem{axm}{Axioma}[chapter]

\theoremstyle{remark}
\newtheorem{rem}{Nota}[chapter]

% --------------------------------------------------------------------------------------------------
%                    Mi preámbulo usual
% --------------------------------------------------------------------------------------------------

\definecolor{mycolor}{RGB}{255,50,0}

\usepackage{physics}

\usepackage{fancybox}
\usepackage{colortbl}
\usepackage{amsbsy}
\usepackage[draft,inline,nomargin]{fixme} \fxsetup{theme=color}
\FXRegisterAuthor{cp}{acp}{\color{blue}CP}
\FXRegisterAuthor{ja}{aja}{\color{mycolor}JA}
\FXRegisterAuthor{cc}{acc}{\color{Purple}CC}


\usepackage[]{lineno}  %\linenumbers
\setlength\linenumbersep{3pt}

\newcommand{\fref}[1]{fig.~\ref{#1}}  \newcommand{\tref}[1]{table~\ref{#1}}
\newcommand{\Fref}[1]{Fig.~\ref{#1}}  \newcommand{\Tref}[1]{Table~\ref{#1}}
\newcommand{\Cref}[1]{Cuadro~\ref{#1}}

\newcommand{\psii}{\psi_i}
\newcommand{\Pk}[1]{\ket{\psi_{#1} }}
\newcommand{\Pb}[1]{\bra{\psi_{#1} }}
\newcommand{\pk}{\ket{\psi}}
\newcommand{\M}{\mathcal{M}^{(N)}}
\newcommand{\E}{\mathcal{E}}
\newcommand{\Erho}{\mathcal{E}(\rho)}
\newcommand{\1}{\mathds{1}}
\newcommand{\ten}{\otimes}
\newcommand{\h}[1]{\colorbox{Yellow}{#1}}
\newcommand{\hi}{\mathcal{H}}
\newcommand{\txt}[1]{\text{#1}}
\newcommand{\here}{\h{\hspace{15cm}} }
\newcommand{\rhoi}{\dyad{\psii}{\psii}}
\newcommand{\ind}[2]{{{}^{#1}_{#2}}}
\newcommand{\rc}[1]{r_{#1}}
\newcommand{\pauli}[2]{\sigma_{#1}\otimes\sigma_{#2}}
\newcommand{\esqueleto}[1]{\textcolor{mycolor}{#1}}

% Para que funcione mejor la numeración {{{
% https://tex.stackexchange.com/questions/43648/why-doesnt-lineno-number-a-paragraph-when-it-is-followed-by-an-align-equation
\newcommand*\patchAmsMathEnvironmentForLineno[1]{%
  \expandafter\let\csname old#1\expandafter\endcsname\csname #1\endcsname
  \expandafter\let\csname oldend#1\expandafter\endcsname\csname end#1\endcsname
  \renewenvironment{#1}%
     {\linenomath\csname old#1\endcsname}%
     {\csname oldend#1\endcsname\endlinenomath}}% 
\newcommand*\patchBothAmsMathEnvironmentsForLineno[1]{%
  \patchAmsMathEnvironmentForLineno{#1}%
  \patchAmsMathEnvironmentForLineno{#1*}}%
\AtBeginDocument{%
\patchBothAmsMathEnvironmentsForLineno{equation}%
\patchBothAmsMathEnvironmentsForLineno{align}%
\patchBothAmsMathEnvironmentsForLineno{flalign}%
\patchBothAmsMathEnvironmentsForLineno{alignat}%
\patchBothAmsMathEnvironmentsForLineno{gather}%
\patchBothAmsMathEnvironmentsForLineno{multline}%
}
% }}}	

% --------------------------------------------------------------------------------------------------
% --------------------------------------------------------------------------------------------------


% Operadores y funciones -------------------------------------------
% Los siguientes son ejemplos de comandos definidos por el usuario
% puede borrarlos, únicamente están para mostrar cómo se construyen con LaTeX
\DeclareMathOperator{\Supp}{Supp}       \DeclareMathOperator{\vol}{Vol}%
\DeclareMathOperator{\Rz}{Re}           \DeclareMathOperator{\Iz}{Im}%

\newcommand{\R}{\mathbb{R}}             \newcommand{\Z}{\mathbb{Z}}%
\newcommand{\C}{\mathbb{C}}             \newcommand{\K}{\mathbb{K}}%
\newcommand{\N}{\mathbb{N}}             \newcommand{\Q}{\mathbb{Q}}%
\newcommand{\Af}{\mathfrak{A}}          \newcommand{\Bf}{\mathfrak{B}}%
\newcommand{\Cf}{\mathfrak{C}}          \newcommand{\Df}{\mathfrak{D}}%
\newcommand{\Ff}{\mathfrak{F}}          \newcommand{\Lf}{\mathfrak{L}}%
\newcommand{\Mf}{\mathfrak{M}}          \newcommand{\Sf}{\mathfrak{S}}%
\newcommand{\Hi}{\mathcal{H}}           \newcommand{\Ba}{\mathcal{B}}%
\newcommand{\nada}{\varnothing}         \newcommand{\To}{\longrightarrow}%
\newcommand{\RR}{[-\infty,+\infty]}     \newcommand{\df}{:=}%
\newcommand{\sani}{$\sigma$\nobreakdash-anillo}
\newcommand{\salg}{$\sigma$\nobreakdash-álgebra}
\newcommand{\ff}{f^{-1}}

%%%%%%%%%%%%%%%%%%%%%%%%%%%%%%%%%%%%%%%%%%%%%%%%%%%%%%%%%%%%%%%%%%%%


% Cuerpo de la tesis -----------------------------------------------

\begin{document}

%% Datos generales del trabajo de graduación
\datosThesis%
{2}%						% física 1; matemática 2
{Protocolo de tesis: 
Mapeos proyectivos en sistemas de varios qubits
}%		% Título del trabajo de graduación
{José Alfredo de León Garrido}%			% autor
{M.Sc. Juan Diego Chang y Dr. Carlos Pineda Zorrilla}%			% asesor
{febrero de 2021}		% mes y año de la orden de impresión
{2}							% femenino 1; masculino 2

%% Datos generales del examen general privado
%\examenPrivado%
%{M.Sc. Edgar Anibal Cifuentes Anléu}%	% director ECFM
%{Ing. José Rodolfo Samayoa Dardón}%		% secretario académico
%{Perengano}%		% examinador 1
%{Zutano}%		% examinador 2
%{Fulano 2}%		% examinador 3

{\onehalfspacing	% interlineado 1 1/2

%\OrdenImpresion{ordenImpresion}		% incluye orden de impresión, guardada en pdf

%\Agrade{agradecimientos}			% Agradecimientos

%\Dedica{dedicatoria}				% Dedicatoria

\par}
 
\frontmatter    % --------------------------------------------------  Hojas preliminares

{\onehalfspacing	% interlineado 1 1/2

\tableofcontents    % Índice general vinculado

%%% \figurasYtablas{ lista_figuras }{ lista_tablas }; con valor 1 se incluye la lista,
%%% cualquier otro valor no la genera
%\figurasYtablas{1}{1}

%%%% INCLUYA LA SIMBOLOGÍA NECESARIA EN ESTE APARTADO
%%% NO CAMBIAR LA DEFINICIÓN DE LA TABLA LARGA


\chapter{LISTA DE SÍMBOLOS}

\begin{longtable}{@{}l@{\extracolsep{\fill}} p{4.75in} @{}}  %%%	NO CAMBIAR ESTA LÍNEA
  \textsf{Símbolo} & \textsf{Significado}\\[12pt]
  \endhead
  $\ket{\psi}$ &  \textit{ket}, vector de estado en la notación de Dirac \\
  $\bra{\psi}$ & \textit{bra}, funcional en la notación de Dirac\\
  $p_i$ & probabilidad $i$-ésima\\
  $\qty{p_i,\ket{¸\psi_i}}$ & ensamble de estados \\
  $\Lambda$ &  operador que actúa sobre el espacio de Hilber\\
  $\expval{\Lambda}$ & valor esperado del operador $\Lambda$ en la notación de Dirac\\
  $\matrixel{\psi_i}{\Lambda}{\psi_j}$ & elemento de matriz $\Lambda_{ij}$\\
  $\braket{\psi}{\phi}$ & \textit{braket}, producto interno en la notación de Dirac\\
  $\dyad{\psi}{\phi}$ & producto externo entre $\ket{\psi}$ y $\ket{\phi}$ en la notación de Dirac \\
  $\rho$ & matriz de densidad\\
  $\Tr$ & traza \\
	$\E$ & canal cuántico \\
	$U$ & operador unitario \\
	$\otimes$ & producto tensorial\\
	$\ket{\psi}\ket{\phi}$ & producto tensorial $\ket{\psi}\otimes\ket{\phi}$\\
	$\1$ & operador identidad \\
	$\sigma_i$ & matrices de Pauli\\
	$\mapsto$ & ``se mapea a''\\
	$\vec\rho$ & matriz de densidad vectorizada\\
	$\mathcal{M}_d$ & espacio de las matrices de $d\times d$\\
	$\mathcal{HS}$ & espacio de Hilbert-Schmidt\\
	$\Lambda^{\dagger}$ & operador adjunto de $\Lambda$\\
	$\delta_{ij}$ & delta de Kronecker\\
	$\taus$ & elementos de la diagonal de un superoperdador PCE en la base de Pauli\\
	$\lambda_i$ & eigenvalores
\end{longtable}
\janote{aquí hay que completar a mano}
  % Lista de símbolos

%%% Haga el diseño que más le guste
\chapter{OBJETIVOS}
\section*{General}
Estudiar los mapeos de borrado de componentes de Pauli (PCE por
sus siglas en inglés) en sistemas de 2 y 3 qubits.


\section*{Específicos}

\begin{enumerate}
\item Estudiar numéricamente la completa positividad de los mapeos 
PCE en sistemas de 2 y 3 qubits.

\item Estudiar las características de los canales PCE.

\item Comparar los canales PCE con otros canales de Pauli que han 
sido previamente estudiados.

\item Desarrollar una herramienta geométrica para entender los 
mapeos PCE.
\end{enumerate}

      % Resumen y objetivos

%%% Haga el diseño que más le guste
\chapter{INTRODUCCIÓN}
\esqueleto{Ningún sistema cuántico en la vida real es un sistema completamente
aislado del resto del universo. En realidad, todos los sistemas cuánticos interactúan,
en menor o mayor grado, con un sistema cuántico externo que se conoce 
como entorno.}


\esqueleto{La decoherencia es un proceso al que irremediablemente están 
sujetos los sistemas cuánticos abiertos. La decoherencia es el colapso de
la superposición de estados de un sistema a sólo uno de los estados de la 
superposición.}


\esqueleto{La teoría de los canales cuánticos en un marco conceptual que puede  
capturar la dinámica de los sistemas abiertos. Un canal cuántico
es una operación que modela un Hamiltoniano o un circuito cuántico. Así mismo, 
el proceso de decoherencia puede modelarse con canales cuánticos.}

\esqueleto{Los sistemas de dos niveles son por excelencia los sistemas cuánticos
más sencillos. Esto los hace los sistemas perfectos para comenzar a estudiar
herramientas para describir la decoherencia. Para 1 qubit, la decoherencia
puede modelarse con canales cuánticos de 1 qubit que han sido ampliamente 
estudiados en el pasado.}

\esqueleto{En este trabajo nuestro objetivo es estudia los canales cuánticos 
que modelan la decoherencia de sistemas de $n$ qubits.}
      % Introducción

\mainmatter     % --------------------------------------------------  Cuerpo del Trabajo de Graduación

\chapter{METODOLOGÍA}
% \esqueleto{
% \begin{itemize}
% \item Hacer un recordatorio del trabajo de prácticas porque es la base 
% teórica de este trabajo
% \item Método numérico para 2 y 3 qubits
% \item Análisis los resultados del numérico
% \item Comparación con los mapeos de Ruskai
% \item Trabajo futuro
% \end{itemize}
% }
\cpnote{Creo que esta seccion está mal. Como esta es lo mismo que la siguiente. 
Yo creo que acá mas bien se deben discutir los métodos que usaras. 
Porfa aclarame eso. }

El primer capítulo contendrá las bases teóricas necesarias para 
el estudio de las operaciones PCE. Se expondrán de manera puntual el 
formalismo de la matriz de densidad y la teoría de los canales cuánticos.
Se utilizará como referencias bibliografías libros especializados en 
el tema: Sakurai \cite{sakurai_napolitano_2017}, 
el texto introductorio estándar para información y computación cuántica de 
Nielsen y Chuang \cite{nielsen_chuang_2011}, Bengtsson 
\cite{bengtsson_zyczkowski_2017} y Preskill \cite{preskill1998lecture}.

En el segundo capítulo se definirán las operaciones PCE, el caso de 
1 qubit y se establecerá el problema para sistemas de $n$ qubits.
Para este capítulo se utilizarán los resultados y el estudio realizado 
durante el trabajo de práctica final, se hará un resumen con los 
aspectos más relevantes ya que son la base de este trabajo. 
Finalmente, en este capítulo se discutirá el uso del método numérico, 
que fue diseñado en la práctica final, para evaluar los casos de 2 y 3 qubits
que son el objetivo de este trabajo.

En el tercer capítulo se presentarán los resultados de 2 y 3 qubits. Se 
analizarán y discutirán los resultados, al mismo tiempo que se desarrollará 
una herramienta geométrica que permita entender de manera sencilla 
los canales cuánticos PCE de 2 y 3 qubits.

En el cuarto capítulo se discutirá la relación de las operaciones PCE con 
los canales diagonales de Pauli constantes sobre los ejes 
\cite{nathanson2007pauli}. Lo que buscamos es saber si los canales cuánticos
PCE están contenidos dentro del conjunto de los canales cuánticos que
estudian Nathanson y Ruskai. 
\cpnote{Yo escribiría porque nos interesa discutir eso, ponlo igual en la siguiente seccion}


      % Cap. 1 

\chapter{OPERACIONES PCE}

\section{Introducción}

\section{Operaciones PCE}

\section{1 qubit}

\section{El problema de $\mathbf{n}$ qubits}

\section{Solución numérica}
      % Cap. 2 

\chapter{CONTENIDOS}

\section*{LISTA DE FIGURAS}

\section*{LISTA DE TABLAS}

\section*{LISTA DE SÍMBOLOS}

\section*{OBJETIVOS}

\section*{INTRODUCCIÓN}

\section*{1 FUNDAMENTOS TEÓRICOS}
\begin{itemize}
\item[1,1] Introducción
\item[1.2] Ensambles de estados cuánticos
\item[1.3] Propiedades de la matriz de densidad
\item[1.4] Canales cuánticos 
\item[1.5] Representaciones de los canales cuánticos
\end{itemize}

\section*{2 OPERACIONES PCE}
\begin{itemize}
\item[2.1] Introducción
\item[2.2] Operaciones PCE
\item[2.3] 1 qubit
\item[2.4] El problema de $n$ qubits
\item[2.5] Solución numérica
\end{itemize}

\section*{3 RESULTADOS DE 2 Y 3 QUBITS}
\begin{itemize}
\item[3.1] Introducción
\item[3.2] Resultados
\item[3.3] Una representación geométrica \janote{Los tableroides}
\item[3.4] Discusión de resultados
\end{itemize}

\section*{4 CANALES DIAGONALES DE PAULI CONSTANTES SOBRE
LOS EJES}
\begin{itemize}
\item[4.1] Introducción
\item[4.2] Canales diagonales de Pauli constantes sobre los ejes
\item[4.3] Relación con los canales cuánticos PCE
\end{itemize}

\section*{CONCLUSIONES Y TRABAJO FUTURO}      % Cap. 3 

{\backmatter     %	Capítulos no van numerados --------------------------------------------------  Apartados finales

%%%% INCLUYA SUS CONCLUSIONES Y RECOMENDACIONES


\chapter{CONCLUSIONES}
\cpnote{Primero, la introduccion de un nuevo tipo de canales que generalizan 
las decoherencias basicas de un qubit. A esto tienen que ir dos frases, y también 
recoerdando al lector que son los PCEs}
La contribución de este trabajo de tesis fue aportar pruebas numéricas
de que los canales cuánticos PCE de $n$ qubits tienen una caracterización propia 
y abrió la posibilidad a preguntas más fundamentales acerca de este 
tipo de canales cuánticos.
Nuestros resultados numéricos de la búsqueda de canales PCE de 2 y 3 qubits 
muestran principalmente dos cosas: (1) que los canales cuánticos PCE pueden 
ordenarse en clases de equivalencia, lo que reduce el número de canales cuánticos
que no están conectados vía operaciones unitarias, y (2) que las operaciones
PCE que satisfacen la condición de completa positividad obedecen reglas muy
específicas. 
\cpnote{Plantea las cosas un poco mas generales. Es decir, puedes hacer un planteamiento
general que contextualice la importancia de tus resultados en general, y que luego 
se aplican a los PCEs. Itntenta plantear un poco las cosas como en la ultima frase de 
este parrafo y luego dices en particular lo qeu hacemos. Como por cada frase de aca
pon una frase que la anteceda tipo la ultima. Quizá vale la pena platicar de esto, 
pareciera confuso}
Por un lado, 
la \textit{regla $2^k$} establece que el número de 
componentes de Pauli invariantes por un canal PCE debe ser una potencia de 2. 
Por otro lado, la \textit{regla espejo} establece que debe existir el mismo número de 
canales cuánticos PCE que dejan invariantes $2^k$ y $2^{2n-k}$ componentes 
de Pauli. Además, probamos que los canales cuánticos PCE no son un subconjunto
de otro tipo de canales cuánticos que se han estudiado antes. Todo esto, la clasificación,
propiedades y la no contención dentro de otro conjunto de canales cuánticos, 
abre la posibilidad y justifica preguntarse si existe una estructura matemática bien 
definida para los canales cuánticos PCE.

\chapter{RECOMENDACIONES}
\begin{enumerate}
	\item Recomendación 1.
	\item Recomendación 2.
	\item Recomendación 3.
\end{enumerate}
     % Conclusiones y recomendaciones

\bibliographystyle{abbrv}
\bibliography{references}   % Bibliografía

}

% Descomentar en el caso de necesitar incluir apéndices
%\appendix			% Apéndices

%\chapter{METODOLOGÍA}
% \esqueleto{
% \begin{itemize}
% \item Hacer un recordatorio del trabajo de prácticas porque es la base 
% teórica de este trabajo
% \item Método numérico para 2 y 3 qubits
% \item Análisis los resultados del numérico
% \item Comparación con los mapeos de Ruskai
% \item Trabajo futuro
% \end{itemize}
% }
\cpnote{Creo que esta seccion está mal. Como esta es lo mismo que la siguiente. 
Yo creo que acá mas bien se deben discutir los métodos que usaras. 
Porfa aclarame eso. }

El primer capítulo contendrá las bases teóricas necesarias para 
el estudio de las operaciones PCE. Se expondrán de manera puntual el 
formalismo de la matriz de densidad y la teoría de los canales cuánticos.
Se utilizará como referencias bibliografías libros especializados en 
el tema: Sakurai \cite{sakurai_napolitano_2017}, 
el texto introductorio estándar para información y computación cuántica de 
Nielsen y Chuang \cite{nielsen_chuang_2011}, Bengtsson 
\cite{bengtsson_zyczkowski_2017} y Preskill \cite{preskill1998lecture}.

En el segundo capítulo se definirán las operaciones PCE, el caso de 
1 qubit y se establecerá el problema para sistemas de $n$ qubits.
Para este capítulo se utilizarán los resultados y el estudio realizado 
durante el trabajo de práctica final, se hará un resumen con los 
aspectos más relevantes ya que son la base de este trabajo. 
Finalmente, en este capítulo se discutirá el uso del método numérico, 
que fue diseñado en la práctica final, para evaluar los casos de 2 y 3 qubits
que son el objetivo de este trabajo.

En el tercer capítulo se presentarán los resultados de 2 y 3 qubits. Se 
analizarán y discutirán los resultados, al mismo tiempo que se desarrollará 
una herramienta geométrica que permita entender de manera sencilla 
los canales cuánticos PCE de 2 y 3 qubits.

En el cuarto capítulo se discutirá la relación de las operaciones PCE con 
los canales diagonales de Pauli constantes sobre los ejes 
\cite{nathanson2007pauli}. Lo que buscamos es saber si los canales cuánticos
PCE están contenidos dentro del conjunto de los canales cuánticos que
estudian Nathanson y Ruskai. 
\cpnote{Yo escribiría porque nos interesa discutir eso, ponlo igual en la siguiente seccion}




\par}               % termina interlineado 1 1/2

\end{document}