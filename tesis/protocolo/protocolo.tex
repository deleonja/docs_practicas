%%
%%	AVISO IMPORTANTE
%%	Formato optimizado para el sistema operativo GNU/Linux 64 bits
%%	usar TexLive 2016 (o superior), http://www.ctan.org/tex-archive/systems/texlive/Images/
%%	usar TeXstudio 2.11, http://texstudio.sourceforge.net/

\documentclass[letterpaper,12pt]{thesisECFM}
\usepackage{macros}

%%	NO OLVIDE INCLUIR FUENTE DE LAS TABLAS Y FIGURAS

% Decomentar para anular recuadros en los hiperenlaces dentro del pdf
% \hypersetup{pdfborder={0 0 0}}

% Teoremas ---------------------------------------------------------
% estos ambientes son para teoremas, lemas, corolarios, otros
% si no los utiliza los puede obviar en su trabajo de graduación
\theoremstyle{plain}
\newtheorem{thm}{Teorema}[section]
\newtheorem{cor}{Corolario}[chapter]
\newtheorem{lem}{Lema}[chapter]
\newtheorem{prp}{Proposición}[chapter]

\theoremstyle{definition}
\newtheorem{exa}{Ejemplo}[chapter]
\newtheorem{defn}{Definición}[chapter]
\newtheorem{axm}{Axioma}[chapter]

\theoremstyle{remark}
\newtheorem{rem}{Nota}[chapter]

% --------------------------------------------------------------------------------------------------
%                    Mi preámbulo usual
% --------------------------------------------------------------------------------------------------

\definecolor{mycolor}{RGB}{255,50,0}

\usepackage{physics}

\usepackage{fancybox}
\usepackage{colortbl}
\usepackage{amsbsy}
\usepackage[draft,inline,nomargin]{fixme} \fxsetup{theme=color}
\FXRegisterAuthor{cp}{acp}{\color{blue}CP}
\FXRegisterAuthor{ja}{aja}{\color{mycolor}JA}
\FXRegisterAuthor{cc}{acc}{\color{Purple}CC}


\usepackage[]{lineno}  %\linenumbers
\setlength\linenumbersep{3pt}

\newcommand{\fref}[1]{fig.~\ref{#1}}  \newcommand{\tref}[1]{table~\ref{#1}}
\newcommand{\Fref}[1]{Fig.~\ref{#1}}  \newcommand{\Tref}[1]{Table~\ref{#1}}
\newcommand{\Cref}[1]{Cuadro~\ref{#1}}

\newcommand{\psii}{\psi_i}
\newcommand{\Pk}[1]{\ket{\psi_{#1} }}
\newcommand{\Pb}[1]{\bra{\psi_{#1} }}
\newcommand{\pk}{\ket{\psi}}
\newcommand{\M}{\mathcal{M}^{(N)}}
\newcommand{\E}{\mathcal{E}}
\newcommand{\Erho}{\mathcal{E}(\rho)}
\newcommand{\1}{\mathds{1}}
\newcommand{\ten}{\otimes}
\newcommand{\h}[1]{\colorbox{Yellow}{#1}}
\newcommand{\hi}{\mathcal{H}}
\newcommand{\txt}[1]{\text{#1}}
\newcommand{\here}{\h{\hspace{15cm}} }
\newcommand{\rhoi}{\dyad{\psii}{\psii}}
\newcommand{\ind}[2]{{{}^{#1}_{#2}}}
\newcommand{\rc}[1]{r_{#1}}
\newcommand{\pauli}[2]{\sigma_{#1}\otimes\sigma_{#2}}
\newcommand{\esqueleto}[1]{\textcolor{mycolor}{#1}}

% Para que funcione mejor la numeración {{{
% https://tex.stackexchange.com/questions/43648/why-doesnt-lineno-number-a-paragraph-when-it-is-followed-by-an-align-equation
\newcommand*\patchAmsMathEnvironmentForLineno[1]{%
  \expandafter\let\csname old#1\expandafter\endcsname\csname #1\endcsname
  \expandafter\let\csname oldend#1\expandafter\endcsname\csname end#1\endcsname
  \renewenvironment{#1}%
     {\linenomath\csname old#1\endcsname}%
     {\csname oldend#1\endcsname\endlinenomath}}% 
\newcommand*\patchBothAmsMathEnvironmentsForLineno[1]{%
  \patchAmsMathEnvironmentForLineno{#1}%
  \patchAmsMathEnvironmentForLineno{#1*}}%
\AtBeginDocument{%
\patchBothAmsMathEnvironmentsForLineno{equation}%
\patchBothAmsMathEnvironmentsForLineno{align}%
\patchBothAmsMathEnvironmentsForLineno{flalign}%
\patchBothAmsMathEnvironmentsForLineno{alignat}%
\patchBothAmsMathEnvironmentsForLineno{gather}%
\patchBothAmsMathEnvironmentsForLineno{multline}%
}
% }}}	

% --------------------------------------------------------------------------------------------------
% --------------------------------------------------------------------------------------------------


% Operadores y funciones -------------------------------------------
% Los siguientes son ejemplos de comandos definidos por el usuario
% puede borrarlos, únicamente están para mostrar cómo se construyen con LaTeX
\DeclareMathOperator{\Supp}{Supp}       \DeclareMathOperator{\vol}{Vol}%
\DeclareMathOperator{\Rz}{Re}           \DeclareMathOperator{\Iz}{Im}%

\newcommand{\R}{\mathbb{R}}             \newcommand{\Z}{\mathbb{Z}}%
\newcommand{\C}{\mathbb{C}}             \newcommand{\K}{\mathbb{K}}%
\newcommand{\N}{\mathbb{N}}             \newcommand{\Q}{\mathbb{Q}}%
\newcommand{\Af}{\mathfrak{A}}          \newcommand{\Bf}{\mathfrak{B}}%
\newcommand{\Cf}{\mathfrak{C}}          \newcommand{\Df}{\mathfrak{D}}%
\newcommand{\Ff}{\mathfrak{F}}          \newcommand{\Lf}{\mathfrak{L}}%
\newcommand{\Mf}{\mathfrak{M}}          \newcommand{\Sf}{\mathfrak{S}}%
\newcommand{\Hi}{\mathcal{H}}           \newcommand{\Ba}{\mathcal{B}}%
\newcommand{\nada}{\varnothing}         \newcommand{\To}{\longrightarrow}%
\newcommand{\RR}{[-\infty,+\infty]}     \newcommand{\df}{:=}%
\newcommand{\sani}{$\sigma$\nobreakdash-anillo}
\newcommand{\salg}{$\sigma$\nobreakdash-álgebra}
\newcommand{\ff}{f^{-1}}

%%%%%%%%%%%%%%%%%%%%%%%%%%%%%%%%%%%%%%%%%%%%%%%%%%%%%%%%%%%%%%%%%%%%


% Cuerpo de la tesis -----------------------------------------------

\begin{document}

%% Datos generales del trabajo de graduación
\datosThesis%
{2}%						% física 1; matemática 2
{Protocolo de tesis: 
Mapeos proyectivos en sistemas de varios qubits
}%		% Título del trabajo de graduación
{José Alfredo de León Garrido}%			% autor
{M.Sc. Juan Diego Chang y Dr. Carlos Pineda Zorrilla}%			% asesor
{febrero de 2021}		% mes y año de la orden de impresión
{2}							% femenino 1; masculino 2

%% Datos generales del examen general privado
%\examenPrivado%
%{M.Sc. Edgar Anibal Cifuentes Anléu}%	% director ECFM
%{Ing. José Rodolfo Samayoa Dardón}%		% secretario académico
%{Perengano}%		% examinador 1
%{Zutano}%		% examinador 2
%{Fulano 2}%		% examinador 3

{\onehalfspacing	% interlineado 1 1/2

%\OrdenImpresion{ordenImpresion}		% incluye orden de impresión, guardada en pdf

%\Agrade{agradecimientos}			% Agradecimientos

%\Dedica{dedicatoria}				% Dedicatoria

\par}
 
\frontmatter    % --------------------------------------------------  Hojas preliminares

{\onehalfspacing	% interlineado 1 1/2

\tableofcontents    % Índice general vinculado

%%% \figurasYtablas{ lista_figuras }{ lista_tablas }; con valor 1 se incluye la lista,
%%% cualquier otro valor no la genera
%\figurasYtablas{1}{1}

%%%% INCLUYA LA SIMBOLOGÍA NECESARIA EN ESTE APARTADO
%%% NO CAMBIAR LA DEFINICIÓN DE LA TABLA LARGA


\chapter{LISTA DE SÍMBOLOS}

\begin{longtable}{@{}l@{\extracolsep{\fill}} p{4.75in} @{}}  %%%	NO CAMBIAR ESTA LÍNEA
  \textsf{Símbolo} & \textsf{Significado}\\[12pt]
  \endhead
  $\ket{\psi}$ &  \textit{ket}, vector de estado en la notación de Dirac \\
  $\bra{\psi}$ & \textit{bra}, funcional en la notación de Dirac\\
  $\qty{p_i,\ket{\psi_i}}$ & ensamble de estados \\
  $p_i$ & $i$-ésima probabilidad\\
  $\braket{\psi}{\phi}$ & \textit{braket}, producto interno en la notación de Dirac\\
  $\Lambda$ &  operador que actúa sobre el espacio de Hilbert\\
  $\Lambda^{\dagger}$ & operador adjunto de $\Lambda$\\
  $\expval{\Lambda}$ & valor esperado de $\Lambda$\\
  $\matrixel{\psi_i}{\Lambda}{\psi_j}$ & elemento de matriz $\Lambda_{ij}$\\
  $\dyad{\psi}{\phi}$ & producto externo entre $\ket{\psi}$ y $\ket{\phi}$ en la notación de Dirac \\
  $\rho$ & matriz densidad\\
  $\rho^{AB}$ & matriz densidad de un sistema compuesto $A$ y $B$\\
  $\Tr \Lambda$ & traza de $\Lambda$\\
  $\Tr_{A}\rho^{AB}$ & traza parcial sobre $A$ de $\rho^{AB}$\\
  	$\1$ & operador identidad \\
	$\E$ & canal cuántico \\
	$D_{\E}$ & matriz de Choi de $\E$\\
	$U$ & operador unitario \\
	$\sigma_i$ & matrices de Pauli\\
	$\ket{\psi}\ket{\phi}$ & producto tensorial $\ket{\psi}\otimes\ket{\phi}$\\
	$\mapsto$ & ``se mapea a''\\
	$\vec\rho$ & matriz densidad vectorizada\\
	$\mathcal{H}$ & espacio de Hilbert\\
	$\mathcal{M}_d$ & espacio de las matrices de $d\times d$\\
	$\mathcal{HS}$ & espacio de Hilbert-Schmidt\\
	$\delta_{ij}$ & delta de Kronecker\\
	$r_{j_1,\ldots,j_n}$ & componentes de la matriz densidad de $n$ qubits en la base
	de matrices de Pauli \\
	$\taus$ & elementos diagonales del superoperdador de una operación 
	PCE en la base de Pauli\\
	$\lambda_i$ & eigenvalores
\end{longtable}
  % Lista de símbolos

%%% Haga el diseño que más le guste
\chapter{OBJETIVOS}
\section*{General}
Estudiar las operaciones de borrado de componentes de Pauli (PCE por
su nombre en inglés ``\textit{Pauli-component-erasing} operations'' 
\cpnote{pon de donde proviene la sigla} 
\janote{Ya}
) en sistemas de 2 y 3 qubits.


\section*{Específicos}

\begin{enumerate}
\item Estudiar numéricamente la completa positividad de las operaciones 
PCE en sistemas de 2 y 3 qubits.

\item Estudiar las características de los canales PCE.
\cpnote{Acá sería un poco mas especifico. que tienes en mente?}
\janote{De acuerdo a lo que platicamos agregué los siguientes dos 
items. Este item lo voy a borrar.}

\item Estudiar las características que debe satisfacer una operación PCE
para ser un canal cuántico.

\item Estudiar la existencia de subconjuntos de canales cuánticos PCE
cuyos elementos sean equivalentes.

\item Desarrollar una herramienta geométrica para entender las
operaciones PCE.

\item Comparar los canales cuánticos PCE con otros canales de Pauli que han 
sido previamente estudiados.
\end{enumerate}

      % Resumen y objetivos

%%% Haga el diseño que más le guste
\chapter{INTRODUCCIÓN}
%\esqueleto{Sistemas abiertos}

Una descripción completa \cpnote{exacta?} de un sistema cuántico requiere
\cpnote{con frecuencia?}
incluir la interacción
con su entorno, es decir, considerar a los sistemas como 
abiertos~\cite{breuer2002theory}. Ningún sistema cuántico en la
naturaleza está completamente aislado del resto del universo\cpnote{Es repetitiva
esta frase con respecto a la primera. Es basicamente la misma idea}.  
Por ejemplo, para describir con generalidad a un átomo en una red óptica, se debe
considerar que el átomo se encuentra inicialmente en un estado \textit{compartido} 
\cpnote{que es un estado compartido??}
con los demás átomos de la red~\cite{pepino2011open}. En consecuencia,
la evolución de este tipo de sistemas no es unitaria, en general, como 
la de los sistemas ideales que no interactúan con su entorno~\cite{preskill1998lecture}. 
\cpnote{Argh aca hay impresiciones.}
En ese sentido, los canales cuánticos proporcionan una herramienta que captura 
la no unitariedad de la dinámica de los sistemas abiertos~\cite{nielsen_chuang_2011}.

%\esqueleto{Qubits y decoherencia de 1 qubit}

La decoherencia es un proceso al que irremediablemente están sujetos los 
sistemas cuánticos abiertos. \cpnote{En el fondo, cual es el mensaje que 
quieres dar en este parrafo? Eso define la primera frase del parrafo. Como veo, 
no esta aun bien pensado}
Este fenómeno es el proceso mediante 
el cual la superposición de estados en el que se encuentra un sistema colapsa a 
sólo uno de los estados de la superposición (pierde su coherencia cuántica)
a causa de la interacción con su entorno~\cite{breuer2002theory}. 
\cpnote{Esta no es la definicion mas general de decoherencia. La podemos discutir.}
Los sistemas de dos niveles son 
los más sencillos y con mucho interés teórico como para estudiar 
la decoherencia de este tipo de sistemas. 
\cpnote{Finalmente de que setrata este parrafo? Mejor vemos primero este parrafo y luego 
sigo leyendo la intro.}
En información y computación 
cuántica se conoce a estos sistemas como qubits, y son de gran importancia 
porque son el análogo cuántico de los bits clásicos en 
la implementación de la computación cuántica~\cite{nielsen_chuang_2011}. 
Algunos ejemplos de sistemas físicos que implementan a un qubit son el espín
del electrón o la polarización de un fotón. 
Existe un tipo de decoherencia de 1 qubit que se puede entender como 
el colapso de su estado cuántico $\ket{\psi}$ a alguno de los dos eigenestados
del operador de espín en la dirección \textit{z}.

%\esqueleto{Operaciones PCE}

Nuestro interés se enfoca en entender la generalización
para sistemas de $n$ qubits de las operaciones que modelan el proceso de 
decoherencia de 1 qubit. Para esto, introduciremos la definición de una 
operación que borra las componentes de Pauli, PCE por sus siglas en inglés
(\textit{Pauli component erasing}). Una operación PCE es una operación lineal 
que preserva o borra por completo las proyecciones de la matriz de densidad de $n$ qubits
sobre la base de productos tensoriales de las matrices de Pauli. Vamos a investigar 
las características en común del subconjunto de las operaciones PCE que son 
completamente positivas y, por consiguiente, canales cuánticos que describen
diferentes tipos de decoherencia de un sistema de $n$ qubits. Vamos a 
discutir cómo nuestros resultados muestran que este tipo particular de 
canales cuánticos, los \textit{canales cuánticos PCE}, podrían poseer 
una estructura matemática. 
      % Introducción

\mainmatter     % --------------------------------------------------  Cuerpo del Trabajo de Graduación

\chapter{METODOLOGÍA}
% \esqueleto{
% \begin{itemize}
% \item Hacer un recordatorio del trabajo de prácticas porque es la base 
% teórica de este trabajo
% \item Método numérico para 2 y 3 qubits
% \item Análisis los resultados del numérico
% \item Comparación con los mapeos de Ruskai
% \item Trabajo futuro
% \end{itemize}
% }
\cpnote{Creo que esta seccion está mal. Como esta es lo mismo que la siguiente. 
Yo creo que acá mas bien se deben discutir los métodos que usaras. 
Porfa aclarame eso. }

El primer capítulo contendrá las bases teóricas necesarias para 
el estudio de las operaciones PCE. Se expondrán de manera puntual el 
formalismo de la matriz de densidad y la teoría de los canales cuánticos.
Se utilizará como referencias bibliografías libros especializados en 
el tema: Sakurai \cite{sakurai_napolitano_2017}, 
el texto introductorio estándar para información y computación cuántica de 
Nielsen y Chuang \cite{nielsen_chuang_2011}, Bengtsson 
\cite{bengtsson_zyczkowski_2017} y Preskill \cite{preskill1998lecture}.

En el segundo capítulo se definirán las operaciones PCE, el caso de 
1 qubit y se establecerá el problema para sistemas de $n$ qubits.
Para este capítulo se utilizarán los resultados y el estudio realizado 
durante el trabajo de práctica final, se hará un resumen con los 
aspectos más relevantes ya que son la base de este trabajo. 
Finalmente, en este capítulo se discutirá el uso del método numérico, 
que fue diseñado en la práctica final, para evaluar los casos de 2 y 3 qubits
que son el objetivo de este trabajo.

En el tercer capítulo se presentarán los resultados de 2 y 3 qubits. Se 
analizarán y discutirán los resultados, al mismo tiempo que se desarrollará 
una herramienta geométrica que permita entender de manera sencilla 
los canales cuánticos PCE de 2 y 3 qubits.

En el cuarto capítulo se discutirá la relación de las operaciones PCE con 
los canales diagonales de Pauli constantes sobre los ejes 
\cite{nathanson2007pauli}. Lo que buscamos es saber si los canales cuánticos
PCE están contenidos dentro del conjunto de los canales cuánticos que
estudian Nathanson y Ruskai. 
\cpnote{Yo escribiría porque nos interesa discutir eso, ponlo igual en la siguiente seccion}


      % Cap. 1 

\chapter{DESCRIPCIÓN DE LOS CAPÍTULOS}
% \esqueleto{
% \begin{itemize}
% \item Cap 1: Fundamentos teóricos (formalismo de la matriz de densidad
% y canales cuánticos) 
% \item Cap 2: Mapeos de borrado de componentes de Pauli 
% \item Cap 3: Resultados 2 y 3 qubits
% \item Cap 4: Canalés cuánticos de Pauli constantes sobre los ejes
% \end{itemize}
% }
% \janote{El capítulo 4 será cortito: exposición de los mapeos de Ruskai y
% el argumento que tenemos para refutar que los PCE sean un subconjunto.}

\section*{Capítulo 1: Fundamentos teóricos}
En este capítulo se definirán las matriz de densidad y los canales cuánticos. 
Se introducirá la matriz de densidad como la nueva herramienta para 
describir a los estados cuánticos y se expondrá la reformulación de 
los postulados de la mecánica cuántica utilizando este nuevo formalismo.
Se introducirá la teoría de los canales cuánticos. Dedicaremos especial 
atención a discutir la condición de completa positividad y presentaremos 
la representación de superoperador y de Kraus de un canal cuántico.
Este capítulo estará basado en las referencias \cite{bengtsson_zyczkowski_2017},
\cite{nielsen_chuang_2011} y \cite{sakurai_napolitano_2017}.

\section*{Capítulo 2: Operaciones PCE}
Se definirán las operaciones PCE y se establecerá el problema de 
estudio para sistemas de $n$ qubits. Se hará una revisión del caso 
de 1 qubit para ofrecer intuición que será útil para entender el 
problema de más qubits. Por último, se discutirá la forma de 
abordar numéricamente el problema de 2 y 3 qubits. Para este
capítulo se utilizará como referencia el informe final de prácticas.

\section*{Capítulo 3: Resultados de 2 y 3 qubits}
En este capítulo se presentarán los resultados numéricos del 
caso de 2 y 3 qubits. Se desarrollará una herramienta geométrica 
para entender y analizar los resultados. Se buscará extraer de 
estos resultados la caracterización general de los canales cuánticos PCE,
o bien, se buscarán las pistas que conduzcan al camino de esta caracterización.

\section*{Capítulo 4: Canales diagonales de Pauli constantes 
sobre los ejes}
En este capítulo se presentarán los canales diagonales de Pauli constantes 
sobre los ejes \cite{nathanson2007pauli} y se estudiará si los canales
cuánticos PCE son un subconjunto de ellos. El objetivo es buscar una 
prueba definitiva para demostrar si el conjunto de los canales 
cuánticos PCE están contenidos, si existe una intersección o 
si son conjuntos excluyentes con los canales diagonales de Pauli.
Con esto queremos explorar si los mapeos de nuestro estudio 
son un caso particular de otros canales cuánticos de Pauli que 
han sido estudiados previamente o si los canales cuánticos PCE 
son canales cuánticos más generales.
      % Cap. 2 

\chapter{CONTENIDOS}

\section*{LISTA DE FIGURAS}

\section*{LISTA DE TABLAS}

\section*{LISTA DE SÍMBOLOS}

\section*{OBJETIVOS}

\section*{INTRODUCCIÓN}

\section*{1 FUNDAMENTOS TEÓRICOS}
\begin{itemize}
\item[1,1] Introducción
\item[1.2] Ensambles de estados cuánticos
\item[1.3] Propiedades de la matriz de densidad
\item[1.4] Canales cuánticos 
\item[1.5] Representaciones de los canales cuánticos
\end{itemize}

\section*{2 OPERACIONES PCE}
\begin{itemize}
\item[2.1] Introducción
\item[2.2] Operaciones PCE
\item[2.3] 1 qubit
\item[2.4] El problema de $n$ qubits
\item[2.5] Solución numérica
\end{itemize}

\section*{3 RESULTADOS DE 2 Y 3 QUBITS}
\begin{itemize}
\item[3.1] Introducción
\item[3.2] Resultados
\item[3.3] Una representación geométrica 
\item[3.4] Discusión de resultados
\end{itemize}

\section*{4 CANALES DIAGONALES DE PAULI CONSTANTES SOBRE
LOS EJES}
\begin{itemize}
\item[4.1] Introducción
\item[4.2] Canales diagonales de Pauli constantes sobre los ejes
\item[4.3] Relación con los canales cuánticos PCE
\end{itemize}

\section*{CONCLUSIONES Y TRABAJO FUTURO}      % Cap. 3 

{\backmatter     %	Capítulos no van numerados --------------------------------------------------  Apartados finales

%%%% INCLUYA SUS CONCLUSIONES Y RECOMENDACIONES


\chapter{CONCLUSIONES}
En este trabajo de tesis propusimos estudiar un nuevo tipo de operaciones 
que generalizan el proceso de decoherencia cuántica para sistemas de $n$ qubits 
(sistemas de dos niveles). Para 1 qubit, el proceso de decoherencia
puede describirse por medio de canales cuánticos bien conocidos como 
el \textit{bit-flip} (inversor de bit), operación que proyecta el estado del 
sistema a alguno de los eigenestados de $\sigma_z$
\cite{bengtsson_zyczkowski_2017,nathanson2007pauli,nielsen_chuang_2011}. 
Para $n$ qubits, 
introdujimos la definición de una operación que borra las componentes de Pauli
(PCE por sus siglas en ingles, \textit{Pauli component erasing}) como una 
operación diagonal de Pauli que preserva o borra por completo 
las proyecciones de la matriz de densidad sobre los elementos de la
base de productos tensoriales de las matrices de Pauli (componentes de Pauli). 
Diseñamos un método 
númerico para implementar la búsqueda de los \textit{canales cuánticos PCE},
\textit{i.e.} operaciones PCE que son completamente positivas, y nuestros 
resultados muestran evidencia que estos canales cuánticos podrían tener
una estructura matemática propia. 
Por un lado, los canales cuánticos PCE pueden clasificarse en clases de equivalencia, 
es decir, subconjuntos dentro de los cuales todos los elementos están 
conectados vía operaciones unitarias. 
Además, se pueden identificar que las operaciones PCE 
que satisfacen la completa positividad obedecen dos reglas, (1) \textit{regla
$\mathit{2^k}$}: preservan una cantidad de componentes de Pauli
que es una potencia de dos, y (2) \textit{regla espejo}:
existe la misma cantidad de canales cuánticos PCE que preservan $2^k$ 
y $2^{2n-k}$ componentes de Pauli.
Además, probamos que los canales cuánticos PCE no son un subconjunto
de otro tipo de canales cuánticos que se han estudiado antes 
\cite{nathanson2007pauli}. 
En resumen, este trabajo de tesis aporta pruebas numéricas
de que los canales cuánticos PCE
poseen una caracterización 
propia, y abre la posibilidad a preguntas más fundamentales acerca de este 
tipo de canales cuánticos.
\cpnote{Esto lo pondria mas arriba. Siento que acá estorba: (una generalización de los canales cuánticos
que describen la decoherencia
para sistemas de $n$ qubits}

% \cpnote{Primero, la introduccion de un nuevo tipo de canales que generalizan 
% las decoherencias basicas de un qubit. A esto tienen que ir dos frases, y también 
% recoerdando al lector que son los PCEs}
% \cpnote{Plantea las cosas un poco mas generales. Es decir, puedes hacer un planteamiento
% general que contextualice la importancia de tus resultados en general, y que luego 
% se aplican a los PCEs. Itntenta plantear un poco las cosas como en la ultima frase de 
% este parrafo y luego dices en particular lo qeu hacemos. Como por cada frase de aca
% pon una frase que la anteceda tipo la ultima. Quizá vale la pena platicar de esto, 
% pareciera confuso}
% 
% \noindent
% \esqueleto{Ideas:
% \begin{itemize}
% 	\item Buscamos generalizar las decoherencias básicas de un qubit [?]
% 	\item Con esto introducimos la definición de una operación PCE
% 	\item Implementamos computacionalmente un método numérico para 
% 	encontrar canales cuánticos PCE
% 	\item Encontramos tales propiedades 
% 	\item Encontramos que se pueden clasificar
% \end{itemize}
% }

\chapter{RECOMENDACIONES}
\begin{enumerate}
	\item Recomendación 1.
	\item Recomendación 2.
	\item Recomendación 3.
\end{enumerate}
     % Conclusiones y recomendaciones

\bibliographystyle{abbrv}
\bibliography{references}   % Bibliografía

}

% Descomentar en el caso de necesitar incluir apéndices
%\appendix			% Apéndices

%\chapter{METODOLOGÍA}
% \esqueleto{
% \begin{itemize}
% \item Hacer un recordatorio del trabajo de prácticas porque es la base 
% teórica de este trabajo
% \item Método numérico para 2 y 3 qubits
% \item Análisis los resultados del numérico
% \item Comparación con los mapeos de Ruskai
% \item Trabajo futuro
% \end{itemize}
% }
\cpnote{Creo que esta seccion está mal. Como esta es lo mismo que la siguiente. 
Yo creo que acá mas bien se deben discutir los métodos que usaras. 
Porfa aclarame eso. }

El primer capítulo contendrá las bases teóricas necesarias para 
el estudio de las operaciones PCE. Se expondrán de manera puntual el 
formalismo de la matriz de densidad y la teoría de los canales cuánticos.
Se utilizará como referencias bibliografías libros especializados en 
el tema: Sakurai \cite{sakurai_napolitano_2017}, 
el texto introductorio estándar para información y computación cuántica de 
Nielsen y Chuang \cite{nielsen_chuang_2011}, Bengtsson 
\cite{bengtsson_zyczkowski_2017} y Preskill \cite{preskill1998lecture}.

En el segundo capítulo se definirán las operaciones PCE, el caso de 
1 qubit y se establecerá el problema para sistemas de $n$ qubits.
Para este capítulo se utilizarán los resultados y el estudio realizado 
durante el trabajo de práctica final, se hará un resumen con los 
aspectos más relevantes ya que son la base de este trabajo. 
Finalmente, en este capítulo se discutirá el uso del método numérico, 
que fue diseñado en la práctica final, para evaluar los casos de 2 y 3 qubits
que son el objetivo de este trabajo.

En el tercer capítulo se presentarán los resultados de 2 y 3 qubits. Se 
analizarán y discutirán los resultados, al mismo tiempo que se desarrollará 
una herramienta geométrica que permita entender de manera sencilla 
los canales cuánticos PCE de 2 y 3 qubits.

En el cuarto capítulo se discutirá la relación de las operaciones PCE con 
los canales diagonales de Pauli constantes sobre los ejes 
\cite{nathanson2007pauli}. Lo que buscamos es saber si los canales cuánticos
PCE están contenidos dentro del conjunto de los canales cuánticos que
estudian Nathanson y Ruskai. 
\cpnote{Yo escribiría porque nos interesa discutir eso, ponlo igual en la siguiente seccion}




\par}               % termina interlineado 1 1/2

\end{document}