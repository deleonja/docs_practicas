\chapter{CONTENIDOS}

\section*{LISTA DE FIGURAS}

\section*{LISTA DE TABLAS}

\section*{LISTA DE SÍMBOLOS}

\section*{OBJETIVOS}

\section*{INTRODUCCIÓN}

\section*{1 FUNDAMENTOS TEÓRICOS}
\begin{itemize}
\item[1,1] Introducción
\item[1.2] Ensambles de estados cuánticos
\item[1.3] Propiedades de la matriz de densidad
\item[1.4] Canales cuánticos 
\item[1.5] Representaciones de los canales cuánticos
\end{itemize}

\section*{2 OPERACIONES PCE}
\begin{itemize}
\item[2.1] Introducción
\item[2.2] Operaciones PCE
\item[2.3] 1 qubit
\item[2.4] El problema de $n$ qubits
\item[2.5] Solución numérica
\end{itemize}

\section*{3 RESULTADOS DE 2 Y 3 QUBITS}
\begin{itemize}
\item[3.1] Introducción
\item[3.2] Resultados
\item[3.3] Una representación geométrica 
\item[3.4] Discusión de resultados
\end{itemize}

\section*{4 CANALES DIAGONALES DE PAULI CONSTANTES SOBRE
LOS EJES}
\begin{itemize}
\item[4.1] Introducción
\item[4.2] Canales diagonales de Pauli constantes sobre los ejes
\item[4.3] Relación con los canales cuánticos PCE
\end{itemize}

\section*{CONCLUSIONES Y TRABAJO FUTURO}