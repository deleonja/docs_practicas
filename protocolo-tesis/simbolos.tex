%%% INCLUYA LA SIMBOLOGÍA NECESARIA EN ESTE APARTADO
%%% NO CAMBIAR LA DEFINICIÓN DE LA TABLA LARGA


\chapter{LISTA DE SÍMBOLOS}

\begin{longtable}{@{}l@{\extracolsep{\fill}} p{4.75in} @{}}  %%%	NO CAMBIAR ESTA LÍNEA
  \textsf{Símbolo} & \textsf{Significado}\\[12pt]
  \endhead
  $:=$ & es definido por\\
  $\cong$ & es isomorfo a\\
  $\Leftrightarrow$ & si y sólo si\\
  $\varnothing$ & conjunto vacío\\
  $E^c$ & complemento de $E$\\
  $\varsubsetneq$ & estrictamente contenido\\
  $E\setminus F$ & diferencia entre $E$ y $F$\\
  $E\Delta F$ & diferencia simétrica entre $E$ y
  $F$\\
  $\mathcal{P} (X)$ & conjunto potencia de $X$\\
  $\chi_E$ & función característica de $E$\\
  $E_n\!\!\uparrow$ & $E_n$ es una sucesión
  creciente\\
  $\mathfrak{L}$ & \salg{} de los conjuntos
  Lebesgue"=medibles\\
  $\mathscr{S}$ & espacio muestral\\
  $\mathfrak{A}$ & \salg{} de eventos\\
  $(\mathscr{S},\mathfrak{A},P)$ & espacio de
  probabilidad\\
  $\mathscr{D}$ & espacio de las funciones de
  prueba\\
  $\mathscr{D}'$ & espacio de las distribuciones\\
  $\delta_0$ & medida de Dirac, función $\delta$ de Dirac o
  $\delta$-función\\
  $\Phi^{\times}$ & espacio antidual de $\Phi$\\
  $\Phi\subset \mathcal{H}\subset \Phi^{\times}$ &
  espacio de Hilbert equipado o tripleta de Gel'fand\\
  $\left\vert \psi \right>$ & vector \emph{ket}\\
  $\left< \psi \right\vert$ & funcional \emph{bra}\\
  $\left< \varphi \vert \psi \right>$ & \emph{braket}
\end{longtable}
