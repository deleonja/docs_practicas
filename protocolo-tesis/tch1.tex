\chapter{METODOLOGÍA}
% \esqueleto{
% \begin{itemize}
% \item Hacer un recordatorio del trabajo de prácticas porque es la base 
% teórica de este trabajo
% \item Método numérico para 2 y 3 qubits
% \item Análisis los resultados del numérico
% \item Comparación con los mapeos de Ruskai
% \item Trabajo futuro
% \end{itemize}
% }
\cpnote{Creo que esta seccion está mal. Como esta es lo mismo que la siguiente. 
Yo creo que acá mas bien se deben discutir los métodos que usaras. 
Porfa aclarame eso. }

El primer capítulo contendrá las bases teóricas necesarias para 
el estudio de las operaciones PCE. Se expondrán de manera puntual el 
formalismo de la matriz de densidad y la teoría de los canales cuánticos.
Se utilizará como referencias bibliografías libros especializados en 
el tema: Sakurai \cite{sakurai_napolitano_2017}, 
el texto introductorio estándar para información y computación cuántica de 
Nielsen y Chuang \cite{nielsen_chuang_2011}, Bengtsson 
\cite{bengtsson_zyczkowski_2017} y Preskill \cite{preskill1998lecture}.

En el segundo capítulo se definirán las operaciones PCE, el caso de 
1 qubit y se establecerá el problema para sistemas de $n$ qubits.
Para este capítulo se utilizarán los resultados y el estudio realizado 
durante el trabajo de práctica final, se hará un resumen con los 
aspectos más relevantes ya que son la base de este trabajo. 
Finalmente, en este capítulo se discutirá el uso del método numérico, 
que fue diseñado en la práctica final, para evaluar los casos de 2 y 3 qubits
que son el objetivo de este trabajo.

En el tercer capítulo se presentarán los resultados de 2 y 3 qubits. Se 
analizarán y discutirán los resultados, al mismo tiempo que se desarrollará 
una herramienta geométrica que permita entender de manera sencilla 
los canales cuánticos PCE de 2 y 3 qubits.

En el cuarto capítulo se discutirá la relación de las operaciones PCE con 
los canales diagonales de Pauli constantes sobre los ejes 
\cite{nathanson2007pauli}. Lo que buscamos es saber si los canales cuánticos
PCE están contenidos dentro del conjunto de los canales cuánticos que
estudian Nathanson y Ruskai. 
\cpnote{Yo escribiría porque nos interesa discutir eso, ponlo igual en la siguiente seccion}


