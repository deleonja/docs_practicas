\chapter{METODOLOGÍA}
% \esqueleto{
% \begin{itemize}
% \item Hacer un recordatorio del trabajo de prácticas porque es la base 
% teórica de este trabajo
% \item Método numérico para 2 y 3 qubits
% \item Análisis los resultados del numérico
% \item Comparación con los mapeos de Ruskai
% \item Trabajo futuro
% \end{itemize}
% }
\cpnote{Creo que esta seccion está mal. Como esta es lo mismo que la siguiente. 
Yo creo que acá mas bien se deben discutir los métodos que usaras. 
Porfa aclarame eso. }

Este trabajo de tesis se puede separar 
en tres partes de acuerdo con la metodología a utilizar durante
la realización del proyecto.
Estas tres partes son: (1) estudiar las herramientas 
teóricas necesarias para abordar el problema de las 
operaciones PCE, (2) evaluar númericamente
la completa positividad de las operaciones PCE de 2 y 3 qubits, y (3) 
comparar los canales cuánticos PCE con los canales diagonales de Pauli 
sobre los ejes estudiados por Nathanson y Ruskai \cite{nathanson2007pauli}.

Para la primera parte se hará una revisión bibliográfica de la literatura 
especializada en las herramientas matemáticas necesarias para este 
proyecto, principalmente la matriz de densidad y la teoría
de las operaciones cuánticas. Como referencia principal se utilizará 
el informe final de prácticas. Otras bibliografías serán utilizadas 
como complementarias para estudiar a profundidad algunas demostraciones,
teoremas y temas puntuales que no se incluyeron en el informe final.

Para la segunda parte se utilizará el método numérico desarrollado durante
las prácticas. Primero se evaluará el caso de 2 qubits y se buscarán las 
características que comparten los canales cuánticos PCE. 
Con este conjunto de características vamos a 
proponer una manera de acotar las operaciones PCE de 3 qubits a evaluar 
numéricamente. Con los resultados del caso de 3 qubits 
se harán algunas revisiones de consistencia con los resultados de 2 qubits 
y se continuará con la búsqueda de la caracterización general que hace 
a una operación PCE un canal cuántico.

Para la tercera parte se estudiará el artículo de Nathanson y Ruskai 
\cite{nathanson2007pauli}. Se buscarán características en común 
o diferencias con los resultados obtenidos de los canales cuánticos PCE
para formular una prueba analítica que demuestre o refute que los 
canales cuánticos PCE son un subconjunto de los canales diagonales 
de Pauli constantes sobre los ejes.

%El primer capítulo contendrá las bases teóricas necesarias para 
%el estudio de las operaciones PCE. Se expondrán de manera puntual el 
%formalismo de la matriz de densidad y la teoría de los canales cuánticos.
%Se utilizará como referencias bibliografías libros especializados en 
%el tema: Sakurai \cite{sakurai_napolitano_2017}, 
%el texto introductorio estándar para información y computación cuántica de 
%Nielsen y Chuang \cite{nielsen_chuang_2011}, Bengtsson 
%\cite{bengtsson_zyczkowski_2017} y Preskill \cite{preskill1998lecture}.
%
%En el segundo capítulo se definirán las operaciones PCE, el caso de 
%1 qubit y se establecerá el problema para sistemas de $n$ qubits.
%Para este capítulo se utilizarán los resultados y el estudio realizado 
%durante el trabajo de práctica final, se hará un resumen con los 
%aspectos más relevantes ya que son la base de este trabajo. 
%Finalmente, en este capítulo se discutirá el uso del método numérico, 
%que fue diseñado en la práctica final, para evaluar los casos de 2 y 3 qubits
%que son el objetivo de este trabajo.
%
%En el tercer capítulo se presentarán los resultados de 2 y 3 qubits. Se 
%analizarán y discutirán los resultados, al mismo tiempo que se desarrollará 
%una herramienta geométrica que permita entender de manera sencilla 
%los canales cuánticos PCE de 2 y 3 qubits.
%
%En el cuarto capítulo se discutirá la relación de las operaciones PCE con 
%los canales diagonales de Pauli constantes sobre los ejes 
%\cite{nathanson2007pauli}. Lo que buscamos es saber si los canales cuánticos
%PCE están contenidos dentro del conjunto de los canales cuánticos que
%estudian Nathanson y Ruskai. 

\janote{El siguiente comentario refería a porqué hablar de los mapeos de Ruskai}
\cpnote{Yo escribiría porque nos interesa discutir eso, ponlo igual en la siguiente seccion}

