\chapter{DESCRIPCIÓN DE LOS CAPÍTULOS}
\esqueleto{
\begin{itemize}
\item Cap 1: Fundamentos teóricos (formalismo de la matriz de densidad
y canales cuánticos) 
\item Cap 2: Mapeos de borrado de componentes de Pauli 
\item Cap 3: Resultados 2 y 3 qubits
\item Cap 4: Canalés cuánticos de Pauli constantes sobre los ejes
\end{itemize}
}
\janote{El capítulo 4 será cortito: exposición de los mapeos de Ruskai y
el argumento que tenemos para refutar que los PCE sean un subconjunto.}

\section*{Capítulo 1: Fundamentos teóricos}
En este capítulo se definirá a la matriz de densidad y a un canal cuántico. 
Se introducirá la matriz de densidad como la nueva herramienta para 
describir a los estados cuánticos y se expondrá la reformulación de 
los postulados de la mecánica cuántica utilizando este nuevo formalismo.
Se introducirá la teoría de los canales cuánticos. Dedicaremos especial 
atención a discutir la condición de completa positividad y presentaremos 
la representación de superoperador y de Kraus de un canal cuántico.
Este capítulo estará basdo en las referencias \cite{bengtsson_zyczkowski_2017},
\cite{nielsen_chuang_2011} y \cite{sakurai_napolitano_2017}.

\section*{Capítulo 2: Operaciones PCE}
Se definirán las operaciones PCE y se establecerá el problema de 
estudio para sistemas de $n$ qubits. Se hará una revisión del caso 
de 1 qubit para ofrecer intuición que será útil para entender el 
problema de más qubits. Por último, se discutirá la forma de 
abordar numéricamente el problema de 2 y 3 qubits. Para este
capítulo se utilizará como referencia el informe final de prácticas.

\section*{Capítulo 3: Resultados de 2 y 3 qubits}
En este capítulo se presentarán los resultados numéricos del 
caso de 2 y 3 qubits. Se desarrollará una herramienta geométrica 
para entender y analizar los resultados. Se buscará extraer de 
estos resultados la caracterización general de los canales cuánticos PCE,
o bien, se buscarán las pistas que conduzcan al camino de esta caracterización.

\section*{Capítulo 4: Canales diagonales de Pauli constantes 
sobre los ejes}
En este capítulo se presentarán los canales diagonales de Pauli constantes 
sobre los ejes \cite{nathanson2007pauli} y se estudiará si los canales
cuánticos PCE son un subconjunto de ellos. El objetivo es buscar una 
prueba definitiva para demostrar que los canales cuánticos PCE que se contienen, 
que existe una intersección o que son conjuntos excluyentes con los canales 
diagonales de Pauli.