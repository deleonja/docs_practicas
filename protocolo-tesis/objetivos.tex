%%% Haga el diseño que más le guste
\chapter{OBJETIVOS}
\section*{General}
Estudiar las operaciones de borrado de componentes de Pauli (PCE por
su nombre en inglés ``\textit{Pauli-component-erasing} operations'' 
\cpnote{pon de donde proviene la sigla} 
\janote{Ya}
) en sistemas de 2 y 3 qubits.


\section*{Específicos}

\begin{enumerate}
\item Estudiar numéricamente la completa positividad de las operaciones 
PCE en sistemas de 2 y 3 qubits.

\item Estudiar las características de los canales PCE.
\cpnote{Acá sería un poco mas especifico. que tienes en mente?}
\janote{De acuerdo a lo que platicamos agregué los siguientes dos 
items. Este item lo voy a borrar.}

\item Estudiar las características que debe satisfacer una operación PCE
para ser un canal cuántico.

\item Estudiar la existencia de subconjuntos de canales cuánticos PCE
cuyos elementos sean equivalentes.

\item Desarrollar una herramienta geométrica para entender las
operaciones PCE.

\item Comparar los canales cuánticos PCE con otros canales de Pauli que han 
sido previamente estudiados.
\end{enumerate}

