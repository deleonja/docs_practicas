\chapter{INTRODUCCIÓN}
% \textcolor{mycolor}{
% \begin{itemize}
% \item La mecánica cuántica y la limitación del formalismo que se 
% aprende en la licenciatura para describir a los sistemas abiertos
% \item Teoría de los canales cuánticos
% \item Mapeos PCE y el estado del estudio, que sería lo de 1
% qubit que se puso en el informe de prácticas
% \item Cuáles son las expectativas de lo que queremos encontrar 
% para 2 y 3 qubits
% \end{itemize}
% }

A diferencia de los sistemas cuánticos cerrados, la dinámica 
de los sistemas abiertos no puede describirse adecuadamente 
con operadores unitarios únicamente.
Un sistema cerrado es un sistema ideal que se considera completamente
aislado del resto del universo. Sin embargo, los sistemas cuánticos 
reales están en alguna medida abiertos a interacciones con su entorno.
Por esa razón, la descripción completa de un sistema cuántico requiere
incluir a su entorno y considerar que este sistema secundario
es inaccesible parcial o totalmente, o que sencillamente no es de interés 
para el observador \cite{schlosshauer2007decoherence}.


%En sus inicios, la teoría cuántica se ocupó de describir de la dinámica 
%de sistemas cuánticos ideales que se consideran completamente aislados 
%de su entorno \cite{feynman1965feyman}. Sin embargo, el rápido progreso
%de las últimas décadas en la manipulación de los sistemas cuánticos de 
%muchos cuerpos ha hecho cada vez más necesaria una formulación 
%más general de la mecánica cuántica en la que se considere a 
%los sistemas cuánticos como sistemas abiertos que pueden sufrir de 
%interacción con su entorno \cite{preskill1998lecture}.
% \cpnote{No es cierto. Desde un inicio si fue importante la decoherencia. 
% Te dejo un link (como comentario en el .tex) que encontre poniendo von neumman decoherence en google. 
% Se me ocurre que le des una leida a la historia de la decoherencia, de varias fuentes. 
% No he leido este libro, pero si he escuchado al autor y creo que puede ser 
% una buena fuente 
% https://www.springer.com/gp/book/9783540357735
% }
% \janote{Reescribí ese primer párrafo. Quedamos que en la introducción 
% de la tesis hablaré sobre la decoherencia.}
% https://plato.stanford.edu/entries/qm-decoherence/#OrthApp

Una propuesta para describir la evolución de los sistemas cuánticos abiertos
es la teoría de los canales cuánticos. Esta teoría proporciona una descripción
discreta de la dinámica de los sistemas abiertos que ocupa el formalismo
de la matriz de densidad para describir a los estados cuánticos
\cite{nielsen_chuang_2011}. La matriz
de densidad posee la ventaja sobre el vector de estado de poder describir
estados puros y mixtos (mezclas estadísticas) \cite{sakurai_napolitano_2017}. 
Otro nombre que reciben 
los canales cuánticos y que describe mejor el tipo de mapeos lineales 
que son es el de operaciones completamente positivas que preservan la traza 
(operaciones CPTP) \cite{bengtsson_zyczkowski_2017}.
%\cpnote{Estas poniendo doble backslash al final del párrafo. No hagas eso pues 
%hace que el espacio sea mayor. Si quieres espacios mas grandes entre 
%parrafos, hay formas más apropiadas de hacer eso. Quito los que vea.}

Por otro lado, los sistemas cuánticos de dos niveles han tomado especial
relevancia en información y computación cuántica, pues son empleados 
como la unidad fundamental de información: el qubit. Por esa razón, es
de interés teórico entender la dinámica de los qubits como sistemas 
abiertos. En este trabajo proponemos la continuación del estudio de 
las operaciones PCE que se comenzó con el trabajo de práctica final.
Las operaciones PCE son operaciones lineales que borran componentes
del vector de Bloch generalizado de un sistema de $n$ qubits. El caso 
de 1 qubit fue estudiado con éxito. De las 8 operaciones PCE posibles para
1 qubit 5 son canales cuánticos: la identidad, las operaciones que borran 2 
componentes del vector de Bloch y el canal completamente depolarizante.
El estudio y caracterización de las operaciones PCE proporcionaría 
un aporte significativo al entendimiento de los canales cuánticos de
sistemas de qubits y de la dinámica de los sistemas abiertos, en general.

Para este trabajo nos proponemos estudiar las operaciones PCE de 
2 y 3 qubits. Con suficiente suerte, entender estos dos casos
puede guiar el camino hacia la caracterización general de las operaciones
PCE de $n$ qubits. Proponemos evaluar numéricamente, de forma sistemática, 
la completa positividad de las operaciones PCE de 2 qubits. El número de
operaciones PCE crece de manera doblemente exponencial con el número $n$ de
qubits ($2^{2^n-1}$). Para $n=3$ el número de 
operaciones PCE es del orden de $10^{18}$. Esto impone una restricción  
para evaluar numéricamente la CP las operaciones PCE a partir de $n=3$.
Por esa razón, buscaremos herramientas geométricas para analizar 
y extraer la mayor cantidad de información relevante de los 
resultados de 1 y 2 qubits y así evaluar de alguna manera razonable
el caso de 3 qubits. Además, revisaremos el 
estudio de otras operaciones de similar naturaleza para compararlos
con las operaciones PCE \cite{nathanson2007pauli}. 
La motivación principal de este trabajo de graduación 
es buscar pistas que conduzcan a la caracterización de las 
operaciones PCE de sistemas de $n$ qubits.


