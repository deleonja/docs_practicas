\chapter{Conclusiones y trabajo futuro}
\section{Conclusiones}
\begin{enumerate}
\item El formalismo de la matriz de densidad proporciona la herramienta 
más general para describir al estado de un sistema cuántico, 
puesto que la matriz de densidad puede describir estados puros y mixtos.
\item Los sistemas cuánticos reales son sistemas abiertos que sufren
de interacción con algún sistema secundario. Por esa razón, el estudio 
de un formalismo para describir la evolución de los sistemas abiertos
es necesario.
\item La teoría de las operaciones cuánticas propone una manera
discreta de describir la dinámica de los sistemas cuánticos abiertos.
\item Una operación cuántica es una operación completamente positiva 
que preserva la traza de la matriz de densidad. 
\item La completa positividad es una condición más robusta a la 
positividad que asegura que una operación cuántica transforma
a todos los estados físicos de un sistema, especialmente a los estados 
entrelazados, en estados positivos.
\item Las operaciones PCE son operaciones lineales que borran las 
componentes de una matriz de densidad escrita en la base de las 
matrices de Pauli.
\item Se estudiaron las operaciones PCE de 1 qubit evalúando numéricamente
la condición de completa positividad y se reprodujeron los 5 canales cuánticos
que han sido bien estudiados por algunos autores
\cite{nielsen_chuang_2011},
\cite{bengtsson_zyczkowski_2017}. Los 5 canales cuánticos PCE 
de 1 qubit son la identidad, el canal totalmente depolarizante y 
un canal que mapea la esfera de Bloch a una línea sobre cualquiera de los ejes.
\end{enumerate}

\pagebreak 

\section{Trabajo futuro}
\begin{enumerate}
\item Estudiar las operaciones PCE en sistemas de más de 1 qubit.
\item Caracterizar las condiciones que debe satisfacer una operación 
PCE para ser una operación completamente positiva y, por consiguiente,
un canal cuántico.
\item Estudiar la relación de las operaciones PCE son otras operaciones 
de Pauli. 
\end{enumerate}