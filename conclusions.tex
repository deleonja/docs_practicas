\chapter{Conclusiones y trabajo futuro}
\section{Conclusiones}

Para abordar el problema de las operaciones que borran componentes 
del vector de Bloch de 1 qubit fue necesario estudiar el formalismo de 
la matriz de densidad y la teoría de las operaciones cuánticas. 
La matriz de densidad es una herramienta que contiene toda la información 
física que se puede extraer de un estado cuántico. La teoría de las 
operaciones cuánticas es un marco teórico para describir 
la evolución dinámica de los sistemas cuánticos abiertos. 
Una operación cuántica es una operación completamente positiva 
que preserva la traza de la matriz de densidad (operación CPTP).
La completa positividad es una condición más robusta a la positividad 
que asegura que una operación cuántica preserva la positividad de todos 
los estados de un sistema, especialmente la de los estados entrelazados.
Con estas herramientas teóricas se pueden abordar una gran variedad
de problemas, sin embargo en este trabajo nos enfocamos en iniciar el 
estudio de un tipo muy específico de operaciones lineales de qubits.

En este proyecto nos enfocamos en estudiar las operaciones PCE 
de 1 qubit que satisfacen la condición de completa positividad
y, por ende, son canales cuánticos. Una operación PCE 
de 1 qubit es una operación lineal que borra componentes del 
vector de Bloch. Existen 8 operaciones PCE de 1 qubit y 
numéricamente se encontró que 5 de ellas son 
canales cuánticos: la identidad, el canal totalmente depolarizante 
y los canales que mapean la esfera de Bloch a una línea sobre 
cualquiera de los ejes. Estos resultados son consistentes 
con los canales cuánticos de 1 qubit que han sido estudiados antes 
por otros autores como Nielsen y Bengtsson 
\cite{bengtsson_zyczkowski_2017,nielsen_chuang_2011}

\section{Trabajo futuro}
Para el trabajo de tesis de graduación proponemos la continuación del
estudio de las operaciones PCE. Proponemos estudiar numéricamente 
la completa positividad de las operaciones PCE en sistemas con 
más de 1 qubit, y con estos resultados lo que se buscaría son pistas 
para la caracterización general que debe cumplir una operación 
PCE para ser un canal cuántico. Por otro lado, también sería importante
revisar si existen otros estudios de canales cuánticos similares a los 
PCE para determinar la generalidad de este tipo de operaciones.
