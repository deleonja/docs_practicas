\chapter{El operador de densidad} % {{{

%Una formulación alternativa, pero equivalente, a la del vector de estado 
%de la mecánica cuántica es posible con el lenguaje de el operador de densidad\cpnote{cita}. 
%La preferencia por el operador de densidad se centra en que provee la descripción
%más general de un estado cuántico ya que contempla los llamados estados puros y,
%también, mezclas estadísticas\cpnote{No se lo que quieres decir en esta
%frase.}. el operador de densidad se puede concebir \cpnote{Por que esta palabra?}
%a partir de la descripción de un ensamble de estados, sin embargo, es
%importante desarrollar una caracterización intrínseca \cpnote{Por qué esta
%palabra?} de la misma. 
La mecánica cuántica es una teoría fundamental de la física que
proporciona la descripción más adecuada y completa de las propiedades
físicas de la naturaleza en la escala atómica y de las partículas 
subatómicas. Este capítulo está dedicado a presentar la herramienta
del operador de densidad y la formulación de la mecánica cuántica 
utilizando este lenguaje. Se presentará primero una definición
del operador de densidad motivada en la descripción de ensambles
de estados, también conocidos como estados mixtos. Luego
se expondrá la caracterización del operador de densidad y la 
formulación de los postulados de la mecánica cuántica utilizando este
formalismo. Finalmente, se presentará el 
operador de densidad reducido, herramienta que permite 
describir subsistemas individuales de un sistema cuántico compuesto.

\section{Ensambles de estados cuánticos} % {{{
El formalismo del operador de densidad proporciona una herramienta
que puede considerar de manera unificada incertezas cuánticas, 
presentes aún cuando el estado de un sistema se conoce 
por completo, e incertezas clásicas, 
debidas a falta de conocimiento \cite{sakurai2010modern}.
En esta sección vamos a definir al operador de densidad asociado a 
un ensamble de estados y luego presentaremos cómo describir la dinámica
y medición de los estados de un sistema cuántico descritos por su operador
de densidad, de manera que esto motivará un camino hacia la reformulación
de todos los postulados de la mecánica cuántica.

Supongamos que un sistema cuántico se encuentra en uno de los estados 
$\ket{\psi _i}$ con probabilidad $p_i$. El conjunto $\{p_i,\ket{\psi _i} \}$ 
se conoce como un ensamble de estados puros. El operador de 
densidad para el sistema se define como \cite{nielsen_chuang_2011}
\begin{equation}
	\rho \equiv \sum _i p_i\dyad{\psi_i}{\psi_i}.
	\label{eq:rho_def}
\end{equation}

Para entender el cálculo de el operador de densidad consideremos el ejemplo
a continuación. Una partícula de espín 1/2 se encuentra en el estado 
$\ket{\psi_1}=\sqrt{\frac{1}{3}}\ket{0} + \sqrt{\frac{2}{3}}\ket{1}$
con probabilidad $p_1=\frac{1}{2}$ y en el estado 
$\ket{\psi_2}=\sqrt{\frac{1}{3}}\ket{0} - \sqrt{\frac{2}{3}}\ket{1}$
con probabilidad $p_2=\frac{1}{2}$. El operador de densidad para este 
sistema se calcula de la siguiente manera
\begin{align}
	\rho 	&= \frac{1}{2}\dyad{\psi_1}{\psi_1} + 
	\frac{1}{2}\dyad{\psi_2}{\psi_2} \nonumber \\
	\rho	&= \frac{1}{2}\qty(\frac{1}{3}\dyad{0}{0} +
					 \frac{\sqrt{2}}{3}\dyad{0}{1} + 
					 \frac{\sqrt{2}}{3}\dyad{1}{0} +
					 \frac{2}{3}\dyad{1}{1}) + \nonumber \\
				&\hspace{5mm} \frac{1}{2}\qty(\frac{1}{3}\dyad{0}{0} -
					 \frac{\sqrt{2}}{3}\dyad{0}{1} -
					 \frac{\sqrt{2}}{3}\dyad{1}{0} +
					 \frac{2}{3}\dyad{1}{1}) \nonumber \\
	\rho	&= \frac{1}{3}\dyad{0}{0} + \frac{2}{3}\dyad{1}{1}. 
	\label{eq:rho-calc-ex}
\end{align}
Ahora consideremos otra partícula de espín $1/2$ cuyo estado es alguno de los
del ensamble $\{ \frac{1}{3}, \ket{0}; \frac{2}{3}, \ket{1} \}$. 
El operador de densidad para este sistema es también \eqref{eq:rho-calc-ex}.
Este ejemeplo muestra cómo distintas mezclas estadísticas de estados puros 
resultan en el mismo operador de densidad.\newline

Ahora nos interesamos en revisar que la dinámica de un sistema cuántico
pueda ser descrito utilizando su operador de densidad. 
Supongamos que la evolución de un sistema 
cerrado se describe por un operador unitario $U$. 
Si el sistema inicialmente se encuentra en alguno de los estados 
$\ket{\psii}$, con probabilidad $p_i$, entonces después de que
la evolución ocurra el sistema se encontrará en el estado $U\ket{\psii}$ 
con probabilidad $p_i$, de manera que la evolución de el operador de densidad
se describe mediante la siguiente ecuación
\begin{align}
	\rho = \sum _ip_i\dyad{\psii}{\psii}
	\xrightarrow[]{U}
	\sum _ip_iU\dyad{\psii}{\psii}U^{\dagger}	=
	U\rho U^{\dagger}.
\end{align}

Por otro lado, las mediciones también se pueden describir con el 
lenguaje del operador de densidad. Supongamos que se realiza una medición, 
que se define por los operadores de medida $M_m$, del estado de un sistema. 
Si el estado inicial del sistema era $\ket{\psii}$ antes de medirlo, 
entonces la probabilidad de obtener el resultado $m$ al realizar
la medición es
\begin{align}
	p\qty(m|i) &= \matrixel{\psii}{M_m^{\dagger}M_m}{\psii}\\
	&= \sum _k \braket{\psii}{k}\matrixel{k}{M_m^{\dagger}M_m}{\psii}\\
	&=\sum _k \matrixel{k}{M_m^{\dagger}M_m}{\psii}\braket{\psii}{k}\\
	&=\Tr \qty(M_m^{\dagger}M_m\dyad{\psii}{\psii}). \label{eq:p(mi)}
\end{align}
Por la ley de la probabilidad total la probabilidad de medir $m$ es
\janote{debería referir hacia un apéndice para explicar esta ley?}
\begin{align}
	p(m) &= \sum_i p(m\vert i)p_i \\
			 &= \sum_i \Tr \qty(M_m^{\dagger}M_m\dyad{\psii}{\psii})p_i \\
			 &= \Tr \qty(M_m^{\dagger}M_m\sum_ip_i\dyad{\psii}{\psii}) \\
			 &= \Tr \qty(M_m^{\dagger}M_m\rho). \label{eq:p(m)}
\end{align}
Si el estado inicial del sistema era $\Pk{i}$ antes de medirlo, 
entonces el estado luego de obtener $m$ como resultado de la medición es
\begin{align}
	\ket{\psi_i^m} = \frac{M_m\Pk{i}}{\sqrt{\matrixel{\psii}{M_m^{\dagger}
	M_m}{\psii}}}.
\end{align}
Por consiguiente, luego de medir $m$ se tiene un ensamble de estados
$\ket{\psi_i^m}$ con probabilidades $p\qty(i\vert m)$. El operador
de densidad para este ensamble es
\begin{align}
	\rho_m &= \sum _i p\qty(i\vert m)\dyad{\psii^m}{\psii^m}
	\nonumber \\
				 &= \sum _i p\qty(i\vert m)
				 \frac{M_m\dyad{\psii}{\psii}M_m^{\dagger}}
				 {\matrixel{\psii}{M_m^{\dagger}M_m}{\psii}}. 
				 \label{eq:rho-after-measurement}
\end{align}
De la teoría de la probabilidad se sabe que
\begin{align}
	p\qty(i\vert m) = \frac{p(m,i)}{p(m)}=
	\frac{p\qty(m\vert i)p_i}{p(m)},
\end{align}
y sustituyendo \eqref{eq:p(mi)} y \eqref{eq:p(m)} en 
\eqref{eq:rho-after-measurement} el operador
de densidad asociado al ensamble de estados 
$\qty{\ket{\psi_i^m},p(i\vert m)}$, que resulta de obtener $m$ como resultado
de la medición, es
\begin{align}
	\rho _m &= \sum _i p_i \frac{M_m\dyad{\psii}{\psii}M_m^{\dagger}}
	{\Tr \qty(M_m^{\dagger}M_m\rho)} \nonumber \\
					&= \frac{M_m\rho M_m^{\dagger}}
	{\Tr \qty(M_m^{\dagger}M_m\rho)}.
\end{align}
Hasta aquí lo que hemos hecho es mostrar que los postulados de la
mecánica cuántica de evolución unitaria de los sistemas cerrados y 
medición se pueden reformular utilizando el operador de densidad.
Antes de mostrar que, en efecto, todos los postulados pueden 
ser reformulados  vamos a presentar una definición del operador 
de densidad independiente del vector de estado. 
\h{\janote{en algún momento mencionar que el operador de densidad
contiene toda la información estadística}}

% }}}
\section{Caracterización del operador de densidad} % {{{
El operador de densidad fue introducido en la sección anterior
como una herramienta para describir mezclas estadísticas de estados cuánticos,
en esta sección vamos a iniciar presentando 
la caracterización del operador de densidad, de modo que
esto completa la formulación del nuevo formalismo de la mecánica cuántica
que adoptaremos en este proyecto. Es importante resaltar que este
formalismo es equivalente al del vector de estado. En vista de ello, 
vamos a concluir reformulando los postulados de la mecánica cuántica 
en el lenguaje del operador de densidad.

Un operador de densidad está caracterizado por el teorema a continuación
\begin{teorema}[\textbf{Caracterización de el operador de densidad}]
Un operador $\rho$ es el operador de densidad asociado a algún ensamble 
$\{p_i, \ket{\psi _i} \}$ si y sólo si satisface las condiciones:
\begin{enumerate}
\item $\Tr \rho = 1$.
\item $\rho \geq 0$.
\end{enumerate}	
\end{teorema}

\begin{proof}
	Supongamos que $\rho = \sum p_i\dyad{\psii}{\psii}$ es un operador de
	densidad. Entonces 
	\begin{align*}
		\Tr \rho &= \sum _i p_i\Tr \qty(\dyad{\psii}{\psii})\\
		&= \sum _i p_i \sum _j \braket{\psi_j}{\psii}\braket{\psii}{\psi_j}\\
		&=\sum _i \sum _j p_i \delta _{ij}^2\\
		&=\sum _i p_i=1,
	\end{align*}
	por consiguiente la condición de traza unitaria de $\rho$ se cumple. 
	Supongamos que $\ket{\phi}$ es un estado arbitrario. Entonces
	\begin{align*}
		\matrixel{\phi}{\rho}{\phi} &= \sum _i p_i \braket{\phi}{\psii}
		\braket{\psii}{\phi} \\
		&=\sum _i p_i \abs{\braket{\phi}{\psii}}^2\\
		&\geq 0,
	\end{align*}
	por consiguiente $\rho$ es una matriz positiva semidefinida. Por el 
	contrario, supongamos que $\rho$ es una matriz que satisface las condiciones
	de traza unitaria y positividad. Dado que $\rho$ es positiva, entonces
	tiene descomposición espectral 
	\begin{align*}
		\rho = \sum _k \lambda _k \dyad{k}{k},
	\end{align*}
	donde $\ket{k}$ satisfacen la relación de ortogonalidad y $\lambda _k$ son
	autovalores de $\rho$ reales y no negativos. Notamos que $\Tr \rho = \sum _k
	\lambda _k = 1$. Por consiguiente, un sistema en el estado $\ket{k}$ con 
	probabilidad $\lambda_k$ tendrá una operador de densidad $\rho$. Dicho 
	de otro modo, el ensamble de estados $\{ p_k,\ket{k}\}$ da lugar a la matriz
	de densidad $\rho$.
\end{proof}

Ahora que hemos establecido las características generales del 
operador de densidad se reformulan los postulados de la mecánica
cuántica.
\begin{itemize}
	\item[] \textbf{Postulado 1.} Un sistema físico tiene asociado un espacio vectorial complejo
	con producto interno que se conoce como el espacio de estados del
	sistema. El sistema está completamente descrito por su operador de densidad,
	que es un operador $\rho$ positivo con traza unitaria, y este actúa sobre 
	el espacio de estados del sistema. Si un sistema cuántico se encuentra
	en el estado $\rho _i$ con probabilidad $p_i$, entonces la matriz de
	densidad del sistema es $\sum p_i\rho_i$.
	\item[] \textbf{Postulado 2.} La evolución de un sistema cuántico cerrado está descrita por una transformación
	unitaria. Es decir, el estado $\rho$ del sistema en un tiempo $t_1$ está 
	relacionado con el estado $\rho'$ del sistema en un tiempo $t_2$ por un operador
	unitario $U$ que depende sólo de los tiempos $t_1$ y $t_2$,
	\begin{equation}
	\rho'=U\rho U^{\dagger}.
	\label{eq:postulate1}
	\end{equation}
	\item[] \textbf{Postulado 3.} Las mediciones cuánticas están descritas por un conjunto de 
	operadores de medición $\{M_m\}$. Estos son operadores que actúan sobre el espacio 
	de estados del sistema a medir. El índice $m$ refiere a los resultados
	de las mediciones que puedan ocurrir en el experimento. Si el estado del sistema
	cuántico es $\rho$ inmediatamente antes de la medición, entonces la probabilidad
	de medir el resultado $m$ está dada por
	\begin{equation}
	p(m)=\tr \qty(M_m^{\dagger}M_m\rho),
	\label{eq:postulate2_prob}
	\end{equation}						
	y el estado del sistema después de la medición es
	\begin{equation}
	\rho'=\frac{M_m\rho M_m^{\dagger}}{\tr \qty(M_m^{\dagger}M_m\rho)}.
	\label{eq:postulate2_rhoPrime}
	\end{equation}	
	Los operadores de medición deben satisfacer la ecuación de completitud
	\begin{equation}
	\sum _m M_m^{\dagger}M_m=\mathbb{1}.
	\label{eq:postulate2_completeness}
	\end{equation}
	\item[] \textbf{Postulado 4.} El espacio de estados de un sistema físico compuesto es el producto tensorial 
	de los espacios de estados de los sistemas físicos que componen al sistema total.
	Además, si los sistemas están enumerados de 1 hasta $n$, y el $i$-ésimo sistema
	se prepara en el estado $\rho_i$, entonces el estado del sistema total es
	$\rho_1\otimes\rho_2\otimes\cdots\otimes\rho_n$.
\end{itemize}

Es nueva formulación de la mecánica cuántica nos permite describir
dos cosas que no sería posible utilizando la formulación del 
vector de estado:
sistemas cuánticos cuyos estados no son conocidos y
subsistemas de un sistema cuántico compuesto como lo veremos
en la siguiente sección. \janote{pulir.}

% }}}
\section{El operador de densidad reducido} % {{{
La aplicación más relevante del operador de densidad es quizás 
la de ser una herramienta para describir subsistemas de un sistema
cuántico compuesto. Esta herramienta se conoce como el operador de
densidad reducido. Es indispensable en el 
análisis de sistemas cuánticos compuestos. Esta sección la 
dedicamos a presentar la definición del operador de densidad 
reducido.

Supongamos que existen dos sistemas $A$ y $B$, cuyo estado está 
descrito por el operador de densidad $\rho^{AB}$. El
operador de densidad reducido para el sistema $A$ se define como
\begin{align}
	\rho^A \equiv \Tr _B\qty(\rho^{AB}),
	\label{eq:partialTrace-def}
\end{align}
donde $\Tr_B$ es un mapeo de operadores que se conoce como
la traza parcial sobre el sistema $B$. La traza parcial
se define como 
\begin{align}
	\Tr_B (\dyad{a_1}{a_2}\otimes \dyad{b_1}{b_2})
	\equiv
	\dyad{a_1}{a_2}\Tr \qty(\dyad{b_1}{b_2}),
	\label{eq:part_trace-def}
\end{align}
donde $\ket{a_1}$ y $\ket{a_2}$ son cualesquiera dos vectores en
$\mathcal{H}_A$, y $\ket{b_1}$ y $\ket{b_2}$ cualesquiera dos vectores
en $\mathcal{H}_B$. La definición de traza parcial, establecida
en \eqref{eq:part_trace-def}, se completa requiriendo que la 
operación de traza parcial sea lineal.

Veamos un ejemplo del cálculo del operador de densidad reducido
para un sistema de 2 qubits que se encuentra en el estado de Bell.
$\qty(\ket{00}-\ket{11})/\sqrt{2}$. Este
sistema tiene un operador de densidad
\begin{align}
	\rho &= \qty(\frac{\ket{00}-\ket{11}}{\sqrt{2}})
	\qty(\frac{\bra{00}-\bra{11}}{\sqrt{2}}) \\
			 &= \frac{\dyad{00}{00}-\dyad{11}{00}-\dyad{00}{11}+\dyad{11}{11}}{2}.
\end{align}
Calculamos ahora el operador de densidad reducido del qubit 1, haciendo
la traza parcial sobre el qubit 2,
\begin{align}
	\rho^A &= \Tr _B(\rho) \\
			 	 &= \frac{\Tr _B(\dyad{00}{00})-\Tr _B(\dyad{11}{00})
			 	 -\Tr _B(\dyad{00}{11})+\Tr _B(\dyad{11}{11})}{2} \\
			 	 &= \frac{\dyad{0}{0} + \dyad{1}{1}}{2} \\
			 	 &= \frac{\mathbb{1}}{2}.
\end{align}
Notemos que el qubit 1 se encuentra en un \h{estado mixto}, es decir
que no contamos con la mayor información sobre ese qubit. 
\janote{ya expliqué qué vergas es un estado mixto?}
Este resultado llama la atención porque el estado del sistema total
es un estado puro. Esta propiedad es una consecuencia
del entrelazamiento cuántico.

La traza parcial es la única operación que describe correctamente 
las cantidades observables para subsistemas que forman parte de un
sistema compuesto. A continuación exponemos la justificación.
Supongamos que $M$ es un observable sobre el sistema $A$ y que
contamos con un aparato de medición que es capaz de realizar mediciones
de $M$. Sea $\tilde{M}$ el observable para la misma medición, pero 
que actúa sobre el sistema compuesto $AB$. Notemos que si el sistema
$AB$ se prepara en el estado $\ket{m}\otimes \ket{\psi}$, donde $\ket{m}$
es un autoestado de $M$ con autovalor $m$, y $\ket{\psi}$ es 
cualquier estado de $B$, entonces el aparato de medición 
debe obtener $m$ como resultado de la medición con probabilidad uno.
De manera que, si $P_m$ es el proyector hacia el autoespacio $m$ del
observable $M$, entonces el proyector correspondiente para $\tilde{M}$
es $P_m\otimes \mathbb{1}_B$. Por consiguiente tenemos
\begin{align}
	\tilde{M} = \sum_m mP_m\otimes \mathbb{1}_B=M\otimes \mathbb{1}_B.
\end{align}
Lo siguiente es mostrar que la traza parcial da correctamente las 
mediciones estadísticas para observaciones realizadas sobre una 
parte del sistema. Supongamos que se realiza una medición sobre
el sistema $A$ descrito por el observable $M$. Para tener 
consistencia física se requiere que cualquier asociación de un estado,
$\rho^A$, al sistema $A$, debe tener la propiedad de que los promedios
en la medición sean los mismo calculados a través de $\rho^A$ o $\rho^{AB}$;
es decir
\begin{align}
	\Tr \qty(M\rho^A) = \Tr \qty(\tilde{M}\rho^{AB}) = 
	\Tr \qty[\qty(M\otimes\mathbb{1}_B)\rho^{AB}].
\end{align}
Notemos que esta ecuación se satisface con la definición
\eqref{eq:partialTrace-def}
\begin{align}
	\Tr \qty[\qty(M\otimes\mathbb{1}_B)\rho^{AB}] 
	&= \sum_{i,j} \bra{i}\tensor\bra{j}	\qty(M\tensor \sum_k\dyad{k}{k})
	\rho^{AB}
	\ket{i}\tensor\ket{j} \nonumber \\
	&= \sum_{i,j,k} \bra{i}\braket{j}{k}M\matrixel{k}{\rho^{AB}}{j}\ket{i}
  \nonumber \\
	&= \sum_{i} \bra{i}M\sum_j \matrixel{j}{\rho^{AB}}{j}\ket{i} \nonumber \\
	&= \sum_i \bra{i}M\Tr_B\qty(\rho^{AB})\ket{i} \nonumber \\
	&= \Tr\qty(M \rho^A).
\end{align}
De hecho, la tracia parcial es la única función que tiene esta propiedad.
Para verificarlo supongamos que $f$ es un mapeo de operadores de 
densidad que actúan sobre el sistema AB hacia operadores de densidad 
que actúan sobre A, tal que 
\begin{align}
	\Tr \qty(Mf\qty(\rho^{AB})) = \Tr \qty[\qty(M\tensor \1)\rho^{AB}]
\end{align}
para cualquier observable $M$. Sea $M_i$ una base ortonormal de operadores
del espacio de los operadores Hermíticos con respecto del producto interno
de Hilbert-Schmidt $\qty(X,Y)\equiv \Tr\qty(XY)$. Entonces escribiendo
$f\qty(\rho ^{AB})$ en esta base
\begin{align}
	f\qty(\rho^{AB}) 
	&= \sum_i M_i\Tr\qty(M_if\qty(\rho^{AB})) \\
	&= \sum_i M_i\Tr\qty[\qty(M_i\tensor \1)\rho^{AB}].
\end{align}
Por consiguiente el único mapeo $f$ que satisface esta condición
es la traza parcial. 
\janote{Revisar este último argumento (ni yo entendí).}

Habiendo justificado el uso de la traza parcial como herramienta para 
describir subsistemas que pertenecen a un sistema compuesto hemos completado
el marco teórico del lenguaje del operador de densidad. Comenzamos
motivando el uso del operador de densidad como herramienta 
para describir ensambles de
estados cuánticos, luego presentamos una definición independiente
de los vectores de estado para caracterizar al operador de densidad y, así,
se reformularon los postulados de la mecánica cuántica.
Finalmente, concluimos introduciendo el operador de densidad reducido y
justificando el uso de la operación de traza parcial. Ahora, dirigimos
nuestra atención hacia la descripción de la dinámica de los estados
de sistemas cuánticos abiertos, el tipo de sistemas que se estudiarán
en este proyecto. Será entonces donde quede claro
la razón por la cual se dedicó un capítulo para introducir 
el lenguaje del operador de densidad.

% }}}
%\section{El espacio de las matrices de densidad}  % {{{
%\janote{Luego de que me dijiste que no mate con este trabajo la tesis
%me parece que esta sección podría estar de más y guardarla para la tesis. 
%Igual lo dejo para que de una vez lo puedas ver.}
%% }}}
%\section{El espacio de Hilbert-Schmidt} % {{{
%% Intro {{{
%A un espacio de Hilbert $\mathcal{H}$ complejo de dimensión $N$
%le acompaña su espacio dual $\mathcal{H}^*$,
%el espacio de las transformaciones lineales de $\mathcal{H}$
%al campo de los números complejos $\mathbb{C}$. Otro espacio disponible 
%a considerar es el de los operadores lineales que actúan sobre $\mathcal{H}$. 
%Cuando este espacio se equipa con 
%\begin{equation}
%\langle A,B\rangle=c\Tr \qty(A^{\dagger}B),
%\label{eq:HS_innerP}
%\end{equation}
%donde $c\in \mathbb{R}$ establece una escala, se conoce como el espacio de
%Hilbert-Schmidt $\mathcal{HS}$. 
%
%El espacio vectorial de los operadores Hermíticos 
%$\mathcal{HM}$ es un subespacio real de $N^2$ dimensiones de $\mathcal{HS}$. 
%$\mathcal{HM}$ también se puede pensar como el álgebra de Lie de $U(n)$. Los
%operadores Hermíticos con traza cero forman un subespacio vectorial de
%$\mathcal{HM}$. Por tanto, es posible encontrar una base ortonormal
%$\sigma_i$ con
%respecto al producto interno \eqref{eq:HS_innerP}. Al agregar la 
%matriz identidad un operador hermítico $A$ puede escribirse de la siguiente
%manera
%\begin{align}
%	A &= \tau_0\sqrt{\frac{2}{N}}\mathbb{1} + \sum _{i=1}^{N^2-1}\tau_i\sigma_i,
%\end{align}
%donde $\tau_0=\frac{\Tr A}{\sqrt{2N}}$ y $\tau_i=\Tr \sigma _iA$.
% 
%Por otro lado, el conjunto de operadores positivos $\mathcal{P}$ es un 
%conjunto convexo contenido dentro de $\mathcal{HM}$ y que
%\begin{align}
%	\dim \qty[\mathcal{P}] = \dim \qty[\mathcal{HM}]=N^2.
%\end{align}
%% }}}
%\subsection{Coordinizando  $\rho$} % {{{
%\cpnote{Esa palabra no existe, creo. Quiza parametrización} 
%
%Finalmente, el conjunto de las matrices de densidad consiste en todos los operadores
%positivos $\rho$ con traza unitaria. Este conjunto se denota como 
%$\M$, donde el exponente enfatiza que el conjunto consiste en 
%matrices de $N\times N$. $\M$ es un conjunto convexo que, en $\mathcal{HM}$, 
%es la intersección del conjunto de los operadores positivos con un 
%hiperplano paralelo al subespacio lineal de los operadores hermíticos
%con traza igual a cero \cite{bengtsson_zyczkowski_2017}.
%
%Una manera posible de \textit{coordinizar} $\M$ es 
%\begin{align}
%	\rho = \frac{1}{N} \mathbb{1} + \sum _{i=1}^{N^2-1} \tau_i\sigma_i,	
%	\label{eq:rho_general}
%\end{align}
%donde la matriz cero se ha cambiado por la matriz
%\begin{align}
%	\rho _{\star} \equiv \frac{1}{N}\mathbb{1},
%	\label{eq:max_mixed_state}
%\end{align}
%que se conoce como el estado máximamente mixto o la `matriz de ignorancia'. 
%Las componentes $\tau_i$ en \eqref{eq:rho_general} se conocen como `coordenadas
%de mezcla'.
%% }}}
%% }}}
%\section{Transformaciones unitarias} % {{{
%
%Teorema de Kadison \cite{bengtsson_zyczkowski_2017}:
%\begin{teorema}[\textbf{Teorema de Kadison}] 
%	Supongamos un mapeo $\Phi:\M \mapsto \M$ que es uno-a-uno y que es afín y
%	que preserva la estructura compleja en el sentido que
%	\begin{align}
%		\Phi \qty(p\rho_1 + \qty(1-p)\rho_2) = p\Phi(\rho_1) + 
%		\qty(1-p)\Phi(\rho_2).
%	\end{align}
%	Entonces el mapeo debe tomar la forma
%	\begin{align}
%		\Phi(\rho) = U\rho U^{-1},
%	\end{align}
%	donde el operador $U$ es unitario o antiunitario.
%\end{teorema}
%
%Este teorema nos permite ver que para que un mapeo preserve la estructura 
%convexa debe ser un mapeo afín y que debe mapear estados putos
%a estados puros. Por ejemplo, para $N=2$ los estados puros forman una
%esfera. Los únicos mapeos afines que preservan la esfera son las rotaciones.

% }}}
% }}}



