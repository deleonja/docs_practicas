\chapter{El operador de densidad} % {{{
% Intro {{{
La mecánica cuántica es una teoría fundamental de la física que
proporciona la descripción más adecuada y completa de las propiedades
físicas de la naturaleza en la escala atómica y de las partículas 
subatómicas. El operador de densidad es una herramienta
que hace posible formular la mecánica cuántica en un lenguaje
que puede considerar de manera unificada incertezas cuánticas, 
presentes aún cuando el estado de un sistema se conoce 
por completo, e incertezas clásicas, 
debidas a falta de conocimiento \cite{sakurai2010modern}.
Este primer capítulo lo dedicamos a presentar esta nueva herramienta.
Comenzaremos estableciendo la definición del operador de densidad
motivada en la descripción de ensambles de estados. Luego
revisaremos las propiedades generales del operador de densidad y la 
formulación de los postulados de la mecánica cuántica utilizando este
lenguaje. Finalmente, en la última sección, presentaremos una aplicación 
del operador de densidad para describir subsistemas individuales 
de un sistema cuántico compuesto: el operador de densidad reducido.

%Una formulación alternativa, pero equivalente, a la del vector de estado 
%de la mecánica cuántica es posible con el lenguaje de la matriz de densidad\cpnote{cita}. 
%La preferencia por la matriz de densidad se centra en que provee la descripción
%más general de un estado cuántico ya que contempla los llamados estados puros y,
%también, mezclas estadísticas\cpnote{No se lo que quieres decir en esta
%frase.}. La matriz de densidad se puede concebir \cpnote{Por que esta palabra?}
%a partir de la descripción de un ensamble de estados, sin embargo, es
%importante desarrollar una caracterización intrínseca \cpnote{Por qué esta
%palabra?} de la misma.  

% }}}

\section{Ensambles de estados cuánticos} % {{{
En esta sección vamos a definir al operador de densidad asociado a 
un ensamble de estados y luego presentaremos cómo describir la dinámica
y la medición de los estados de un sistema cuántico utilizando 
su operador de densidad, 
de manera que esto motivará un camino hacia la reformulación
de todos los postulados de la mecánica cuántica.

Consideremos un sistema cuántico que se encuentra en alguno de los estados 
$\Pk{i}$ con probabilidad $p_i$. El conjunto de todos los estados
conforman un ensamble de estados del sistema. Supongamos que realizamos 
una medición de algún observable $A$ sobre el ensamble. Podríamos 
preguntarnos cuál es el valor promedio de $A$ cuando se 
reproduce un número muy grande de mediciones \cite{sakurai2010modern}. 
Esto está dado por el valor promedio del ensamble de $A$, es decir
\begin{align}
	\expval{A} &= \sum_i p_i \matrixel{\psii}{A}{\psii} \\
	&= \sum_i p_i \matrixel{\psii}{A}{\psii} \\
	&= \sum_ip_i \sum _j\sum _k
	\bra{\psii}\dyad{\phi_j}{\phi_j}A\dyad{\phi_k}{\phi_k}\Pk{i},
\end{align}
donde $\ket{\phi_j}$ y $\ket{\phi_k}$ son dos bases distintas del
espacio de Hilbert del sistema,
\begin{align}
	&= \sum _j\sum _k \qty(\sum_ip_i \braket{\phi_k}{\psii}
	\braket{\psii}{\phi_j})	\matrixel{\phi_j}{A}{\phi_k}.
\end{align}
Esto motiva la definición del operador de densidad $\rho$ 
de la siguiente manera
\begin{equation}
	\rho \equiv \sum _i p_i\dyad{\psi_i}{\psi_i}.
	\label{eq:rho_def}
\end{equation}
Los elementos de la matriz $\rho$ tienen la forma 
\begin{align}
	\matrixel{\phi_k}{\rho}{\phi_j} = 
	\sum_ip_i \braket{\phi_k}{\psii}\braket{\psii}{\phi_j},
\end{align}
de manera que el operador de densidad contiene toda la información 
física que se puede obtener de un ensamble de estados.

Dos ensambles de estados distintos pueden tener el mismo operador
de densidad. Para ver esto consideremos dos ensambles distintos de 
una partícula de espín 1$/$2. El primero que vamos a considerar es 
$\qty{ p_1=1/2, \Pk{1}=1/\sqrt{3}\ket{0}+\sqrt{2/3}\ket{1};
p_2=1/2, \Pk{1}=1/\sqrt{3}\ket{0}-\sqrt{2/3}\ket{1}}$. El operador
de densidad asociado se calcula como
\begin{align}
	\rho 	&= \frac{1}{2}\dyad{\psi_1}{\psi_1} + 
	\frac{1}{2}\dyad{\psi_2}{\psi_2} \nonumber \\
	\rho	&= \frac{1}{2}\qty(\frac{1}{3}\dyad{0}{0} +
					 \frac{\sqrt{2}}{3}\dyad{0}{1} + 
					 \frac{\sqrt{2}}{3}\dyad{1}{0} +
					 \frac{2}{3}\dyad{1}{1}) + \nonumber \\
				&\hspace{5mm} \frac{1}{2}\qty(\frac{1}{3}\dyad{0}{0} -
					 \frac{\sqrt{2}}{3}\dyad{0}{1} -
					 \frac{\sqrt{2}}{3}\dyad{1}{0} +
					 \frac{2}{3}\dyad{1}{1}) \nonumber \\
	\rho	&= \frac{1}{3}\dyad{0}{0} + \frac{2}{3}\dyad{1}{1}. 
	\label{eq:rho-calc-ex}
\end{align}
Ahora consideremos el ensamble $\qty{ \frac{1}{3}, \ket{0}; 
\frac{2}{3}, \ket{1}}$ y notemos que el operador de densidad 
\eqref{eq:rho-calc-ex} es también el operador de densidad asociado a
este ensamble. En general, los autovectores y autovalores de un
operador de densidad sólo indican uno de los muchos posibles ensambles
que tienen asociado un operador de densidad específico. Por ello, 
no hay razón para pensar que existe algún ensamble privilegiado.

%Este ejemplo es útil para revisar el cálculo explícito de
%la matriz de densidad, específicamente para un estado mixto. Sin
%embargo, si ahora consideramos el ensamble $\{ \frac{1}{3}, \ket{0};
%\frac{2}{3}, \ket{1} \}$ notemos que \eqref{eq:rho-calc-ex} también es la
%matriz de densidad que describe este ensamble. Esto muestra que a más de un
%estado mixto \cpnote{Los estados mixtos son matrices de densidad. Quizá lo que
%quieres decir es que a mas de un ensamble le puede\ldots} le puede corresponder la misma matriz de densidad.\cpnote{En el
%codigo tienes un newline. No se para que lo tengas. Si quieres avisar explícitamente
%que viene un parrafo puede poner el comando ``par'', pero creo que es equivalente
%a una linea en blanco. El newline puede causar problemas. Lo comento por ahora.}
%  \newline


Ahora nos interesa revisar que la dinámica de un sistema cuántico
pueda ser descrito utilizando su operador de densidad. 
Supongamos que la evolución de un sistema 
cerrado se describe por un operador unitario $U$. 
Si el sistema inicialmente se encuentra en alguno de los estados 
$\ket{\psii}$, con probabilidad $p_i$, entonces después de que
la evolución ocurra el sistema se encontrará en el estado $U\ket{\psii}$ 
con probabilidad $p_i$, por consiguiente la evolución del operador de densidad
se describe mediante la siguiente ecuación
\begin{align}
	\rho = \sum _ip_i\dyad{\psii}{\psii}
	\xrightarrow[]{U}
	\sum _ip_iU\dyad{\psii}{\psii}U^{\dagger}	=
	U\rho U^{\dagger}.
\end{align}

Por otro lado, las mediciones también se pueden describir con el 
lenguaje del operador de densidad. Supongamos que se realiza una medición, 
que se define por los operadores de medida $M_m$, del estado de un sistema. 
Si el estado inicial del sistema era $\ket{\psii}$ antes de medirlo, 
entonces la probabilidad de obtener el resultado $m$ al realizar
la medición es
\begin{align}
	p\qty(m|i) &= \matrixel{\psii}{M_m^{\dagger}M_m}{\psii}\\
	&= \sum _k \braket{\psii}{k}\matrixel{k}{M_m^{\dagger}M_m}{\psii}\\
	&=\sum _k \matrixel{k}{M_m^{\dagger}M_m}{\psii}\braket{\psii}{k}\\
	&=\Tr \qty(M_m^{\dagger}M_m\dyad{\psii}{\psii}). \label{eq:p(mi)}
\end{align}
Por la ley de la probabilidad total la probabilidad de medir $m$ es
\janote{Carlos, debería referir hacia un apéndice para explicar esta ley?}
\begin{align}
	p(m) &= \sum_i p(m\vert i)p_i \\
			 &= \sum_i \Tr \qty(M_m^{\dagger}M_m\dyad{\psii}{\psii})p_i \\
			 &= \Tr \qty(M_m^{\dagger}M_m\sum_ip_i\dyad{\psii}{\psii}) \\
			 &= \Tr \qty(M_m^{\dagger}M_m\rho). \label{eq:p(m)}
\end{align}
Si el estado inicial del sistema era $\Pk{i}$ antes de medirlo, 
entonces el estado luego de obtener $m$ como resultado de la medición es
\begin{align}
	\ket{\psi_i^m} = \frac{M_m\Pk{i}}{\sqrt{\matrixel{\psii}{M_m^{\dagger}
	M_m}{\psii}}}.
\end{align}
Por consiguiente, luego de medir $m$ se tiene un ensamble de estados
$\ket{\psi_i^m}$ con probabilidades $p\qty(i\vert m)$. El operador
de densidad para este ensamble es
\begin{align}
	\rho_m &= \sum _i p\qty(i\vert m)\dyad{\psii^m}{\psii^m}
	\nonumber \\
				 &= \sum _i p\qty(i\vert m)
				 \frac{M_m\dyad{\psii}{\psii}M_m^{\dagger}}
				 {\matrixel{\psii}{M_m^{\dagger}M_m}{\psii}}. 
				 \label{eq:rho-after-measurement}
\end{align}
De la teoría de la probabilidad se sabe que
\begin{align}
	p\qty(i\vert m) = \frac{p(m,i)}{p(m)}=
	\frac{p\qty(m\vert i)p_i}{p(m)},
\end{align}
y, sustituyendo \eqref{eq:p(mi)} y \eqref{eq:p(m)} en 
\eqref{eq:rho-after-measurement}, el operador
de densidad asociado al ensamble de estados 
$\qty{\ket{\psi_i^m},p(i\vert m)}$, producto de obtener $m$ como resultado
de la medición, es
\begin{align}
	\rho _m &= \sum _i p_i \frac{M_m\dyad{\psii}{\psii}M_m^{\dagger}}
	{\Tr \qty(M_m^{\dagger}M_m\rho)} \nonumber \\
					&= \frac{M_m\rho M_m^{\dagger}}
	{\Tr \qty(M_m^{\dagger}M_m\rho)}.
\end{align}
Hasta aquí lo que hemos hecho es mostrar que los postulados de la
mecánica cuántica de evolución unitaria de los sistemas cerrados y 
medición se pueden reformular utilizando el operador de densidad.
Antes de proceder a formular todos los postulados en el lenguaje del 
operador de densidad vamos a presentar las propiedades generales
que esta herramienta posee. 
\cpnote{Aun no dices como es la formula para obtener el valor $i$ para la
matriz de densidad. }
\janote{No sé a qué haces referencia.}

% }}}

\section{Propiedades generales del operador de densidad} % {{{
%El operador de densidad fue introducido en la sección anterior
%como una herramienta para describir mezclas estadísticas de estados cuánticos,
%en esta sección vamos a iniciar presentando 
%la caracterización del operador de densidad, de modo que
%esto completa la formulación del nuevo formalismo de la mecánica cuántica
%que adoptaremos en este proyecto. Es importante resaltar que este
%formalismo es equivalente al del vector de estado. En vista de ello, 
%vamos a concluir reformulando los postulados de la mecánica cuántica 
%en el lenguaje del operador de densidad.

En esta sección vamos a establecer una caracterización del 
operador de densidad más abstracta que la introducida en 
la sección anterior y vamos a demostrar que es equivalente 
a la definición motivada en los ensambles de estados. 
Esto va a completar los fundamentos matemáticos
sobre la que recae la formulación de la mecánica cuántica 
utilizando el operador de densidad. 

%Intuitivamente, es útil definir a la matriz de densidad a partir de la 
%descripción de una mezcla estadística de estados. Sin embargo, la matriz 
%de densidad es un concepto más general e independiente de los ensambles de
%estados. Por ello, la matriz de densidad está caracterizada por el siguiente
%teorema.  
\cpnote{Está rara tu linea de pensamiento. Quieres hacer una definición independiente
del ensamble, pero usas las propiedades de ensamble para probar lo que dices. 
Creo que tu cabeza esta como en un nudo acá. Piensa que quieres decir, si quieres 
lo platicamos, y despues si lo escribes.}
\janote{Sí. Fui a leer el Sakurai y entiendo a lo que te refieres. Por eso 
reformulé el enfoque de este párrafo introductorio.}

Los operadores lineales que son operadores de densidad están 
caracterizados por el siguiente teorema:
\begin{teorema}[\textbf{Caracterización del operador de densidad}]
Un operador $\rho$  que actúa sobre el espacio de Hilbert de un sistema 
es el operador de densidad asociado a algún ensamble 
$\{p_i, \ket{\psi _i} \}$ si y sólo si satisface las condiciones:
\begin{enumerate}
\item $\Tr \rho = 1$.
\item $\rho \geq 0$.
\end{enumerate}	
\label{teo:density-operator}
\end{teorema}

\begin{proof}
	Ya que el teorema \ref{teo:density-operator} contempla un \textit{si
	y sólo si} se debe demostrar que un operador $\rho$ que cumple con 
	las condiciones establecidas en el teorema es el operador 
	de densidad asociado a un esamble de estados; además, también debemos
	mostrar la proposición inversa: un ensamble de estados
	tiene un operador de densidad que cumple con tales condiciones.
	
	Vamos a comenzar con la primera proposición. 
	Supongamos que $\rho = \sum p_i\dyad{\psii}{\psii}$ es un operador de
	densidad. Entonces 
	\begin{align*}
		\Tr \rho &= \sum _i p_i\Tr \qty(\dyad{\psii}{\psii})\\
		&= \sum _i p_i \sum _j \braket{\psi_j}{\psii}\braket{\psii}{\psi_j}\\
		&=\sum _i \sum _j p_i \delta _{ij}^2\\
		&=\sum _i p_i=1,
	\end{align*}
	por consiguiente la condición de traza unitaria de $\rho$ se cumple. 
	Supongamos que $\ket{\phi}$ es un estado arbitrario. Entonces
	\begin{align*}
		\matrixel{\phi}{\rho}{\phi} &= \sum _i p_i \braket{\phi}{\psii}
		\braket{\psii}{\phi} \\
		&=\sum _i p_i \abs{\braket{\phi}{\psii}}^2\\
		&\geq 0,
	\end{align*}
	por consiguiente $\rho$ es una matriz positiva semidefinida. 
	%\cpnote{Acá ya termino la prueba, te toca cortar la demostración en el texto.} 
 	
 	De manera inversa,
 	%\cpnote{No quieres decir por el contrario, sino quiza algo como ``de manera inversa'' }
 	supongamos que $\rho$ es una matriz que satisface las condiciones
	de traza unitaria y positividad. Dado que $\rho$ es positiva, entonces
	tiene descomposición espectral \cite{nielsen_chuang_2011} 
	\cpnote{cita} \janote{Listo}
	\begin{align*}
		\rho = \sum _k \lambda _k \dyad{k}{k},
	\end{align*}
	donde $\ket{k}$ satisfacen la relación de ortogonalidad y $\lambda _k$ son
	autovalores de $\rho$ reales y no negativos. Notamos que $\Tr \rho = \sum _k
	\lambda _k = 1$. Por consiguiente, un sistema en el estado $\ket{k}$ con 
	probabilidad $\lambda_k$ tendrá un operador de densidad $\rho$. Dicho 
	de otro modo, el ensamble de estados $\{ \lambda_k,\ket{k}\}$ 
	da lugar a la matriz de densidad $\rho$. 
	%\cpnote{Acá no es $p_k$ sino $\lambda_k$}.
\end{proof}

Es útil contar con un criterio para saber si un operador de densidad 
está asociado a un estado puro o mixto. Ese criterio es el siguiente:
sea $\rho$ un operador de densidad, entonces
$\Tr \qty(\rho^2)\leq 1$, donde la igualdad se satisface únicamente
para estados puros.

\begin{proof}
	Si $\rho$ es un operador de densidad, entonces $\sum_i^N \lambda_i=1$, 
	con $\lambda_i$ los autovalores de $\rho$ y $N$ la dimensión 
	del espacio de Hilbert del sistema. Se sigue que 
	$\Tr \qty(\rho^2) = \sum_i^N \lambda_i^2$. Por otro lado, 
	$1=\qty(\sum_i^N \lambda_i)^2=\lambda_1^2+\lambda_2^2+\ldots+\lambda_N^2+C$,
	donde $C$ son los términos producto entre distintas $\lambda_i$ y sus
	potencias. Si 
	$\qty(\sum_i \lambda_i)^2\geq\lambda_1^2+\lambda_2^2+\ldots+\lambda_N^2$,
	entonces
	$1\geq\lambda_1^2+\lambda_2^2+\ldots+\lambda_N^2$. Véase que 
	la igualdad se cumple cuando $N=1$, i.e. cuando el sistema 
	se encuentra en un estado puro. 
\end{proof}

Ahora que hemos establecido las características generales del 
operador de densidad vamos a exponer la formulación
de  los postulados de la mecánica cuántica en el lenguaje del 
operador de densidad.
\begin{itemize}
	\item[] \textbf{Postulado 1.} Un sistema físico tiene asociado un espacio vectorial complejo
	con producto interno que se conoce como el espacio de Hilbert $\hi$ del
	sistema. El sistema está completamente descrito por su operador de densidad,
	que es un operador $\rho$ positivo con traza unitaria, y este actúa sobre 
	$\hi$. 
%	Si un sistema cuántico se encuentra
%	en el estado $\rho _i$ con probabilidad $p_i$, entonces la matriz de
%	densidad del sistema es $\sum p_i\rho_i$.
\cpnote{Veo problemático usar las $\rho_i$ para definir rho. Quiza lo dejaria
sin esa ultima frase.}
\janote{De acuerdo.}
	\item[] \textbf{Postulado 2.} La evolución de un sistema cuántico cerrado está descrita por una transformación
	unitaria. Es decir, el estado $\rho$ del sistema en un tiempo $t_1$ está 
	relacionado con el estado $\rho'$ del sistema en un tiempo $t_2$ por un operador
	unitario $U$ que depende sólo de los tiempos $t_1$ y $t_2$,
	\begin{equation}
	\rho'=U\rho U^{\dagger}.
	\label{eq:postulate1}
	\end{equation}
	\item[] \textbf{Postulado 3.} Las mediciones cuánticas están descritas
por un conjunto de 
	operadores de medición $\{M_m\}$. Estos son operadores que actúan sobre el espacio 
	de estados del sistema a medir. El índice $m$ refiere a los resultados
	de las mediciones que puedan ocurrir en el experimento. Si el estado del sistema
	cuántico es $\rho$ inmediatamente antes de la medición, entonces la probabilidad
	de medir el resultado $m$ está dada por
	\begin{equation}
	p(m)=\Tr \qty(M_m^{\dagger}M_m\rho),
	\label{eq:postulate2_prob}
	\end{equation}						
	y el estado del sistema después de la medición es
	\begin{equation}
	\rho'=\frac{M_m\rho M_m^{\dagger}}{\tr \qty(M_m^{\dagger}M_m\rho)}.
	\label{eq:postulate2_rhoPrime}
	\end{equation}	
	Los operadores de medición deben satisfacer la ecuación de completitud
	\begin{equation}
	\sum _m M_m^{\dagger}M_m=\mathbb{1}.
	\label{eq:postulate2_completeness}
	\end{equation}
\cpnote{Quizá por acá también citaría y mencionaría la relación con las
mediciones de von Neumann}
\janote{Listo.}
	Un caso especial de este postulado es el de las mediciones proyectivas.
	Una medición proyectiva está descrita por un observable
	$A$, que es un operador Hermítico que actúa sobre $\hi$. $A$ tiene una
	descomposición espectral 
	\begin{align}
		A = \sum _a \lambda_aP_a,
	\end{align}
	donde $P_a$ es el proyector al autoespacio de $A$ con autovalor $\lambda_a$.
	De acuerdo con este postulado si el sistema se encuentra en el estado
	$\rho$, la probabilidad de medir $\lambda_a$ es
	\begin{align}
		p(a) = \Tr \qty(P_a^{\dagger}P_a\rho) = \Tr \qty(P_a\rho).
	\end{align}
	Dado que se midió $\lambda_a$ el estado del sistema inmediatamente 
	luego de realizar la medición es 
	\begin{align}
		\rho'&=\frac{P_a\rho P_a}{\Tr \qty(P_a \rho)}.
	\end{align}
	\item[] \textbf{Postulado 4.} El espacio de Hilbert de un sistema físico 
	compuesto es el producto tensorial de los espacios de Hilbert 
	individuales de cada uno de los sistemas que componen al sistema total.
	Es decir, si el sistema total se compone de $N$ subsistemas, entonces
	\begin{align}
		\hi_{\txt{total}} = \hi_1\ten \hi_2 \ten \ldots \ten \hi_N.
	\end{align}
%	\h{Además, si los sistemas están enumerados de 1 hasta $n$, y el 
%	$i$-ésimo sistema	se prepara en el estado $\rho_i$, 
%	entonces el estado del sistema total 
%	$\rho_1\otimes\rho_2\otimes\cdots\otimes\rho_n$.}
	Los estados del sistema compuesto están descritos por matrices de 
$N\times N$ Hermíticas, positivas y de traza unitaria que actúan 
sobre el espacio de Hilbert del sistema total.
\cpnote{Veo problemas acá. 
Por ejemplo, si dos qubits están en un estado de Bell, cada qubit está en el estado 
máximamente mixto, pero el estado global no es el producto tensorial de estos estados.}
\janote{Entiendo. Estoy dejando fuera los estados no factorizables. Redacté
de nuevo este postulado.}
\end{itemize}

Es nueva formulación de la mecánica cuántica nos permite describir
dos cosas que no sería posible utilizando la formulación del 
vector de estado:
sistemas cuánticos cuyos estados son mixtos y
subsistemas individuales de un sistema cuántico compuesto, como lo veremos
en la siguiente sección. 

% }}}
\section{El operador de densidad reducido} % {{{
Una de las aplicaciones más relevantes del operador de densidad es  
la de una herramienta que permite
describir subsistemas individuales de un sistema
cuántico compuesto: el operador de
densidad reducido. Es indispensable en el 
análisis de sistemas cuánticos compuestos. Esta sección la 
dedicamos a presentar la definición de esta herramienta
y justificar el uso de la traza parcial como única operación
para definir al operador de densidad reducido.

Supongamos que existen dos sistemas $A$ y $B$, cuyo estado está 
descrito por el operador de densidad $\rho^{AB}$. El
operador de densidad reducido para el sistema $A$ se define como
\begin{align}
	\rho^A \equiv \Tr _B\qty(\rho^{AB}),
	\label{eq:partialTrace-def}
\end{align}
donde $\Tr_B$ es un mapeo de operadores que se conoce como
la traza parcial sobre el sistema $B$. La traza parcial
se define como 
\begin{align}
	\Tr_B (\dyad{a_1}{a_2}\otimes \dyad{b_1}{b_2})
	\equiv
	\dyad{a_1}{a_2}\Tr \qty(\dyad{b_1}{b_2}),
	\label{eq:part_trace-def}
\end{align}
donde $\ket{a_1}$ y $\ket{a_2}$ son cualesquiera dos vectores en
$\mathcal{H}_A$, y $\ket{b_1}$ y $\ket{b_2}$ cualesquiera dos vectores
en $\mathcal{H}_B$. La definición de traza parcial, establecida
en \eqref{eq:part_trace-def}, se completa requiriendo que sea 
una operación lineal.

\cpnote{Me gustaría que entendieras la motivación física de esta definición.
Revisa el box del libro de Chuang, entiendelo, y escribe no la prueba pero si
el resultado fundamentental que muestran ahi. Si no lo entiendes, lo
discutimos. }
\janote{Muy bien. A continuación:}

La razón por la que se utiliza la traza parcial para definir al operador
de densidad reducido es porque
es la única operación que describe correctamente 
las mediciones de los observables sobre subsistemas que forman parte de un
sistema compuesto. A continuación exponemos la justificación.

Supongamos que $M$ es un observable sobre el sistema $A$ y que
contamos con un aparato de medición que es capaz de realizar mediciones
de $M$. Sea $\tilde{M}$ el observable para la misma medición, pero 
es el operador que actúa sobre el sistema compuesto $AB$. 
Notemos que si el sistema
$AB$ se prepara en el estado $\ket{m}\otimes \ket{\psi}$, donde $\ket{m}$
es un autoestado de $M$ con autovalor $m$, y $\ket{\psi}$ es 
cualquier estado de $B$, entonces el aparato de medición 
debe obtener $m$ como resultado de la medición con probabilidad igual a uno.
De manera que, si $P_m$ es el proyector hacia el autoespacio $m$ del
observable $M$, entonces el proyector correspondiente para $\tilde{M}$
es $P_m\otimes \mathbb{1}_B$. Por consiguiente tenemos
\begin{align}
	\tilde{M} = \sum_m mP_m\otimes \mathbb{1}_B=M\otimes \mathbb{1}_B.
\end{align}
Lo que sigue es probar que la traza parcial da correctamente las 
mediciones estadísticas para observaciones realizadas sobre una 
parte del sistema. Supongamos que se realiza una medición sobre
el sistema $A$ descrito por el observable $M$. Para tener 
consistencia física se requiere que cualquier asociación de un estado,
$\rho^A$, al sistema $A$, debe tener la propiedad de que los promedios
en la medición sean los mismo calculados a través de $\rho^A$ o $\rho^{AB}$,
del estado del subsistema o del estado del sistema completo;
es decir que (recordemos que la probabilidad de una medición 
se describe mediante la ecuación \eqref{eq:postulate2_prob}
y que los proyectores satisfacen $P^2=P$)
\begin{align}
	\Tr \qty(M\rho^A) = \Tr \qty(\tilde{M}\rho^{AB}) = 
	\Tr \qty[\qty(M\otimes\mathbb{1}_B)\rho^{AB}].
	\label{eq:partial-trace}
\end{align}
Notemos que esta ecuación se satisface con la definición
\eqref{eq:partialTrace-def}
\begin{align}
	\Tr \qty[\qty(M\otimes\mathbb{1}_B)\rho^{AB}] 
	&= \sum_{i,j} \bra{i}\ten\bra{j}	\qty(M\ten \sum_k\dyad{k}{k})
	\rho^{AB}
	\ket{i}\ten\ket{j} \nonumber \\
	&= \sum_{i,j,k} \bra{i}\braket{j}{k}M\matrixel{k}{\rho^{AB}}{j}\ket{i}
  \nonumber \\
	&= \sum_{i} \bra{i}M\sum_j \matrixel{j}{\rho^{AB}}{j}\ket{i} \nonumber \\
	&= \sum_i \bra{i}M\Tr_B\qty(\rho^{AB})\ket{i} \nonumber \\
	&= \Tr\qty(M \rho^A).
\end{align}
De hecho, la tracia parcial es la única función que tiene esta propiedad.
Para verificarlo supongamos que $f$ es un mapeo de operadores de 
densidad que actúan sobre el sistema AB hacia operadores de densidad 
que actúan sobre A, tal que 
\begin{align}
	\Tr \qty(Mf\qty(\rho^{AB})) = \Tr \qty[\qty(M\ten \1)\rho^{AB}]
\end{align}
para cualquier observable $M$. Sea $M_i$ una base ortonormal de operadores
del espacio de los operadores Hermíticos con respecto del producto interno
de Hilbert-Schmidt $\qty(X,Y)\equiv \Tr\qty(XY)$. Entonces escribiendo
$f\qty(\rho ^{AB})$ en esta base
\begin{align}
	f\qty(\rho^{AB}) 
	&= \sum_i \Tr\qty[M_if\qty(\rho^{AB})]M_i \\
	&= \sum_i \Tr\qty[\qty(M_i\ten \1)\rho^{AB}]M_i.
\end{align}
Por consiguiente, el único mapeo $f$ que satisface esta condición
es la traza parcial, determinada en \eqref{eq:partial-trace}.

Antes de concluir revisaremos un ejemplo del cálculo del 
operador de densidad reducido que evidencia una de las
\textit{extrañas} consecuencias del
entrelazamiento cuántico.
Consideremos un sistema de 2 qubits que se encuentra en el estado de Bell
$\qty(\ket{00}-\ket{11})/\sqrt{2}$, un estado entrelazado. El operador
de densidad para el sistema compuesto es
\begin{align}
	\rho &= \qty(\frac{\ket{00}-\ket{11}}{\sqrt{2}})
	\qty(\frac{\bra{00}-\bra{11}}{\sqrt{2}}) \\
			 &= \frac{\dyad{00}{00}-\dyad{11}{00}-\dyad{00}{11}+\dyad{11}{11}}{2}.
\end{align}
Calculamos ahora el operador de densidad reducido para el qubit 1.  
Haciendo la traza parcial sobre el qubit 2 se tiene
\begin{align}
	\rho^A &= \Tr _B(\rho) \\
			 	 &= \frac{\Tr _B(\dyad{00}{00})-\Tr _B(\dyad{11}{00})
			 	 -\Tr _B(\dyad{00}{11})+\Tr _B(\dyad{11}{11})}{2} \\
			 	 &= \frac{\dyad{0}{0} + \dyad{1}{1}}{2} \\
			 	 &= \frac{\mathbb{1}}{2}.
\end{align}
Notemos entonces que el estado del qubit 1 se encuentra en un estado mixto
$\qty(\Tr \qty(\mathbb{1}/2)<1)$, 
es decir un estado del cual no contamos con la mayor información 
sobre el qubit 1. 
Aunque el estado del sistema compuesto es un estado puro el estado
del primero, y también del segundo qubit, es un estado mixto. 

Con el operador de densidad reducido hemos completado
el marco teórico del lenguaje del operador de densidad necesario
para el objetivo de este proyecto. Comenzamos
motivando el operador de densidad como herramienta 
para describir ensambles de
estados cuánticos, luego presentamos las características
del operador de densidad y la reformulación de los postulados 
de la mecánica cuántica.
Finalmente, concluimos introduciendo el operador de densidad reducido y
justificando el uso de la operación de traza parcial. Ahora, dirigimos
nuestra atención hacia la descripción de la dinámica de los estados
de sistemas cuánticos abiertos, el tipo de sistemas que son de interés
para este proyecto.

% }}}

%\section{El espacio de las matrices de densidad}  % {{{
%\janote{Luego de que me dijiste que no mate con este trabajo la tesis
%me parece que esta sección podría estar de más y guardarla para la tesis. 
%Igual lo dejo para que de una vez lo puedas ver.}
%\cpnote{Esta sección se ve poco pulida. Me gustaría que le dieras una iterada antes de 
%revisarla de nuevo. Trata de dejar lo importante (por ejemplo el factor de escala no
%y otras cosas no. Si no necesitas simbolos, es mejor no meterlos. Quizá dejo aca la
%revision pues lo que haces luego depende un poco de lo que haces acá.}
%% }}}
%\section{El espacio de Hilbert-Schmidt} % {{{
%% Intro {{{
%A un espacio de Hilbert $\mathcal{H}$ complejo de dimensión $N$
%le acompaña su espacio dual $\mathcal{H}^*$,
%el espacio de las transformaciones lineales de $\mathcal{H}$
%al campo de los números complejos $\mathbb{C}$. Otro espacio disponible 
%a considerar es el de los operadores lineales que actúan sobre $\mathcal{H}$. 
%Cuando este espacio se equipa con 
%\begin{equation}
%\langle A,B\rangle=c\Tr \qty(A^{\dagger}B),
%\label{eq:HS_innerP}
%\end{equation}
%donde $c\in \mathbb{R}$ establece una escala, se conoce como el espacio de
%Hilbert-Schmidt $\mathcal{HS}$. 
%
%El espacio vectorial de los operadores Hermíticos 
%$\mathcal{HM}$ es un subespacio real de $N^2$ dimensiones de $\mathcal{HS}$. 
%$\mathcal{HM}$ también se puede pensar como el álgebra de Lie de $U(n)$. Los
%operadores Hermíticos con traza cero forman un subespacio vectorial de
%$\mathcal{HM}$. Por tanto, es posible encontrar una base ortonormal
%$\sigma_i$ con
%respecto al producto interno \eqref{eq:HS_innerP}. Al agregar la 
%matriz identidad un operador hermítico $A$ puede escribirse de la siguiente
%manera
%\begin{align}
%	A &= \tau_0\sqrt{\frac{2}{N}}\mathbb{1} + \sum _{i=1}^{N^2-1}\tau_i\sigma_i,
%\end{align}
%donde $\tau_0=\frac{\Tr A}{\sqrt{2N}}$ y $\tau_i=\Tr \sigma _iA$.
% 
%Por otro lado, el conjunto de operadores positivos $\mathcal{P}$ es un 
%conjunto convexo contenido dentro de $\mathcal{HM}$ y que
%\begin{align}
%	\dim \qty[\mathcal{P}] = \dim \qty[\mathcal{HM}]=N^2.
%\end{align}
%% }}}
%\subsection{Coordinizando  $\rho$} % {{{
%\cpnote{Esa palabra no existe, creo. Quiza parametrización} 
%
%Finalmente, el conjunto de las matrices de densidad consiste en todos los operadores
%positivos $\rho$ con traza unitaria. Este conjunto se denota como 
%$\M$, donde el exponente enfatiza que el conjunto consiste en 
%matrices de $N\times N$. $\M$ es un conjunto convexo que, en $\mathcal{HM}$, 
%es la intersección del conjunto de los operadores positivos con un 
%hiperplano paralelo al subespacio lineal de los operadores hermíticos
%con traza igual a cero \cite{bengtsson_zyczkowski_2017}.
%
%Una manera posible de \textit{coordinizar} $\M$ es 
%\begin{align}
%	\rho = \frac{1}{N} \mathbb{1} + \sum _{i=1}^{N^2-1} \tau_i\sigma_i,	
%	\label{eq:rho_general}
%\end{align}
%donde la matriz cero se ha cambiado por la matriz
%\begin{align}
%	\rho _{\star} \equiv \frac{1}{N}\mathbb{1},
%	\label{eq:max_mixed_state}
%\end{align}
%que se conoce como el estado máximamente mixto o la `matriz de ignorancia'. 
%Las componentes $\tau_i$ en \eqref{eq:rho_general} se conocen como `coordenadas
%de mezcla'.
%% }}}
%% }}}
%\section{Transformaciones unitarias} % {{{
%
%Teorema de Kadison \cite{bengtsson_zyczkowski_2017}:
%\begin{teorema}[\textbf{Teorema de Kadison}] 
%	Supongamos un mapeo $\Phi:\M \mapsto \M$ que es uno-a-uno y que es afín y
%	que preserva la estructura compleja en el sentido que
%	\begin{align}
%		\Phi \qty(p\rho_1 + \qty(1-p)\rho_2) = p\Phi(\rho_1) + 
%		\qty(1-p)\Phi(\rho_2).
%	\end{align}
%	Entonces el mapeo debe tomar la forma
%	\begin{align}
%		\Phi(\rho) = U\rho U^{-1},
%	\end{align}
%	donde el operador $U$ es unitario o antiunitario.
%\end{teorema}
%
%Este teorema nos permite ver que para que un mapeo preserve la estructura 
%convexa debe ser un mapeo afín y que debe mapear estados putos
%a estados puros. Por ejemplo, para $N=2$ los estados puros forman una
%esfera. Los únicos mapeos afines que preservan la esfera son las rotaciones.

% }}}
% }}}