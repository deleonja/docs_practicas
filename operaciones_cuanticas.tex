\chapter{Operaciones cuánticas} % {{{

Adoptando la descripción de estados cuánticos en términos de la matriz de 
densidad $\rho$ podemos describir la dinámica de ellos como
\begin{align*}
	\rho '=\mathcal{E} (\rho).
\end{align*}
El mapeo $\mathcal{E}$ define a una operación cuántica. Esta es la 
representación matemática de la evolución dinámica que ocurre como
resultado de un proceso físico, $\rho$ es el estado inicial del 
sistema y $\mathcal{E}(\rho)$ el estado final.

Una manera natural de describir la dinámica de un sistema cuántico
abierto es considerarlo que resulta de la interacción del sistema
de interés, el sistema principal, y un entorno, que juntos forman
un sistema cuántico cerrado. En general, el estado final 
del sistema $\E (\rho)$ no está relacionado al estado inicial
mediante una transformación unitaria. Vamos a asumir que
el estado del sistema completo se encuentra en un estado
producto, $\rho \otimes \sigma$, donde $\sigma$ es el 
estado en el que se encuentra el entorno. Luego de la
transformación $U$ el sistema ya no interactúa con el 
entorno y por ello realizamos una operación de traza parcial sobre 
el entorno para obtener el estado reducido del sistema principal:
\begin{align}
	\E (\rho) = \Tr _{env}\qty[U \qty(\rho \otimes \sigma)U^{\dagger}].
	\label{eq:PTraceE(rho)}
\end{align}
Esta definición de operación cuántica puede tener problemas
de generalidad dado que se hizo la suposición de que el 
estado inicial del sistema total era un estado producto. En 
general este no es el caso, sin embargo el formalismo de las 
operaciones cuánticas describe también la dinámica cuántica
de sistemas que no se encuentran inicialmente en un estado
producto.

Una definición más apropiada de las operaciones cuánticas es
definirlo como la clase de mapeos que surgen como resultado de
los siguientes procesos: algún sistema inicial se prepara en un
estado cuántico desconocido $\rho$ y luego se pone en contacto
con otros estados preparados en estados estándar, permitiendo 
la interacción mediante alguna evolución unitaria y luego
se desecha alguna parte del sistema total, dejando así 
sólo al sistema final en algún estado $\rho'$. En conclusión, 
una operación cuántica $\E$ es un mapeo de $\rho$ a $\rho'$.

\section{Representación de suma de operadores}
Las operaciones cuánticas se pueden representar de una manera
conocida como la representación de operadores de suma. Esta
representación es reescribir la ecuación \eqref{eq:PTraceE(rho)}
en término de operadores que actúan sobre el sistema de Hilbert
principal. EL resultado está motivado por el siguiente cálculo. 
Supongamos que $\ket{e_k}$ es una base ortonormal para el
espacio de estados del entorno y sea $\sigma = \dyad{e_0}{e_0}$
el estado inicial del entorno. No hay perdida de generalidad si 
asumimos que el entorno comienza en un estado producto. De 
esta manera, la ecuación \eqref{eq:PTraceE(rho)} se convierte en
\begin{align*}
	\E (\rho) &= \sum _k \bra{e_k}U\qty[\rho \otimes \dyad{e_0}{e_0}]U^{\dagger}
	\ket{e_0} \\
						&= \sum _k E_k\rho E_k^{\dagger},
\end{align*}
donde $E_k\equiv \matrixel{e_k}{U}{e_k}$ es un operador que actúa
sobre el espacio de estados del sistema principal. Los operadores
$E_k$ se conocen como elementos de operación para la operación 
cuántica $\E$. 

\begin{align*}
	1 &= \Tr \qty(\Erho) \\
		&= \Tr \qty(\sum _k E_k\rho E_k^{\dagger}) \\
		&= \Tr \qty(\sum _k E_k^{\dagger}E_k \rho),
\end{align*}
esto se debe cumplir para cualquier $\rho$, entonces 
\begin{align}
	\sum _k E_k^{\dagger}E_k = \mathbb{1}.
\end{align}
Esta ecuación la satisfacen las operaciones cuánticas que preservan
la traza. No obstante, también existen operaciones cuánticas que 
no preservan la traza, las cuales $\sum _kE_k^{\dagger}E_k\leq \mathbb{1}$,
pero describen procesos en los que información extra sobre lo que ocurrió
en el proceso se obtiene por medición. 

La representación en operadores de suma es importante porque ofrece
una manera intrínseca de caracterizar la dinámica del sistema principal. 
Este formalismo describe la dinámica del sistema principal sin hacerse
necesario considerar de manera explícita las propiedades del entorno. 
Esto provee de implicaciones teóricas considerables. 

\subsection{Enfoque axiomático}
Ahora vamos a adoptar una ruta distinta para las operaciones cuánticas
en la que vamos a intentar enunciar axiomas motivados físicamente que
esperamos que cumplan las operaciones cuánticas. Esta ruta es por supuesto
más abstracta que la anterior, pero es esta misma abstracción lo que hace
este enfoque muy poderoso.

Para proceder con este enfoque vamos a definir a las operaciones 
cuánticas de acuerdo con un conjunto de axiomas justificados en
bases físicas. Luego, vamos a probar que un mapeo $\E$ satisface
estos axiomas si y sólo si tiene una representación de operadores
de suma. De esta manera se provee del puente teórico entre 
la formulación axiomática y la formulación motivada físicamente.

Vamos definir una operación cuántica $\E$ como un mapeo del conjunto
de los operadores de densidad del espacio de entrada $Q_1$ al 
conjunto de los operadores de densidad del espacio de salida $Q_2$,
con las siguientes tres propiedades axiomáticas:
\begin{description}
	\item[A1] \label{axiom1}
	$\Tr \qty[\E (\rho)]$ es la probabilidad de que un proceso
	representado por $\E$ suceda, cuando $\rho$ es el estado inicial. 
	De esa manera, $0\leq \Tr \qty[\E (\rho)]\leq 1$ para cualquier
	estado $\rho$.
	\item[A2] $\E$ es un mapeo lineal convexo sobre el conjunto de las 
	matrices de densidad. Es decir que, para un conjunto
	de probabilidades $\{ p_i\}$,
	\begin{align}
		\E \qty(\sum _i p_i\rho_i) = \sum _i p_i \E \qty(\rho_i).
		\label{eq:qtmOp-a2}
	\end{align}
	\item[A3] $\E$ es un mapeo completamente positivo. Es decir, si $\E$ 
	mapea operadores de densidad del sistema $Q_1$ en operadores de
	densidad del sistema $Q_2$, entonces $\E (A)$ debe de ser positivo
	para cualquier operador $A$. Además, si se introduce un sistema
	extra $R$ de dimensión arbitraria, debe ser cierto que $\qty(
	\mathbb{1}\otimes \E)(A)$ es un operador positivo para cualquier
	operador $A$ sobre el sistema combinado $RQ_1$, donde $\mathbb{1}$
	denota el mapeo de la identidad sobre el sistema $R$.
\end{description}

Una operación cuántica física es aquella que satisface el 
requisito de que las probabilidades nunca suman más de 1,
$\Tr \qty[\E (\rho)]\leq 1$.

La segunda propiedad proviene de un requisito físico sobre las
operaciones cuánticas. Supongamos que la entrada $\rho$ de la 
operación cuántica se obtiene aleatoriamente seleccionando un 
estado del ensamble $\qty{ p_i,\rho_i}$ de estados cuánticos, 
es decir que $\rho = \sum_i p_i\rho_i$. Entonces se espera que
el estado resultante, $\E (\rho)/\Tr \qty[\E (\rho)] =
\E (\rho)/p\qty(\E)$ corresponda a una selección aleatoria del
ensamble $\qty{p\qty(i\vert\E), \E (\rho)/\Tr \qty[\E (\rho)]}$,
donde $p\qty(i\vert\E)$ es la probabilidad de que el estado
preparado fuese $\rho_i$, dado que el proceso representado por
$\E$ ocurrió. De esa manera, se debe querir que
\begin{align}
	\E (\rho) = p\qty(\E)\sum _i p\qty(i\vert \E)\frac{\E (\rho)}
	{\Tr \qty[\E (\rho)]},
	\label{eq:e(rho)-bayes}
\end{align}
donde $p\qty(\E)=\Tr \qty[\E (\rho)]$ is la probabilidad de que 
el proceso descrito por $\E$ suceda en alguna entrada de $\rho$.
Por la regla de Bayes,
\begin{align}
	p\qty(i\vert \E)=p\qty(\E \vert i) \frac{p_i}{p\qty(\E)}
	=\frac{\Tr \qty[\E (\rho)]p_i}{p\qty(\E)},
\end{align}
de manera que \eqref{eq:e(rho)-bayes} se reduce a \eqref{eq:qtmOp-a2}.

La tercera propiedad se origina de que no sólo $\E (\rho)$ debe ser
una matriz de densidad válida tanto cuánto $\rho$ sea válida, pero 
además, si $\rho _{RQ}$ es la matriz de densidad de un sistema 
de dos partes $R$ y $Q$, si $\E$ actúa sólamente sobre $Q$, 
entonces $\E(\rho_{RQ})$ debe ser también una matriz de densidad 
válida del sistema completo. Supongamos que introducimos un sistema 
$R$ finito dimensional. Sea $\mathbb{1}$ el mapeo identidad sobre
el sistema $R$. Entonces el mapeo $\mathbb{1}\otimes \E$ debe
enviar operadores positivos hacia operadores positivos.

El siguiente teorema establece la equivalencia entre este enfoque 
axiomático y los modelos de sistema-entorno y la representación
de operadores de suma con el que se comenzó
la discusión de las operaciones cuánticas:
\begin{teorema}
	El mapeo $\E$ satisface los axiomas \textbf{A1}, \textbf{A2} y 
	\textbf{A3} si y sólo si
	\begin{align}
		\E(\rho) = \sum _i E_i\rho E_i^{\dagger},
	\end{align}
	para algún conjunto de operadores $\qty{E_i}$ que mapean el espacio
	de Hilbert de entrada hacia el espacio de Hilbert de salida, y
	$\sum _kE_k^{\dagger}E_k\leq \mathbb{1}$.
\end{teorema}


\section{Ejemplos de operaciones cuánticas}

\section{Mediciones y POVM's}

\section{Matrix reshaping y reshuffling}

\section{Mapeos positivos y completamente positivos}

\section{Representaciones del entorno}

\section{Mapeos de un qubit}


% }}}

