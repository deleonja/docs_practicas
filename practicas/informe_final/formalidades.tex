\section*{Objetivo General}
Identificar los mapeos CP  que preservan la traza que borran
componentes arbitrarias de la matriz de densidad $\rho$ de 
un sistema de $n$ qubits. 	

\section*{Objetivos Específicos}
\begin{itemize}
\item Entender la definición y propiedades de la matriz de densidad, así como
	  los fundamentos de la mecánica cuántica utilizando este lenguaje.
\item Entender la condición de completa positividad con la que deben cumplir 
	  los mapeos que actúan sobre matrices de densidad para asegurar que estos 
	  transforman estados cuánticos válidos en estados válidos de sistemas 
	  cuánticos abiertos.
\item Entender los mapeos que borran componentes de la matriz de densidad
	  de un sistema de 1 qubit.
\item Escribir un programa que construya de numéricamente los mapeos que
	  borran componentes arbitrarias de la matriz de densidad de un sistema 
	  de $n$ qubits.
\item Discriminar numéricamente los mapeos no físicos del conjunto de mapeos
	  que borran componentes de la matriz de densidad verificando 
	  la condición de completa positividad mediante el uso del teorema de Choi. 
\end{itemize}

\section*{Introducción}
Durante la licenciatura, los cursos de mecánica cuántica se limitan a 
tratar la descripción matemática de los sistemas cuánticos cerrados,
sin embargo, los sistemas abiertos son de gran interés teórico
y experimental para la investigación actual. Los sistemas cuánticos 
reales son sistemas que están en alguna medida abiertos a 
interacciones con su entorno. Por tal la razón, la descripción 
completa de un sistema cuántico requiere incluir a su entorno 
y considerar que este sistema secundario es inaccesible parcial 
o totalmente, o que no es de interés para el observador 
\cite{schlosshauer2007decoherence}.

La teoría de las operaciones cuánticas propone un formalismo para
describir la evolución de los sistemas abiertos de manera discreta
\cite{nielsen_chuang_2011}. 
Quizás un nombre más descriptivo para este formalismo es 
``operaciones completamente positivas que preservan la traza",
pues la condición de completa positividad (CP) es el rasgo característico 
de una operación cuántica. La CP es una condición más fuerte  
a la de la positividad y que, en el contexto de la
mecánica cuántica y los sistemas abiertos, asegura que, aún 
cuando el sistema principal se encuentre inicialmente enlazado 
con un sistema secundario, una operación cuántica preservará 
la positividad del estado del sistema. Las operaciones cuánticas 
son una herramienta que permiten ir más allá de la descripción 
de la evolución los sistemas ideales. 

El problema que nos interesa estudiar en el marco de la teoría de 
las operaciones cuánticas es el de caracterizar a las
operaciones que borran componentes de la matriz de densidad
de un sistema de qubits escrita en la base de los productos 
tensoriales de las matrices de Pauli. Un qubit es un sistema cuántico de dos niveles, 
como una partícula de espín 1/2 o la polarización de la luz. 
Durante este trabajo de prácticas se estudió el caso de 1 qubit de las 
operaciones que borran las componentes del vector de Bloch. Esto 
servirá como preámbulo y preparación de un estudio posterior 
de sistemas de más qubits y la caracterización general que este
tipo de operaciones debe cumplir para ser canales cuánticos.
Uno de los objetivos para este trabajo fue reproducir los
resultados para el caso de 1 qubit que fueran 
consistentes con la caracterización general de 
los canales cuánticos de 1 qubit que otros autores han 
estudiado previamente 
\cite{bengtsson_zyczkowski_2017,nielsen_chuang_2011}.

La estructura de este informe es la siguiente. En el capítulo 1
se presenta la revisión y estudio de la bibliografía que se realizó 
sobre el formalismo de la matriz de densidad de la mecánica cuántica, 
herramienta utilizada para describir 
al estado de un sistema cuántico. En el capítulo 2 se expone 
la revisión y estudio bibliográfico del formalismo de las 
operaciones cuánticas. Se estudió la completa positividad, 
dos representaciones distintas y los ejemplos más relevantes 
de 1 qubit de las operaciones cuánticas. En el último capítulo
se formula el enunciado del problema de las operaciones que borran 
componentes (información) de la matriz de densidad de un sistema de 
qubits. Luego, se presentan los métodos analítico y numérico 
para resolver el problema de 1 qubit, así como los resultados 
que se obtuvieron, y se comparan con lo que se encuentra actualmente
en la literatura de las operaciones cuánticas. Este trabajo 
es el preludio de la tesis de licenciatura, en la que se continuará 
estudiando este problema para sistemas de más de 1 qubit.


\section*{Justificación}
Estudiar los conceptos de la matriz de densidad y de las operaciones completamente
positivas que preservan la traza constituye una herramienta que permite 
entender la teoría cuántica desde un punto de vista distinto al que se estudia
en los cursos de licenciatura y, al mismo tiempo, comprender los fundamentos 
para estudiar los sistemas cuánticos abiertos. 

Por otra parte, este proyecto ofrece la oportunidad de poner en práctica habilidades
computacionales aprendidas en cursos de licenciatura, así como de 
aprender nuevas habilidades que sean necesarias para cumplir con los objetivos
de este trabajo. La adquisición de este tipo de habilidades son de provecho para la formación 
profesional en diversas áreas de investigación en física. 

El trabajo propuesto para estas prácticas finales es el inicio del proyecto
que se busca continuar en el trabajo de graduación de licenciatura. 
Estas prácticas servirán de preparación teórica y, al mismo tiempo, 
para investigar el caso más sencillo del estudio de las operaciones 
que borran componentes de la matriz de densidad de un sistema 
de qubits.

En fin, este proyecto es importante para estudiar los fundamentos teóricos
de un área de investigación importante en información cuántica. 
Por otro lado, este proyecto servirá como un preludio para un futuro trabajo de
tesis de graduación y un proyecto de investigación, cuya experiencia a adquirir durante
dicho proyecto complementaría la formación de un estudiante de licenciatura. 


