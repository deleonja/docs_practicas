\chapter{Conclusiones y trabajo futuro}
\section{Conclusiones}
Este trabajo de prácticas sirvió para establecer el marco teórico dentro del 
cual se puede abordar el estudio de las operaciones lineales que borran 
componentes del vector de Bloch de un sistema de qubits. Se estudió 
el formalismo de la matriz de densidad ya que es una herramienta 
para la descripción de los sistemas abiertos. Este formalismo
proporciona  una herramienta matemática en la que contiene 
toda la información física disponible de un subsistema: la 
matriz de densidad reducida.
Para la descripción de la evolución de los estados
cuánticos se estudió la teoría de las operaciones cuánticas. Las 
operaciones cuánticas se definen como operaciones completamente
positivas que preservan la traza (operaciones CPTP) y son una 
forma discreta para describir la dinámica de los sistemas abiertos.
Con estas herramientas teóricas una gran variedad de problemas 
se pueden abordar, sin embargo este trabajo se enfocó en ser
el inicio del estudio de un tipo muy específico de operaciones lineales de qubits.

El problema que se estableció en este trabajo fue el de estudiar 
las operaciones PCE en sistemas de qubits que cumplen con las condiciones
para ser canales cuánticos. Específicamente, se estudió el caso de
1 qubit. Una operación PCE (\textit{Pauli component erasing}) es una 
operación lineal que borra las componentes del vector de Bloch 
(generalizado, en el caso de $n$ qubits). El problema en el contexto
de los sistemas cerrados es trivial, todas las operaciones PCE representan
dinámicas físicas que podrían atravesar los estados cuánticos. Sin embargo,
para el caso de los sistemas abiertos la condición de completa positividad 
debe ser cumplida por una operación PCE para que describa una evolución
física. Se diseñó un método numérico para evaluar la condición de CP 
de las operaciones PCE y se encontró que 5 de las 8 operaciones PCE 
de 1 qubit son operaciones cuánticas: la identidad, el canal totalmente
depolarizante y tres canales que mapean la esfera de Bloch a una 
línea sobre cada uno de los ejes. El estudio de las operaciones PCE 
de 1 qubit sentaron las bases para el estudio de una caracterización 
general de las operaciones PCE que se pretende continuar en el trabajo
de graduación de licenciatura.

% Para abordar el problema de las operaciones que borran componentes 
% del vector de Bloch de 1 qubit fue necesario estudiar el formalismo de 
% la matriz de densidad y la teoría de las operaciones cuánticas. 
% \cpnote{Sigue estando mal\ldots esto es una lista. Yo quizá quitaría 
% el resto del parrafo menos la ultima frase. trata de que sea fluido}
% \janote{Ok, reescribí en los primeros 2 párrafos y los que estaban 
% se borran}
% La matriz de densidad es una herramienta que contiene toda la información 
% física que se puede extraer de un estado cuántico. La teoría de las 
% operaciones cuánticas es un marco teórico para describir 
% la evolución dinámica de los sistemas cuánticos abiertos. 
% Una operación cuántica es una operación completamente positiva 
% que preserva la traza de la matriz de densidad (operación CPTP).
% La completa positividad es una condición más robusta a la positividad 
% que asegura que una operación cuántica preserva la positividad de todos 
% los estados de un sistema, especialmente la de los estados entrelazados.
% Con estas herramientas teóricas se pueden abordar una gran variedad
% de problemas, sin embargo en este trabajo nos enfocamos en iniciar el 
% estudio de un tipo muy específico de operaciones lineales de qubits.
% 
% En este proyecto nos enfocamos en estudiar las operaciones PCE 
% de 1 qubit que satisfacen la condición de completa positividad
% y, por ende, son canales cuánticos. Una operación PCE 
% de 1 qubit es una operación lineal que borra componentes del 
% vector de Bloch. Existen 8 operaciones PCE de 1 qubit y 
% numéricamente se encontró que 5 de ellas son 
% canales cuánticos: la identidad, el canal totalmente depolarizante 
% y los canales que mapean la esfera de Bloch a una línea sobre 
% cualquiera de los ejes. Estos resultados son consistentes 
% con los canales cuánticos de 1 qubit que han sido estudiados antes 
% por otros autores como Nielsen y
% Bengtsson~\cite{bengtsson_zyczkowski_2017,nielsen_chuang_2011}.

\section{Trabajo futuro}
% \janote{Modifiqué este párrafo para que no pareciera lista también}
Para el trabajo de tesis de graduación proponemos continuar con el
estudio de las operaciones PCE. El objetivo a largo plazo es caracterizar
de manera general a las operaciones PCE en sistemas de qubits. 
A corto plazo se debe estudiar el caso de 2 qubits y analizar esos 
resultados para determinar pistas que guíen el camino a: (1) un
estudio sistemático del caso de 3 qubits, y (2) las características generales
que deben cumplir las operaciones PCE para ser canales cuánticos.
Por otro lado, puede ser útil revisar los estudios recientes sobre operaciones
de Pauli para estudiar las similitudes que puedan guardar con las operaciones 
PCE. Aún resta trabajo por hacer para conseguir la caracterización general
de las operaciones PCE, pero con este trabajo de prácticas se consiguió 
establecer el marco teórico y el inicio del estudio de este tipo de operaciones.

