\documentclass[11pt]{beamer}

\usetheme{Madrid}


\usepackage[utf8]{inputenc}
\usepackage[spanish]{babel}
\usepackage{amsmath}
\usepackage{amsfonts}
\usepackage{amssymb}
\usepackage{graphicx}
\usepackage{textpos}
\usepackage{bbold}
\usepackage{physics}


\title[Mapeos que borran componentes de $\rho$]{Mapeos CPTP que
borran componentes de la matriz de densidad de un sistema de 
qubits}
\author[J. A. de León]{José Alfredo de León}
%\setbeamercovered{transparent} 
\setbeamertemplate{navigation symbols}{} 
%\logo{\includegraphics[height=1.5cm]{logoECFM.png}\vspace{0.5pt}} 
\institute[ECFM-USAC]{Escuela de Ciencias Físicas y Matemáticas, USAC} 
\date[xx.09.2020]{xx de septiembre, 2020} 
 

\AtBeginSection[]
{
\begin{frame}{Outline}
\tableofcontents[currentsection]
\end{frame}
}

\begin{document}

\begin{frame}
\titlepage
\end{frame}

\addtobeamertemplate{frametitle}{}{%
\begin{textblock*}{100mm}(.90\textwidth,6.9cm)
%\includegraphics[height=1.2cm]{logoECFM.png}
\end{textblock*}}

\section{Fundamentos teóricos}
\begin{frame}{Qubits}
Los qubits son la unidad básica de información de la computación cuántica.
\end{frame}

\begin{frame}{La matriz de densidad}
La matriz de densidad $\rho$ es un operador positivo con traza unitaria. 
\end{frame}

\begin{frame}{La esfera de Bloch}
Hola
\end{frame}

\begin{frame}{Mapeos CPTP}
¿Qué son los mapeos CPTP y por qué son de interés para el problema?
\end{frame}


\section{Motivación del problema}


\section{Resultados}


\section{Trabajo futuro}


\end{document}
