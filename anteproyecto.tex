\documentclass[11pt, spanish, letterpage]{article} % {{{
\usepackage[T1]{fontenc}
\usepackage[utf8]{inputenc}
\usepackage[letterpaper]{geometry}
\geometry{verbose,tmargin=2cm,bmargin=2.5cm,lmargin=2cm,rmargin=2cm}
%\pagestyle{plain}
\setlength{\parskip}{\baselineskip}%espacio entre parrafos
\setlength{\parindent}{0mm}
\usepackage{graphicx}
%\usepackage{setspace}
\usepackage{tabulary}
\usepackage{amsmath}
\usepackage{amsfonts}
\usepackage{amssymb}
\usepackage{amsthm}
\usepackage{physics}
\usepackage{wrapfig}% para colocar figuras en diferentes posiciones
\usepackage{bbold}

\usepackage{fancybox}
\usepackage{colortbl}
\usepackage{amsbsy}
\usepackage[draft,inline,nomargin]{fixme} \fxsetup{theme=color}
\FXRegisterAuthor{cp}{acp}{\color{blue}CP}
\FXRegisterAuthor{ja}{aja}{\color{orange}JA}

\usepackage{lipsum}
\usepackage{babel}
\usepackage{multirow}
\usepackage{array}

\usepackage[]{lineno}  %\linenumbers
%\setlength\linenumbersep{3pt}

\renewcommand{\baselinestretch}{1} % interlinado
\addto\shorthandsspanish{\spanishdeactivate{~<>}}
\date{}
\spanishdecimal{.}
\usepackage{multicol}%para escribir en muchas columnas
%para que no corte palabras
\usepackage[none]{hyphenat}
\usepackage{times}
%\onehalfspacing

\usepackage{hyperref}
%\usepackage{biblatex}

%---ejercicios, problemas -teoremas
%---Problemas encerrados-Bonitos
\usepackage[framemethod=tikz]{mdframed}
\mdfsetup{skipabove=\topskip,skipbelow=\topskip}
\newcounter{problem}[section]
\newenvironment{problem}[1][]{%
%\stepcounter{problem}%
\ifstrempty{#1}%
{\mdfsetup{%
frametitle={%
	\tikz[baseline=(current bounding box.east),outer sep=0pt]
	\node[anchor=east,rectangle,fill=brown!50]
{\strut Problema~\theproblem};}}

}%
{\mdfsetup{%
frametitle={%
	\tikz[baseline=(current bounding box.east),outer sep=0pt]
	\node[anchor=east,rectangle,fill=brown!50]
{\strut Problema ~\theproblem:~#1};}}%

}%
\mdfsetup{innertopmargin=5pt,linecolor=black!50,%
	linewidth=2pt,topline=true,
	frametitleaboveskip=\dimexpr-\ht\strutbox\relax,}
	
\begin{mdframed}[]\relax%
}{\end{mdframed}}

\newenvironment{solution}% environment name
{\colorbox{gray}{~~\textbf{\textcolor{white}{Solución:}}~~}~~}%
{}
%-----end------------

%\newtheorem{example}{Ejemplo}[chapter]
%\newtheorem{ejercicio}{Ejercicio}[chapter]
%%---
\newcommand{\Ev}{\mathbf{E}}
\newcommand{\rv}{\mathbf{r}}
\newcommand{\ru}{\hat{\rv}}
\usepackage{tabulary}
%---paquetes para fisica
\usepackage{physics}%facilita la escritura de operadores usados en fisica
%-paquete para unidades en el sistema internacional
\usepackage[load=prefix, load=abbr, load=physical]{siunitx}
\newunit{\gram}{g }%gramos
\newunit{\velocity}{ \metre / \Sec }%unidades de velocidad sistema internacional
\newunit{\acceleration}{ \metre / \Sec^2 }%unidades de aceleracion sistema internacional
\newunit{\entropy}{ \joule / \kelvin }%unidades de entropia sisteme internacional
%--definiendo constantes fisicas en el SI
\newcommand{\accgravity}{9.8 \metre / \Sec^2}
%---diferencial inexacta
\newcommand{\dbar}{\mathchar'26\mkern-12mu d}

\oddsidemargin 0in
\textwidth 6.5in
\topmargin -0.5in
\textheight 8.5in
% }}}
\begin{document}
\begin{titlepage} % {{{ Suppresses displaying the page number on the title page and the subsequent page counts as page 1                                  
\newcommand{\HRule}{\rule{\linewidth}{0.5mm}} % Defines a new command for horizontal lines, change thickness here                             

\center % Centre everything on the page                                                                                                       

%------------------------------------------------                                                                                             
%       Title                                                                                                                                 
%------------------------------------------------                                                                                             
	
\HRule\\[0.6cm]

{\huge\bfseries Estudio de mapeos proyectivos en sistemas\\
de varios qubits}\\[0.5cm] % Title of your document                                                 

\HRule\\[2cm]

%------------------------------------------------                                                                                             
%       Author(s)                                                                                                                             
%------------------------------------------------                                                                                             


\Large{\textbf{José Alfredo de León Garrido}}\\ [2cm] % Your name                                                                                          

%------------------------------------------------                                                                                             
%       Headings                                                                                                                              
%------------------------------------------------                                                                                             

\textsc{\LARGE Universidad de San Carlos de Guatemala\\ Escuela de Ciencias Físicas y Matemáticas\\ Licenciatura en Física}\\[2cm]

\textsc{\huge Anteproyecto}\\
\textsc{\Large Año de prácticas}\\[2cm]

\textsc{\Large Supervisado por: \textbf{Dr. Carlos Pineda (IF-UNAM) y\\M.Sc. Juan Diego Chang (ICFM-USAC)}}
                                                                                                      

%------------------------------------------------                                                                                             
%       Date                                                                                                                                  
%------------------------------------------------                                                                                             
\vfill\vfill\vfill % Position the date 3/4 down the remaining page
\vfill\vfill\vfill

{\large 13 de enero 2020} % Date, change the \today to a set date if you want to be precise                                                              

%------------------------------------------------                                                                                             
%       Logo                                                                                                                                  
%------------------------------------------------                                                                                             

%\vfill\vfill                                                                                                                                 
%\includegraphics[width=0.2\textwidth]{placeholder.jpg}\\[1cm] % Include a department/university logo - this will require the graphicx packag\                                                                                                                                                  

%----------------------------------------------------------------------------------------                                                     

\vfill % Push the date up 1/4 of the remaining page                                                                                           

\end{titlepage} % }}}
\section{Descripción general de la Institución} % {{{
\subsection{Instituto de Investigación de Ciencias Físicas y Matemáticas -
	Universidad de San Carlos de Guatemala (ICFM-USAC)}  % {{{
% \cpnote{Creo que esta mal el titulo, falta Investigación}

El Instituto de Investigación de Ciencias Físicas y Matemáticas (ICFM-USAC) es
la unidad de la Escuela de Ciencias Físicas y Matemáticas de la Universidad de
San Carlos (ECFM-USAC) que promueve y realiza estudios avanzados en áreas 
científicas, fundamentales y aplicadas, de las ciencias físicas y matemáticas.
El ICFM-USAC se proyecta como una plataforma regional de excelencia dedicada a
la investigación y difusión del conocimiento en física y matemática. Las 
principales líneas de trabajo del ICFM-USAC son:
\begin{itemize}
	\item La investigación en ciencia básica y aplicada.
	\item La promoción de la investigación en ciencia básica y aplicada en el ámbito universitario.
	\item La difusión y divulgación del conocimiento generado por la investigación en ciencias físicas y matemáticas. 
	\item La actualización de programas académicos de ciencias físicas y matemáticas.
\end{itemize}
% }}}
\subsection{Instituto de Física - Universidad Nacional Autónoma de México (IFUNAM)} % {{{
% \cpnote{De donde sacaste esta inforamcion?}
% \janote{De \href{https://www.fisica.unam.mx/es/historia.php}{aquí.}}
Creado en 1938, el Instituto de Física (IFUNAM) ha crecido y madurado como
institución académica para convertirse en uno de los centros de investigación
en física más importantes de México con un sólido prestigio internacional. En
el IFUNAM se realiza una parte muy significativa de la investigación en física
que se lleva a cabo en México, y se cultivan la docencia y formación de
recursos humanos como actividades fundamentales.

El Instituto de Física ha jugado una papel fundamental en el desarrollo de la
física en México, lo cual se refleja la calidad de sus aportaciones científicas
y en la publicación de cerca de 6100 artículos, la mayoría en revistas de
circulación internacional, además de otros múltiples productos de
investigación. Como resultado de lo anterior, sus académicos han obtenido un
gran número de premios y distinciones.

En el IFUNAM se desarrolla investigación de frontera en una amplia gama de
temas que abarcan la totalidad de las escalas observadas en el universo: desde
las diminutas escalas del microcosmos hasta los amplios horizontes de la
cosmología. Las líneas de investigación del IFUNAM son: \begin{itemize}
	\item Física de altas energías, física nuclear, astropartículas y cosmología.
	\item Óptica y física cuántica.
	\item Nanociencias y materia condensada.
	\item Física aplicada y temas interdisciplinarios.
\end{itemize}
% \cpnote{Pienso que tienes mal el uso de mayusculas en los items anteriores}

% }}}
% }}}
\section{Descripción del grupo de trabajo} % {{{
% \cpnote{Creo que te falta describir el trabajo de Juan Diego} 
El proyecto es dirigido por el Dr. Carlos Pineda y el M.Sc. Juan Diego Chang.
% como parte del proyecto... \janote{tal vez agregar aquí este trabajo parte de
% qué proyecto es. Creo que tú sabes mejor, Carlos. }

\subsection{M.Sc. Juan Diego Chang}
El profesor Chang es egresado de la licenciatura en Física de la Universidad
del Valle de Guatemala. Posteriormente realizó estudios de maestría en la 
Universidad Cergy-Pontoise, en Francia, en donde realizó estudios de física
matemática en sistemas integrables. Actualmente trabaja como profesor
e investigador en la Escuela de Ciencias Físicas y Matemáticas de la
Universidad de San Carlos de Guatemala. Su intereses de investigación se 
centran en la modelación matemática de fenómenos sociales y biológicos. 

\subsection{Grupo de Información y Óptica Cuántica (GIOC-UNAM)}
El grupo de Información y Óptica Cuántica de la Universidad Nacional Autónoma
de México (GOIC-UNAM) está compuesto por investigadores y estudiantes de
IIMAS-UNAM, ICN-UNAM y IF-UNAM. 

Desde su fundación, GIOC empezó como una serie de reuniones regulares con doble
propósito: presentar los avances principales de investigación tanto al grupo
como también a la comunidad internacional, y entender los aspectos básicos de
la mecánica cuántica y de la óptica cuántica en un nivel de posgrado. 

Las líneas de investigación en las que trabaja el grupo son:
\begin{itemize}
	\item Sistemas de espín
	\item Matrices aleatorias en información cuántica
	\item Markovianidad en sistemas cuánticos
	\item Información cuántica relativista
	\item Dinámica de rango
\end{itemize}
% }}}

\newpage
\section{Descripción General del Proyecto}  % {{{
El proyecto consiste en estudiar y entender un tipo de mapeos lineales, que actúan
sobre matrices de densidad $\rho$ de un sistema de qubits, que preservan la traza y
borran componentes de $\rho$. Además, se desea entender la condición de completa
positividad de las transformaciones lineales que actúan sobre estados cuánticos y 
verificar qué subconjunto de los mapeos de nuestro interés cumplen con esta condición.
Esta condición asegura que estados cuánticos físicos sean transformados en estados 
cuánticos físicos. 

Para el estudio de este problema son necesarios conocimientos específicos del
álgebra lineal, así como de los fundamentos de la descripción de los sistemas
cuánticos abiertos. Los conceptos de álgebra lineal adecuados permiten
escribir los postulados de la mecánica cuántica utilizando el lenguaje de la
matriz de densidad. A su vez, esta formulación establece los conceptos
fundamentales para describir la dinámica de los sistemas cuánticos abiertos, el
tipo de sistemas que son de interés en el problema de este proyecto.

\subsection{Qubits} % {{{
Un bit cuántico, conocido también como \textit{qubit}, es la unidad fundamental
de información de la computación e información cuántica, de la misma manera que
el bit lo es para la computación e información clásica. Un qubit es un sistema
cuántico de dos niveles que está descrito por una matriz de densidad que
actúa sobre el espacio de Hilbert del espacio de estados del
sistema. La matriz debe ser de traza unitaria, hermítica y positiva
semidefinida para describir a un sistema físico \cite{nielsen_chuang_2011}. 

Para un sistema de $n$ qubits el espacio de Hilbert
$\mathcal{H}_{\text{total}}$ se construye a partir del producto tensorial de
los espacios de Hilbert de cada uno de los qubits que componen al sistema, es
decir $\mathcal{H}_{\text{total}}=\qty(\mathbb{C}^2)^{\otimes n}$. Así, la 
matriz de densidad $\rho$ debe ser un operador que actúe sobre el espacio
$\mathcal{H}_{\text{total}}$. 

% }}}
\subsection{El espacio de las matrices de densidad} % {{{
El conjunto de las matrices de densidad es un conjunto convexo que se encuentra 
dentro del espacio vectorial de las matrices Hermíticas.  
%\cpnote{No entiendo esa ultima frase} \janote{La había copiado textualmente del 
%Geometry of Quantum States, pero quizás sonaba hasta redundante.}
Por ello, es necesario estudiar el espacio de Hilbert-Schmidt y algunas de 
sus propiedades. Habiendo conocido la estructura del espacio en el que viven las
matrices de densidad se deberá estudiar una manera conveniente de
escribir la matriz de densidad de tal manera que se consiga una mejor 
descripción del estado cuántico. Finalmente, es de nuestro interés estudiar
el tipo de transformaciones que actúan sobre las matrices de densidad tales 
que preservan la estructura convexa del espacio.
% }}}
\newpage
\subsection{Canales cuánticos}%\janote{Cambié a 'canales cuánticos' porque Wikipedia
%menciona que 'operaciones cuánticas' se refiere a mapeos CP que no aumentan la 
%traza y canales cuánticos a mapeos CPTP.}  % {{{
%\cpnote{Usas la palabra sistemas de manera inconsistente. Si quieres plaitcamos
%alternativas. Refina}

Para este proyecto son de interés los sistemas cuánticos abiertos. Estos son aquellos
que interactúan con un sistema cuántico externo. Son importantes objetos de estudio
ya que, en la práctica, no existen sistemas cuánticos que estén completamente aislados
de su entorno.  A diferencia de los sistemas cuánticos cerrados, la dinámica de los
sistemas abiertos no está descrita por operadores unitarios. Sin embargo, el objeto de 
estudio de este proyecto son sistemas abiertos cuya dinámica puede describirse
utilizando el formalismo de los canales cuánticos. 

Este formalismo permite describir adecuadamente los cambios discretos de estado,
es decir, transformaciones entre un estado inicial $\rho$ y un estado final $\rho'$
sin referencia explícita al paso del tiempo \cite{nielsen_chuang_2011}.
En especial, será necesario entender la condición de completa positividad (CP
por sus siglas en inglés) con la que estas transformaciones que actúan sobre 
matrices de densidad deben de cumplir para ser mapeos físicos. Para ello se 
hace necesario estudiar el teorema de Choi, que establece una condición equivalente
de la completa positividad. 

% }}}
% }}}
\section{Objetivos} % {{{
\subsection{Objetivo General}
Identificar los mapeos CP  que preservan la traza que borran
componentes arbitrarias de la matriz de densidad $\rho$ de un sistema de $n$
qubits. 	

\subsection{Objetivos específicos}
\begin{itemize}
\item Entender la definición y propiedades de la matriz de densidad, así como
	  los fundamentos de la mecánica cuántica utilizando este lenguaje.
\item Entender la condición de completa positividad con la que deben cumplir 
	  los mapeos que actúan sobre matrices de densidad para asegurar que estos 
	  transforman estados cuánticos válidos en estados válidos de sistemas 
	  cuánticos abiertos.
\item Entender los mapeos que borran componentes de la matriz de densidad
	  de un sistema de 1 qubit.
\item Escribir un programa que construya de numéricamente los mapeos que
	  borran componentes arbitrarias de la matriz de densidad de un sistema 
	  de $n$ qubits.
%\cpnote{Falta el item que te dice que vas a hacer algo para discriminar los mapas
%que si son buenos y los que no.} 
\item Discriminar numéricamente los mapeos no físicos del conjunto de mapeos
	  que borran componentes de la matriz de densidad verificando 
	  la condición de completa positividad mediante el uso del teorema de Choi.
%\janote{Ya que enuncio aquí al teorema de Choi agregué un enunciado al último
%párrafo de la sección 3.3} 
\end{itemize}

% }}}
\section{Justificación del Proyecto}% {{{
%\cpnote{un papel. Se ve que te falto una pasada mas a la redaccion} 
%\janote{Qué terrible lo que había escrito en esta sección. Lo redacté todo de nuevo.}

Estudiar los conceptos de matriz de densidad y mapeos completamente
positivos que preservan la traza constituye una herramienta que permite 
entender la teoría cuántica desde un punto de vista distinto al que se estudia
en los cursos de licenciatura y, al mismo tiempo, comprender los fundamentos 
para estudiar sistemas cuánticos abiertos. 

Por otra parte, este proyecto ofrece la oportunidad de poner en práctica habilidades
computacionales aprendidas en cursos de licenciatura, así como de 
aprender nuevas habilidades que sean necesarias para cumplir con los objetivos
de este trabajo. La adquisición de este tipo de habilidades son de provecho para la formación 
profesional en diversas áreas de investigación en física. 

En fin, este proyecto es importante para estudiar los fundamentos teóricos
de un área de investigación importante en información cuántica. 
Por otro lado, este proyecto servirá como un preludio para un futuro trabajo de
tesis de graduación y un proyecto de investigación, cuya experiencia a adquirir durante
dicho proyecto complementaría la formación de un estudiante de licenciatura.  

% }}}
\section{Metodología}% {{{
La metodología del proyecto consiste en estudiar primero los conceptos 
necesarios para luego abordar el problema que se ha planteado para este
trabajo. Por consiguiente, se estudiarán capítulos seleccionados de
las referencias \cite{bengtsson_zyczkowski_2017} y \cite{nielsen_chuang_2011},
haciendo un resumen de los conceptos más importantes y realizando los
ejercicios sugeridos en la bibliografía. Luego, se realizará una 
implementación computacional para construir los mapeos que borran componentes
de la matriz de densidad de un sistema de $n$ qubits y, también, para
verificar cuáles de estos mapeos son canales cuánticos. 
% }}}

\newpage

\section{Cronograma} % {{{
\begin{center}
\begingroup
\setlength{\tabcolsep}{10pt} % Default value: 6pt
\renewcommand{\arraystretch}{1.2} % Default value: 1
\begin{tabular}{|c|c|c|c|c|c|c|c|c|}
\hline
\textbf{Tareas} & \textbf{Enero} & \textbf{Febrero} & \textbf{Marzo} & \textbf{Abril} & \textbf{Mayo} & \textbf{Junio} \\ \hline
Tarea 1 & \cellcolor[gray]{0.5} &  &  &  &  &	\\ \hline
Tarea 2 & \cellcolor[gray]{0.5} & & & & &	\\ \hline
Tarea 3 & & \cellcolor[gray]{0.5} & & & &	\\ \hline
Tarea 4 & & \cellcolor[gray]{0.5} & & & &	\\ \hline
Tarea 5 & & & \cellcolor[gray]{0.5} & & &	\\ \hline
Tarea 6 & & & \cellcolor[gray]{0.5} & & &	\\ \hline
Tarea 7 & & & \cellcolor[gray]{0.5} & \cellcolor[gray]{0.5} & \cellcolor[gray]{0.5} &	\\ \hline
Tarea 8 & & & & & & \cellcolor[gray]{0.5}	\\ \hline
\end{tabular}
\endgroup
\end{center}

\begin{itemize}
	\item \textbf{Tarea 1:} Repasar los conceptos de álgebra lineal que se ocupan
	en la mecánica cuántica y estudiar los postulados de la mecánica cuántica 
	utilizando el lenguaje de la matriz de densidad.
	\item \textbf{Tarea 2:} Estudiar los conceptos básicos del formalismo de las
	operaciones cuánticas.
	\item \textbf{Tarea 3:} Estudiar el espacio de las matrices de densidad.
	\item \textbf{Tarea 4:} Estudiar sobre la purificación de estados cuánticos mixtos.
	\item \textbf{Tarea 5:} Estudiar sobre las condiciones de positividad y completa
	positividad de las operaciones cuánticas.
	\item \textbf{Tarea 6:} Estudiar, reproducir y entender los mapeos que borran 
	componentes de la matriz de densidad de 1 qubit.
	\item \textbf{Tarea 7:} Escribir un programa que construya numéricamente los mapeos
	que borran componentesde la matriz de densidad de sistemas de $n$ qubits.
	\item \textbf{Tarea 8:} Elaboración del informe final. 
\end{itemize}
% }}}
\bibliographystyle{abbrv}
\bibliography{references}
\end{document}
