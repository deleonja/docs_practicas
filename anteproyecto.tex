\documentclass[11pt, spanish, letterpage]{article} % {{{
\usepackage[T1]{fontenc}
\usepackage[utf8]{inputenc}
\usepackage[letterpaper]{geometry}
\geometry{verbose,tmargin=2cm,bmargin=2.5cm,lmargin=2cm,rmargin=2cm}
%\pagestyle{plain}
\setlength{\parskip}{\baselineskip}%espacio entre parrafos
\setlength{\parindent}{0mm}
\usepackage{graphicx}
%\usepackage{setspace}
\usepackage{tabulary}
\usepackage{amsmath}
\usepackage{amsfonts}
\usepackage{amssymb}
\usepackage{amsthm}
\usepackage{physics}
\usepackage{wrapfig}% para colocar figuras en diferentes posiciones
\usepackage{bbold}

\usepackage{fancybox}
\usepackage{colortbl}
\usepackage{amsbsy}
\usepackage[draft,inline,nomargin]{fixme} \fxsetup{theme=color}
\FXRegisterAuthor{cp}{acp}{\color{blue}CP}
\FXRegisterAuthor{ja}{aja}{\color{orange}JA}

\usepackage{lipsum}
\usepackage{babel}
\usepackage{multirow}
\usepackage{array}

\usepackage[]{lineno}  \linenumbers
\setlength\linenumbersep{3pt}

\renewcommand{\baselinestretch}{1} % interlinado
\addto\shorthandsspanish{\spanishdeactivate{~<>}}
\date{}
\spanishdecimal{.}
\usepackage{multicol}%para escribir en muchas columnas
%para que no corte palabras
\usepackage[none]{hyphenat}
\usepackage{times}
%\onehalfspacing

\usepackage{hyperref}
%\usepackage{biblatex}

%---ejercicios, problemas -teoremas
%---Problemas encerrados-Bonitos
\usepackage[framemethod=tikz]{mdframed}
\mdfsetup{skipabove=\topskip,skipbelow=\topskip}
\newcounter{problem}[section]
\newenvironment{problem}[1][]{%
%\stepcounter{problem}%
\ifstrempty{#1}%
{\mdfsetup{%
frametitle={%
	\tikz[baseline=(current bounding box.east),outer sep=0pt]
	\node[anchor=east,rectangle,fill=brown!50]
{\strut Problema~\theproblem};}}

}%
{\mdfsetup{%
frametitle={%
	\tikz[baseline=(current bounding box.east),outer sep=0pt]
	\node[anchor=east,rectangle,fill=brown!50]
{\strut Problema ~\theproblem:~#1};}}%

}%
\mdfsetup{innertopmargin=5pt,linecolor=black!50,%
	linewidth=2pt,topline=true,
	frametitleaboveskip=\dimexpr-\ht\strutbox\relax,}
	
\begin{mdframed}[]\relax%
}{\end{mdframed}}

\newenvironment{solution}% environment name
{\colorbox{gray}{~~\textbf{\textcolor{white}{Solución:}}~~}~~}%
{}
%-----end------------

%\newtheorem{example}{Ejemplo}[chapter]
%\newtheorem{ejercicio}{Ejercicio}[chapter]
%%---
\newcommand{\Ev}{\mathbf{E}}
\newcommand{\rv}{\mathbf{r}}
\newcommand{\ru}{\hat{\rv}}
\usepackage{tabulary}
%---paquetes para fisica
\usepackage{physics}%facilita la escritura de operadores usados en fisica
%-paquete para unidades en el sistema internacional
\usepackage[load=prefix, load=abbr, load=physical]{siunitx}
\newunit{\gram}{g }%gramos
\newunit{\velocity}{ \metre / \Sec }%unidades de velocidad sistema internacional
\newunit{\acceleration}{ \metre / \Sec^2 }%unidades de aceleracion sistema internacional
\newunit{\entropy}{ \joule / \kelvin }%unidades de entropia sisteme internacional
%--definiendo constantes fisicas en el SI
\newcommand{\accgravity}{9.8 \metre / \Sec^2}
%---diferencial inexacta
\newcommand{\dbar}{\mathchar'26\mkern-12mu d}

\oddsidemargin 0in
\textwidth 6.5in
\topmargin -0.5in
\textheight 8.5in
% }}}
\begin{document}
\begin{titlepage} % {{{ Suppresses displaying the page number on the title page and the subsequent page counts as page 1                                  
\newcommand{\HRule}{\rule{\linewidth}{0.5mm}} % Defines a new command for horizontal lines, change thickness here                             

\center % Centre everything on the page                                                                                                       

%------------------------------------------------                                                                                             
%       Title                                                                                                                                 
%------------------------------------------------                                                                                             
	
\HRule\\[0.6cm]

{\huge\bfseries Estudio de canales cuánticos que borran componentes arbitrarias de la matriz de densidad $\boldsymbol{\rho}$ de un sistema de $\mathbf{n}$ qubits}\\[0.5cm] % Title of your document                                                 

\HRule\\[2cm]

%------------------------------------------------                                                                                             
%       Author(s)                                                                                                                             
%------------------------------------------------                                                                                             


\Large{\textbf{José Alfredo de León Garrido}}\\ [2cm] % Your name                                                                                          

%------------------------------------------------                                                                                             
%       Headings                                                                                                                              
%------------------------------------------------                                                                                             

\textsc{\LARGE Universidad de San Carlos de Guatemala\\ Escuela de Ciencias Físicas y Matemáticas\\ Licenciatura en Física}\\[2cm]

\textsc{\huge Anteproyecto}\\
\textsc{\Large Año de prácticas}\\[2cm]

\textsc{\Large Supervisado por: \textbf{Dr. Carlos Pineda (IF-UNAM) y\\M.Sc. Juan Diego Chang (ICFM-USAC)}}
                                                                                                      

%------------------------------------------------                                                                                             
%       Date                                                                                                                                  
%------------------------------------------------                                                                                             
\vfill\vfill\vfill % Position the date 3/4 down the remaining page
\vfill\vfill\vfill

{\large 13 de enero 2020} % Date, change the \today to a set date if you want to be precise                                                              

%------------------------------------------------                                                                                             
%       Logo                                                                                                                                  
%------------------------------------------------                                                                                             

%\vfill\vfill                                                                                                                                 
%\includegraphics[width=0.2\textwidth]{placeholder.jpg}\\[1cm] % Include a department/university logo - this will require the graphicx packag\                                                                                                                                                  

%----------------------------------------------------------------------------------------                                                     

\vfill % Push the date up 1/4 of the remaining page                                                                                           

\end{titlepage} % }}}
\section{Descripción general de la Institución} % {{{
\subsection{Instituto de Investigación de Ciencias Físicas y Matemáticas -
	Universidad de San Carlos de Guatemala (ICFM-USAC)}  % {{{
% \cpnote{Creo que esta mal el titulo, falta Investigación}

El Instituto de Investigación de Ciencias Físicas y Matemáticas (ICFM-USAC) es
la unidad de la Escuela de Ciencias Físicas y Matemáticas de la Universidad de
San Carlos (ECFM-USAC) que promueve y realiza estudios avanzados en áreas 
científicas, fundamentales y aplicadas, de las ciencias físicas y matemáticas.
El ICFM-USAC se proyecta como una plataforma regional de excelencia dedicada a
la investigación y difusión del conocimiento en física y matemática. Las 
principales líneas de trabajo del ICFM-USAC son:
\begin{itemize}
	\item La investigación en ciencia básica y aplicada.
	\item La promoción de la investigación en ciencia básica y aplicada en el ámbito universitario.
	\item La difusión y divulgación del conocimiento generado por la investigación en ciencias físicas y matemáticas. 
	\item La actualización de programas académicos de ciencias físicas y matemáticas.
\end{itemize}
% }}}
\subsection{Instituto de Física - Universidad Nacional Autónoma de México (IFUNAM)} % {{{
% \cpnote{De donde sacaste esta inforamcion?}
% \janote{De \href{https://www.fisica.unam.mx/es/historia.php}{aquí.}}
Creado en 1938, el Instituto de Física (IFUNAM) ha crecido y madurado como
institución académica para convertirse en uno de los centros de investigación
en física más importantes de México con un sólido prestigio internacional. En
el IFUNAM se realiza una parte muy significativa de la investigación en física
que se lleva a cabo en México, y se cultivan la docencia y formación de
recursos humanos como actividades fundamentales.

El Instituto de Física ha jugado una papel fundamental en el desarrollo de la
física en México, lo cual se refleja la calidad de sus aportaciones científicas
y en la publicación de cerca de 6100 artículos, la mayoría en revistas de
circulación internacional, además de otros múltiples productos de
investigación. Como resultado de lo anterior, sus académicos han obtenido un
gran número de premios y distinciones.

En el IFUNAM se desarrolla investigación de frontera en una amplia gama de
temas que abarcan la totalidad de las escalas observadas en el universo: desde
las diminutas escalas del microcosmos hasta los amplios horizontes de la
cosmología. Las líneas de investigación del IFUNAM son: \begin{itemize}
	\item Física de altas energías, física nuclear, astropartículas y cosmología.
	\item Óptica y física cuántica.
	\item Nanociencias y materia condensada.
	\item Física aplicada y temas interdisciplinarios.
\end{itemize}
% \cpnote{Pienso que tienes mal el uso de mayusculas en los items anteriores}

% }}}
% }}}
\section{Descripción del grupo de trabajo} % {{{
% \cpnote{Creo que te falta describir el trabajo de Juan Diego} 
El proyecto es dirigido por el Dr. Carlos Pineda y el M.Sc. Juan Diego Chang.
% como parte del proyecto... \janote{tal vez agregar aquí este trabajo parte de
% qué proyecto es. Creo que tú sabes mejor, Carlos. }

\subsection{M.Sc. Juan Diego Chang}
El profesor Chang es egresado de la licenciatura en Física de la Universidad
del Valle de Guatemala. Posteriormente realizó estudios de maestría en la 
Universidad Cergy-Pontoise, en Francia, en donde realizó estudios de física
matemática en sistemas integrables. Actualmente trabaja como profesor
e investigador en la Escuela de Ciencias Físicas y Matemáticas de la
Universidad de San Carlos de Guatemala. Su intereses de investigación se 
centran en la modelación matemática de fenómenos sociales y biológicos. 

\subsection{Grupo de Información y Óptica Cuántica (GIOC-UNAM)}
El grupo de Información y Óptica Cuántica de la Universidad Nacional Autónoma
de México (GOIC-UNAM) está compuesto por investigadores y estudiantes de
IIMAS-UNAM, ICN-UNAM y IF-UNAM. 

Desde su fundación, GIOC empezó como una serie de reuniones regulares con doble
propósito: presentar los avances principales de investigación tanto al grupo
como también a la comunidad internacional, y entender los aspectos básicos de
la mecánica cuántica y de la óptica cuántica en un nivel de posgrado. 

Las líneas de investigación en las que trabaja el grupo son:
\begin{itemize}
	\item Sistemas de espín
	\item Matrices aleatorias en información cuántica
	\item Markovianidad en sistemas cuánticos
	\item Información cuántica relativista
	\item Dinámica de rango
\end{itemize}
% }}}
\section{Descripción General del Proyecto}  % {{{
El proyecto consiste en estudiar y entender un tipo de mapeos lineales, que actúan
sobre matrices de densidad $\rho$ de un sistema de qubits, que preservan la traza y
borran componentes de $\rho$. Además, se desea entender la condición de completa
positividad de las transformaciones lineales que actúan sobre estados cuánticos y 
verificar qué subconjunto de los mapeos de nuestro interés cumplen con esta condición.
Esta condición asegura que estados cuánticos físicos sean transformados en estados 
cuánticos físicos. 

Para el estudio de este problema son necesarios conocimientos específicos del
álgebra lineal, así como de los fundamentos de la descripción de los sistemas
cuánticos abiertos. Los conceptos de álgebra lineal adecuados permiten
escribir los postulados de la mecánica cuántica utilizando el lenguaje de la
matriz de densidad. A su vez, esta formulación establece los conceptos
fundamentales para describir la dinámica de los sistemas cuánticos abiertos, el
tipo de sistemas que son de interés en el problema de este proyecto.

\subsection{Qubits} % {{{
Un bit cuántico, conocido también como \textit{qubit}, es la unidad fundamental
de información de la computación e información cuántica, de la misma manera que
el bit lo es para la computación e información clásica. Un qubit es un sistema
cuántico de dos niveles que está descrito por una matriz de densidad que
actúa sobre el espacio de Hilbert del espacio de estados del
sistema. La matriz debe ser de traza unitaria, hermítica y positiva
semidefinida para describir a un sistema físico \cite{nielsen_chuang_2011}. 

Para un sistema de $n$ qubits el espacio de Hilbert
$\mathcal{H}_{\text{total}}$ se construye a partir del producto tensorial de
los espacios de Hilbert de cada uno de los qubits que componen al sistema, es
decir $\mathcal{H}_{\text{total}}=\qty(\mathbb{C}^2)^{\otimes n}$. Así, la 
matriz de densidad $\rho$ debe ser un operador que actúe sobre el espacio
$\mathcal{H}_{\text{total}}$. 

% }}}
\subsection{El espacio de las matrices de densidad} % {{{
El conjunto de las matrices de densidad es un conjunto convexo que se encuentra 
dentro del espacio vectorial de las matrices Hermíticas. La razón para
considerar a este conjunto convexo como el espacio de estados de un sistema
físico es que existen ventajas al considerar vectores que también son
matrices \cite{bengtsson_zyczkowski_2017}. 
\cpnote{No entiendo esa ultima frase}.

Por ello, será importante estudiar el espacio de Hilbert-Schmidt y algunas de 
sus propiedades. Así también, se debe estudiar de qué manera conviene 
escribir la matriz de densidad de tal manera que se consiga una mejor 
descripción del estado cuántico. Finalmente, es de nuestro interés estudiar
el tipo de transformaciones que actúan sobre las matrices de densidad tales 
que preservan la estructura convexa del espacio.
% }}}
\subsection{Operaciones cuánticas}  % {{{
\cpnote{Usas la palabra sistemas de manera inconsistente. Si quieres plaitcamos
alternativas. Refina}
El tipo de sistemas cuánticos que son de interés para este proyecto,
sistemas de qubits, son sistemas que no se consideran aislados. Es decir,
son sistemas que podrían interactuar con otros sistemas que no sean el 
objeto de estudio. A este tipo de sistemas se les conoce como sistemas 
cuánticos abiertos. Para estudiar esta clase de sistemas se requiere estudiar
el formalismo de las operaciones cuánticas que permite describir la 
dinámica de los sistemas cuánticos abiertos. Este formalismo permite 
describir adecuadamente los cambios discretos de estado, es decir, 
transformaciones entre un estado inicial $\rho$ y un estado final $\rho'$
sin referencia explícita al paso del tiempo \cite{nielsen_chuang_2011}.
En especial, será necesario entender la condición de completa positividad (CP
por sus siglas en inglés) con la que estas transformaciones que actúan sobre 
matrices de densidad deben de cumplir para ser mapeos físicos.  

% }}}
% }}}
\section{Objetivos} % {{{
\cpnote{Refina porfa los objetivos específicos}
\subsection{Objetivo General}
Identificar los mapeos CP  que preservan la traza que borran
componentes arbitrarias de la matriz de densidad $\rho$ de un sistema de $n$
qubits. 	

\subsection{Objetivos específicos}
\begin{itemize}
\item Entender el concepto y las propiedades de la matriz de densidad.
\item Entender los fundamentos de la mecánica cuántica en el lenguaje de la
	  matriz de densidad.
\item Entender la condición de completa positividad con la que deben cumplir 
	  los mapeos que actúan sobre matrices de densidad para transformar estados
      cuánticos válidos en estados válidos de sistemas cuánticos abiertos.
\item Entender los mapeos que borran componentes de la matriz de densidad
	  de un sistema de 1 qubit.
\item Escribir un programa que construya de manera numérica los mapeos que
	  borran componentes arbitrarias de la matriz de densidad de un sistema 
	  de $n$ qubits y, además, que verifique la completa positividad de estos
	  mapeos para discriminar los mapeos no físicos. 
\cpnote{Falta el item que te dice que vas a hacer algo para discriminar los mapas
que si son buenos y los que no.}
\end{itemize}

% }}}
\section{Justificación del Proyecto}% {{{
El estudio de los sistemas cuánticos abiertos juega \cpnote{un papel. Se ve que te falto 
una pasada mas a la redaccion} una pieza fundamental en la teoría de la
Información Cuántica \cpnote{Porqupe mayusculas? Le paro y dale una iterada mas
porfa}. También contribuye al conocimiento que se
está produciendo en el camino a la implementación de las computadoras
cuánticas, sobre las cuáles hay muchas expectativas puestas por la comunidad
científica. Por otro lado, el estudio de los mapeos CP que preservan la traza
de nuestro interés también revelan nuevos resultados producto de la mecánica
cuántica.   

Como en cualquier otra área de investigación se debe considerar los
conocimientos de matemática y física que son necesarios para dar solución al
problema de estudio. El estudio de los sistemas cuánticos abiertos requiere de
conocimientos de álgebra lineal, mecánica cuántica con el lenguaje de la matriz
de densidad y las operaciones cuánticas para describir la dinámica de este tipo
de sistemas cuánticos.

El estudio de este problema permite tener una aproximación a la experiencia de
investigación en Mecánica Cuántica. Así también, los conocimientos que se
requiere aprender previo a la resolución del problema en cuestión son
fundamentales en el área de Información Cuántica, y en la ECFM la Licenciatura
en Física no cuenta con ningún curso de Información Cuántica. Es por ello, que
este trabajo de prácticas será de mucho beneficio para continuar esta área de
investigación en el futuro.  

Haciendo una prueba. Hola qué pex. 

% }}}
\section{Metodología}% {{{
La metodología del proyecto consiste primero en estudiar los temas de interés,
según los conocimientos mencionados antes, de las referencias
\cite{bengtsson_zyczkowski_2017} y \cite{nielsen_chuang_2011}. Se leerá el
material detenidamente de tal manera que se consiga una comprensión completa de
los conceptos. Tras la lectura se realizan los ejercicios sugeridos en las
bibliografías. 

Después de obtener los conocimientos teóricos necesarios se llevará a cabo una
implementación computacional para construir numéricamente los mapeos CP que
preservan la traza de esstudio y así dilucidar el camino a una descripción
analítica de ellos. 
% }}}
\newpage
\section{Cronograma} % {{{
\begin{center}
	\begingroup
	\setlength{\tabcolsep}{10pt} % Default value: 6pt
	\renewcommand{\arraystretch}{1.2} % Default value: 1
        \begin{tabular}{|c|c|c|c|c|c|c|c|c|}
                \hline
                \textbf{Tareas} & \textbf{Enero} & \textbf{Febrero} & \textbf{Marzo} & \textbf{Abril} & \textbf{Mayo} & \textbf{Junio} \\ \hline
                Tarea 1 & \cellcolor[gray]{0.5} &  &  &  &  &	\\ \hline
                Tarea 2 & \cellcolor[gray]{0.5} & & & & &	\\ \hline
                Tarea 3 & \cellcolor[gray]{0.5} & & & & &	\\ \hline
                Tarea 4 & & \cellcolor[gray]{0.5} & & & &	\\ \hline
                Tarea 5 & & \cellcolor[gray]{0.5} & & & &	\\ \hline
                Tarea 6 & & & \cellcolor[gray]{0.5} & & &	\\ \hline
                Tarea 7 & & & \cellcolor[gray]{0.5} & & &	\\ \hline
                Tarea 8 & & & \cellcolor[gray]{0.5} & \cellcolor[gray]{0.5} & \cellcolor[gray]{0.5} &	\\ \hline
				Tarea 9 & & & & & & \cellcolor[gray]{0.5}	\\ \hline
        \end{tabular}
    \endgroup
\end{center}

\begin{itemize}
	\item \textbf{Tarea 1:} Repasar los conceptos de álgebra lineal que se ocupan en la mecánica cuántica y los postulados de la mecánica cuántica con el lenguaje de la matriz de densidad.
	\item \textbf{Tarea 2:} Estudiar los conceptos básicos de del formalismo de las operaciones cuánticas para describir la dinámica de los sistemas cuánticos abiertos.
	\item \textbf{Tarea 3:} Estudiar los conceptos básicos de las medidas de distancia en la teoría de procesamiento de información cuántica. 
	\item \textbf{Tarea 4:} Estudiar la estructura y propiedades del espacio de las matrices de densidad $\mathcal{M}^{(n)}$.
	\item \textbf{Tarea 5:} Estudiar los conceptos de purificación de estados cuánticos mixtos.
	\item \textbf{Tarea 6:} Estudiar, con un el rigor matemático apropiado, las propiedades de las operaciones cuánticas. 
	\item \textbf{Tarea 7:} Estudiar, reproducir y entender los mapeos que borran componentes de la matriz de densidad $\rho$ de 1 qubit.
	\item \textbf{Tarea 8:} Construir numéricamente los mapeos que borran componentes de la matriz de densidad $\rho$ de sistemas con dos o más qubits.
	\item \textbf{Tarea 9:} Elaboración del informe final. 
\end{itemize}
% }}}
\bibliographystyle{abbrv}
\bibliography{references}
\end{document}
