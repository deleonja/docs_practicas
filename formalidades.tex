\section*{Objetivo General}
Identificar los mapeos CP  que preservan la traza que borran
componentes arbitrarias de la matriz de densidad $\rho$ de 
un sistema de $n$ qubits. 	

\section*{Objetivos Específicos}
\begin{itemize}
\item Entender la definición y propiedades de la matriz de densidad, así como
	  los fundamentos de la mecánica cuántica utilizando este lenguaje.
\item Entender la condición de completa positividad con la que deben cumplir 
	  los mapeos que actúan sobre matrices de densidad para asegurar que estos 
	  transforman estados cuánticos válidos en estados válidos de sistemas 
	  cuánticos abiertos.
\item Entender los mapeos que borran componentes de la matriz de densidad
	  de un sistema de 1 qubit.
\item Escribir un programa que construya de numéricamente los mapeos que
	  borran componentes arbitrarias de la matriz de densidad de un sistema 
	  de $n$ qubits.
\item Discriminar numéricamente los mapeos no físicos del conjunto de mapeos
	  que borran componentes de la matriz de densidad verificando 
	  la condición de completa positividad mediante el uso del teorema de Choi. 
\end{itemize}

\section*{Introducción}
\section*{Justificación}
Estudiar los conceptos de matriz de densidad y mapeos completamente
positivos que preservan la traza constituye una herramienta que permite 
entender la teoría cuántica desde un punto de vista distinto al que se estudia
en los cursos de licenciatura y, al mismo tiempo, comprender los fundamentos 
para estudiar sistemas cuánticos abiertos. 

Por otra parte, este proyecto ofrece la oportunidad de poner en práctica habilidades
computacionales aprendidas en cursos de licenciatura, así como de 
aprender nuevas habilidades que sean necesarias para cumplir con los objetivos
de este trabajo. La adquisición de este tipo de habilidades son de provecho para la formación 
profesional en diversas áreas de investigación en física. 

En fin, este proyecto es importante para estudiar los fundamentos teóricos
de un área de investigación importante en información cuántica. 
Por otro lado, este proyecto servirá como un preludio para un futuro trabajo de
tesis de graduación y un proyecto de investigación, cuya experiencia a adquirir durante
dicho proyecto complementaría la formación de un estudiante de licenciatura. 