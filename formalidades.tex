\section*{Objetivo General}
Identificar los mapeos CP  que preservan la traza que borran
componentes arbitrarias de la matriz de densidad $\rho$ de 
un sistema de $n$ qubits. 	

\section*{Objetivos Específicos}
\begin{itemize}
\item Entender la definición y propiedades de la matriz de densidad, así como
	  los fundamentos de la mecánica cuántica utilizando este lenguaje.
\item Entender la condición de completa positividad con la que deben cumplir 
	  los mapeos que actúan sobre matrices de densidad para asegurar que estos 
	  transforman estados cuánticos válidos en estados válidos de sistemas 
	  cuánticos abiertos.
\item Entender los mapeos que borran componentes de la matriz de densidad
	  de un sistema de 1 qubit.
\item Escribir un programa que construya de numéricamente los mapeos que
	  borran componentes arbitrarias de la matriz de densidad de un sistema 
	  de $n$ qubits.
\item Discriminar numéricamente los mapeos no físicos del conjunto de mapeos
	  que borran componentes de la matriz de densidad verificando 
	  la condición de completa positividad mediante el uso del teorema de Choi. 
\end{itemize}

\section*{Introducción}
La descripción matemática de la evolución de los sistemas cuánticos 
ha sido ampliamente estudiada con los formalismos 
de Schrödinger y de Heissenberg. 
No obstante, estas descripciones están limitadas
exclusivamente para sistemas ideales, sistemas 
que no sufren de ninguna de interacción con 
su entorno. Aunque en las últimas décadas se ha 
conseguido un enorme progreso experimental en la manipulación 
de sistemas cuánticos individuales, sigue siendo de gran interés el 
estudio de los sistemas de muchos cuerpos. 
Es en el caso de manipular este tipo de sistemas  en el que 
se vuelve casi imposible aislar al sistema por completo de su entorno, 
o sólo a una parte de él del resto del sistema. 
De esa cuenta se hace necesario un marco teórico 
para describir la dinámica de este tipo de sistemas, 
conocidos como sistemas abiertos.

La teoría de las operaciones cuánticas propone un formalismo para
describir la evolución de los sistemas abiertos de manera discreta. 
Quizás un nombre más apropiado para este formalismo es 
``operaciones completamente positivas", pues la condición 
de completa positividad (CP) es la que hace 
fundamentalmente distinta a una operación cuántica
del operador de evolución. La CP es una condición más robusta 
a la de la positividad de una matriz  y que, en el contexto de la
mecánica cuántica y los sistemas abiertos, asegura que, aún 
cuando el sistema principal se encuentra inicialmente acoplado 
a un sistema secundario y el sistema completo esté en 
un estado entrelazado, una operación cuántica seguirá
dando como resultado un estado físico, para el sistema total 
y cada una de sus partes. Las operaciones cuánticas son una 
propuesta teórica que permite estudiar la dinámica de los 
sistemas de muchos cuerpos, considerando que en el laboratorio 
estos sistemas pueden sufrir de interacción con sistemas secundarios.

El problema que nos interesa estudiar, utilizando el marco teórico de 
las operaciones cuánticas, es el de caracterizar a las
operaciones que borran información arbitraria del estado de un sistema
de qubits. Un qubit es un sistema cuántico de dos niveles 
como una partícula de espín 1/2 o la polarización de la luz. 
Las operaciones cuánticas de 1 qubit han sido estudiadas y 
se pueden encontrar en la literatura. Por lo cual, también 
se pueden encontrar la operaciones de nuestro interés para 
el caso de 1 qubit. Uno de los objetivos para este trabajo fue 
el de reproducir los resultados que ya existen sobre estas operaciones
para 1 qubit.

La estructura de este informe es la siguiente. En el capítulo 1
se presenta la revisión y estudio de la bibliografía que se realizó 
sobre el formalismo de la matriz de densidad de la mecánica cuántica, 
herramienta utilizada por las operaciones cuánticas para describir 
al estado de un sistema cuántico. En el capítulo 2 se expone 
la revisión y estudio bibliográfico del formalismo de las 
operaciones cuánticas. Se estudió la completa positividad, 
dos representaciones distintas y los ejemplos más relevantes 
de 1 qubit de las operaciones cuánticas. En el último capítulo
se formula el enunciado del problema de las operaciones que borran 
componentes (información) de la matriz de densidad de un sistema de 
qubits; luego, se presentan los métodos analítico y numérico 
para resolver el problema de 1 qubit, así como los resultados 
que se obtuvieron, y se comparan con lo que se encuentra actualmente
en la literatura de las operaciones cuánticas. Este trabajo 
es el preludio de la tesis de licenciatura, en la que se continuará 
estudiando este problema para sistemas de más de 1 qubit.


\section*{Justificación}
Estudiar los conceptos de matriz de densidad y mapeos completamente
positivos que preservan la traza constituye una herramienta que permite 
entender la teoría cuántica desde un punto de vista distinto al que se estudia
en los cursos de licenciatura y, al mismo tiempo, comprender los fundamentos 
para estudiar sistemas cuánticos abiertos. 

Por otra parte, este proyecto ofrece la oportunidad de poner en práctica habilidades
computacionales aprendidas en cursos de licenciatura, así como de 
aprender nuevas habilidades que sean necesarias para cumplir con los objetivos
de este trabajo. La adquisición de este tipo de habilidades son de provecho para la formación 
profesional en diversas áreas de investigación en física. 

En fin, este proyecto es importante para estudiar los fundamentos teóricos
de un área de investigación importante en información cuántica. 
Por otro lado, este proyecto servirá como un preludio para un futuro trabajo de
tesis de graduación y un proyecto de investigación, cuya experiencia a adquirir durante
dicho proyecto complementaría la formación de un estudiante de licenciatura. 