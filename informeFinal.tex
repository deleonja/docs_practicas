\documentclass[12pt]{report}
\usepackage[spanish]{babel}
\usepackage[utf8]{inputenc}
\usepackage[T1]{fontenc}
\usepackage{makeidx}
\usepackage{graphicx}
\usepackage{subfig}
\usepackage{amsmath}
\usepackage{amsfonts}
\usepackage{amssymb}
\usepackage{authblk} % para la manipulación de autores y afiliación
\usepackage[pdftex]{hyperref}
\usepackage{multirow}
\usepackage{multicol}
\usepackage{float}
\usepackage{booktabs}
\usepackage{colortbl}
\usepackage{bbold}
\usepackage{physics}


\usepackage[]{lineno}  \linenumbers
\setlength\linenumbersep{3pt}

%\usepackage{hyperref}
%\usepackage{commath}
\decimalpoint
\renewcommand{\tablename}{Tabla}
\oddsidemargin 0in
\textwidth 6.5in
\topmargin -0.5in
\textheight 8.5in

%<stdio.h>

\begin{document}

\begin{titlepage} % Suppresses displaying the page number on the title page and the subsequent page counts as page 1                                  
\newcommand{\HRule}{\rule{\linewidth}{0.6mm}} % Defines a new command for horizontal lines, change thickness here                             

\center % Centre everything on the page                                                                                                       

%------------------------------------------------                                                                                             
%       Title                                                                                                                                 
%------------------------------------------------                                                                                             

\HRule\\[0.4cm]

{\LARGE\bfseries Estudio de canales cuánticos que borran\\
				 componentes arbitrarias de la matriz densidad $\boldsymbol{\rho}$\\
				 de un sistema de $\mathbf{n}$ qubits}\\[0.4cm] % Title of your document                                                 

\HRule\\[2cm]

%------------------------------------------------                                                                                             
%       Author(s)                                                                                                                             
%------------------------------------------------                                                                                             


\Large{\textbf{José Alfredo de León Garrido}}\\[2cm]

%------------------------------------------------                                                                                             
%       Headings                                                                                                                              
%------------------------------------------------                                                                                             

\textsc{\Large Universidad de San Carlos de Guatemala\\
		Escuela de Ciencias Físicas y Matemáticas\\
		Licenciatura en Física}\\[2cm]

\textsc{{\Large\bfseries Informe final}}\\
\textsc{\large Año de prácticas}\\[2cm]

\textsc{\large Supervisado por:\\
		\textbf{Dr. Carlos Pineda (IF-UNAM) y\\
		M.Sc. Juan Diego Chang (ICFM-USAC)}}


%------------------------------------------------                                                                                             
%       Date                                                                                                                                  
%------------------------------------------------                                                                                             
\vfill\vfill\vfill % Position the date 3/4 down the remaining page
\vfill\vfill\vfill

{\large 13 de enero 2020} % Date, change the \today to a set date if you want to be precise                                                              

%------------------------------------------------                                                                                             
%       Logo                                                                                                                                  
%------------------------------------------------                                                                                             

%\vfill\vfill                                                                                                                                 
%\includegraphics[width=0.2\textwidth]{placeholder.jpg}\\[1cm] % Include a department/university logo - this will require the graphicx packag\                                                                                                                                                  

%----------------------------------------------------------------------------------------                                                     

\vfill % Push the date up 1/4 of the remaining page  
\end{titlepage}

\newtheorem{definition}{Definición}[section]

\newtheorem{teorema}{Teorema}[section]

\newtheorem{property}{Propiedad}[section]

\section*{Objetivo General}



\section*{Objetivos Específicos}


\section*{Introducción}


\section*{Justificación}


\chapter{La matriz densidad}
Una formulación alternativa de la mecánica cuántica es posible utilizando la herramienta de
la matriz densidad. Esta es una formulación alternativa al enfoque de los vectores de estado,
pero es matemáticamente equivalente y, de hecho, más conveniente para cierto tipo de problemas
en mecánica cuántica. 

\section{Ensambles de estados cuánticos}
El lenguaje de la matriz densidad permite describir sistemas cuánticos cuyos estados no son 
conocidos por completo. Supongamos que un sistema cuántico se encuentra en uno de los estados 
$\{\ket{\psi _i}\}$, con respectivas probabilidades $p_i$. Llamamos al conjunto $\{p_i, 
\ket{\psi _i} \}$ un ensamble de estados puros. La matriz densidad para dicho sistema está 
definida como
\begin{equation}
	\rho \equiv \sum _i p_i\dyad{\psi_i}{\psi_i}.
	\label{eq:rho_def}
\end{equation}

Todos los postulados de la mecánica cuántica pueden ser reformulados en el lenguaje de la matriz
densidad. Los mismos resultados se obtienen si se utiliza el lenguaje de la matriz densidad o el
lenguaje del vector de estado. Sin embargo, dependiendo el problema uno de los lenguajes será más
conveniente y práctico de utilizar que el otro. 

Un sistema cuántico cuyo estado $\ket{\psi}$ es completamente conocido se dice que se encuentra en
un estado puro. De cualquier otra manera, $\rho$ se encuentra en un estado mixto; se dice que está
en una mezcla de distintos estados puros en el ensamble para $\rho$. \newpage

\section{Propiedades generales de la matriz densidad}
Es útil y conveniente dar una descripción de la matriz densidad que no esté basado en el vector de
estado. La matriz densidad está caracterizada por el siguiente teorema: 
\begin{teorema}
Un operador $\rho$ es la matriz densidad asociada a un ensamble $\{p_i, \ket{\psi _i} \}$ si y
sólo si comple con las siguientes condiciones:
\begin{enumerate}
\item $\rho$ tiene traza unitaria.
\item $\rho$ es una matriz positiva semidefinida. 
\end{enumerate}	
\end{teorema}

Este teorema permite caracterizar la matriz densidad de manera intrínsica a la matriz misma. Con 
esta definición es posible reformular los postulados de la mecánica cuántica con el lenguaje de 
la matriz densidad. 
\begin{itemize}
	\item \textbf{Postulado 1:} Cualquier sistema físico tiene asociado un espacio vectorial complejo
								con producto interno que se conoce como el espacio de estados del
								sistema. El sistema está completamente descrito por su operador densidad,
								que es un operador positivo $\rho$ con traza unitaria, que actúa sobre 
								el espacio de estados del sistema. Si un sistema cuántico se encuentra
								en el estado $\rho _i$ con probabilidad $p_i$, entonces el operador densidad
								del sistema es $\sum p_i\rho_i$.
	\item \textbf{Postulado 2:} La evolución de un sistema cuántico cerrado está descrita por una transformación
								unitaria. Es decir, el estado $\rho$ del sistema en el tiempo $t_1$ está 
								relacionado al estado $\rho'$ del sistema en el tiempo $t_2$ por un operador
								unitario $U$ que depende sólo de los tiempos $t_1$ y $t_2$,
								\begin{equation}
								\rho'=U\rho U^{\dagger}.
								\label{eq:postulate1}
								\end{equation}
	\item \textbf{Postulado 3:} Las mediciones cuánticas están descritas por un conjunto $\{M_m\}$ de 
								operadores de medición. Estos son operadores que actúan sobre el espacio 
								de estados del sistema que se mide. El índice $m$ se refiere a los resultados
								de las mediciones que puedan ocurrir en el experimento. Si el estado del sistema
								cuántico es $\rho$ inmediatamente antes de la medición, entonces la probabilidad
								de medir el resultado $m$ está dado por
								\begin{equation}
								p(m)=\tr \qty(M_m^{\dagger}M_m\rho),
								\label{eq:postulate2_prob}
								\end{equation}						
								y el estado del sistema después de la medición es
								\begin{equation}
								\rho'=\frac{M_m\rho M_m^{\dagger}}{\tr \qty(M_m^{\dagger}M_m\rho)}.
								\label{eq:postulate2_rhoPrime}
								\end{equation}	
								Los operadores de medición satisfacen la ecuación 
								\begin{equation}
								\sum _m M_m^{\dagger}M_m=\mathbb{1}.
								\label{eq:postulate2_completeness}
								\end{equation}
	\item \textbf{Postulado 4:} El espacio de estados de un sistema físico compuesto es el producto tensorial 
								de los espacios de estado de los sistemas físicos que componen al sistema total.
								Además, si los sistemas están enumerados de 1 hasta $n$, y el $i$-ésimo sistema
								está preparado en el estado $\rho_i$, entonces el estado del sistema total es
								$\rho_1\otimes\rho_2\otimes\cdots\otimes\rho_n$.
\end{itemize}

\chapter{Conclusiones y trabajo futuro}



\bibliographystyle{abbrv}
\bibliography{references}

\end{document}