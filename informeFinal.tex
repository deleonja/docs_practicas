\documentclass[11pt,dvipsnames]{report}
\usepackage[spanish]{babel}
\usepackage[utf8]{inputenc}
\usepackage[T1]{fontenc}
\usepackage{makeidx}
\usepackage{graphicx}
\usepackage{subfig}
\usepackage{amsmath}
\usepackage{amsfonts}
\usepackage{amssymb}
\usepackage{authblk} % para la manipulación de autores y afiliación
\usepackage[pdftex]{hyperref}
\usepackage{multirow}
\usepackage{multicol}
\usepackage{float}
\usepackage{xcolor}
\usepackage{booktabs}
\usepackage{colortbl}
\usepackage{bbold}
\usepackage{physics}
\usepackage{mathtools}

% Theorems, proofs, etc
\usepackage{amsthm}

\usepackage{fancybox}
\usepackage{colortbl}
\usepackage{amsbsy}
\usepackage[draft,inline,nomargin]{fixme} \fxsetup{theme=color}
\FXRegisterAuthor{cp}{acp}{\color{blue}CP}
\FXRegisterAuthor{ja}{aja}{\color{RedViolet}JA}

\usepackage[]{lineno}  \linenumbers
\setlength\linenumbersep{3pt}

%\usepackage{hyperref}
%\usepackage{commath}
\decimalpoint
\renewcommand{\tablename}{Tabla}
\oddsidemargin 0in
\textwidth 6.5in
\topmargin -0.5in
\textheight 8.5in

\newcommand{\psii}{\psi_i}
\newcommand{\Pk}[1]{\ket{\psi_{#1}}}
\newcommand{\Pb}[1]{\bra{\psi_{#1}}}
\newcommand{\pk}{\ket{\psi}}
\newcommand{\M}{\mathcal{M}^{(N)}}


\begin{document}

\begin{titlepage} % Suppresses displaying the page number on the title page and the subsequent page counts as page 1                                  
\newcommand{\HRule}{\rule{\linewidth}{0.6mm}} % Defines a new command for horizontal lines, change thickness here                             

\center % Centre everything on the page                                                                                                       

%------------------------------------------------                                                                                             
%       Title                                                                                                                                 
%------------------------------------------------                                                                                             

\HRule\\[0.4cm]

% Title of your document                                                 
{\LARGE\bfseries Mapeos proyectivos en sistemas de varios qubits}\\[0.4cm] 

\HRule\\[2cm]

%------------------------------------------------                                                                                             
%       Author(s)                                                                                                                             
%------------------------------------------------                                                                                             


\Large{\textbf{José Alfredo de León Garrido}}\\[2cm]

%------------------------------------------------                                                                                             
%       Headings                                                                                                                              
%------------------------------------------------                                                                                             

\textsc{\Large Universidad de San Carlos de Guatemala\\
		Escuela de Ciencias Físicas y Matemáticas\\
		Licenciatura en Física}\\[2cm]

\textsc{{\Large\bfseries Informe final}}\\
\textsc{\large Año de prácticas}\\[2cm]

\textsc{\large Supervisado por:\\
		\textbf{Dr. Carlos Pineda (IF-UNAM) y\\
		M.Sc. Juan Diego Chang (ICFM-USAC)}}


%------------------------------------------------                                                                                             
%       Date                                                                                                                                  
%------------------------------------------------                                                                                             
\vfill\vfill\vfill % Position the date 3/4 down the remaining page
\vfill\vfill\vfill

{\large xx de noviembre, 2020} % Date, change the \today to a set date if you want to be precise                                                              

%------------------------------------------------                                                                                             
%       Logo                                                                                                                                  
%------------------------------------------------                                                                                             

%\vfill\vfill                                                                                                                                 
%\includegraphics[width=0.2\textwidth]{placeholder.jpg}\\[1cm] % Include a department/university logo                                                                                                                                      

%----------------------------------------------------------------------------------------                                                     

\vfill % Push the date up 1/4 of the remaining page  
\end{titlepage}

\newtheorem{definition}{Definición}[section]

\newtheorem{teorema}{Teorema}[section]

\newtheorem{property}{Propiedad}[section]

\section*{Objetivo General}



\section*{Objetivos Específicos}


\section*{Introducción}


\section*{Justificación}


\chapter{La matriz de densidad}
Una formulación alternativa, pero equivalente, a la del vector de estado 
de la mecánica cuántica es posible con el lenguaje de la matriz de densidad. 
La preferencia por la matriz de densidad se centra en que provee la descripción
más general de un estado cuántico ya que contempla los llamados estados puros y,
también, mezclas estadísticas. La matriz de densidad se puede concebir 
a partir de la descripción de un ensamble de estados, sin embargo, es importante
desarrollar una caracterización intrínseca de la misma. 

\section{Ensambles de estados cuánticos}
Supongamos que un sistema cuántico se encuentra en cada uno de los estados 
$\ket{\psi _i}$ con probabilidades $p_i$, respectivamente. El conjunto 
$\{p_i,\ket{\psi _i} \}$ es un ensamble de estados puros. La matriz de 
densidad de dicho sistema está definida como
\begin{equation}
	\rho \equiv \sum _i p_i\dyad{\psi_i}{\psi_i}.
	\label{eq:rho_def}
\end{equation}

Por ejemplo, una partícula de espín 1/2 se encuentra en el estado 
$\ket{\psi_1}=\sqrt{\frac{1}{3}}\ket{0} + \sqrt{\frac{2}{3}}\ket{1}$
con probabilidad $p_1=\frac{1}{2}$ y en el estado 
$\ket{\psi_2}=\sqrt{\frac{1}{3}}\ket{0} - \sqrt{\frac{2}{3}}\ket{1}$
con probabilidad $p_2=\frac{1}{2}$. La matriz de densidad es
\begin{align}
	\rho 	&= \frac{1}{2}\dyad{\psi_1}{\psi_1} + 
	\frac{1}{2}\dyad{\psi_2}{\psi_2} \nonumber \\
	\rho	&= \frac{1}{2}\qty(\frac{1}{3}\dyad{0}{0} +
					 \frac{\sqrt{2}}{3}\dyad{0}{1} + 
					 \frac{\sqrt{2}}{3}\dyad{1}{0} +
					 \frac{2}{3}\dyad{1}{1}) + \nonumber \\
				&\hspace{5mm} \frac{1}{2}\qty(\frac{1}{3}\dyad{0}{0} -
					 \frac{\sqrt{2}}{3}\dyad{0}{1} -
					 \frac{\sqrt{2}}{3}\dyad{1}{0} +
					 \frac{2}{3}\dyad{1}{1}) \nonumber \\
	\rho	&= \frac{1}{3}\dyad{0}{0} + \frac{2}{3}\dyad{1}{1}. 
	\label{eq:rho-calc-ex}
\end{align}
Este ejemplo es útil para revisar el cálculo explícito de
la matriz de densidad, específicamente para un estado mixto. Sin
embargo, si ahora consideramos el ensamble $\{ \frac{1}{3}, \ket{0};
\frac{2}{3}, \ket{1} \}$ notemos que \eqref{eq:rho-calc-ex} también es la
matriz de densidad que describe este ensamble. Esto muestra que a más de un
estado mixto le puede corresponder la misma matriz de densidad.\newline

Supongamos que la evolución de un sistema cuántico cerrado se describe por
un operador unitario $U$. Si el sistema inicialmente se encuentra en alguno
de los estados $\ket{\psii}$ con probabilidad $p_i$, entonces después de que
la evolución haya ocurrido el sistema se encontrará en el estado $U\ket{\psii}$ 
con probabilidad $p_i$. De esa manera, la evolución de la matriz de densidad se
describe con la siguiente ecuación 
\begin{align}
	\rho = \sum _ip_i\dyad{\psii}{\psii}
	\xrightarrow[]{U}
	\sum _ip_iU\dyad{\psii}{\psii}U^{\dagger}	=
	U\rho U^{\dagger}.
\end{align}
Las mediciones también se pueden describir con el lenguaje de la matriz de
densidad. Supongamos que se realiza una medición definida por los
operadores de medida $M_m$. Si el estado inicial era $\ket{\psii}$, 
entonces la probabilidad de obtener el resultado $m$ es
\begin{align}
	p\qty(m|i) &= \matrixel{\psii}{M_m^{\dagger}M_m}{\psii}\\
	&= \sum _k \braket{\psii}{k}\matrixel{k}{M_m^{\dagger}M_m}{\psii}\\
	&=\sum _k \matrixel{k}{M_m^{\dagger}M_m}{\psii}\braket{\psii}{k}\\
	&=\Tr \qty(M_m^{\dagger}M_m\dyad{\psii}{\psii}).
\end{align}

%Un sistema cuántico cuyo estado $\ket{\psi}$ es completamente conocido se dice que se encuentra en
%un estado puro. De cualquier otra manera, $\rho$ se encuentra en un estado mixto; se dice que está
%en una mezcla de distintos estados puros en el ensamble para $\rho$. \newpage

\section{Una definición intrínseca de la matriz de densidad}
Intuitivamente, es útil definir a la matriz de densidad a partir de la 
descripción de una mezcla estadística de estados. Sin embargo, la matriz 
de densidad es un concepto más general e independiente de los ensambles de
estados. Por ello, la matriz de densidad está caracterizada por el siguiente
teorema.  
\begin{teorema}[\textbf{Caracterización de la matriz de densidad}]
Un operador $\rho$ es la matriz de densidad asociada a algún ensamble 
$\{p_i, \ket{\psi _i} \}$ si y sólo si satisface las condiciones:
\begin{enumerate}
\item $\Tr \rho = 1$.
\item $\rho \geq 0$.
\end{enumerate}	
\end{teorema}

\begin{proof}
	Supongamos que $\rho = \sum p_i\dyad{\psii}{\psii}$ es una matriz de
	densidad. Entonces 
	\begin{align*}
		\Tr \rho &= \sum _i p_i\Tr \qty(\dyad{\psii}{\psii})\\
		&= \sum _i p_i \sum _j \braket{\psi_j}{\psii}\braket{\psii}{\psi_j}\\
		&=\sum _i \sum _j p_i \delta _{ij}^2\\
		&=\sum _i p_i=1,
	\end{align*}
	por consiguiente la condición de traza unitaria de $\rho$ se cumple. 
	Supongamos que $\ket{\phi}$ es un estado arbitrario. Entonces
	\begin{align*}
		\matrixel{\phi}{\rho}{\phi} &= \sum _i p_i \braket{\phi}{\psii}
		\braket{\psii}{\phi} \\
		&=\sum _i p_i \abs{\braket{\phi}{\psii}}^2\\
		&\geq 0,
	\end{align*}
	por consiguiente $\rho$ es una matriz positiva semidefinida. Por el 
	contrario, supongamos que $\rho$ es una matriz que satisface las condiciones
	de traza unitaria y positividad. Dado que $\rho$ es positiva, entonces
	tiene descomposición espectral 
	\begin{align*}
		\rho = \sum _k \lambda _k \dyad{k}{k},
	\end{align*}
	donde $\ket{k}$ satisfacen la relación de ortogonalidad y $\lambda _k$ son
	autovalores de $\rho$ reales y no negativos. Notamos que $\Tr \rho = \sum _k
	\lambda _k = 1$. Por consiguiente, un sistema en el estado $\ket{k}$ con 
	probabilidad $\lambda_k$ tendrá una matriz de densidad $\rho$. Dicho 
	de otro modo, el ensamble de estados $\{ p_k,\ket{k}\}$ da lugar a la matriz
	de densidad $\rho$.
\end{proof}

\section{Revisión de los postulados de la Mecánica Cuántica}
La equivalencia entre el lenguaje del vector de estado y la matriz de 
densidad es posible reformular los postulados de la mecánica cuántica 
de la siguiente manera.
\begin{itemize}
	\item[] \textbf{Postulado 1.} Un sistema físico tiene asociado un espacio vectorial complejo
	con producto interno que conocido como el espacio de estados del
	sistema. El sistema está completamente descrito por su matriz de densidad,
	que es un operador positivo $\rho$ con traza unitaria, que actúa sobre 
	el espacio de estados del sistema. Si un sistema cuántico se encuentra
	en el estado $\rho _i$ con probabilidad $p_i$, entonces la matriz de
	densidad del sistema es $\sum p_i\rho_i$.
	\item[] \textbf{Postulado 2.} La evolución de un sistema cuántico cerrado está descrita por una transformación
	unitaria. Es decir, el estado $\rho$ del sistema en el tiempo $t_1$ está 
	relacionado con el estado $\rho'$ del sistema en el tiempo $t_2$ por un operador
	unitario $U$ que depende sólo de los tiempos $t_1$ y $t_2$,
	\begin{equation}
	\rho'=U\rho U^{\dagger}.
	\label{eq:postulate1}
	\end{equation}
	\item[] \textbf{Postulado 3.} Las mediciones cuánticas están descritas por un conjunto de 
	operadores de medición $\{M_m\}$. Estos son operadores que actúan sobre el espacio 
	de estados del sistema que se mide. El índice $m$ se refiere a los resultados
	de las mediciones que puedan ocurrir en el experimento. Si el estado del sistema
	cuántico es $\rho$ inmediatamente antes de la medición, entonces la probabilidad
	de medir el resultado $m$ está dada por
	\begin{equation}
	p(m)=\tr \qty(M_m^{\dagger}M_m\rho),
	\label{eq:postulate2_prob}
	\end{equation}						
	y el estado del sistema después de la medición es
	\begin{equation}
	\rho'=\frac{M_m\rho M_m^{\dagger}}{\tr \qty(M_m^{\dagger}M_m\rho)}.
	\label{eq:postulate2_rhoPrime}
	\end{equation}	
	Los operadores de medición deben satisfacer la ecuación de completitud,
	\begin{equation}
	\sum _m M_m^{\dagger}M_m=\mathbb{1}.
	\label{eq:postulate2_completeness}
	\end{equation}
	\item[] \textbf{Postulado 4.} El espacio de estados de un sistema físico compuesto es el producto tensorial 
	de los espacios de estado de los sistemas físicos que componen al sistema total.
	Además, si los sistemas están enumerados de 1 hasta $n$, y el $i$-ésimo sistema
	está preparado en el estado $\rho_i$, entonces el estado del sistema total es
	$\rho_1\otimes\rho_2\otimes\cdots\otimes\rho_n$.
\end{itemize}

\section{El operador de densidad reducido}
Supongamos que tenemos dos sistemas $A$ y $B$. Un estado en el espacio de 
estados total está descrito por el operador de densidad $\rho^{AB}$. El
operador de densidad reducido para el sistema $A$ está definido por
\begin{align}
	\rho^A \equiv \Tr _B\qty(\rho^{AB}),
\end{align}
donde $\Tr_B$ es la traza parcial sobre el sistema $B$. La traza parcial
se define como 
\begin{align}
	\Tr_B (\dyad{a_1}{a_2}\otimes \dyad{b_1}{b_2})
	\equiv
	\dyad{a_1}{a_2}\Tr \qty(\dyad{b_1}{b_2}),
	\label{eq:part_trace-def}
\end{align}
donde $\ket{a_1}$ y $\ket{a_2}$ son cualesquiera dos vectores de estado en
$\mathcal{H}_A$, y $\ket{b_1}$ y $\ket{b_2}$ cualesquiera dos vectores
en $\mathcal{H}_B$. Además de la condición \eqref{eq:part_trace-def} se
requiere que la operación de traza parcial sea lineal.


\section{El espacio de las matrices de densidad} 
\janote{Luego de que me dijiste que no mate con este trabajo la tesis
me parece que esta sección podría estar de más y guardarla para la tesis. 
Igual lo dejo para que de una vez lo puedas ver.}
\section{El espacio de Hilbert-Schmidt}
A un espacio de Hilbert $\mathcal{H}$ complejo de dimensión $N$
le acompaña su espacio dual $\mathcal{H}^*$,
el espacio de las transformaciones lineales de $\mathcal{H}$
al campo de los números complejos $\mathbb{C}$. Otro espacio disponible 
a considerar es el de los operadores lineales que actúan sobre $\mathcal{H}$. 
Cuando este espacio se equipa con 
\begin{equation}
\langle A,B\rangle=c\Tr \qty(A^{\dagger}B),
\label{eq:HS_innerP}
\end{equation}
donde $c\in \mathbb{R}$ establece una escala, se conoce como el espacio de
Hilbert-Schmidt $\mathcal{HS}$. 

El espacio vectorial de los operadores Hermíticos 
$\mathcal{HM}$ es un subespacio real de $N^2$ dimensiones de $\mathcal{HS}$. 
$\mathcal{HM}$ también se puede pensar como el álgebra de Lie de $U(n)$. Los
operadores Hermíticos con traza cero forman un subespacio vectorial de
$\mathcal{HM}$. Por tanto, es posible encontrar una base ortonormal
$\sigma_i$ con
respecto al producto interno \eqref{eq:HS_innerP}. Al agregar la 
matriz identidad un operador hermítico $A$ puede escribirse de la siguiente
manera
\begin{align}
	A &= \tau_0\sqrt{\frac{2}{N}}\mathbb{1} + \sum _{i=1}^{N^2-1}\tau_i\sigma_i,
\end{align}
donde $\tau_0=\frac{\Tr A}{\sqrt{2N}}$ y $\tau_i=\Tr \sigma _iA$.
 
Por otro lado, el conjunto de operadores positivos $\mathcal{P}$ es un 
conjunto convexo contenido dentro de $\mathcal{HM}$ y que
\begin{align}
	\dim \qty[\mathcal{P}] = \dim \qty[\mathcal{HM}]=N^2.
\end{align}

\subsection{Coordinizando $\rho$}
Finalmente, el conjunto de las matrices de densidad consiste en todos los operadores
positivos $\rho$ con traza unitaria. Este conjunto se denota como 
$\M$, donde el exponente enfatiza que el conjunto consiste en 
matrices de $N\times N$. $\M$ es un conjunto convexo que, en $\mathcal{HM}$, 
es la intersección del conjunto de los operadores positivos con un 
hiperplano paralelo al subespacio lineal de los operadores hermíticos
con traza igual a cero \cite{bengtsson_zyczkowski_2017}.

Una manera posible de \textit{coordinizar} $\M$ es 
\begin{align}
	\rho = \frac{1}{N} \mathbb{1} + \sum _{i=1}^{N^2-1} \tau_i\sigma_i,	
	\label{eq:rho_general}
\end{align}
donde la matriz cero se ha cambiado por la matriz
\begin{align}
	\rho _{\star} \equiv \frac{1}{N}\mathbb{1},
	\label{eq:max_mixed_state}
\end{align}
que se conoce como el estado máximamente mixto o la `matriz de ignorancia'. 
Las componentes $\tau_i$ en \eqref{eq:rho_general} se conocen como `coordenadas
de mezcla'.

\section{Transformaciones unitarias}

Teorema de Kadison \cite{bengtsson_zyczkowski_2017}:
\begin{teorema}[\textbf{Teorema de Kadison}] 
	Supongamos un mapeo $\Phi:\M \mapsto \M$ que es uno-a-uno y que es afín y
	que preserva la estructura compleja en el sentido que
	\begin{align}
		\Phi \qty(p\rho_1 + \qty(1-p)\rho_2) = p\Phi(\rho_1) + 
		\qty(1-p)\Phi(\rho_2).
	\end{align}
	Entonces el mapeo debe tomar la forma
	\begin{align}
		\Phi(\rho) = U\rho U^{-1},
	\end{align}
	donde el operador $U$ es unitario o antiunitario.
\end{teorema}

Este teorema nos permite ver que para que un mapeo preserve la estructura 
convexa debe ser un mapeo afín y que debe mapear estados putos
a estados puros. Por ejemplo, para $N=2$ los estados puros forman una
esfera. Los únicos mapeos afines que preservan la esfera son las rotaciones.



\chapter{Purificación de estados mixtos}

\section{Productos tensoriales y reducción de estados}

\section{Descomposición de Schmidt}

\section{Purificación de estados}



\chapter{Operaciones cuánticas}

\section{Ejemplos de operaciones cuánticas}

\section{Mediciones y POVM's}

\section{Matrix reshaping y reshuffling}

\section{Mapeos positivos y completamente positivos}

\section{Representaciones del entorno}

\section{Mapeos de un qubit}



\chapter{Mapeos que borran componentes arbitrarias de $\boldsymbol{\rho}$}



\chapter{Método numérico}



\chapter{El caso de 2 qubits}



\chapter{Conclusiones y trabajo futuro}



\bibliographystyle{abbrv}
\bibliography{references}

\end{document}
