% !TeX spellcheck = en_US
\documentclass[12pt]{report}
\usepackage[spanish]{babel}
\usepackage[utf8]{inputenc}
\usepackage[T1]{fontenc}
\usepackage{makeidx}
\usepackage{graphicx}
\usepackage{subfig}
\usepackage{amsmath}
\usepackage{amsfonts}
\usepackage{amssymb}
\usepackage{authblk} % para la manipulación de autores y afiliación
\usepackage[pdftex]{hyperref}
\usepackage{multirow}
\usepackage{multicol}
\usepackage{float}
\usepackage{booktabs}
\usepackage{colortbl}
\usepackage{bbold}
\usepackage{physics}


\usepackage[]{lineno}  \linenumbers
\setlength\linenumbersep{3pt}

%\usepackage{hyperref}
%\usepackage{commath}
\decimalpoint
\renewcommand{\tablename}{Tabla}
\oddsidemargin 0in
\textwidth 6.5in
\topmargin -0.5in
\textheight 8.5in

%<stdio.h>

\begin{document}

\begin{titlepage} % Suppresses displaying the page number on the title page and the subsequent page counts as page 1                                  
\newcommand{\HRule}{\rule{\linewidth}{0.6mm}} % Defines a new command for horizontal lines, change thickness here                             

\center % Centre everything on the page                                                                                                       

%------------------------------------------------                                                                                             
%       Title                                                                                                                                 
%------------------------------------------------                                                                                             

\HRule\\[0.4cm]

{\LARGE\bfseries Estudio de canales cuánticos que borran\\
				 componentes arbitrarias de la matriz densidad $\boldsymbol{\rho}$\\
				 de un sistema de $\mathbf{n}$ qubits}\\[0.4cm] % Title of your document                                                 

\HRule\\[2cm]

%------------------------------------------------                                                                                             
%       Author(s)                                                                                                                             
%------------------------------------------------                                                                                             


\Large{\textbf{José Alfredo de León Garrido}}\\[2cm]

%------------------------------------------------                                                                                             
%       Headings                                                                                                                              
%------------------------------------------------                                                                                             

\textsc{\Large Universidad de San Carlos de Guatemala\\
		Escuela de Ciencias Físicas y Matemáticas\\
		Licenciatura en Física}\\[2cm]

\textsc{{\Large\bfseries Informe final}}\\
\textsc{\large Año de prácticas}\\[2cm]

\textsc{\large Supervisado por:\\
		\textbf{Dr. Carlos Pineda (IF-UNAM) y\\
		M.Sc. Juan Diego Chang (ICFM-USAC)}}


%------------------------------------------------                                                                                             
%       Date                                                                                                                                  
%------------------------------------------------                                                                                             
\vfill\vfill\vfill % Position the date 3/4 down the remaining page
\vfill\vfill\vfill

{\large 13 de enero 2020} % Date, change the \today to a set date if you want to be precise                                                              

%------------------------------------------------                                                                                             
%       Logo                                                                                                                                  
%------------------------------------------------                                                                                             

%\vfill\vfill                                                                                                                                 
%\includegraphics[width=0.2\textwidth]{placeholder.jpg}\\[1cm] % Include a department/university logo                                                                                                                                      

%----------------------------------------------------------------------------------------                                                     

\vfill % Push the date up 1/4 of the remaining page  
\end{titlepage}

\newtheorem{definition}{Definición}[section]

\newtheorem{teorema}{Teorema}[section]

\newtheorem{property}{Propiedad}[section]

\section*{Objetivo General}



\section*{Objetivos Específicos}


\section*{Introducción}


\section*{Justificación}


\chapter{La matriz densidad}
Una formulación alternativa de la mecánica cuántica es posible utilizando la herramienta de
la matriz densidad. Esta es una formulación alternativa al enfoque de los vectores de estado,
pero es matemáticamente equivalente y, de hecho, más conveniente para cierto tipo de problemas
en mecánica cuántica. 

\section{Ensambles de estados cuánticos}
El lenguaje de la matriz densidad permite describir sistemas cuánticos cuyos estados no son 
conocidos por completo. Supongamos que un sistema cuántico se encuentra en uno de los estados 
$\{\ket{\psi _i}\}$, con respectivas probabilidades $p_i$. Llamamos al conjunto $\{p_i, 
\ket{\psi _i} \}$ un ensamble de estados puros. La matriz densidad para dicho sistema está 
definida como
\begin{equation}
	\rho \equiv \sum _i p_i\dyad{\psi_i}{\psi_i}.
	\label{eq:rho_def}
\end{equation}

Todos los postulados de la mecánica cuántica pueden ser reformulados en el lenguaje de la matriz
densidad. Los mismos resultados se obtienen si se utiliza el lenguaje de la matriz densidad o el
lenguaje del vector de estado. Sin embargo, dependiendo el problema uno de los lenguajes será más
conveniente y práctico de utilizar que el otro. 

Un sistema cuántico cuyo estado $\ket{\psi}$ es completamente conocido se dice que se encuentra en
un estado puro. De cualquier otra manera, $\rho$ se encuentra en un estado mixto; se dice que está
en una mezcla de distintos estados puros en el ensamble para $\rho$. \newpage

\section{Definición de $\boldsymbol{\rho}$ independiente del vector de estado}
Es útil y conveniente dar una descripción de la matriz densidad que no esté basado en el vector de
estado. La matriz densidad está caracterizada por el siguiente teorema: 
\begin{teorema}
Un operador $\rho$ es la matriz densidad asociada a un ensamble $\{p_i, \ket{\psi _i} \}$ si y
sólo si comple con las siguientes condiciones:
\begin{enumerate}
\item $\rho$ tiene traza unitaria.
\item $\rho$ es una matriz positiva semidefinida. 
\end{enumerate}	
\end{teorema}

Este teorema permite caracterizar la matriz densidad de manera intrínsica a la matriz misma. El 
conjunto de las matrices densidad será denotado como $\mathcal{M}^{(n)}$. Este es un conjunto convexo
que se encuentra dentro del espacio vectorial de las matrices Hermíticas y sus estados putos son 
matrices de densidad que satisfacen la ecuación $\rho^2=\rho$. Es decir, las estados puros forman 
un espacio proyectivo complejo. 

\section{Postulados de la mecánica cuántica}
Con esta definición es posible reformular los postulados de la mecánica cuántica con el lenguaje de 
la matriz densidad. 
\begin{itemize}
	\item \textbf{Postulado 1:} Cualquier sistema físico tiene asociado un espacio vectorial complejo
	con producto interno que se conoce como el espacio de estados del
	sistema. El sistema está completamente descrito por su operador densidad,
	que es un operador positivo $\rho$ con traza unitaria, que actúa sobre 
	el espacio de estados del sistema. Si un sistema cuántico se encuentra
	en el estado $\rho _i$ con probabilidad $p_i$, entonces el operador densidad
	del sistema es $\sum p_i\rho_i$.
	\item \textbf{Postulado 2:} La evolución de un sistema cuántico cerrado está descrita por una transformación
	unitaria. Es decir, el estado $\rho$ del sistema en el tiempo $t_1$ está 
	relacionado al estado $\rho'$ del sistema en el tiempo $t_2$ por un operador
	unitario $U$ que depende sólo de los tiempos $t_1$ y $t_2$,
	\begin{equation}
	\rho'=U\rho U^{\dagger}.
	\label{eq:postulate1}
	\end{equation}
	\item \textbf{Postulado 3:} Las mediciones cuánticas están descritas por un conjunto $\{M_m\}$ de 
	operadores de medición. Estos son operadores que actúan sobre el espacio 
	de estados del sistema que se mide. El índice $m$ se refiere a los resultados
	de las mediciones que puedan ocurrir en el experimento. Si el estado del sistema
	cuántico es $\rho$ inmediatamente antes de la medición, entonces la probabilidad
	de medir el resultado $m$ está dado por
	\begin{equation}
	p(m)=\tr \qty(M_m^{\dagger}M_m\rho),
	\label{eq:postulate2_prob}
	\end{equation}						
	y el estado del sistema después de la medición es
	\begin{equation}
	\rho'=\frac{M_m\rho M_m^{\dagger}}{\tr \qty(M_m^{\dagger}M_m\rho)}.
	\label{eq:postulate2_rhoPrime}
	\end{equation}	
	Los operadores de medición satisfacen la ecuación 
	\begin{equation}
	\sum _m M_m^{\dagger}M_m=\mathbb{1}.
	\label{eq:postulate2_completeness}
	\end{equation}
	\item \textbf{Postulado 4:} El espacio de estados de un sistema físico compuesto es el producto tensorial 
	de los espacios de estado de los sistemas físicos que componen al sistema total.
	Además, si los sistemas están enumerados de 1 hasta $n$, y el $i$-ésimo sistema
	está preparado en el estado $\rho_i$, entonces el estado del sistema total es
	$\rho_1\otimes\rho_2\otimes\cdots\otimes\rho_n$.
\end{itemize}

\section{El espacio de Hilbert-Schmidt}
Partiendo de un de un espacio de Hilbert $\mathcal{H}$ complejo de dimensión $n$ existe un espacio de 
Hilbert dual $\mathcal{H}^*$ que se define como el espacio de las transformaciones lineales de $\mathcal{H}$
al campo de los números complejo $\mathbb{C}$. Otro espacio de interés es el espacio de los operadores 
lineales que actúan sobre $\mathcal{H}$. Este espacio está equipado con el producto interno
\begin{equation}
\langle A,B\rangle=c\Tr \qty(A^{\dagger}B),
\label{eq:HS_innerP}
\end{equation}
donde $c\in \mathbb{R}$ establece una escala. A este espacio se le conoce como el espacio de
Hilbert-Schmidt $\mathcal{HS}$. Dado un espacio de Hilbert $\mathcal{H}$ uno siempre cuenta también
con el espacio complejo de $n^2$ dimensiones $\mathcal{HS}$, también conocido como el álgebra de las 
matrices complejas.

Un operador $A$ puede diagonalizarse por un cambio de base unitario si y sólo si $A$ es normal, es decir,
si y sólo si $\comm{A}{A^{\dagger}}=0$. Cualquier operador normal se puede escribir de la forma
\begin{equation}
A=\sum _i=1^rz_i\dyad{e_i}{e_i},
\end{equation} 
donde la suma va sobre los proyectores ortogonales que corresponden a los $r$ eigenvalores $z_i$ que
son distintos de cero. Los eigenvectores $\ket{e_i}$ generan al subespacio lineal que se conoce como
el rango del operador $A$. Además, los eigenvectores con eigenvalor cero forman al subespacio que se
conoce como el kernel de $A$. Por consiguiente, el espacio de Hilbert se puede escribir como suma
directa de estos dos subespacios. 

El espacio vectorial $\mathcal{HM}$ de los operadores Hermíticos es un subespacio real de $n^2$ dimensiones
del espacio $\mathcal{HS}$. $\mathcal{HM}$ también se puede pensar como el álgebra de Lie de $U(n)$. El 



\begin{teorema}
Los conjuntos $\ket{\overset{\sim}{\psi}_i}$ y $\ket{\overset{\sim}{\phi}_j}$ generan la misma matriz densidad si
y sólo si 
\begin{equation}
\ket{\overset{\sim}{\psi}_i}=\sum u_{ij}\ket{\overset{\sim}{\phi}_j},
\end{equation}
donde $u_{ij}$ es una matriz unitaria de números complejos.
\end{teorema}

La matriz densidad también es una herramienta útil para describir subsistemas de sistemas cuánticos compuestos.
Tal descripción es proporcionada por la matriz densidad reducida. 

Supongamos que el estado de dos sistemas $A$ y $B$ está descrito por la matriz $\rho ^{AB}$. La matriz densidad del
sistema $A$ se define como
\begin{equation}
\rho ^A=\Tr _B\qty(\rho ^{AB}),
\label{eq:reduced_DM}
\end{equation}
donde la operación $\Tr _B$ es la traza parcial sobre el sistema $B$. La traza parcial se define como
\begin{equation}
\Tr _B\qty(\dyad{a_1}{a_2}\otimes \dyad{b_1}{b_2}) \equiv \dyad{a_1}{a_2}\Tr \qty(\dyad{b_1}{b_2}),
\end{equation}
donde $\ket{a_1}$ y $\ket{a_2}$ son dos vectores cualquiera en el espacio de estados de $A$ y $\ket{b_1}$ y $\ket{b_2}$
son dos vectores cualquiera en el espacio de estados de $B$. 


\section{Estados mixtos}

\section{Transformaciones unitarias}

\section{El espacio de las matrices densidad como un conjunto convexo}



\chapter{Purificación de estados mixtos}

\section{Productos tensoriales y reducción de estados}

\section{Descomposición de Schmidt}

\section{Purificación de estados}



\chapter{Operaciones cuánticas}

\section{Ejemplos de operaciones cuánticas}

\section{Mediciones y POVM's}

\section{Matrix reshaping y reshuffling}

\section{Mapeos positivos y completamente positivos}

\section{Representaciones del entorno}

\section{Mapeos de un qubit}



\chapter{Mapeos que borran componentes arbitrarias de $\boldsymbol{\rho}$}



\chapter{Método numérico}



\chapter{El caso de 2 qubits}



\chapter{Conclusiones y trabajo futuro}



\bibliographystyle{abbrv}
\bibliography{references}

\end{document}
