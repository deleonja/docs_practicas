\chapter{Mapeos proyectivos}  
% Intro {{{
\janote{Introducción. Motivar y justificar el estudio de las operaciones
que borran componentes de la matriz de densidad de un 1 qubit. Discutir
que entender los canales cuánticos que borran componentes de la 
matriz de densidad ayudaría a entender mejor la teoría 
de las operaciones CPTP y, también, la mecánica cuántica porque al fin y al 
cabo estamos estudiando qué permite y qué no la mecánica 
cuántica.}

\cpnote{La justificacion me gustaria ver en pseudocodigo antes de que la afines. 
Menciona un poco también la relación con los otros mapeos, pero muy por encima, como 
los de Ruskai}

% }}}
\section{El caso de 1 qubit} % {{{
\janote{Introducción sobre los canales cuánticos de 1 qubit. Recordar 
la forma de la matriz de densidad escrita en la representación 
de Pauli y su asociación con la esfera de Bloch.}

\janote{Establecer el problema que nos interesa para 1 qubit:
mapeos que borran y dejan invariantes todas las combinaciones 
posibles de las componentes del vector de Bloch. 
Discutir que la componente $r_0$ la dejamos intacta para 
preservar la traza. Hablar de la representación geométrica 
de las operaciones que estudiamos.}

\janote{Presentar la forma analítica de calcular la representación
en forma de superoperador de alguno de los 8 mapeos posibles. 
A diferencia de lo que ya se hizo en el capítulo anterior, voy a presentar
el cálculo a partir de la forma diagonal del mapeo (también para ir 
dando las herramientas para el algoritmo numérico)}

\janote{Hablar sobre la equivalencia entre los 3 canales que dejan
1 componente invariante del vector de Bloch y sobre la identidad
y el totalmente depolarizante como canales triviales y extremos 
de nuestros canales.}

\janote{Resumir los resultados y comparar con los resultados que 
exponen en el Geometry of Quantum States de la forma general de
cualquier canal cuántico de 1 qubit. Allí tienen la forma general
de cualquier canal cuántico sobre 1 qubit, creo que es un buen
check para mostrar que entendí el caso de 1 qubit.}

\cpnote{Me parece bien, pero porfa no te extiendas de más en este capitulo}

Recordemos que la matriz de densidad de 1 qubit se escribe como
\begin{align}
\rho = \frac{1}{2}\sum_{i=0}^3r_i\sigma_i
=\frac{1}{2}\mqty(1+r_3 & r_1-ir_2\\
r_1-ir_2 & 1-r_3),
\label{eq:dm-1q-pauli}
\end{align}
donde el vector definido como $\qty(r_1,r_2,r_3)$ 
determina un punto en la esfera de Bloch como se 
muestra en la \Fref{}. 

Vamos a continuación a establecer el tipo operaciones cuánticas
que son de interés para este proyecto. Nos interesa estudiar 
las operaciones cuánticas que borran un número arbitrario 
de componentes de la matriz de densidad de 1 qubit 
escrita en la base de las matrices de Pauli 
como en \eqref{eq:dm-1q-pauli}, excepto la primera 
componente. Es decir, la componente $r_0=1$ es invariante 
por cualquiera de estas operaciones, para así asegurar que 
$\Tr\qty(\rho)=1$ se mantiene invariante.

De esa manera, tenemos tres casos según el número de componentes
que borra la operación. Estas operaciones se pueden entender 
de manera geométrica de la siguiente manera. 
\begin{enumerate}
\item Borrar 0 componentes $r_i$. Esto es igual a dejar invariante 
por completo a la esfera de Bloch. 
\item Borrar 1 componente $r_i$. Esto es equivalente a deformar la bola 
de Bloch en un disco. Un ejemplo así discutimos en el capítulo 
anterior. 
\item Borrar 2 componentes $r_i$. Hay 3 opciones. Esto es equivalente
a colapsar dos dimensiones de la bola de Bloch y deformarla a
una línea sobre alguno de los ejes. 
\item Borrar las 3 componentes $r_i$. Esta operación geométricamente 
es igual a colapsar las 3 dimensiones de la esfera de Bloch. 
Esta operación es el caso límite del canal depolarizante $(p=1)$
del capítulo anterior. 
\end{enumerate}
En principio, con los ejemplos que se desarrollaron en el capítulo 
anterior ya sabemos cuáles de estas operaciones son operaciones
cuánticas y cuáles no. Por lo tanto, vamos a justificar porque las 
operaciones son equivalentes. Son equivalentes porque 
sólo ocurre un cambio de base local. 
\begin{align}
P \E P\qty(\rho),
\end{align}
con $P$ el cambio de base local.
% }}}
\section{Solución numérica} % {{{
\janote{Introducción para justificar porqué nos interesa reproducir
lo de 1 qubit de forma numérica (porque queremos estudiar 2+ qubits).}

\janote{Enunciar el algoritmo.}

\janote{Hablar de que se implementó en Mathematica y por 
aquí meter la info al repositorio y añadir la idea que tienes 
con eso.}

\janote{Concluir comparando los resultados del numérico con
lo de la sección anterior.}

\janote{En el siguiente capítulo (último último) hablamos sobre
seguir trabajando en este proyecto para la tesis, 
pero ahora con más qubits, y concluir discutiendo 
que se cumplieron con los objetivos planteados
para este trabajo de prácticas.}
\cpnote{Si, todo bien}

Algoritmo. \textbf{Entrada:} Lista con elementos de la diagonal de
la operación. \textbf{Salida:} matriz de la operación.
\begin{enumerate}
\item Construir la matriz diagonal.
\item Efectuar la operación $T\qty[\E]_{\sigma}T$ para hacer
el cambio de base.
\end{enumerate}

Se implementó un paquete en Mathematica con las funciones:
\begin{enumerate}
\item \textit{Reshuffle}: implementa el procedimiento de 
\textit{Reshuffle} sobre una matriz de dimensión $N\times N$,
con $N=2^k, k\in \mathbb{Z}^+$.

\item PCE: Su nombre viene de \textit{``Pauli component-erasing''}
(PCE) e implementa el cálculo de la representación matricial, en la 
base computacional, de
un superoperador $\E$ a partir de los elementos de la diagonal
de la matriz en la base de productos tensoriales de las matrices 
de Pauli.
\end{enumerate}.


% }}}


