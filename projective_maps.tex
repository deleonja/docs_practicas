\chapter{Mapeos proyectivos}  
\janote{Introducción. Motivar y justificar el estudio de las operaciones
que borran componentes de la matriz de densidad de un 1 qubit. Discutir
que entender los canales cuánticos que borran componentes de la 
matriz de densidad ayudaría a entender mejor la teoría 
de las operaciones CPTP y, también, la mecánica cuántica porque al fin y al 
cabo estamos estudiando qué permite y qué no la mecánica 
cuántica.}

\section{El caso de 1 qubit}
\janote{Introducción sobre los canales cuánticos de 1 qubit. Recordar 
la forma de la matriz de densidad escrita en la representación 
de Pauli y su asociación con la esfera de Bloch.}

\janote{Establecer el problema que nos interesa para 1 qubit:
mapeos que borran y dejan invariantes todas las combinaciones 
posibles de las componentes del vector de Bloch. 
Discutir que la componente $r_0$ la dejamos intacta para 
preservar la traza. Hablar de la representación geométrica 
de las operaciones que estudiamos.}

\janote{Presentar la forma analítica de calcular la representación
en forma de superoperador de alguno de los 8 mapeos posibles. 
A diferencia de lo que ya se hizo en el capítulo anterior, voy a presentar
el cálculo a partir de la forma diagonal del mapeo (también para ir 
dando las herramientas para el algoritmo numérico)}

\janote{Hablar sobre la equivalencia entre los 3 canales que dejan
1 componente invariante del vector de Bloch y sobre la identidad
y el totalmente depolarizante como canales triviales y extremos 
de nuestros canales.}

\janote{Resumir los resultados y comparar con los resultados que 
exponen en el Geometry of Quantum States de la forma general de
cualquier canal cuántico de 1 qubit. Allí tienen la forma general
de cualquier canal cuántico sobre 1 qubit, creo que es un buen
check para mostrar que entendí el caso de 1 qubit.}

\section{Solución numérica}
\janote{Introducción para justificar porqué nos interesa reproducir
lo de 1 qubit de forma numérica (porque queremos estudiar 2+ qubits).}

\janote{Enunciar el algoritmo.}

\janote{Hablar de que se implementó en Mathematica y por 
aquí meter la info al repositorio y añadir la idea que tienes 
con eso.}

\janote{Concluir comparando los resultados del numérico con
lo de la sección anterior.}

\vspace{2cm}
\janote{En el siguiente capítulo (último último) hablamos sobre
seguir trabajando en este proyecto para la tesis, 
pero ahora con más qubits, y concluir discutiendo 
que se cumplieron con los objetivos planteados
para este trabajo de prácticas.}
