\chapter{Mapeos proyectivos}  
% Intro {{{
\janote{Introducción. Motivar y justificar el estudio de las operaciones
que borran componentes de la matriz de densidad de un 1 qubit. Discutir
que entender los canales cuánticos que borran componentes de la 
matriz de densidad ayudaría a entender mejor la teoría 
de las operaciones CPTP y, también, la mecánica cuántica porque al fin y al 
cabo estamos estudiando qué permite y qué no la mecánica 
cuántica.}

\cpnote{La justificacion me gustaria ver en pseudocodigo antes de que la afines. 
Menciona un poco también la relación con los otros mapeos, pero muy por encima, como 
los de Ruskai}

% }}}
\section{El caso de 1 qubit} % {{{
\janote{Introducción sobre los canales cuánticos de 1 qubit. Recordar 
la forma de la matriz de densidad escrita en la representación 
de Pauli y su asociación con la esfera de Bloch.}

\janote{Establecer el problema que nos interesa para 1 qubit:
mapeos que borran y dejan invariantes todas las combinaciones 
posibles de las componentes del vector de Bloch. 
Discutir que la componente $r_0$ la dejamos intacta para 
preservar la traza. Hablar de la representación geométrica 
de las operaciones que estudiamos.}

\janote{Presentar la forma analítica de calcular la representación
en forma de superoperador de alguno de los 8 mapeos posibles. 
A diferencia de lo que ya se hizo en el capítulo anterior, voy a presentar
el cálculo a partir de la forma diagonal del mapeo (también para ir 
dando las herramientas para el algoritmo numérico)}

\janote{Concluir presentando las 8 operaciones de 1 qubit que nos intersan.}
\cpnote{Me parece bien, pero porfa no te extiendas de más en este capitulo}

Los canales cuánticos de 1 qubit son especialmente interesantes porque 
son fáciles de visualizar en la esfera de Bloch, como se presentó
en la sección \ref{sec:CC's-1q-ejemplos}, y eso hace que 
aporten mucha intuición física para entender la teoría de
las operaciones cuánticas. 
Particularmente, recordemos 
las operaciones $\E_z$ y $\E_{xz}$ (Figs. \ref{fig:qtm-op-motivation}
y \ref{fig:QC-ex2}) que borran las componentes en $z$, y 
$x$ y $z$ del vector de Bloch, respectivamente. Ambas operaciones
transforman estados de 1 qubit en estados de 1 qubit. No obstante, 
vimos que $\E_z$ no es una operación completamente positiva porque 
$\E_z\otimes\1 _2$ transforma al estado máximamente entrelazado de 
dos qubits en una matriz que no es una matriz de densidad;
por consiguiente, $\E_z$ no es un canal cuántico. 
Motivados en estas dos operaciones, nos parece interesante 
estudiar las operaciones que borran las componentes del 
vector de Bloch de la matriz de densidad de 1 qubit.

Para ser más precisos, el problema que abordaremos es
el siguiente: determinar el subconjunto de canales cuánticos
del conjunto de operaciones lineales que borran componentes 
del vector de Bloch de la matriz de densidad de 1 qubit.
Para esto, es necesario recordar que la matriz de densidad 
de 1 qubit se puede representar en la base de las matrices 
de Pauli como
\begin{align}
\rho = \frac{\1}{2}+\sum_{i=1}^{3}r_i\sigma_i,
\label{eq:ch3-rho-1q}
\end{align}
con $r_i$ las componentes del vector de Bloch y $\sigma_i$ las 
matrices de Pauli. Por lo tanto, nuestro objetivo es encontrar
las operaciones CP que borran un número arbitrario de 
$r_i$ de \eqref{eq:ch3-rho-1q}. Puesto que una matriz 
$\rho$, de la forma \eqref{eq:ch3-rho-1q} con cualquier
conjunto de $r_i$ iguales a cero, satisface
(i) $\Tr(\rho)$, (ii) $\rho=\rho^{\dagger}$ y (iii) $\rho\geq0$,
entonces la única condición que resta por 
evaluar para determinar si las operaciones de nuestro estudio
son canales cuánticos es la completa positividad. 
En el resto de esta sección presentaremos una manera 
analítica y sistemática de calcular la forma matricial de una operación
y evaluar si es completamente positiva.
%
%
%Los mapeos que deseamos estudiar son especialmente sencillos
%de visualizar en la esfera de Bloch: mapeos que borran componentes
%arbitrarias del vector de Bloch de una matriz de densidad $\rho$ 
%de la forma \eqref{eq:ch3-rho-1q}. Explotando la representación 
%de la esfera de Bloch los mapeos de nuestro interés son cuatro:
%\begin{enumerate}
%\item La identidad, una operación que deja intacta a la esfera de Bloch.
%\item Tres mapeos que colapsa una dimensión de la esfera de Bloch. Es decir,
%la esfera de Bloch la deforma a un disco. 
%\item Tres mapeos que colapsan dos dimensiones. Es decir, la esfera
%de Bloch se deforma a una línea sobre uno de los ejes. 
%\item Un mapeo que colapsa las tres dimensiones. La esfera de Bloch
%colapsa a un punto en el origen. Este canal se conoce como el 
%completamente depolarizante, dado que es el caso límite $(p=1)$
%del canal depolarizante que presentamos en la sección 2.4.2.
%\end{enumerate}

%\subsection{Cálculo analítico}
%\begin{enumerate}
%\item Se calcula la forma matricial $\qty[\E]_{\sigma}$
%de la operación en la base de las matrices de Pauli, base en 
%la cual la operación es diagonal. 
%\item Se hace un cambio de base $P\qty[\E]_{\sigma}P^{-1}$ 
%para encontrar la forma matricial $\E$ de la operación 
%en la base computacional.
%\item Se calcula $\E^R$ para calcular la matriz de Choi $D_{\E}$
%de la operación.
%\item Se evalúa la positividad semidefinida de $D_{\E}$; si cumple,
%entonces se concluye que $\E$ es un canal cuántico. 
%\end{enumerate}
%Debe notarse que la clase de operaciones
%que estamos estudiando mantiene invariante la traza unitaria,
%Hermiticidad y positividad de la matriz de densidad del sistema de 
%qubits que se está considerando. La única condición que resta 
%evaluar a las operaciones es la de completa positividad. Por
%ejemplo, las operaciones que borran componentes de la matriz 
%de densidad de 1 qubit es un mapeo afín de matrices de 
%densidad de 1 qubit, pero no todas son CP.

Vamos a abordar una vez más la operación $\E_z$
para mostrar el procedimiento análitico para 
calcular la forma matricial de una operación 
y evaluar su completa positividad. La diferencia 
de este procedimiento con lo que mostramos en el
capítulo anterior será que está formulado para 
poderse implementar de manera numérica.
Recordemos que $\E_z$ transforma a las componentes 
de la matriz de densidad de 1 qubit como
\begin{align}
\qty(1,r_1,r_2,r_3)\longrightarrow\qty(1,r_1,r_2,0),
\end{align}
por lo que la forma matricial de $\E_z$ debe ser diagonal 
en la base de Pauli 
$\{ \vec{\sigma}_0,\vec{\sigma}_1,\vec{\sigma}_2,\vec{\sigma}_3\}$, 
que son las matrices de Pauli y $\sigma_0=\1$ como matrices vectorizadas.
Denotemos $\qty[\E_z]_{\sigma}$ a la forma matricial de
$\E_z$ en la base de Pauli. 
Los elementos de la diagonal $(\qty[\E_{z}]_{\sigma})_{ii}$ 
deben satisfacer la ecuación $r_i'=(\qty[\E_{z}]_{\sigma})_{ii}r_i$, 
con $r_0=1$; así se determina 
\begin{align}
\qty[\E_z]_{\sigma}=
\mqty(
 1 & 0 & 0 & 0 \\
 0 & 1 & 0 & 0 \\
 0 & 0 & 1 & 0 \\
 0 & 0 & 0 & 0 
).
\end{align}
El siguiente paso es hacer un cambio de base para escribir
a $\qty[\E_z]_{\sigma}$ en la base computacional.
Para ello, construimos a la matriz de cambio de base $P$
yuxtaponiendo a los vectores en la base de Pauli,
\begin{align}
P=\mqty(
 1 & 0 & 0 & 1 \\
 0 & 1 & -i & 0 \\
 0 & 1 & i & 0 \\
 1 & 0 & 0 & -1 
).
\end{align}
Se efectúa el cambio de base, siguiendo la expresión 
$P\qty[\E_z]_{\vec{\sigma}}P^{-1}$, y tenemos
\begin{align}
\mqty(
 1 & 0 & 0 & 1 \\
 0 & 1 & -i & 0 \\
 0 & 1 & i & 0 \\
 1 & 0 & 0 & -1 
)
\mqty(
 1 & 0 & 0 & 0 \\
 0 & 1 & 0 & 0 \\
 0 & 0 & 1 & 0 \\
 0 & 0 & 0 & 0 
)
\mqty(
 \frac{1}{2} & 0 & 0 & \frac{1}{2} \\
 0 & \frac{1}{2} & \frac{1}{2} & 0 \\
 0 & \frac{i}{2} & -\frac{i}{2} & 0 \\
 \frac{1}{2} & 0 & 0 & -\frac{1}{2} 
)
&=
\mqty(
 \frac{1}{2} & 0 & 0 & \frac{1}{2} \\
 0 & 1 & 0 & 0 \\
 0 & 0 & 1 & 0 \\
 \frac{1}{2} & 0 & 0 & \frac{1}{2} 
).
\end{align}
De esta manera, hemos llegado a la misma forma matricial de $\E_z$
en \eqref{eq:Ez-matrix} (sección \ref{sec:ch2-matrixForm}). 
Lo que falta por hacer es determinar si $\E_z$ es CP, 
para ello, necesitamos calcular la matriz
de Choi $D_{\E_z}$, siguiendo la ecuación \eqref{eq:R-4ind}, y 
evaluar si es positiva. La manera en la que evaluaremos 
si $D_{\E_z}$ es positiva será calculando los eigenvalores 
y evaluando que todos sean no negativos. 
De hecho, esto fue lo que hicimos en 
la ecuación \eqref{eq:ch2-Choi-Ez},
calculamos los eigenvalores de $D_{\E_z}$ y  
encontramos que uno de ellos es igual a $-1/2$, 
por lo cual se concluye que $\E_z$ no es un canal cuántico.

Para ser prácticos vamos a introducir un nombre para
las operaciones de estudio: ``Operaciones
que borran componentes de Pauli'', u ``operaciones PCE",
por su nombre en inglés (\textit{Pauli-components-erasing}). 
Las operaciones PCE de 1 qubit que vamos a estudiar 
se muestran en el \Cref{cuadro:operacionesPCE-1q}.
\begin{table}
\centering
\begin{tabular}{|c|l|} 
\hline
\textbf{Operación PCE} & 
\textbf{Transformación de las componentes} $\mathbf{r_i}$ \\
\hline
$\E_{}$ & \hspace{1.2cm}$\qty(1,r_1,r_2,r_3)\longrightarrow\qty(1,r_1,r_2,r_3)$ \\ 
\hline 
$\E_{x}$ & \hspace{1.2cm}$\qty(1,r_1,r_2,r_3)\longrightarrow\qty(1,0,r_2,r_3)$ \\ 
\hline 
$\E_{y}$ & \hspace{1.2cm}$\qty(1,r_1,r_2,r_3)\longrightarrow\qty(1,r_1,0,r_3)$ \\ 
\hline 
$\E_{z}$ & \hspace{1.2cm}$\qty(1,r_1,r_2,r_3)\longrightarrow\qty(1,r_1,r_2,0)$ \\ 
\hline 
$\E_{yz}$ & \hspace{1.2cm}$\qty(1,r_1,r_2,r_3)\longrightarrow\qty(1,r_1,0,0)$ \\ 
\hline 
$\E_{xz}$ & \hspace{1.2cm}$\qty(1,r_1,r_2,r_3)\longrightarrow\qty(1,0,r_2,0)$ \\ 
\hline 
$\E_{xy}$ & \hspace{1.2cm}$\qty(1,r_1,r_2,r_3)\longrightarrow\qty(1,0,0,r_3)$ \\ 
\hline 
$\E_{xyz}$ & \hspace{1.2cm}$\qty(1,r_1,r_2,r_3)\longrightarrow\qty(1,0,0,0)$ \\ 
\hline
\end{tabular}   
\caption{Operaciones PCE de 1 qubit. 
Los subíndices de $\E$ indican las componentes $r_i$ 
del vector de Bloch que la operación borra.}
\label{cuadro:operacionesPCE-1q}
\end{table}

Hemos establecido que el problema que vamos a 
estudiar es el de determinar los canales cuánticos 
del conjunto de las operaciones PCE de 1 qubit. 
Por esa razón, mostramos un procedimiento analítico
y sistemático para determinar si una operación PCE 
es completamente positiva. 
En la siguiente sección presentaremos el método 
numérico que implementamos para resolver problema.

% }}}
\section{Método numérico} % {{{
\janote{Introducción para justificar porqué nos interesa reproducir
lo de 1 qubit de forma numérica (porque queremos estudiar 2+ qubits).}

\janote{Enunciar el algoritmo.}

\janote{Hablar de que se implementó en Mathematica y por 
aquí meter la info al repositorio y añadir la idea que tienes 
con eso.}

\janote{Concluir con los resultados del numérico.}

\janote{En el siguiente capítulo (último último) hablamos sobre
seguir trabajando en este proyecto para la tesis, 
pero ahora con más qubits, y concluir discutiendo 
que se cumplieron con los objetivos planteados
para este trabajo de prácticas.}
\cpnote{Si, todo bien}

En la sección anterior se enunció que las operaciones
PCE de 1 qubit son 8. Un poco de paciencia y tiempo es lo que 
se requeriría para encontrar, a papel y lápiz, cuáles de esas operaciones son
canales cuánticos. Sin embargo, es de nuestro interés estudiar las 
operaciones PCE de sistemas de más de 1 qubit en el futuro. 
Por ese motivo, se diseñó un método numérico para resolver 
el problema para un sistema de $n$ qubits.
En principio, esto apuntaría a que con el método numérico 
diseñado, el problema está resuelto; sin embargo, 
el número de operaciones PCE, según el número $n$ de qubits
en el sistema, es $2^{4^n-1}$ y la dimensión de las matrices 
a manipular es $4^n\times4^n$. Apenas para 3 qubits, el método numérico
tendría que manipular $\sim 9\times10^{18}$ matrices de 
dimensión $64\times64$, por lo cual esto impone desde 
ya una restricción sobre la capacidad del método numérico. 

Se utilizó el lenguaje de Wolfram para implementar 
rutinas numéricas que construyan la forma matricial 
de las operaciones PCE de 1 qubit y evalúen si son CP,
reproduciendo el procedimiento analítico de la sección anterior. 
Por consiguiente, se programaron las siguientes funciones:
\begin{enumerate}
\item PauliToComp: construye la matriz de cambio de base $P$
para transformar a una matriz de la base de Pauli
a la base computacional.
\item PCE: calcula la forma matricial de una operación PCE en la
base computacional, a partir de las componentes $r_i'$ 
de la matriz de densidad transformada por la operación.
\item Reshuffle: realiza el reordenamiento de \textit{reshuffle} de una matriz.
\item PTest: determina la positividad semidefinida de una matriz 
a partir de sus eigenvalores.
\end{enumerate}
Estas funciones se escribieron en un paquete llamado ``quantumJA.m".

Ya que contamos con las funciones necesarias, mostramos
a continuación el algoritmo que se debe seguir para construir 
una operación PCE y determinar si es CP.

\vspace{1em} \hrule \vspace{1em}
\textbf{Canales cuánticos PCE}.
Algoritmo para construir la forma matricial de una operación 
PCE y determinar si es CP.
\textbf{Entrada:} Elementos de la diagonal de $\qty[\E]_{\sigma}$
(1 si $r_i$ se mantiene invariante, 0 si se borra). 
\textbf{Salida:} ``\textit{True}'' si la operación $\E$ es un canal cuántico,
``\textit{False}'' si no.
\textbf{Utiliza:} PCE, Reshuffle y PTest.
\begin{enumerate}
\item Calcular la forma matricial de $\E$ en la base computacional.
\item Efectuar el \textit{reshuffle} de $\E$ para determinar $D_{\E}$.
\item Determinar si $D_{\E}$ es positiva semidefinida.
\end{enumerate}
\hrule \vspace{1em}

\janote{Por aquí debería ir el link al repositorio. Pendiente.}

Se ejecutó el método numérico y se encontraron los
resultados que se muestran en el \Cref{c:resultados-1q}.
De las 8 operaciones PCE se encontraron 5 canales cuánticos:
(1) la identidad, (2) las tres operaciones 
que deforman la esfera de Bloch a una línea sobre cada uno 
de los ejes, y (3) el canal completamente depolarizante, 
que colapsa la esfera de Bloch a un punto en el origen.

\begin{table}
\centering
\begin{tabular}{|l|c|c|c|}
\hline
\textbf{No. de $\mathbf{r_i}$ invariantes} & \textbf{1} & \textbf{2} &  \textbf{4}\\
\hline 
 & $\E_{xyz}$ & $\E_{xy}$ & $\E_{\{\}}$ \\ 
\textbf{Canales PCE} &  & $\E_{yz}$ &  \\ 
 & & $\E_{xz}$ &  \\ 
\hline
\textbf{Cantidad de canales PCE} & 1 & 3 & 1 \\
\hline
\end{tabular} 
\caption{Canales PCE de 1 qubit. Resultados
obtenidos con el método numérico.}
\label{c:resultados-1q}
\end{table}

Nuestros resultados coinciden con lo que se encuentra 
en la literatura sobre los canales cuánticos 
de 1 qubit \cite{bengtsson_zyczkowski_2017}
\cite{nielsen_chuang_2011}. Bengtsson y 
Źyczkowski exponen que si un canal cuántico $\E$, de 1 qubit,
transforma la esfera de Bloch a un elipsoide, de
tal manera que el vector de Bloch $\vec{r}$ 
se transforma según  
\begin{align}
\vec{r}\ '=\vec{\eta}\cdot\vec{r}+\vec{\kappa},
\end{align}
donde las componentes de $\vec{\eta}$ determinan 
la forma del elipsoide y $\vec{\kappa}$ las coordenadas del 
centro geométrico del elipsoide, entonces
la forma matricial de $\E$ es, en general,
\begin{align}
\E =
\mqty(
1+\eta_z+\kappa_z & 0 & 0 & 1-\eta_z+\kappa_z \\
\kappa_x+i\kappa_y & \eta_x+\eta_y & \eta_x-\eta_y & \kappa_x+i\kappa_y\\
\kappa_x-i\kappa_y & \eta_x-\eta_y & \eta_x+\eta_y & \kappa_x-i\kappa_y\\
1-\eta_z-\kappa_z & 0 & 0 & 1+\eta_z-\kappa_z
).
\end{align}

Similar a nuestro trabajo, ellos estudian el caso en el que 
$\vec{\kappa}=0$, en el que se deben satisfacer 
las condiciones de Fujiwara-Algoet
\begin{align}
\qty(1\pm\eta_z)^2\geq\qty(\eta_x\pm\eta_y)^2,
\label{eq:fujiwara-algoet}
\end{align}
para que la operación $\E$ sea completamente positiva. En el 
caso particular de las operaciones PCE las 
componentes $\eta_i$ deben ser iguales a 1 o 0. 
El subconjunto de los canales PCE está caracterizado por
0, 1 o 3 componentes $\eta_i$ iguales a 1. Por ejemplo, 
cuando $\eta_x=1$ y $\eta_y=\eta_z=0$ se tiene,
sustituyendo en \eqref{eq:fujiwara-algoet}, $1\geq\eta_x^2$. 
Por consiguiente, se puede concluir que
la operación PCE que deforma  a la esfera de 
Bloch a una línea sobre el eje $x$ es un canal cuántico. Por otro lado, 
cuando $\eta_x=\eta_z=1$ y $\eta_y=0$ y se sustituye 
en \eqref{eq:fujiwara-algoet} se tiene $\qty(1\pm\eta_z)^2\geq\eta_x^2$. 
Para la condición en la que se restan los términos dentro 
del paréntesis, en el lado izquierdo de la desigualdad, se obtiene que
es falsa, por lo tanto la operación PCE que deforma a la esfera de 
Bloch a un disco sobre los ejes $x$ y $z$ no es CP. De manera
similar, se puede revisar que el resto de nuestros
resultados (\Cref{c:resultados-1q}) satisfacen las condiciones
de Fujiwara-Algoet.

En resumen, se estudiaron las operaciones PCE de 1 qubit, operaciones 
que borran las componentes $r_i$ de una matriz de densidad 
escrita en la representación de las matrices de Pauli. 
Se implementó un método numérico para resolver
el problema. No obstante, el método está diseñado  
para resolver el problema de sistemas con más de 1 qubit.
Se encontró que 5 de las 8 operaciones PCE son canales cuánticos.
Son canales PCE la identidad, el canal completamente depolarizante 
y los canales que transforman a la esfera de Bloch a una línea sobre 
cualquiera de los 3 ejes. Los resultados coinciden con lo que se
encuentra en la literatura reciente. En el futuro, nos gustaría 
caracterizar a los canales PCE de sistemas de $n$ qubits.
% }}}



