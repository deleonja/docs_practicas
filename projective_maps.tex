\chapter{Mapeos proyectivos}  
% Intro {{{
\janote{Introducción. Motivar y justificar el estudio de las operaciones
que borran componentes de la matriz de densidad de un 1 qubit. Discutir
que entender los canales cuánticos que borran componentes de la 
matriz de densidad ayudaría a entender mejor la teoría 
de las operaciones CPTP y, también, la mecánica cuántica porque al fin y al 
cabo estamos estudiando qué permite y qué no la mecánica 
cuántica.}

\cpnote{La justificacion me gustaria ver en pseudocodigo antes de que la afines. 
Menciona un poco también la relación con los otros mapeos, pero muy por encima, como 
los de Ruskai}

% }}}
\section{El caso de 1 qubit} % {{{
\janote{Introducción sobre los canales cuánticos de 1 qubit. Recordar 
la forma de la matriz de densidad escrita en la representación 
de Pauli y su asociación con la esfera de Bloch.}

\janote{Establecer el problema que nos interesa para 1 qubit:
mapeos que borran y dejan invariantes todas las combinaciones 
posibles de las componentes del vector de Bloch. 
Discutir que la componente $r_0$ la dejamos intacta para 
preservar la traza. Hablar de la representación geométrica 
de las operaciones que estudiamos.}

\janote{Presentar la forma analítica de calcular la representación
en forma de superoperador de alguno de los 8 mapeos posibles. 
A diferencia de lo que ya se hizo en el capítulo anterior, voy a presentar
el cálculo a partir de la forma diagonal del mapeo (también para ir 
dando las herramientas para el algoritmo numérico)}

\janote{Hablar sobre la equivalencia entre los 3 canales que dejan
1 componente invariante del vector de Bloch y sobre la identidad
y el totalmente depolarizante como canales triviales y extremos 
de nuestros canales.}

\janote{Resumir los resultados y comparar con los resultados que 
exponen en el Geometry of Quantum States de la forma general de
cualquier canal cuántico de 1 qubit. Allí tienen la forma general
de cualquier canal cuántico sobre 1 qubit, creo que es un buen
check para mostrar que entendí el caso de 1 qubit.}

\cpnote{Me parece bien, pero porfa no te extiendas de más en este capitulo}

\janote{\h{Aún no tiene mucha forma lo de este capítulo.} Ando 
todavía en la primera iterada.}

Los canales cuánticos de 1 qubit se hacen interesantes porque 
son fáciles de visualizar en la esfera de Bloch, como se presentó
en la sección \ref{sec:CC's-1q-ejemplos}, y eso hace que 
aporten mucha intuición física para entender las operaciones cuánticas. 
Particularmente, recordemos 
las operaciones $\E_z$ y $\E_{xz}$ (Figs. \ref{fig:qtm-op-motivation}
y \ref{fig:QC-ex2}) que borran la componentes en $z$, y 
$x$ y $z$ del vector de Bloch, respectivamente. Ambas operaciones
transforman estados de 1 qubit en estados de 1 qubit. No obstante, 
vimos que $\E_z$ no es una operación completamente positiva porque 
$\E_z\otimes\1 _2$ transforma al estado máximamente entrelazado de 
dos qubits en una matriz que no es una matriz de densidad;
por consiguiente, $\E_z$ no es un canal cuántico. 
Motivados en estas dos operaciones, nos parece interesante 
estudiar las operaciones que borran las componentes del 
vector de Bloch de la matriz de densidad de 1 qubit.

Para ser más precisos, el problema que abordaremos es
el siguiente: determinar el subconjunto de canales cuánticos
del conjunto de operaciones lineales que borran componentes 
del vector de Bloch de la matriz de densidad de 1 qubit.
Para esto, es necesario recordar que la matriz de densidad 
de 1 qubit se puede representar en la base de las matrices 
de Pauli como
\begin{align}
\rho = \frac{\1}{2}+\sum_{i=1}^{3}r_i\sigma_i,
\label{eq:ch3-rho-1q}
\end{align}
con $r_i$ las componentes del vector de Bloch y $\sigma_i$ las 
matrices de Pauli. Por lo tanto, nuestro objetivo es encontrar
las operaciones CP que borran un número arbitrario de 
$r_i$ de \eqref{eq:ch3-rho-1q}. Puesto que una matriz de 
densidad $\rho$ de la forma \eqref{eq:ch3-rho-1q} satisface
(1) $\Tr(\rho)$, (2) $\rho=\rho^{\dagger}$ y (3) $\rho\geq0$, 
una matriz $\rho$ con cualquier número de $r_i$ iguales a cero,
en la forma de \eqref{eq:ch3-rho-1q}, también es una matriz
de densidad. Es decir, la única condición que resta por 
evaluar a las operaciones de nuestro estudio, para determinar
si son canales cuánticos, es la completa positividad. 
En el resto de esta sección presentaremos la manera 
analítica de calcular la forma matricial de una operación
y evaluar si es completamente positiva.
%
%
%Los mapeos que deseamos estudiar son especialmente sencillos
%de visualizar en la esfera de Bloch: mapeos que borran componentes
%arbitrarias del vector de Bloch de una matriz de densidad $\rho$ 
%de la forma \eqref{eq:ch3-rho-1q}. Explotando la representación 
%de la esfera de Bloch los mapeos de nuestro interés son cuatro:
%\begin{enumerate}
%\item La identidad, una operación que deja intacta a la esfera de Bloch.
%\item Tres mapeos que colapsa una dimensión de la esfera de Bloch. Es decir,
%la esfera de Bloch la deforma a un disco. 
%\item Tres mapeos que colapsan dos dimensiones. Es decir, la esfera
%de Bloch se deforma a una línea sobre uno de los ejes. 
%\item Un mapeo que colapsa las tres dimensiones. La esfera de Bloch
%colapsa a un punto en el origen. Este canal se conoce como el 
%completamente depolarizante, dado que es el caso límite $(p=1)$
%del canal depolarizante que presentamos en la sección 2.4.2.
%\end{enumerate}

%\subsection{Cálculo analítico}
%\begin{enumerate}
%\item Se calcula la forma matricial $\qty[\E]_{\sigma}$
%de la operación en la base de las matrices de Pauli, base en 
%la cual la operación es diagonal. 
%\item Se hace un cambio de base $P\qty[\E]_{\sigma}P^{-1}$ 
%para encontrar la forma matricial $\E$ de la operación 
%en la base computacional.
%\item Se calcula $\E^R$ para calcular la matriz de Choi $D_{\E}$
%de la operación.
%\item Se evalúa la positividad semidefinida de $D_{\E}$; si cumple,
%entonces se concluye que $\E$ es un canal cuántico. 
%\end{enumerate}
%Debe notarse que la clase de operaciones
%que estamos estudiando mantiene invariante la traza unitaria,
%Hermiticidad y positividad de la matriz de densidad del sistema de 
%qubits que se está considerando. La única condición que resta 
%evaluar a las operaciones es la de completa positividad. Por
%ejemplo, las operaciones que borran componentes de la matriz 
%de densidad de 1 qubit es un mapeo afín de matrices de 
%densidad de 1 qubit, pero no todas son CP.

Vamos a presentar el procedimiento análitico a seguir para 
determinar si una operación es un canal cuántico desarrollando
de nuevo la operación $\E_z$, pero ahora la haremos con 
el espíritu de implementar una solución numérica, 
que presentaremos en la siguiente sección.
Recordemos que las componentes de la matriz de densidad
de 1 qubit  se transforman bajo la acción de $\E_z$ como
\begin{align}
\qty(1,r_1,r_2,r_3)\longrightarrow\qty(1,r_1,r_2,0),
\end{align}
por lo que la forma matricial de $\E_z$ debe ser diagonal 
en la base de las matrices de Pauli vectorizadas (incluyendo
a $\sigma_0=1$ en ese conjunto).
Denotemos como $\qty[\E_z]_{\sigma}$ a la forma matricial de
$\E_z$ en la base de Pauli. 
Los elementos de la diagonal $(\qty[\E_{z}]_{\sigma})_{ii}$ 
satisfacen la ecuación $r_i'=(\qty[\E_{z}]_{\sigma})_{ii}r_i$, 
con $r_0=1$, por lo que se calcula
\begin{align}
\qty[\E_z]_{\sigma}=
\mqty(
 1 & 0 & 0 & 0 \\
 0 & 1 & 0 & 0 \\
 0 & 0 & 1 & 0 \\
 0 & 0 & 0 & 0 
).
\end{align}
El siguiente paso es hacer un cambio de base para escribir
a $\qty[\E_z]_{\sigma}$ en la base computacional.
Para ello, construimos a la matriz de cambio de base $P$
yuxtaponiendo las matrices vectorizadas de los elementos
de la base de Pauli,
\begin{align}
P=\mqty(
 1 & 0 & 0 & 1 \\
 0 & 1 & -i & 0 \\
 0 & 1 & i & 0 \\
 1 & 0 & 0 & -1 
).
\end{align}
Se efectúa el cambio de base, siguiendo la expresión 
$P\qty[\E_z]_{\vec{\sigma}}P^{-1}$, y tenemos
\begin{align}
\mqty(
 1 & 0 & 0 & 1 \\
 0 & 1 & -i & 0 \\
 0 & 1 & i & 0 \\
 1 & 0 & 0 & -1 
)
\mqty(
 1 & 0 & 0 & 0 \\
 0 & 1 & 0 & 0 \\
 0 & 0 & 1 & 0 \\
 0 & 0 & 0 & 0 
)
\mqty(
 \frac{1}{2} & 0 & 0 & \frac{1}{2} \\
 0 & \frac{1}{2} & \frac{1}{2} & 0 \\
 0 & \frac{i}{2} & -\frac{i}{2} & 0 \\
 \frac{1}{2} & 0 & 0 & -\frac{1}{2} 
)
&=
\mqty(
 \frac{1}{2} & 0 & 0 & \frac{1}{2} \\
 0 & 1 & 0 & 0 \\
 0 & 0 & 1 & 0 \\
 \frac{1}{2} & 0 & 0 & \frac{1}{2} 
).
\end{align}
De esta manera, hemos llegado a la misma forma matricial de $\E_z$
de \eqref{eq:Ez-matrix} (sección \ref{sec:ch2-matrixForm}). 
Lo que falta por hacer es calcular la matriz
de Choi $D_{\E_z}$, siguiendo la ecuación \eqref{eq:R-4ind}, y 
evaluar si es positiva. La positividad se puede evaluar 
calculando los eigenvalores y evaluando que todos 
sean no negativos. De hecho, esto fue lo que hicimos en 
la ecuación \eqref{eq:ch2-Choi-Ez};
se calcularon los eigenvalores de $D_{\E_z}$ y se 
encontró que uno de ellos es igual a $-1/2$, por lo cual $\E_z$ no 
es un canal cuántico.

Por último, vamos a enunciar las operaciones que 
deseamos estudiar. Por practicidad, vamos a nombrar 
a las operaciones de estudio como ``operaciones
que borran componentes de Pauli'', u operaciones PCE
por su nombre en inglés (\textit{Pauli-components-erasing}). 
Las operaciones PCE de 1 qubit que vamos a estudiar 
se muestran en el \Cref{cuadro:operacionesPCE-1q}.
\begin{table}
\centering
\begin{tabular}{|c|l|} 
\hline
\textbf{Operación PCE} & 
\textbf{Transformación de las componentes} $\mathbf{r_i}$ \\
\hline
$\E_{}$ & \hspace{1.2cm}$\qty(1,r_1,r_2,r_3)\longrightarrow\qty(1,r_1,r_2,r_3)$ \\ 
\hline 
$\E_{x}$ & \hspace{1.2cm}$\qty(1,r_1,r_2,r_3)\longrightarrow\qty(1,0,r_2,r_3)$ \\ 
\hline 
$\E_{y}$ & \hspace{1.2cm}$\qty(1,r_1,r_2,r_3)\longrightarrow\qty(1,r_1,0,r_3)$ \\ 
\hline 
$\E_{z}$ & \hspace{1.2cm}$\qty(1,r_1,r_2,r_3)\longrightarrow\qty(1,r_1,r_2,0)$ \\ 
\hline 
$\E_{yz}$ & \hspace{1.2cm}$\qty(1,r_1,r_2,r_3)\longrightarrow\qty(1,r_1,0,0)$ \\ 
\hline 
$\E_{xz}$ & \hspace{1.2cm}$\qty(1,r_1,r_2,r_3)\longrightarrow\qty(1,0,r_2,0)$ \\ 
\hline 
$\E_{xy}$ & \hspace{1.2cm}$\qty(1,r_1,r_2,r_3)\longrightarrow\qty(1,0,0,r_3)$ \\ 
\hline 
$\E_{xyz}$ & \hspace{1.2cm}$\qty(1,r_1,r_2,r_3)\longrightarrow\qty(1,0,0,0)$ \\ 
\hline
\end{tabular}   
\caption{Operaciones PCE de 1 qubit. 
Los subíndices de $\E$ indican las componentes $r_i$ 
del vector de Bloch que la operación borra.}
\label{cuadro:operacionesPCE-1q}
\end{table}

Hemos establecido el problema que vamos a estudiar: determinar
los canales cuánticos del conjunto de las operaciones PCE. Presentamos
el procedimiento analítico para determinar si una operación PCE es 
completamente positiva, y también presentamos todas las operaciones
PCE de 1 qubit. En la siguiente presentaremos el método numérico para 
resolver este problema.

% }}}
\section{Método numérico} % {{{
\janote{Introducción para justificar porqué nos interesa reproducir
lo de 1 qubit de forma numérica (porque queremos estudiar 2+ qubits).}

\janote{Enunciar el algoritmo.}

\janote{Hablar de que se implementó en Mathematica y por 
aquí meter la info al repositorio y añadir la idea que tienes 
con eso.}

\janote{Concluir comparando los resultados del numérico con
lo de la sección anterior.}

\janote{En el siguiente capítulo (último último) hablamos sobre
seguir trabajando en este proyecto para la tesis, 
pero ahora con más qubits, y concluir discutiendo 
que se cumplieron con los objetivos planteados
para este trabajo de prácticas.}
\cpnote{Si, todo bien}

En el futuro próximo, es de nuestro interés estudiar las 
operaciones PCE de sistemas de más de 1 qubit. Por esa 
razón, el método numérico se diseñó para resolver el 
problema para cualquier número de qubits.
En principio, esto podría apuntar a que una vez  se diseñe 
el método numérico el problema está resuelto, sin embargo, 
el número de operaciones PCE según el número $n$ de qubits
en el sistema es $2^{4^n-1}$ y la dimensión de las matrices 
es $4^n\times4^n$. Apenas para 3 qubits, el método numérico
tendría que revisar $\sim 9\times10^{18}$ matrices de 
dimensión $64\times64$, lo que impone desde ya una restricción
sobre la capacidad del método numérico. 

Para implementar el método numérico de diseñaron 
rutinas numéricas en el lenguaje de Wolfram. Se utilizaron
las funciones de Álgebra Lineal de  Wolfram para 
implementar las rutinas que fueron necesarias.
En vista del desarrollo analítico de la sección anterior
se anticipó la necesidad de funciones para:
\begin{enumerate}
\item calcular la forma matricial de una operación PCE en la
base computacional a partir de la transformación de las 
componentes $r_i$ de la matriz de densidad,
\item realizar el procedimiento de \textit{reshuffle} de una matriz,
\item determinar la positividad semidefinida de una matriz.
\end{enumerate}

\subsection{Algoritmo}

Algoritmo para verificar si una operación $\E$ es un canal cuántico.
\textbf{Entrada:} Elementos de la diagonal de $\qty[\E]_{\sigma}$. 
\textbf{Salida:} ``\textit{True}'' si $\E$ es un canal cuántico, 
``\textit{False}'' si no.
\textbf{Necesita:} PauliCh, Reshuffle y PTest.
\begin{enumerate}
\item Calcular la forma matricial de $\E$ en la base computacional.
\item Efectuar el \textit{reshuffle} de $\E$ para determinar $D_{\E}$.
\item Determinar si $D_{\E}$ es positiva semidefinida.
\end{enumerate}

Se construyó el paquete ``quantumJA.m'' en Mathematica 
con las siguientes funciones:
\begin{enumerate}
\item PauliToComp: según el número de qubits en el sistema,
construye la matriz de cambio de base $P$
para transformar una operación $\qty[\E]_{\sigma}$ 
de la base de matrices de Pauli a la base computacional.
\item PauliCh: a partir de la acción de una operación lineal $\E$ 
sobre las componentes de una matriz de densidad $\rho$, 
de un sistema de qubits y
escrita en la base de matrices de Pauli,
calcula la forma matricial de $\E$ en la base computacional.
\item Reshuffle: implementa el reordenamiento de 
\textit{reshuffle} de una matriz $A$ de dimensión $N\times N$,
con $N=2^k, k\in \mathbb{Z}^+$.
\item PTest: evalúa la positividad semidefinida de una matriz a partir
de sus eigenvalores.
\end{enumerate}.

\subsection{Resultados}
Introducimos la siguiente notación para identificar las operaciones de Pauli
de 1 qubit que borran componentes de la siguiente manera:
\begin{align*}
\E_{i},
\end{align*}
donde $i$ es el conjunto de componentes del vector de Bloch
que la operación borra. Por ejemplo, $\E_{xyz}$ es el canal de 
Pauli que borra todas las componentes del vector de Bloch 
de 1 qubit.

Los resultados obtenidos con el método numérico se muestran
en el \Cref{c:resultados-1q}. De las 8 operaciones de Pauli de 1 qubit que 
borran componentes 5 de ellas son CP: (1) la identidad, 
que deja a la esfera de Bloch invariante; (2) las tres operaciones 
que deforman la esfera de Bloch a una línea sobre cada uno 
de los ejes; y (3) la operación que colapsa la esfera de Bloch 
a un punto en el origen. Las tres operaciones de Pauli que borran 
componentes que no satisfacen la condición de completa positividad
son las operaciones que deforman la esfera de Bloch a un 
disco sobre cualquier par de ejes.
\begin{table}
\centering
\begin{tabular}{l|c|c|c}
\textbf{No. de comp. invariantes} & \textbf{1} & \textbf{2} &  \textbf{4}\\
\hline 
 & $\E_{xyz}$ & $\E_{xy}$ & $\E_{\{\}}$ \\ 
\textbf{Canales cuánticos} &  & $\E_{yz}$ &  \\ 
 & & $\E_{xz}$ &  \\ 
\hline
\textbf{Cantidad de canales c.} & 1 & 3 & 1 \\
\end{tabular} 
\caption{Canales de Pauli de 1 qubit que borran componentes. Resultados
obtenidos con el método numérico.}
\label{c:resultados-1q}
\end{table}

Estos resultados son coherentes con lo que se encuentra 
ampliamente en la literatura sobre los canales cuánticos 
de 1 qubit \cite{nielsen_chuang_2011}
\cite{bengtsson_zyczkowski_2017}. Por un lado, Bengtsson y 
Źyczkowski 

% }}}



